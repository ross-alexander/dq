\begin{skill*}{Healer}{1.12}{healer}
The healer skill is pseudo-magical and Healers are able to cure all
physical ills and perform miracles.  It is a highly skilled profession
and Healers are not common but even so their existence means the
health and life span of people in the DQ world are considerably better
than their medieval counterpart.

A healer's empathy often gives them a distaste for causing pain to
others.

A healer will charge whatever their client can afford for their lower
Ranked abilities. The charge for a miracle (the performance of an
ability Rank 8 or greater) will normally exceed 2000 Silver Pennies.

A healer may also use their abilities upon animals which they have
Beastmaster (\S\ref{beastmaster}) familiarity. However non sentients
cannot be resurrected.

The section \S\ref{adventure:health} Health and Fitness details the
effects of injury and illness on patients.

\subsubsection{Field Operations}

Once a healer begins work on curing a patient the condition of the
patient is ``stabilised'' while that curing is continuing. This means
that no Endurance or Fatigue loss will occur for the condition that is
being cured. Other afflictions will be unaffected. After each healing
attempt a pulse effectively passes prior to any other attempt
beginning.

\subsection{Benefits}

A healer gains specific abilities at each Rank as per the following
table:
\begin{Enumerate}
\setcounter{enumi}{-1}
\item
Empathy, Non-tactile empathy (optional)

\item
Cure Infection, Disease, Headaches, Fever

\item
Soothe Pain, Prolong Life

\item
Heal Endurance, Transfer Fatigue

\item
Neutralise Poison, Graft Skin

\item
Repair Muscle, Preserve Dead

\item
Repair Bones

\item
Repair Tissues and Organs

\item
Resurrect the Dead

\item
Regenerate Limbs and Joints

\item
Regenerate Trunk, Head and Vital Organs
\end{Enumerate}

NB. A healer must choose at Rank 0 whether or not to learn Non-tactile
empathy.

\subsection{Restrictions}

A healer must expend as many Fatigue Points as the Rank at which they
acquired the ability they are using (except non-tactile empathy).

A healer may use any of their abilities (with the exception of
resurrection) upon themselves.

A healer must ``lay hands'' (place their hands) on an entity on whom
they are to use any of their abilities (except non-tactile empathy).

A healer has the following modifications to their combat strike chances:

\emph{Tactical Empathy:}

\begin{tabularx}{\linewidth}{lX}
- 1 / 2 Rank & Close Combat strike chance \\
- 1 / 5 Rank & Melee Combat strike chance \\
\end{tabularx}

\emph{Non-Tactile Empathy:}

\begin{tabularx}{\linewidth}{lX}
- 1 / Rank  & Close Combat strike chance \\
-1 / 2 Rank & Melee Combat strike chance \\
\end{tabularx}

\subsection{Ability Descriptions}

\subsubsection{Empathy}
\basechance{automatic}
\timetaken{5 seconds}
When a healer lays on hands they immediately invoke empathy.

A healer uses empathy to identify which of the healing abilities is
required to heal the patient.

The healer automatically detects the surface emotions of the entity
being healed. An entity's surface emotions are those which currently
occupy their conscious mind. The GM informs the healer of the general
feelings of the being with which they have empathy.

\subsubsection{Non-Tactile Empathy}

\basechance{Perception + 10 / Rank}
\timetaken{5 seconds}
If the healer has learnt non-tactile empathy they may attempt to
detect the surface emotions of an entity no more than (2 \x Rank) feet
away from them at a cost of 1 Fatigue Point. If the entity actively
resists then subtract twice the target's Willpower from this success
chance.

\subsubsection{Cure Infection, Disease, Headaches, Fever and Graft Skin}
\label{healer:cure}
\basechance{15 \x Rank + Patient's Endurance}
\timetaken{30 minutes - 2 / Rank}
A healer cures fevers and diseases, neutralises poisons and grafts
skin in much the same manner that medicines and antidotes do.

If the healing attempt is unsuccessful the patient subtracts 10 from
their next die roll to see if they naturally recovers from their
affliction.

\subsubsection{Neutralise Poison}

\basechance{90 + Rank or 50 - 5 \x Difference in Rank (see below)}
\timetaken{5 seconds}
A healer may neutralise the effects of a natural venom or the effects
of a synthetic poison created by an alchemist of equal or lesser
Rank. If a synthetic poison is produced by an alchemist of greater
Rank they must roll the 2nd Base Chance above.

\subsubsection{Soothe Pain}

\basechance{90\% + Rank}
\timetaken{60 seconds - 5 / Rank}
\duration{Rank squared hours}
When a healer uses their soothe pain ability, they numb their
patient's nervous system so that it will not transmit pain sensations
to their brain. The ability also has a soporific effect upon the
patient, so that they will not inadvertently injure themselves while
unable to distinguish hurtful actions. The GM may, at their
discretion, permit the healer to use this ability as if they had fed
or injected their patient with a local or general anaesthetic,
tranquilliser, etc.

\subsubsection{Prolong Life}
\basechance{90\% + Rank}
\timetaken{60 seconds - 5 / Rank}
When a healer uses the prolong life ability add D10 \x (Healer's Rank +
Patient's Endurance) days to the life of their patient. A patient's
life may not be prolonged to over three times their natural life. An
entity with a prolonged life has a reduced chance of resurrection.

\subsubsection{Heal Endurance and Transfer Fatigue}

\basechance{90\% + Rank}
\timetaken{Time: 11 minutes - 1 / Rank}
Heal Endurance will cure the patient of [D + Rank - 5] Endurance
points.  It will not heal damage associated with a specific grievous
injury.

When a healer transfers fatigue the patient gains one Fatigue point
for each point the healer expends (above the fatigue cost to use the
ability).

An entity may never have more Fatigue or Endurance Points than the
value of the relevant characteristic and excess points cured have no
effect upon the patient.

\subsubsection{Repair Damage}

\basechance{90\% + Rank}
\timetaken{50 hours - 3 / Rank}
A healer may repair torn, damaged, or broken muscles, bone, tissues
and organs.  Generally these abilities will be used to repair the
effects of grievous injuries.

At least one half of a muscle, bone, or organ to be repaired must
remain in the patient's body if the healer is to use one of these
abilities. Tissue may be grown from existing material in or on the
patient's body.

A healer can act as a cosmetic surgeon. First, they sedate their
patient with the soothe pain ability. They then slice and reshape the
skin, muscles, and bones which are deemed unsightly, and make them
whole with the appropriate repair ability. Unless the healer has also
learnt regeneration, it is best that they work with a partner.

\subsubsection{Preserve Dead}

\basechance{90\% + Rank}
\timetaken{60 minutes - 5 / Rank}
A healer can suspend the time limit on resurrection by preserving the
dead body of a being. Each time a healer uses the preserve dead
ability, the body will not ``age'' for a number of days equal to the
healer's Rank. This ability may be repeated by the same healer on the
same body.

\subsubsection{Resurrection}

\timetaken{60 minutes - 5 / Rank}
\emph{Base Chance:} Patient's Endurance + 8\% / Rank \\
\hspace*{2em} minimum = Rank \\
\hspace*{2em} maximum = 90 + Rank \\
\hspace*{2em} regardless of the total modifiers.

\emph{Base Chance Modifiers:}\\
\begin{tabularx}{\linewidth}{rX}
+5	& healer is life aspected \\
+5	& patient is life-aspected \\
-5	& healer is death-aspected \\
-5	& patient is death-aspected \\
-1	& per year (or fraction) the patient's life has been prolonged \\
-1	& per day of regeneration it would normally require to make the body whole \\
-10	& body is whole but has suffered Damage Points equal to or greater
 than twice its Endurance (including after death damage) \\
-10	& per unsuccessful resurrection attempt since patient died \\
\end{tabularx}

\begin{effects}
A resurrection will cure the body of all ills and damage done to it
provided that Rank 8 healing or below would been sufficient had the
patient been alive. For example, poison and non-specific wounds will
be cured automatically.

If the resurrection is successful, the patient is resurrected with
their body whole. Their Endurance Characteristic is decreased by one,
though all of their other characteristics remain as before they died.

After a resurrection the patient will have 1 Endurance point and 0
Fatigue. The Endurance is considered to be grievous damage and the
Fatigue loss is deemed to be due to tiredness. This means that the
Fatigue loss may only be recovered by sleep, rest, hot meals or some
form of fatigue transfer and the endurance loss may be cured by a
Healer, magic or by letting the body heal itself naturally.

If the resurrection is unsuccessful the patient is not resurrected and
their Endurance characteristic is decreased by one. The body is
preserved for one full day after the attempt. When an entity's
Endurance is reduced to zero or less, that entity may no longer be
resurrected.

If the roll for resurrection is equal to or greater than (90 + Rank),
the healer has summoned a malignant spirit, rather than the patient's
life-force.  The spirit will drain the healer's Endurance
characteristic by [D - 5]. The spirit will then return to the
netherworld.
\end{effects}

\smallskip
\emph{Restrictions:}
\begin{Enumerate}
\item
A healer may attempt the resurrection of an entity who is less than 10
\x Rank hours dead.

\item
A healer must have a body part at least the size of a torso to attempt
the resurrection of an entity. A healer will not succeed if they
attempt the resurrection of a living being from a severed body part
(there is only one life force). If a body is completely destroyed
(perhaps burned), which prevents the resurrection of the entity, that
thing may become a revenant.

\item
If the patient has wounds that require regeneration (Rank 9 or 10)
healing, these need to be healed separately.

\item
Most vital organs will need to be healed prior to the resurrection
otherwise the body will died again immediately (the notable exception
being the eyes).

\item
The healer need not know what the patient looked like since the
healing of the body is governed by its own characteristics. Hence any
changes that had been made to the body (for example facial changes or
embedded items) will be gone after the resurrection.

\item
A player may take no action with their character's dead body.

\item
In rare instances a healer may be able to resurrect a life force into
a different body. The resurrected entity has the physical
characteristics of the new body and the mental characteristics from
the life force.  It will take some months for the entity to get used
to the new body and this will effect base chances of physical
abilities.  The GM will advise the specifics..

\item
A body that has been animated (\eg Zombie) may still be resurrected
provided it is no longer animated and all the other conditions have
been made (for example length of death, condition, etc.). Note that
the Healer Preserve Dead will not affect Zombies.
\end{Enumerate}

\subsubsection{Regeneration}

\basechance{90\% + Rank}
\timetaken{1 week (per organ or body part)}
A healer can regenerate every portion of an entity's body including
vital organs or severed body parts.

A healer must regenerate each vital organ or body part separately. An
entity's vital organs are the heart, stomach, viscera (liver, small
and large intestines and kidneys), genetalia, brain, and eyes.

A regenerated vital organ will immediately begin to function if enough
of the rest of the entity's body is in working order. Otherwise, the
vital organ will be dormant until the healer can repair or regenerate
the necessary body parts. The regeneration time does not need to be
consecutive, but the damaged part will not function until the
regeneration is complete.

\subsection{Potion Costs}
\label{healer:potions}
A healer can manufacture certain potions in conjunction with an
alchemist. See Alchemist (\ref{alchemist}) for more details.

\begin{tabularx}{\linewidth}{Xr}
			& \textbf{Base} \\
\textbf{Potion}		& \textbf{Value} \\
Cure Disease		& 600 \\
Cure Fever		& 600 \\
(Graft) Skin Salve	& 650 \\
Neutralise Poison (specify type) & 700 \\
Cure Endurance Points	& 1500 \\
Prolong Life		& 2500 \\
\end{tabularx}
\end{skill*}

\begin{table*}[h]
\begin{center}
\begin{tabular}{|l|l|l|l|} \hline
\textbf{Ability}	& \textbf{Base Chance}	& \textbf{Time} \\ \hline
Empathy			& Automatic		& 5 seconds \\
Non-tactile empathy 	& PC + 10 / Rank	& 5 seconds \\
Cure Infection, Disease, Headaches, Fever \& Graft Skin	& (15 \x Rank) + Patients EN	& 30 minutes - 2 / Rank \\
Soothe Pain		& 90 + Rank	& 60 - 5 / Rank seconds \\
Prolong Life		& 90 + Rank	& 90 + Rank or 50 - 5 \x Difference in Rank \\
Heal Endurance		& 90 + Rank	& 11 minutes - 1 / Rank \\
Transfer Fatigue	& 90 + Rank	& 11 minutes - 1 / Rank \\
Neutralise Poison	& 90 + or 50 - 5 \x Difference in Rank	& 5 seconds \\
Repair Muscle, Bones, Tissues and Organs & 90 + Rank	& 50 hours - 3 / Rank \\
Preserve Dead 		& 90 + Rank	& 60 seconds - 5 / Rank \\
Resurrect the Dead	& special	& 60 minutes - 5 / Rank \\
Regenerateration	& 90 + Rank	& 1 week \\  \hline
\end{tabular}
\end{center}
\end{table*}
\newpage
