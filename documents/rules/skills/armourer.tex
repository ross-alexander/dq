\begin{skill}{Armourer}{1.1}{armourer}

\subsection{Restrictions}

The skill is related to that of weaponsmith, and an armourer who is a
more skilled weaponsmith expends only three quarters of the necessary
Experience Points to acquire or improve this skill.  The reverse is
also true.  An armourer's progress in their skill is inhibited by a
low Manual Dexterity, and aided by a high Manual Dexterity.  An
armourer has an increased Experience Point cost of 5\% for each point
of Manual Dexterity less than 16.  An armourer decreases their
Experience Point cost by 5\% for each point of Manual Dexterity greater
than 20.

\subsection{Benefits}

\subsubsection{An armourer acquires the ability to make one category of
armour every two ranks.}

Some categories require other categories as prerequisites and cannot
be learned before their prerequisites.  All armourers begin with the
cloth category at rank 0.

%\subsubsection{Prerequisites}

\begin{tabularx}{\linewidth}{@{}l@{\hspace{0.5em}}Xr@{}}
\multicolumn{3}{c}{\textbf{Categories} \hfill \textbf{Prerequisites}} \\
1 &	Cloth						& None  \\
2 &	Leather (leather, soft leather and furs)	& Cloth \\
3 &	Scale (scale and full scale)			& Cloth \\
4 &	Chain mail					& Cloth \\
5 &	Partial plate					& Chain \\
6 &	Plate I (full plate and heavy plate)		& Chain \\
7 &	Plate II (improved plate, all jousting armour)	& Plate I \\
8 &	Dragon skin					& Scale, \\
  &							& Leather \\
9 &	Mithril						& Chain \\
\end{tabularx}

An Armourer may gain additional categories after achieving rank 10 by
the expenditure of 10,000 Experience Points per category.

\subsubsection{An Armourer can make increasingly effective armour as their
rank increases.}


An armourer may positively affect any of the 4 attributes of armour
(Weight factor, Protection, Agility Modifier and Stealth Modifier) or
any combination thereof.  Some of the attributes are harder to affect,
and this is reflected in the number of ranks an Armourer must have
to do so.  Also, some of the attributes have maximums (\eg the
Agility Modifier may not be decreased beyond 0).  The ranks required
and the attribute maximums are:

\begin{Description}

\item[Weight] 1/2 a factor per 3 full ranks. Never lighter than WT 1.
(This attribute may not be affected for the cloth, leather or mithril
categories).

\item[Protection] +1 per 4 full ranks.  This attribute may not be
affected for cloth, furs or soft leather, and no more than 1 additional
point of protection may be added to hard leather).

\item[Agility Modifier] 1 per 6 full ranks.  Never better than 0.

\item[Stealth Modifier] +1\% per rank.  Never better than +5\%.

\end{Description}

\textbf{Note:} These effects are not cumulative.  For example a rank 7
Armourer could make a suit of armour with 1 less weight factor and 1\%
better stealth, or 1/2 a weight factor less and 1 point more
protection, or any of the other non-cumulative combinations.  An
Armourer may always make a suit of armour at a lower effective rank
than their true rank.

Armour statistics shown on the Alusian Armour Chart are for armours
manufactured with an effective rank of 0, \ie of the mass-produced,
off the peg variety.  The Armourer who made them may have been of
greater rank but the level of skill used was elementary.

\subsubsection{The time and cost required for an Armourer to construct
a suit of armour is dependent on the effective Rank used and the
category of armour.}

\begin{Description}
\item[Time]
The time required to construct a suit of armour is the following --
1 + Rank / 2 \x number of days below.

\begin{tabularx}{\linewidth}{Xr}
Cloth or leather  & \textonehalf{} day \\
Scale & 4 days \\
Chain mail & 6 days \\
Partial plate & 10 days \\
Plate I & 12 days \\
Plate II & 15 days \\
Dragon skin & 16 days \\
Mithril & 20 days \\
\end{tabularx}

The fitting time for the armour (the time spent with the Armourer by
the wearer-to-be) is a number of hours equal to the base number of day
(\eg 6 hours to fit a suit of chain mail).  The hours need not be
consecutive but all must be done in the first half of manufacture
time.

\item[Cost] 80\% of the Base Cost as shown on the Armour Chart \x
(Effective Rank + 1) silver pennies.  \textbf{Note:} this is the cost
to the Armourer, not the sale price.

\end{Description}

\subsubsection{Fixing and Modifying Armour}

The time taken to repair a suit of armour damaged by a Grievous blow,
or to modify a suit to fit a new, but appropriately sized wearer, is
usually no more than the armour's Base time, for example 6 days for a
suit of chain mail.  The cost of the repairs or modifications is
usually 5\% -- 10\% of the original cost of the armour.  The Armourer
who is repairing the damage must be of at least equal rank to the
Effective Rank with which the armour was made.

\subsubsection{An Armourer is treated as a Merchant of their armourer
rank when attempting to buy or value armour from categories with
which they are familiar.}

If the armourer is not familiar with an armour category they act as a
Merchant of half their rank (rounded down).

\subsection{Costs}

\subsubsection{An Armourer can only perform their skill in a properly
maintained workshop.}

It costs 2000 silver pennies to construct a workshop and 500 silver
pennies per year to maintain it with tool and materials.  A basic tool
kit will cost (100 + 100 \x rank) silver pennies.  It costs only 20\%
of the above amount to add to a Weaponsmith's workshop so as to make
it usable by an Armourer as well.  The reverse is also true.  A
workshop may be rented at the rate of 10 silver pennies a day.

\end{skill}
