\begin{skill}{Spy}{1.1}{spy}

Spies practice their trades covertly, in order to avail themselves of
the well-guarded wealth of the powerful. The spy represents themselves
as one worthy of their victim's trust to gain access to valuable
information.  They will continue their impersonation until the victim
is sucked dry of everything of value or until the spy is discovered.

If a spy character wishes to use their skill while not accompanied by
the rest of the party, the GM should run a solo adventure (unless the
task the spy sets themself is very easy).  A spy had best not be
captured after discovery: the traditional punishment for an exposed
spy was to draw and quarter the prisoner.

\subsection{Restrictions}

A spy must be able to read and write in one language at Rank 4 if they
wish to advance beyond Rank 2.

When a character is both a spy and a thief, the player may use the
better of the two percentages to perform a given ability.

\subsection{Benefits}

\subsubsection{If a character's Rank as a spy is greater then their
Rank as a thief, the character expends one-half the necessary
Experience Points to acquire or improve the latter skill.}

The reverse is also true.

\subsubsection{A spy can pick locks or open safes with the aid of their
tools.}
\label{spy:picklocks}

The time a spy must spend to implement the pick lock ability is (240 -
20 \x Rank) seconds, and (30 - 2 \x Rank) minutes to use the open safe
ability.

If the GM's roll on percentile dice is equal to or less than the
success percentage the spy has opened the safe or picked the lock. If
the roll is greater than the success percentage, the safe or lock
resists the spy's best efforts.  If any trap remains in place when a
spy attempts to open a safe or pick a lock, it is triggered by that
action.

\begin{Description}

\item[For Spy to Pick Lock] (MD + 4 \x Rank) - (6 \x Lock Rank)

\item[For Spy to Open Safe] (MD + 3 \x Rank) - (7 \x Safe Rank)

\end{Description}

\subsubsection{A spy may attempt to detect traps and should the spy
succeed, may try to remove them.}
\label{spy:removetraps}

A spy may make one attempt to detect traps (which requires 10 seconds)
in a particular location per day.  A spy must spend (24 - 2 \x Rank)
minutes to use the remove trap ability.

The GM must make one percentile roll for each trap to see if the spy
detects it. If the roll is less than or equal to the success
percentage, the spy notices the location of the trap. If the roll is
above the success percentage, they remain blissfully unaware of the
trap's presence.

\begin{Description}

\item[For Spy to Detect Trap] (2 \x Perception + 7 \x Rank)

\item[For Spy to Remove Trap] (MD + 7 \x Rank) - (5 \x Trap Rank)

\end{Description}

When a spy attempts to remove a trap, the GM rolls percentile dice.
If the roll is less than or equal to the success percentage the spy
has removed the trap without triggering it. If the spy has a trap
container, they may store the removed trap. If the GM's roll is
greater than the success percentage, the trap is triggered (see
\S\ref{mechanician:traps}).

\subsubsection{A spy can sometimes detect a secret or hidden aperture.}

Any character can try to find a secret or hidden aperture if they spend
time sounding and searching the appropriate wall, floor, or ceiling. A
spy has a (2 \x Perception + 5 \x Rank)\% chance of noticing that a
secret or hidden aperture is within (5 + Rank) feet of them.

If the GM's roll on percentile dice is equal to or less than the
success percentage, the spy senses that at least one hidden or secret
door is in their detection area (but is not told how many).  If the
roll is greater than the success percentage, the spy does not notice
the aperture(s).

\subsubsection{A spy can attempt to pick the pocket of another being
without being detected.}
\label{spy:pickpockets}

\index{pickpocket, spy!pickpocket}

A spy has a base success percentage equal to (3 \x Manual Dexterity +
6 \x Rank)\% to pickpocket a being.  The following modifiers are
applied to the success percentage:

\smallskip
\begin{tabularx}{\linewidth}{Xr}
The victim is unconscious		& +50\% \\
The victim is sleeping or stunned	& +25\% \\
The victim cannot see well in current circumstances (\eg human at night) & +10\% \\
The victim is inebriated		& +5\% \\
The pickpocket attempt is made in an uncrowded area and the victim has at least a slight suspicion of the spy's intentions & -15\% \\
The object to be pickpocketed is in a sealed pocket, pouch or compartment & -20\% \\
The object to be pickpocketed is affixed to the victim's person or is something used constantly during the day by the victim	& -30\% \\
The object to be pickpocketed makes noise when removed	& -25\% \\
The victim wears metal armour or garments		& -5\% \\
The victim is an assassin, thief or spy: Subtract (5 \x Victim's Rank)\% & \\
\end{tabularx}

It is assumed that the spy attempting to pickpocket is not handicapped
by their physical condition; if they are, the GM should modify the
success percentage accordingly.

If the GM's roll of percentile dice is equal to or less than the
success percentage, the spy filches the object they desire without the
victim noticing.  If the roll is between one and two the success
percentage, the spy is detected by the victim just after the object
has been removed from its storage place. If the roll is equal to or
greater than twice the success percentage, the spy is caught with
their hand in the victim's pocket.

\subsubsection{A spy will develop a photographic memory as they gain
experience.}

A spy's success percentage to employ their photographic memory ability
is (2 \x Perception + 12 \x Rank)\%.  A spy may use the ability
without error for up to (1 + Rank) days.  When a spy uses the ability
after the error-free time limit is expired, reduce their Rank for
success percentage calculation (only) by one for each day over that
time limit.

If the GM's roll on percentile dice is equal to or less than the
success percentage, the spy can recall visual details, such as those
of a room or a piece of parchment, etc., if they observed it for the
requisite length of time.  A spy must have observed the object in
question for (120 - 10 \x Rank) seconds to use the ability.  If the
roll is greater than the success percentage, the spy's memory has more
and more gaps in it as the roll approaches 100.  If the spy is
attempting to recall past their error-free time limit, the GM
introduces erroneous information into the memory gaps as the roll
approaches 100.

A spy tests their photographic memory ability whenever they try to
verbally describe an object or place, whenever they call on their
memory to gain a mental image of the object or place, or whenever they
record it in writing. If a spy fails to recall an object or place
once, they may not use the ability again to try to recall the image of
that object or place.

\subsubsection{A spy increases their chance of performing an action
involving stealth by 2\% per Rank.}

\index{stealth!spy}

\subsubsection{A spy may use their photographic memory ability to recall
spoken phrases.}

Even if a spy does not know the language used, they can reproduce the
phrases phonetically.  Additionally, when a spy concentrates for (60 -
5 \x Rank) seconds, they can extend their range of vision and hearing
to (100 + 5 \x Rank)\% of what it normally is.

\end{skill}
