\begin{skill}{Thief}{1.1}{thief}

Thieves practice their trades covertly, in order to avail themselves
of the well-guarded wealth of the powerful.  The thief has a task to
accomplish: the (hopefully) undisturbed removal of property from a
supposedly secure place of storage.  A thief usually seeks monetary
rewards for their efforts, and a thief cultivates contacts in the
underworld of their area of operations.  These contacts will enable
them to discover where the choicest items are stored, and aid them in
disposing of their ill-gotten gains.

If a thief character wishes to use their skill while not accompanied
by the rest of the party, the GM should run a solo adventure (unless
the task the thief sets themself is very easy). A thief who is caught
in the act of burglary is liable to the stiff penalties of medieval
times: a hand is removed for the first (known) offence, a second time
merits the removal of the other hand or the eye opposite the missing
hand, with a greater degree of dismemberment for each succeeding
offence.

\subsection{Restrictions}

A thief must be able to read and write in one language at Rank 3 if
they want to advance beyond Rank 3.

When a character is both a spy and a thief, the player may use the
better of the two percentages to perform a given ability.

\subsection{Benefits}

\subsubsection{If a character's Rank as a thief is greater then their
Rank as a spy, the character expends one-half the necessary Experience
Points to acquire or improve the latter skill.}

The reverse is also true.

\subsubsection{A thief can pick locks or open safes with the aid of
tools.}
\label{thief:picklocks}

The time a thief must spend to implement the pick lock ability is (120
- 10 \x Rank) seconds, and (15 - Rank) minutes to use the open safe
ability.

If the GM's roll on percentile dice is equal to or less than the
success percentage the thief has opened the safe or picked the
lock. If the roll is greater than the success percentage, the safe or
lock resists the thief's best efforts.  If any trap remains in place
when a thief attempts to open a safe or pick a lock, it is triggered
by that action.

\begin{Description}

\item[For Thief to Pick Lock] (2 \x  MD + 6 \x  Rank) - (6  \x  Lock  Rank)

\item[For Thief to Open Safe] (2 \x MD + 5 \x Rank) - (7 \x Safe Rank)

\end{Description}

\subsubsection{A thief may attempt to detect traps and should the thief
succeed, may try to remove them.}
\label{thief:removetraps}

A thief may make one attempt to detect traps (which requires 10
seconds) in a particular location per day.  A thief must spend (12 -
Rank) minutes to use their remove trap ability.

The GM must make one percentile roll for each trap to see if the thief
detects it. If the roll is less than or equal to the success
percentage, the thief notices the location of the trap. If the roll is
above the success percentage, they remain blissfully unaware of the
trap's presence.

\begin{Description}

\item[For Thief to Detect Trap] (Perception + 11 \x Rank)

\item[For Thief to Remove Trap] (2 \x MD + 11 \x Rank) - (5 \x Trap Rank)

\end{Description}

When a thief attempts to remove a trap, the GM rolls percentile dice.
If the roll is less than or equal to the success percentage the thief
has removed the trap without triggering it. If the thief has a trap
container, they may store the removed trap. If the GM's roll is
greater than the success percentage, the trap is triggered (see
\S\ref{mechanician:traps}).

\subsubsection{A thief can sometimes detect a secret or hidden
aperture.}

Any thief can try to find a secret or hidden aperture if they
spend time sounding and searching the appropriate wall, floor, or
ceiling. A thief has a (2 \x Perception + 5 \x Rank)\% chance of
noticing that a secret or hidden aperture is within (5 + Rank) feet of
them.

If the GM's roll on percentile dice is equal to or less than the
success percentage, the thief senses that at least one hidden or
secret door is in their detection area (but is not told how many).  If
the roll is greater than the success percentage, the thief does not
notice the aperture(s).

\subsubsection{A thief can attempt to pick the pocket of another being
without being detected.}
\label{thief:pickpockets}

A thief has a base success percentage equal to (3 \x Manual Dexterity
+ 6 \x Rank)\% to pickpocket a being.  The following modifiers are
applied to the success percentage:

\smallskip
\begin{tabularx}{\linewidth}{Xr}
The victim is unconscious		& +50\% \\
The victim is sleeping or stunned	& +25\% \\
The victim cannot see well in current circumstances (\eg human at night) & +10\% \\
The victim is inebriated		& +5\% \\
The pickpocket attempt is made in an uncrowded area and the victim has at least a slight suspicion of the thief's intentions & -15\% \\
The object to be pickpocketed is in a sealed pocket, pouch or compartment & -20\% \\
The object to be pickpocketed is affixed to the victim's person or is something used constantly during the day by the victim	& -30\% \\
The object to be pickpocketed makes noise when removed	& -25\% \\
The victim wears metal armour or garments		& -5\% \\
The victim is an assassin, thief or spy: Subtract (5 \x Victim's Rank)\% & \\
\end{tabularx}

It is assumed that the thief attempting to pickpocket is not
handicapped by their physical condition; if they are, the GM should
modify the success percentage accordingly.

If the GM's roll of percentile dice is equal to or less than the
success percentage, the thief filches the object they desire without
their victim noticing.  If the roll is between one and two the success
percentage, the thief is detected by the victim just after the object
has been removed from its storage place. If the roll is equal to or
greater than twice the success percentage, the thief is caught with
their hand in the victim's pocket.

\subsubsection{A thief will develop a photographic memory as they
gain experience.}

A thief's success percentage is (Perception + 10 \x Rank)\%.  A thief
may use the ability without error for up to (1 + Rank) days.  When a
thief uses the ability after the error-free time limit is expired,
reduce the Rank for success percentage calculation (only) by one for
each day over that time limit.

If the GM's roll on percentile dice is equal to or less than the
success percentage, the thief can recall visual details, such as those
of a room or a piece of parchment, etc., if they observed it for the
requisite length of time.  A thief must have observed the object in
question for (240 - 20 \x Rank) seconds to use the ability.  If the
roll is greater than the success percentage, the thief's memory has
more and more gaps in it as the roll approaches 100.  If the thief is
attempting to recall past their error-free time limit, the GM
introduces erroneous information into the memory gaps as the roll
approaches 100.

A thief tests their photographic memory ability whenever they try to
verbally describe an object or place, whenever they call on their
memory to gain a mental image of the object or place, or whenever they
record it in writing. If a thief fails to recall an object or place
once, they may not use the ability again to try to recall the image of
that object or place.

\subsubsection{A thief increases their chance to perform an action
involving stealth by 1\% per Rank.}

\index{stealth!thief}

\subsubsection{A thief can, as long as they may find a purchase
sufficient to bear their weight, climb any structure.}

\index{climbing!thief}

The success chance when climbing on a structure not made for that
purpose is (4 \x MD + 10 \x Rank) - (Structure Height in Feet / 10)\%.
Round the structure height down.  If the GM's roll is greater than the
success percentage, the thief has fallen in climbing the structure.
To determine the height at which the thief falls, roll D100.  Round
the number off to the nearest 10\% (a roll of 5 is rounded down), and
multiply the height the thief sought to attain by that percentage.
See \S\ref{falling} for falling damage.

\end{skill}