\begin{college}[1.1]{rune}{Rune Magics}{RU}

The College of Rune Magics is concerned with the use of special
symbols of power to shape mana into desired forms.  A Rune is a
graphic symbol representing some actual, elemental, or mystical force.
In rare cases, additional Runes may be developed or discovered which
employ parts of existing Runes.  However, much of the power of the
Runes derives from their constant usage over many centuries, and most
useful Runes will be known to all Adepts of this College (or at least
be readily available to them with very little research).

In addition to the power of the Runes themselves, part of the power of
this College derives from the use of special materials to construct
the Runewands and Runesticks into which the Runes are usually
inscribed. The Runewand table and the Runestick Chart describe the
special properties of various types of Runewands and Runesticks.

\subsection{Restrictions}

Adepts of the College of Rune Magics may use their talent magic
without restriction and may use some spells by merely inscribing the
appropriate Rune on an item to be enchanted.  In most cases, spells
and rituals of this College require the Adept to employ Runesticks or
their personal Runewand in casting the spell or performing the ritual.

The MA requirement for this College is 18.

\subsection{Runewards and Runesticks}

As part of their initiation into the mysteries of this College, the
Adept is required to prepare a Runewand for themselves.  Usually, one or
more of their teachers will participate in the endeavour as well.  If
the Runewand being manufactured is of exceptionally costly materials,
the Adept will be required to go into debt to pay for those materials,
but in most cases the material will be of some cheap, common wood, and
the Adept's labour during their apprenticeship will be sufficient to
cover the cost.

Runewands are of three types: Rods, Staffs, and Sceptres.  All three
operate in basically the same manner.  However, in addition to its
magical properties, a Staff may be used as a normal weapon having
exactly the same characteristics as a quarterstaff.  A Sceptre may
also be used as a weapon having the characteristics of a ceremonial
mace. A Rod may never be used as a weapon, since it is often nothing
more than a switch or hollow tube, looking much like the traditional
magic wand.

When a character is initiated into the College of Rune Magics, the
character's player rolls D100 and consults the Runewand Table to
determine the type of Runewand the Adept receives from his
teachers. An Adept may later equip themselves with a different type of
Runewand or create (or purchase) another Runewand of the same type to
replace a previously possessed Runewand that has been damaged,
destroyed, or stolen.  An Adept may own any number of Runewands, but
may use only one at a time.

Runewands are created by performing the Ritual of Fashioning Runewand
(Q-2). Runewands are considered magical weapons for all purposes and
definitions.

In order to use a Runewand manufactured by someone else, the Adept
must successfully read the Runes inscribed on the Runewand, using
Talent T-3.  If they fail to do so, they may not use the
Runewand. Even if they successfully read the Runes inscribed on a
Runewand which they did not manufacture, the Adept still suffers a
penalty when using the Runewand. The Base Chance for any spell or
ritual performed with the aid of that Runewand is reduced by 20.

Runesticks are small sticks carved of various woods or soft materials
(which do not interfere with the flow of mana) and incised with Runes
appropriate to the purpose of the Runesticks.  Unlike Runewands,
Runesticks are not multi-purpose tools which can be used for a variety
of spells or rituals.  They are specifically created to work with a
single spell or ritual.  Exception: the Warding Rune is used in a
variety of spells and rituals, and Runesticks containing this Rune may
be used in any of them.  Runesticks are fashioned and prepared using
the Ritual of Fashioning Runesticks (Q-1) and may be manufactured of
any material listed on the Runestick Chart.

An Adept may use Runesticks fashioned by someone else, but they must
first successfully read the Runes incised on them.  The Base Chance is
reduced by 10 when an Adept attempts to perform a ritual or cast a
spell with Runesticks not of their manufacture.

In some cases, the Adept may have to draw or carve a Rune into some
object to be enchanted instead of using Runesticks or a Runewand to
perform the magic.  In order to write the Rune, the Adept may use any
substance that will mark the surface of the object to be enchanted.
However, some substances will work better than others at creating the
desired enchantment.  Any tool may be used to carve a Rune into a
substance, so long as the tool is hard enough to do the job and it is
not composed of Cold Iron. Exception: the Adept may use a tool
containing Cold Iron if the Cold Iron is neutralised as per
\S\ref{magic:restrictions}.

Rune Mages are not expert wood carvers.  The minimum size, beyond
which the stick would be too small to inscribe Runes on, is 15cm
\x 2cm \x 1cm and weighs 2oz.

\subsection{Base Chance Modifiers}
\label{rune:modifiers}

The Base Chance of performing a talent, spell, or ritual of the
College of Rune Magics is modified by the addition of the following
numbers.

The talent, spell or ritual requires the use of Runesticks, and the
Runesticks used by the Adept are:

\begin{tabularx}{\linewidth}{Xr}
Made of Gilded metal   &  20 \\
Made of Silvered metal &  15 \\
Made of Mistletoe      &  10 \\
Made of Ashwood        &   8 \\
Made of Oak            &   8 \\
Made of Cedarwood      &   5 \\
Made of Aspenwood      &   3 \\
Made of Chestnut       &   3 \\
Made of Pinewood       &  -5 \\
Made of Yarrow         &  -5 \\
Made of Yew            &  -5 \\
Manufactured by someone other than the Adept  & -10 \\
\end{tabularx}

The talent, spell or ritual requires the use of a Runewand, and the
Runewand used by the Adept is:

\begin{tabularx}{\linewidth}{Xr}
A Truesilver Sceptre           &  25 \\
A Gilded Sceptre               &  22 \\
A Silver Sceptre               &  20 \\
A Copper Rod                   &  18 \\
An Ebony Rod                   &  14 \\
An Ivory Rod                   &  12 \\
An Ash Staff                   &  10 \\
A Cedar Rod                    &  10 \\
A Blackthorn Staff             &   8 \\
A Bronze Sceptre               &   8 \\
An Oak Staff                   &   5 \\
A Willow Rod                   &   2 \\
Manufactured by someone other than the Adept & -20 \\
\end{tabularx}

The talent, spell or ritual requires the drawing of a Rune on an
object to be enchanted and the substance used to draw the Rune is:

\begin{tabularx}{\linewidth}{Xr}
Fresh blood from a dragon           & 50 \\
Fresh blood from a member of a character race (costs 1 fatigue) & 10 \\
Ink impregnated with particles of platinum* & 15 \\
Ink impregnated with particles of gold** & 10 \\
Ink impregnated with particles of silver*** & 5 \\
Fresh blood from a mammalian being & 5 \\
\end{tabularx}

*Average cost of 22 Silver Pennies for enough ink to draw one Rune
(\ie to cast one spell).

**Average cost of 16 Silver Pennies for enough ink to draw one Rune.

***Average cost of 5 Silver Pennies for enough ink to draw one Rune.

In all cases, the weight of one vial with sufficient ink to draw one
Rune is 5 ounces. It normally takes one minute to draw a Rune.

In most cases, these modifiers are not cumulative.  As an exception,
modifiers for the type of Runewand or Runestick used are added to the
modifiers pertaining to the substance used as an ink.

In addition, the Cast Chance is affected by all modifiers listed in
\S\ref{magic:modifiers}, except that the number of hours spent
preparing a spell has no effect on the Cast Chance of the spell, as is
the case in other Colleges.  Rune Magicians may not engage in Ritual
Spell Preparation as outlined in \S\ref{magic:preparation}.

The Rank of an Adept in the Ritual of Fashioning Runewand and the
Ritual of Fashioning Runestick do not affect the efficacy of a spell
cast using the Runesticks or Runewand created via those rituals. It
affects only the actual performance of the rituals of fashioning.

\subsection{Talents}

\begin{talent}[T-1]{Read Dead Languages}

\multiple{150}
\begin{effects}
All Adepts of the College of Rune Magics have a 10\% chance of knowing
a dead language or dialect encountered at Rank 0.  There is a 40\%
chance (+ 3 / Rank) that the Adept will be able to puzzle out the
language if they do not know it.  If the Adept puzzles out or knows
the language (from clues such as root words in known languages), they
will have rank with the language equal to half their Rank with this
talent (round up) after spending 20 hours (-30 minutes / Rank with
this talent) mastering the nuances of the language. A dead language
may be studied in this manner via written word (tomes, inscriptions,
etc.), and hence the language is known only in written form. There
must be sufficient text to allow the Adept to achieve the requisite
Rank (players should record the languages and rank they know using
this talent, and also those that they have failed to master).

An Adept may increase Rank with a dead language that they have
mastered by spending one week per Rank practising the language (at no
Experience cost) until they know it at the same Rank as their Rank
with this talent. If they wish to gain Rank with a dead language which
is greater than their Rank with this talent, they must follow the
normal procedure to gain Rank with a language as described in
\S\ref{languages}.
\end{effects}
\end{talent}

\begin{talent}[T-2]{Decipher Codes and Ciphers}

\multiple{150}
\basechance{25\% + 3\% / Rank}
\begin{effects}
Any Adept of this College has a 25\% chance (+3 / Rank) of being able
to break any code or cipher created by an Adept of this college using
T-4 if they study it for 12 hours (-1 / Rank, +1 / Rank of the Adept
who created the code or cipher with T-4).  The Adept's chances of
breaking such a cipher or code are decreased by 3 for every Rank the
Adept who created the code or cipher had with T-4 (Create Codes and
Ciphers) of this College.  The Adept must maintain concentration for
the required period of time in order to attempt to break the cipher or
code.  If the Adept maintains concentration, a D100 role can be made
by the GM to determine the Adept's success.  The Adept's Rank in
breaking non-magical codes and ciphers is equal to their rank in this
talent.
\end{effects}
\end{talent}

\begin{talent}[T-3]{Read Runesticks}

\multiple{100}
\begin{effects}
Any Adept of this College has a 30\% chance (+3 / Rank, -3 / Rank of the
Runestick created with Q-1) to successfully read the Runesticks of
another Adept of the College of Rune magics without assistance.  If
assisted by whoever created the Runesticks, there is a 70\% chance
(+3 / Rank, -3 / Rank of the Runestick created with Q-1) of detecting
misinformation given in the guise of assistance.  Only Runesticks
which have been successfully read may be employed to cast a spell or
perform a ritual of this College.  Runesticks which have been
``successfully'' read, but incorrectly assessed due to false assistance,
may be used, but they will automatically backfire.  The GM always
rolls to determine the success or failure at the moment this talent is
applied.  An Adept automatically reads the runes (though not
necessarily correctly) when assisted by the creator of the Runes.
\end{effects}
\end{talent}

\begin{talent}[T-4]{Create Codes and Ciphers}
\multiple{150}
\begin{effects}
An Adept of the College of Rune Magics can create a magically enhanced
code or cipher which a normal character or NPC would have only a 1\%
chance of breaking, but which another Adept of this College would have
a greater chance of breaking.  The Rank of the code or cipher's
creator modifies all attempts to break the code or cipher by -3 /
Rank.  It takes 1 hour to create a code or cipher.
\end{effects}
\end{talent}

\begin{talent}[T-5]{Summon Wand}

\range{10 feet + 10 / Rank}
\multiple{200}
\basechance{40\% + 3\% / Rank}
\begin{effects}
An Adept can summon to themselves any Runewand they have created that
is within 10 feet (+ 10 / Rank).  The Adept simply wills the wand to
come to them (requires a Pass Action in combat) and the Runewand leaps
into their hand.  The Base Chance to successfully use this talent is
40\% (+3 / Rank).  Note: this talent can operate only if the path
between the Adept and their Runewand is not blocked by anything
through which the Runewand could not normally pass (such as a wall or
person).  Only the Runewand is summoned; nothing surrounding or
attached to it travels with it.
\end{effects}
\end{talent}

\subsection{General Knowledge Spells}

\begin{spell}[G-1]{Learn Rune}

\range{15 feet + 15 / Rank}
\duration{Immediate}
\multiple{100}
\basechance{45\%}
\resist{Active}
\target{Entity or Object}
\begin{effects}
The Adept must point their Runewand at the object of the spell.  If
successful the Adept will gain sufficient information to create a Rune
that may be used to represent the generic type of the object of the
spell.
\end{effects}
\end{spell}

\begin{spell}[G-2]{Darkness}

\range{15 feet + 15 / Rank}
\duration{ 15 minutes / Rank (\x 1 if unranked)}
\multiple{75}
\basechance{45\%}
\resist{None}
\target{Volume}
\begin{effects}
The Adept creates a volume in which non-magical light is partially
suppressed.  The volume will be 1000 (+ 500 /Rank) cubic feet, and may
be in any one contiguous area the Adept desires, provided that no
dimension is smaller than one foot. The entire volume must be visible
and within range at time of casting, and may not be moved.  At Ranks
0--5 the amount of light within the volume is reduced to 10\% (appears
as though lit on a cloudy night), at Ranks 6--10 it is reduced to 5\%
(as though a windowless room), at Ranks 11--15 it is reduced to 1\%
(so dark that night vision like that of a cat will take about a minute
to adjust), and at Ranks 16--20 all light is banished (\ie totally
dark).  Although infravision works off heat and elvish and dwarvish
visions work in total darkness, it is still not possible to see at all
at ranks 16--20.  It will not aid in providing bonuses for casting
purposes, though it will reduce penalties due to natural light by up
to 5\% + 1\% / Rank.  If the lighting conditions are lower than that
provided by the spell, no effect will be apparent. Note that because
light is only being suppressed, it may still pass through, and no
shadows are generated. If it is possible to see through a Darkness,
all beyond it is perfectly visible. This spell can engender
silhouettes of lit objects against the darkness, though not create
shadows.  Any of this volume may be overridden by a higher ranked
Spell of Light, or neutralised (back to original conditions) by an
equal rank.  In all cases, the darkness will emanate from the tip of
the Adept's Runewand, but will last for only so long as the Runewand
remains unbroken and in the Adept's possession.
\end{effects}
\end{spell}

\begin{spell}[G-3]{Light}

\range{15 feet + 15 / Rank}
\duration{15 minutes / Rank (\x 1 if unranked)}
\multiple{75}
\basechance{50\%}
\resist{None}
\target{Volume}
\begin{effects}
One 10-foot cube (1000 cubic feet) area is illuminated.  The lighted
area may be any shape (even pencil thin), but must emanate from the
tip of the Adept's Runewand and will last for only so long as the
Runewand remains unbroken and in the Adept's possession.  At Ranks
0--5 the amount of darkness within the volume is reduced to 10\%
(appears as though lit on a cloudy day), at Ranks 6--10 it is reduced
to 5\% (as though lit on a sunny day), at Ranks 11--15 it is reduced
to 1\% (similar to daytime in a desert), and at Ranks 16--20 all
darkness is banished (\ie totally light), so it is impossible to see
into or whilst within the volume (unless under the effects of a
Resistance to Light).  It will not aid in providing bonuses for
casting purposes, though it will reduce penalties due to natural
darkness by up to 5\% (+ 1\% / Rank).

\end{effects}
\end{spell}

\begin{spell}[G-4]{Pyrogenesis}

\range{Touch of Runewand}
\duration{Immediate}
\multiple{75}
\basechance{40\%}
\resist{Passive}
\target{Object}
\begin{effects}
One small flammable object or entity may be caused to burst into flame
by the touch of the Adept's Runewand.  Thereafter, the flames are
fuelled by the object or entity.  They may be extinguished
normally. Note this spell can only be used to light matches and cause
insects and small furry animals no larger than a mouse to burst into
flames.
\end{effects}
\end{spell}

\begin{spell}[G-5]{Curse}
\index{curse!rune}
\range{Touch of Runewand}
\duration{Permanent until dispelled}
\multiple{400}
\basechance{35\%}
\resist{Active, Passive}
\target{Entity}
\begin{effects}
The Adept may curse one target (who must be touched by the Runewand)
with any of the possible minor curses listed (following) that he has
the necessary Rank to employ (Rank: Possible curse). The touch is
automatic unless the target is actively avoiding being touched, in
which case the target cannot be touched and the spell cannot take
effect.  The spell must be prepared normally.  The effects are
permanent until dispelled by anyone casting a Rune College General
Knowledge Counterspell.

\begin{Description}
\item[1--5]
The Adept may afflict the target with hallucinations that will reduce
the target's Perception by 5 in addition to any specific effects.  The
GM and the Adept must work out the exact nature of the hallucination
at the time that the curse is made.  Hallucinations should, however,
be of a minor, generalised nature: seeing coloured lights in the
distance, hearing sounds like the clanking of weaponry, smelling meat
cooking from time to time, and so forth.

\item[6--10]
The Adept may afflict the target with increasing physical debilitation
that will decrease Physical Strength by 1 immediately, and will
subtract 1 from Endurance at the end of each day until the target
reaches 4 Endurance or the curse is dispelled.

\item[11--15]
The Adept may afflict the target with total loss of any one sense
(sight, smell, touch, hearing, taste).  The loss of sense takes place
immediately.

\item[16--20]
The Adept may afflict the target with extreme paranoia and
nightmares. The target will recover only one fatigue point per hour
from taking a nap, and only 2 per hour from sleeping.  In addition,
the target will feel hagridden and imagine themselves pursued by
phantasms.  They will, unless the curse is first dispelled, eventually
become more and more estranged from reality, distrustful of friends
and companions, and obsessed with the idea of destroying their enemies
(who they think are ``all around''). If the curse is not dispelled
within D10 \x [target's Willpower - 2 \x spell Rank] days, the target
will completely lose touch with reality. They will then plot to
destroy their friends in the belief that they are ``out to get them''
and will exhibit other bizarre behaviour.  They will be cured of the
advanced stage of this affliction only by having the curse dispelled
and then spending a number of days equal to the Adept' s Rank \x D10
in rest and recuperation.
\end{Description}
\end{effects}
\end{spell}

\begin{spell}[G-6]{Illusion}

\range{5 feet + 5 / Rank}
\duration{Permanent until dispelled}
\multiple{200}
\basechance{35\%}
\resist{Special}
\target{Runestick}
\begin{effects}
The Adept places a single Runestick carved with the Rune for this
spell and a Rune representing an object or entity of their choice on
the ground.  The visual Illusion must be contained within 5 (+ 5 /
Rank) adjacent one foot cubes.  After a successful cast the stick will
appear to all except the Adept to be the same object or entity as the
Rune incised on the Runestick.  The illusion lasts until dispelled by
the appropriate counterspell or the stick is moved.  The image will be
static and will remain even when touched.
\end{effects}
\end{spell}

\begin{spell}[G-7]{Control Entity}

\range{15 feet + 15 / Rank}
\duration{Special}
\multiple{500}
\basechance{15\%}
\resist{Active, Passive}
\target{Entity}
\begin{effects}
Three Runesticks containing the binding Rune must be physically bound
onto the entity to be controlled.  10\% is added to the Base Chance if
the Rune representing the generic type of the entity to be bound has
been carved onto the Runesticks. This binding cannot be done in
combat, although the target may be physically restrained while the
Adept attaches the Runesticks.  In some cases, the Adept may be able
to induce the entity to put the Runesticks on itself voluntarily (via
trickery, for example).  Once the sticks are in place, the spell can
be cast to determine whether or not the sticks function.  The target
must be visible to the caster in order for the spell to be cast.  Once
cast, this spell remains in effect until the sticks are no longer
bound to the target.  However, every Rank + 1 days after the sticks
have been placed, the entity gets a further resistance check.
Successful resistance means that the entity is no longer under the
spell's effects and the sticks cease to function.  Failure to resist
means that the sticks continue to work. The Base Chance of resisting
(for the rechecks only) is equal to the entity's usual passive magic
resistance versus Rune College spells minus the Adept's rank in this
spell.  Otherwise, the target may not remove the runesticks themselves
unless so commanded by the binder.  Until the sticks are removed, the
target will freely do the bidding of the Adept, acting in all ways as
their loyal servant (even to the extent of fighting anyone trying to
remove the sticks from them).
\end{effects}
\end{spell}

\begin{spell}[G-8]{Purification}

\range{Touch}
\duration{Immediate}
\multiple{100}
\basechance{30\%}
\resist{None}
\target{Liquid}
\begin{effects}
The Adept may turn any aqueous substance into potable water by
touching the substance with a Runestick that has had a Purification
Rune incised into it.  The Adept may purify 1 quart (+ 1 / Rank) by
volume with this spell.  This spell may be used to neutralise poison
in solution.  Note: This spell is not intended for use in combat and
will not work if the Runestick is forced into an entity's bloodstream.
\end{effects}
\end{spell}

\begin{spell}[G-9]{Runelock}

\range{5 feet + 5 / Rank}
\duration{Permanent until dispelled}
\multiple{200}
\basechance{30\%}
\resist{None}
\target{Portal}
\begin{effects}
This spell may be cast over any portal (door or window) that can
normally be opened or closed and is in sight.  It effectively locks
the portal with an unpickable lock.  The spell can be dispelled by
anyone casting the Rune College General Counterspell.  The portal may
still be forced open by brute strength.  In this case, the Physical
Strength of all the figures attempting to force the portal are
totalled and multiplied by the Difficulty Factor of the task.  The
Difficulty Factor is always a function of the Rank of the spell:

\begin{tabular}{ll}
\textbf{Rank} & \textbf{Difficulty} \\
1--5	& 2.0 \\
6--10	& 1.5 \\
11-20	& 1.0 \\
\end{tabular}

In order to place a Runelock on a portal, (which takes a minute), the
Adept must draw or paint the Runelock Rune on the portal. They may
only then cast the spell.  Note: The opening spell of the College of
Ensorcelments and Enchantments will open a Runelocked portal, but will
not dissipate the Runelock.
\end{effects}
\end{spell}

\subsection{General Knowledge Rituals}

\begin{ritual}[Q-1]{Fashioning Runesticks}

\multiple{100}
\basechance{55\% + 3\% / Rank}
\begin{effects}
The Adept must use this ritual to actually carve the appropriate
Rune(s) on a stick fashioned of any material listed in
\S\ref{rune:runesticktable} (the Runestick Chart).  At the time the
Adept fashions the Runestick, the Adept's player must announce what
Runes are being cut into the stick (\ie what spell or ritual the
Runestick can be used to perform). Careful records must be kept of the
number and type of Runesticks carried by a character.  Whenever an
Adept wishes to cast a spell or perform a ritual using the runesticks
in their possession, they prepare the spell or performs the ritual
normally using the runestick(s) only to cast a spell once it is
prepared. Until a spell is cast, the Runestick is unaffected by the
preparation.  However, once a Cast Check is made, whether successful
or not, the Runesticks used in that spell or ritual are used up. They
retain the power necessary to keep the spell or ritual in effect for
its normal duration, but are otherwise of no magical value.  They can
never be ``recharged'' or reused, although the materials they comprise
may be refashioned (see note).  Once they have been fashioned as part
of this ritual, Runesticks are permanently imbued with the power of
the spell or ritual represented by the Rune(s) cut into them.  Only
one spell or ritual may be imbued in each stick.  It takes a length of
time equal to that listed on the ``Time'' column of the Runestick
Chart (- 1 / Rank, but with a minimum of 1 minute) to perform this
ritual.  Only one stick may be produced per ritual.  It costs 10
fatigue points (- 1 / three Ranks or fraction) to perform the ritual,
and the Base Chance of successfully performing it is 55\% (+ 3 /
Rank), and there is no backfire.

\textbf{Note:} The actual materials in a Runestick, once used in a
spell, may be used to fashion new Runesticks for future use.  This
requires a new ritual of Fashioning Runesticks, during which the
Runesticks are refashioned (carved or cast). Runesticks used in a
spell or ritual that backfires are destroyed (burned up), and the
materials may not be refashioned into new Runesticks or used for any
purpose, magical or otherwise. Only the actual stick itself may be
reused (as opposed to the ink, etc.). The Base Chance of the stick
being reusable is equal to the Adept's Magical Aptitude + 5 / Rank
with this ritual. If the Cast Check is successful then the stick can
be refashioned by going through the ritual again (as listed). If the
Cast Check is unsuccessful then the stick in question is absolutely
useless.

\textbf{Clarification:}
If a Runestick is fashioned using this ritual, but is not fashioned
successfully, then the stick becomes useless and cannot be refashioned
or used again. If, however, the ritual was successful, then the
Runestick can be used as intended.  Once a Cast Check has been made
against a Runestick, then that stick cannot be used again until it has
been refashioned.  If the Adept desires to refashion it, then before
they begin the ritual he first finds out if the stick is reusable
using the Cast Chance given above. If the stick is reusable then the
Adept can proceed with the ritual.
\end{effects}
\end{ritual}

\begin{ritual}[Q-2]{Fashioning Runewand}

\multiple{300}
\basechance{30\% + 3\% / Rank}
\begin{effects}
The Adept may employ this ritual to create a Runewand out of any of
the materials listed on the Runewand Table.  The implement is
fashioned by inscribing Runes into the material's surface, which
describe its use, name, and history. Once the Runewand has been
fashioned and consecrated in this ritual, it remains fully effective
unless and until it is broken or otherwise destroyed.  It takes from
one to four weeks to perform this ritual, depending on the type of
material used to fashion the runewand.  The total time necessary to
fashion a Runewand is decreased by 1 day for each Rank the Adept has
with this ritual (minimum of 1 day to perform the ritual), and the
ritual costs 10 Endurance (- 1 / two Ranks with this ritual). The
Endurance loss will heal normally.  The Adept may interrupt the ritual
to eat and attend to housekeeping (maximum two hours per day) and to
sleep (maximum eight hours per day), but any break of longer than 10
hours results in the ritual failing and the materials used being
ruined. Any Endurance expended on this ritual is expended upon
completion of the ritual, not during its course.  Once the ritual is
completed, the Adept determines whether or not it was successful. The
Base Chance for this ritual is 30\% (+3 / Rank).  All materials used
in an unsuccessful ritual (or a ritual that backfires) are destroyed
or ruined. If the ritual is successful, the Adept may use the Runewand
thereafter to cast spells and perform rituals that require the use of
a Runewand.

In addition, the Adept may store a maximum of 1 Fatigue Point in the
Runewand at Rank 0, and an additional 1 Fatigue Point for every 2 or
fraction Ranks they have with the ritual of Fashioning Runewand at the
time the Runewand is fashioned.  Fatigue is stored in a Runewand
simply by touching the Runewand and willing one or more Fatigue Points
to enter the Runewand.  Twice the Fatigue Points stored in the
Runewand are subtracted from the Adept.  Once stored in the Runewand,
Fatigue Points remain there indefinitely and can be used by the Adept
to cast spells at any time that they are holding the Runewand while
making a Cast Check.  An Adept may add Fatigue to a Runewand any
number of times, so long as the Runewand has the capacity remaining to
store the Fatigue each time the wand is ``recharged''.
\end{effects}
\end{ritual}

\begin{ritual}[Q-3]{Warding with Runesticks}
\multiple{200}
\begin{effects}
The Adept sets up a pattern of Runesticks inscribed with the Ward Rune
(as fashioned by Q-1).  This pattern may consist of three, five or
seven sticks composed of any material listed in
\S\ref{rune:runesticktable}.  If three Runesticks are used, the Base
Chance for this ritual is 20\%.  If five sticks are used, the Base
Chance is 30\%.  If seven sticks are used, the Base Chance is 40\%.
All Base Chances are increased by 5 per Rank.  This ritual takes two
hours (-10 minutes / Rank, with a minimum of 10 minutes) to complete.
During the ritual, the Adept must place the Runesticks containing the
Runeward symbol in a roughly circular configuration around the area to
be warded (the Adept must remain inside the area while the ritual is
being prepared).  At the end of the ritual, if it is successful, a
Runeward exists that will help to protect those inside it from magic.
No magical item (amulet, weapon, etc.)  can be brought into the warded
area, though items already inside the warded area can be taken out.
The area to be warded is a sphere with a diameter, in feet, of up to
10 times the number of sticks used.

Any magical creature or Adept attempting to enter the warded area must
make a Passive Resistance check, or it will be unable to enter the
area.  The entity's Magic Resistance is unaffected if the ritual used
only three Runesticks.  A five stick ward reduces the entity's Magic
Resistance by the Adept's Rank with this ritual, and reduces Magic
Resistance by twice their Rank for a seven stick ward. In addition, if
the Runesticks are all made of Rowan then an entity which is wholly or
partially of another plane (such as demons, devils, imps, hellhounds)
decreases its Magic Resistance by 3 times the Adept's Rank when it
attempts to enter the warded area. The Runeward is automatically and
permanently broken if any magical entity or magic user succeeds in
passing it. However, so long as it is in effect, all targeted spells
cast into (not out of) the warded area have a chance of being
dissipated harmlessly when striking the ward according to the number
of sticks used in the ward: by 10 if it is a three stick ward; by 20
if it is a five stick ward and by 30 if it is a seven stick ward, plus
twice the Adept's rank in this ritual.  Backfire from this ritual
results not only in the destruction of the Runesticks, but in D10
damage to the Adept's Endurance as well.

The same Runesticks which are used for this ritual may be used in the
Ritual of Healing (Q-4) and may also be used in conjunction with the
Runestick(s) necessary to the casting of some other spell of this
College to create a ward as described in \S\ref{college:ward}.
\end{effects}
\end{ritual}

\begin{ritual}[Q-4]{Healing}
\multiple{150}
\basechance{50\% + 4\% / Rank}
\begin{effects}
The Adept creates a warded area by setting up a Runeward as described
in Q-3. However, only the seven Runestick Runeward may be used. The
Runeward is set up around the entity to be healed.  The ritual lasts
seven hours, at the end of which the entity to be healed is cured of
all Fatigue and Endurance losses, plus any non-magical diseases,
fevers, or infections from which the entity may suffer.  It is
possible for the ritual to backfire.  If it does so, the entity being
healed immediately goes to 0 Fatigue and -1 Endurance (unless the
patient is already below this).  The Adept must expend 10 Fatigue to
employ this ritual. Any types of material listed in
\ref{rune:runesticktable} may be used to make Runesticks used in this
ritual except for Elder and Yew.  In addition, if the Runesticks used
in this ritual are made of Walnut or Elm, the number of hours the
ritual requires is reduced to five.

\end{effects}
\end{ritual}

\begin{ritual}[Q-5]{Runes of Sight}
\multiple{300}
\begin{effects}
The Adept may gain insight into the future by casting the Runes of
Sight (Runesticks which have Runes cut into them representing the
cosmic balance). It takes one hour to cast these Runes and the Adept
may perform no other action during that time.  The performance of this
ritual allows the Adept to exercise one of the following functions
during its course:

\begin{Description}
\item[Limited Precognition]
This action is executed as a ritual, but with the same results as for
the Spell of Limited Precognition of the Mind College with a Base
Chance of 30\% (+ 2 / Rank).

\item[Divining Enchantment]
This action is executed as a ritual in the same manner as the Ritual
of Divination (R-1) of the College of Naming Incantations. It has a
Base Chance of 55\% (+ 4 / Rank).
\end{Description}

Only one of these two options may be performed at each casting of the
Ritual. It requires three sticks incised with the appropriate Runes to
perform this ritual.
\end{effects}
\end{ritual}

\begin{ritual}[Q-6]{Sending}
\range{10 miles + 5 / Rank}
\multiple{250}
\basechance{30\% + 5\% / Rank}
\resist{Passive}
\begin{effects}
The Adept must paint their forehead with a Sending Rune before
retiring to sleep at night.  They will then require a five hour period
of sleep with no disturbances sufficient to wake him or the ritual
will fail.  The target of the spell is likewise required to be asleep
for five undisturbed hours or the ritual will not work.  The time
asleep counts as resting for Fatigue recovery purposes.  During the
time asleep, the Adept will be in communication with one entity of
their choice that they have seen and studied sufficiently (as per
College of Ensorcelments and Enchantments Spell of Location for ``seen
and studied'').  Alternatively, the Adept may employ the target's
Individual True Name if it is known.  If the Cast Check is successful
and the target fails to resist then it will answer all questions asked
of it in a yes / no fashion. This ritual does not allow communication
with entities on other planes of existence. The questions that are to
be asked of the target must be formulated before the Adept goes to
sleep.  Upon completion of the five hour ritual the Adept may receive
the answers to Rank \x 5 questions (\x 1 if unranked).
\end{effects}
\end{ritual}

\subsection{Special Knowledge Spells}

\begin{spell}[S-1]{Runewall Spell}

\range{15 feet + 15 / Rank}
\duration{30 minutes + 30 / Rank}
\multiple{250}
\basechance{50\%}
\resist{Passive}
\target{Area}
\begin{effects}
The Adept places a single Runestick incised with the warding Rune on
the ground and performs the spell.  If the spell is successful, the
stick metamorphoses into a translucent, shimmering wall of force 1
inch thick, 10 feet high, and 20 feet long that may be shaped by the
Adept (and no one else) into any shape of their devising (\eg circle,
dome, etc.).  The Adept may alter the height or length of the wall by
1 foot per Rank.  The wall cannot be created touching an entity,
although it may encompass them.  Any entity coming into contact with
the wall must resist or be thrown back (falling prone).  In addition,
if the Runestick used to manufacture the wall was made of Elder,
anyone who fails to resist suffers [D - 2] + 1 / Rank.
\end{effects}
\end{spell}

\begin{spell}[S-2]{Torment}

\range{15 feet + 15 / Rank}
\duration{15 seconds + 15 / Rank}
\multiple{250}
\basechance{15\%}
\resist{Active, Passive}
\target{Entity}
\begin{effects}
The Adept can, by pointing a Runestick inscribed with the Pain Rune at
one entity, cause that entity extreme pain.  Each pulse that the Adept
continues to point the Runestick at the entity (requiring a pass
action) it suffers 1 point of damage and for the entire duration has
all its Base Chances reduced by Rank\%.  For Mind Mages, the reduction
is reduced by 5 (+ 1 / Rank with Resist Pain).  Moreover, if the
Runestick is made of Yarrow, the entity suffers a further Rank\%
reduction to Base Chances and - 1 / pulse off TMR (while the stick is
still being pointed at them).
\end{effects}
\end{spell}

\begin{spell}[S-3]{Creating Rune Shield}

\range{Touch}
\duration{1 hour + 1 / Rank}
\multiple{200}
\basechance{40\%}
\resist{None}
\target{Runestick}
\begin{effects}
The Adept must use a Runestick inscribed with the appropriate shield
Rune.  Upon successful cast, the stick is transformed into the shield
of whatever type the Rune incised on the Runestick indicated (except
Main Gauche).  This shield may then be used by anyone (no strength or
MD limitations) and provides an extra 5\% + Rank defence extra to the
shield type.  Note that this defence is as per shield rules thus only
protecting from two of the front three hexes.  Moreover, if the
Runestick is made of Walnut, any Grievous Blow through the shield
will, instead of harming the wielder, merely smashes the shield,
causing it to revert to a now broken Runestick.
\end{effects}
\end{spell}

\begin{spell}[S-4]{Visitation}

\range{1 mile + 1 / Rank}
\duration{Concentration: Maximum 1 hour + 1 / Rank}
\multiple{300}
\basechance{15\%}
\resist{None}
\target{Entity}
\begin{effects}
The Adept must cast the Runes of Far-seeing (three matched Runesticks)
on the ground before them while performing the spell.  If successful,
the Adept is able to send a ghost-like image of themselves instantly
to any location within range that the Adept has physically occupied at
least once in the past.  They are present in that location in all ways
except bodily (\ie the Adept may communicate and use all their
senses while the image is there, but may not be harmed by any
attack). The image ``mimics'' the actual actions of the Adept, and may
move no more than 10 feet (+ 10 / rank) from the spot where it
materialised, which may be anywhere at the location the Adept
wishes. Since the Adept's consciousness is in the image, which is
non-physical, they may not cast any spells (although they may appear
to should they so desire).

Thus, if the Adept desired the image to talk, the body will also speak
the words wherever it is physically located.  Also, if the Adept is in
a location which would prevent them from moving, the image may not
move either.  When the visitation time has expired (or anytime prior
that the Adept wished), the image quickly fades and travels back to
the Adept.  This image has an ``aura'' which, if detected by a Detect
Aura talent, may give the compass direction at which the Adept would
be located, but not the distance.
\end{effects}
\end{spell}

\begin{spell}[S-5]{Truth}

\range{15 feet + 15 / Rank}
\duration{1 hour + 1 / Rank}
\multiple{300}
\basechance{30\%}
\resist{Passive}
\target{Entity}
\begin{effects}
Prior to casting this spell the Adept must first draw a Truth Rune on
the forehead (or over the brainpan) of the spell's target (which may
be themselves). The spell may only be cast over one target entity of the
Adept's choosing and the Adept must touch the target to cast it.  If
unsuccessful, a new Truth Rune must be drawn on the target before the
spell can be attempted again. A successful spell that is not
successfully resisted causes the target to be unable to speak a
falsehood for the duration of the spell. The target must not knowingly
say anything false, but may refuse to answer a question put to him.

In addition, the Truth Rune enables the target to see the true nature
of all things.  This results in the GM modifying the target's
Perception roll because the wearer of the Rune is more likely to see
through deceptions (\eg magical traps).  The roll should be modified
by two times the Adept's rank with this spell.
\end{effects}
\end{spell}

\begin{spell}[S-6]{Banishment}

\range{Touch}
\duration{Immediate}
\multiple{250}
\basechance{30\%}
\resist{Active, Passive}
\target{Summonable}
\begin{effects}
The Adept may banish any one entity from another dimension to its own
plane of existence. In order to do so the Adept must touch the target
entity with their Runewand at the moment the spell is completed.  If
successful, the spell results in the entity immediately returning to
its own dimension unless the entity successfully resists. The touch is
automatic unless the target is actively avoiding being touched, in
which case the target cannot be touched and the spell cannot take
effect. The spell must be prepared normally. The target returns to a
random spot, in an appropriate medium, on its own plane.  The exact
whereabouts is GM's discretion, however, entities banished at
approximately the same time will appear in approximately the same
area.
\end{effects}
\end{spell}

\begin{spell}[S-7]{Smite}

\range{Touch}
\duration{Immediate}
\multiple{200}
\basechance{25\%}
\resist{Passive}
\target{Entity}
\begin{effects}
The Adept must, at the moment the spell is cast, touch the intended
target with the Runewand. If the target fails to successfully resist,
it suffers [D + 1] + 1 / Rank damage.  The target takes half damage if
it fails to resist.  The damage is similar to the effect of an
electric shock, so halve the damage if the target is an insulator
(entities are generally not insulators). The touch is automatic unless
the target is actively avoiding being touched, in which case the
target cannot be touched and the spell cannot take effect.  The spell
must be prepared normally. If the target fails to resist then they
become stunned.
\end{effects}
\end{spell}

\begin{spell}[S-8]{Creating Runeweapon}

\range{Touch}
\duration{1 hour + 1 / Rank}
\multiple{200}
\basechance{20\%}
\resist{None}
\target{Runestick}
\begin{effects}
The Adept must use a Runestick incised with the Death Rune and with a
Rune representing the type of weapon they wish to create.  The Adept
holds the Runestick while casting the spell.  Upon successfully
completing the cast, the Runestick transforms into a magical weapon of
whatever type the Rune incised on the Runestick indicated (estoc,
dagger, glaive, etc.).  Since the substance of the weapon is magical,
the Adept can wield the weapon without suffering the penalties
associated with cold iron.  Further, the weapon is usable against
those entities normally affected only by magical weapons, but
otherwise has the same properties as a normal weapon of the same
type. An entity other than the Adept using a Runeweapon has -10\% on
strike chance. There is never any chance of the weapon breaking.  The
duration of this spell is decreased to 1 minute (+ 1 / Rank) if the
Runestick used is of Yew.

If Runesticks of Yew are used, and at least one point of effective
damage is inflicted on a target, the wound is poisoned, causing [D -
5] (+ 1 for every 3 or fraction ranks) damage per pulse for D10 pulses
due to poison. The target can only have one poison in effect at any
one time, \ie poison from different strikes is not cumulative.  The
normal rules for using poisoned weapons apply.  The poison is
considered a nature poison for purposed of antidotes.

\end{effects}
\end{spell}

\subsection{Special Knowledge Rituals}

\begin{ritual}[R-1]{Casting the Runes}

\multiple{500}
\basechance{5\% + 5\% / Rank}
\begin{effects}
The Adept must prepare a piece of paper or vellum on which are written
the Runes of Doom. The entire ritual of preparation takes one hour. At
the end of the hour, the Adept chooses which of the demons from the
College of Greater Summonings will be the executor of the doom and
also writes this name on the paper.  The Adept's player must actually
write this information down, since it will only come into play in the
future.  Once the ritual is prepared, the Adept then passes the sheet
of paper on to the victim whose name is written on the paper.  The
victim must voluntarily accept the paper (though they need not know
what is on it).  Once they accept it, the demon named on the paper
hunts them down and kills them. Even if the demon is destroyed it
returns as soon as it is able and continues the hunt (see College of
Greater Summonings for how demons recover from injury and ``death'' in
their own dimension).  Only by passing the paper on to another entity
who voluntarily accepts it can the doom be transferred.  If the paper
is destroyed, the doom can never be lifted or transferred.  If the
ritual backfires, the Adept loses [D + 2] Endurance.
\end{effects}
\end{ritual}

\begin{ritual}[R-2]{Creeping Doom}

\multiple{450}
\basechance{20\% + 4\% / Rank}
\begin{effects}
The Adept creates 13 Runesticks by carving the appropriate
maledictions into human bones.  They then perform a ritual over them
(duration 1 hour) and bury the sticks beneath the dwelling of
someone they wish to curse. It is best if the victim's name is carved
in the bones as well, otherwise others in the house may become ill
instead. For each month that the bones remain in or under the victim's
dwelling, they must make a Resistance Check, the Base Chance for which
is composed of the victim's Endurance multiplied by the Difficulty
Rating of the resistance.

\begin{tabular}{ll}
\textbf{Rank} & \textbf{Difficulty} \\
1--5	& 4.0 \\
6--10	& 3.0 \\
11-15	& 2.5 \\
16--18	& 2.0 \\
19--20	& 1.5 \\
\end{tabular}

If the victim fails to resist, they suffer a wasting disease and loses
[D - 3] Endurance points for the purposes of future resistance (only).
If they fail to resist for three straight months, they die.

Generally, the victim of these maledictions does not know exactly what
is wrong with them.  Should they discover the bones, they may remove
the curse by removing the bones from the house.  Other means of ending
a curse do not normally suffice, although the sufferer would show
immediate improvement upon leaving the house and sleeping elsewhere
for a few weeks.  There is no chance of this ritual backfiring.
\end{effects}
\end{ritual}

\begin{ritual}[R-3]{Creating Rune Portal}

\multiple{400}
\begin{effects}
The Adept must place a Runestick with the Portal Rune on it in the
ground and perform a half hour ritual.  Once the ritual is
successfully executed, the Runestick may not be moved without
destroying the portal.  It becomes the terminus for a future attempt
at teleportation. Once the terminus has been established, the Adept
may, at any future time, use another Runestick which was carved at the
same time and is a mate of the Runestick used as the terminus to
travel back to the original terminus.  There can only be one ``mate''
for the terminus, and it too must contain the Portal Rune and must be
of the same material as the terminus Runestick. In order to teleport
to the terminus, the Adept simply places the mate in or on the ground
and performs the ritual.  If unsuccessful, both Runesticks are
destroyed.  If successful, the mate of the terminus is activated and
becomes a terminus too.  Thereafter, anyone who touches one terminus
is teleported to the other terminus and appears standing within 5 feet
of that terminus (travel time is one pulse). A terminus may be used
any number of times until destroyed, but may never be moved without
destroying the Portal. A terminus can be destroyed by casting a
Special Knowledge Counterspell of the College of Rune Magics over
it. Once a terminus is destroyed, the mate of the Runestick used to
form the terminus no longer functions (though this will not
necessarily be known until someone tries to use the mate).  If a
terminus is destroyed during the pulse while an entity is in transit
between the two, the entity is destroyed and its molecules dispersed
over known space.

The Base Chance for this ritual is Magical Aptitude + 3 / Rank, and is
decreased by 1 for every 5 miles separating the two Runesticks and
increased by 15 if the Runesticks are shaped from Willow wood.
\end{effects}
\end{ritual}

\begin{ritual}[R-4]{Binding Elements}

\duration{2 hours + 2 / Rank}
\multiple{500}
\basechance{MA + 3\% / Rank}
\casttime{30 minutes}
\begin{effects}
The Adept may gain control of any element by using this ritual.  They
must possess a Runestick containing the Binding Rune and the Rune
representing the element to be bound and they must touch the element
with that Runestick at the conclusion of the ritual.  The Adept may
bind 500 pounds of earth (+ 500 / Rank), 500 gallons of water (+ 500 /
Rank), 1000 cubic feet of air (+ 500 / Rank), or all fire within a 10
foot radius (+ 15 feet / Rank). They may do anything with the element
except form an elemental. This ritual may not be used over an area
occupied by an elemental and cannot be used in any way to control an
elemental.
\end{effects}
\end{ritual}

\subsection{Runewand Table}
\label{rune:runewandtable}
\index{tables!runewand table}

\begin{inset}{\small}{center}
\begin{tabular}{l@{\hspace{1.0em}}l@{\hspace{0.75em}}c@{\hspace{0.75em}}c@{\hspace{0.75em}}c}
Dice 	& Runewand	& Weight & Cost & Time \\ \hline
01--15	& Oak Staff	& 3	& 55	& 2 \\
16--30	& Blackthorn Staff & 3	& 60	& 2 \\
31--45	& Ash Staff	& 3	& 60	& 3 \\
46--55	& Willow Rod	& 1	& 55	& 1 \\
56--65	& Cedar Rod	& 1	& 75	& 2 \\
66--73	& Ivory Rod	& 1	& 80	& 2 \\
74--81	& Ebony Rod	& 1	& 90	& 3 \\
82--89	& Copper Rod	& 1	& 55	& 3 \\
90--97	& Bronze Sceptre & 5	& 90	& 3 \\
98*	& Silver Sceptre & 5	& 200	& 4 \\
99*	& Gilded Sceptre & 5	& 500	& 4 \\
00*	& Truesilver Sceptre &5 & 900	& 4 \\
\end{tabular}
\end{inset}

\begin{Description}
\item[Runewand]
Type of Runewand.
\item[Weight]
The average weight in pounds of a Runewand made of this material.

\item[Cost]
The cost in Silver Pennies of the materials (including incense, oils,
etc.) used in the preparation of the Runewand if it is manufactured by
the Adept and not purchased.

\item[Time]
The amount of time in weeks required to create a Runewand of this
type, given the necessary materials and tools.
\end{Description}

* This type of Runewand must be paid for by the Adept out of the
proceeds of their first six months of adventuring or the money lenders
from whom they gained the wherewithal to have the item made will send
one or more debt collectors to collect.

\subsection{Runestick Chart}
\label{rune:runesticktable}
\index{tables!runestick table}

\begin{inset}{\small}{center}
\begin{tabular}{lccc}
Material	& Weight & Cost & Time \\ \hline
Ashwood		& 1.0	& 1	& 15 \\
Aspenwood	& 1.0	& 2	& 10 \\
Cedarwood	& 1.0	& 2	& 10 \\
Chestnut	& 1.0	& 1	& 10 \\
Elder Wood	& 1.0	& 2	& 10 \\
Elm wood	& 1.0	& 2	& 10 \\
Gilded Metal	& 2.0	& 80	& 240 \\
Mistletoe	& 0.5	& 1	& 10 \\
Oak		& 2.0	& 2	& 20 \\
Pinewood	& 0.5	& 1	& 5 \\
Rowan		& 1.0	& 4	& 15 \\
Silvered Metal	& 2.0	& 10	& 240 \\
Walnut		& 1.0	& 1	& 15 \\
Willow Wood	& 0.5	& 1	& 15 \\
Yarrow		& 1.0	& 1	& 10 \\
Yew		& 1.0	& 2	& 10 \\
\end{tabular}
\end{inset}

\begin{Description}
\item[Material]
The type of material used to make the Runestick.  Weight: The weight
in ounces of one Runestick made of this material.
\item[Cost]
The cost of the materials (in Silver Pennies) necessary to make one
Runestick of this material, not including cost of tools.
\item[Time]
The amount of time in minutes necessary to manufacture one Runestick
of this type.  Some spells and rituals require the use of Runesticks
made of specific materials.  In other cases, any Runestick on this
table will do, but the modifiers in \S\ref{rune:modifiers} apply.
\end{Description}
\end{college}
