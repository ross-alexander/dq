\begin{mmgroup}{Dragons}

Dragons are the most ferocious creatures in the DragonQuest worlds.
They have a long, thin, tapering body (about 25 feet for mature
males).  They are generally reptilian in form, with sharp claws, a
pointed tail, leathery wings, large fangs, a long neck, and spiked
ridges along their backs. Their eyes glow with a shine of intelligence
inherent in no ordinary reptile, however.  Dragons are seven-hex
monsters.

Dragons have heavy scales all over their bodies, with the exception of
their undersides which are generally softer.  As some dragons age,
however, they accumulate and sleep on a hoard of gem stones that will
become embedded in them, making them as nearly invulnerable from below
as from above.  A dragon's Armor Protection Rating will be between 10
and 12 from the top, and will be between 2 and 15 on the bottom,
depending on the level of encrustation. There is a 50\% chance that a
dragon will have one vulnerable spot along the underside, regardless
of encrustation.  If a character knows the location of a weak spot, he
has a 20\% chance of hitting it on any successful Strike Check.  The
Armor Protection Rating at this spot will be 1 or 2.

All dragons are highly intelligent. Most dragons will be able to speak
1–5 human tongues at Rank 10, and 10-15 other human and humanoid
languages at Rank 6–8.  The least intelligent of dragons will be as
bright as the average human, and most will be ultra-intelligent by
human standards.

With the exception of golden dragons, dragons are highly malicious,
loving to cause as much pain and destruction as they can.  They enjoy
playing with humans, manipulating and outwitting them. Their
intelligence, however, gives them a sense of caution, and a dragon
will not hesitate to fly away from or attempt to verbally conciliate a
more powerful opponent.  In times of rage, however, they sometimes
become reckless, and it is at these times that they are the most
vulnerable.

Most dragons are greedy, and as they age they will accumulate a
tremendous hoard within their lair. Their treasure will usually be
composed of gold, gems, and other items on which the dragon will make
its bed. All dragons except the Black Dragon can occasionally be
persuaded to reveal information or perform a service for a character
if enough wealth is offered.  All save the Golden Dragon will attempt
merely to steal the treasure offered, if possible, unless it is well
guarded or the character protects himself well in some other way, for
dragons hate servitude.  By the time a dragon reaches maturity, the
wealth accumulated even in their hides will be worth a huge fortune.

Dragons love puzzles and word games and anything else that challenges
their intellect.  They love riddles and trick questions.  They also
enjoy flattery, although they will see through it almost all the
time. Nevertheless, they will be better disposed toward a flatterer
than to one who is insolent.  In general, dragons are very
hot-tempered and quick to respond to insult.

Dragons have incredibly acute senses of hearing, smell and sight. They
can see perfectly in the dark, and they have a 90\% chance per pulse of
detecting physically hidden characters.  They have a 75\% chance per
pulse of detecting the presence of invisible or otherwise magically
hidden creatures.  They will not know the exact location of invisible
creatures, although they will be able to guess well enough to hit the
character with their breath weapon (if they have one and want to use
it).

Dragons of all types generally prefer to live in caves, narrow at
their open ends, but gradually widening into long, deep caverns.  The
mouth of the cavern will usually just be large enough for the dragon
to pass with folded wings, while the main cavem will be spacious
enough for the dragon to turn easily.  A dragon's lair will usually
contain a number of wards to snare the unwary before they can approach
the dragon.  Dragons have a fierce territorial imperative, attacking
any creature that intrudes upon the area surrounding their lair, be it
human, another dragon, or some other powerful creature.

The area around a dragon's lair will often be a wasteland, devastated
by the creature.  Dragon lairs themselves will reek horribly, with
solid rock floors melted and scarred by the crcature's acidic
excretions.  The air surrounding a dragon is noxious; a dragon's
breath is foul, and its aroma sickening.  Because of their smell, all
creatures fight with 5 taken off their Base Chance to hit the dragon.
Golden Dragons are the exception to the above, with pleasant-smelling
lairs surrounded by normal countryside.

Dragons can fly according to the speed for their respective types, or
they can crawl, although comparatively slowly.  They can also hover
motionless in the air, their wings beating furiously, creating blasts
of wind beneath them.

Dragons are usually encountered alone, although rarely (10\%) a lair
will be occupied by a female with [D − 6] young dragons (40\%) or [D -
2] eggs (60\%).

Dragons' blood is highly corrosive; any time a weapon penetrates a
dragon's armor and does damage to the creature itself there is a 30\%chance that any weapon will be rendered useless, -10\% per magical Rank
inherent in the weapon.  In addition there is a 30\% chance that some
of the blood will splatter onto the wielder of the weapon if the
weapon was used in Melee or Close Combat, doing [D + 2] damage.  Armor
will absorb this type of damage, but reduce the Armor's Protection
Rating by 1 each time it is hit by the blood.

A dragon's gaze is transfixing, and any creature that looks into a
dragon's eyes must roll 3 x Willpower or less on D100 or remain
paralyzed until the dragon removes his gaze.

All dragons are able to induce fear at will in those confronting
them. Characters must roll 3 x Willpower or less on D100 or run away
in panic, dropping weapons and packs in headlong flight.  Once a
character has successfully resisted panic, he will never have to check
again for the duration of the encounter.

There is an 80\% chance that any dragon encountered in its lair will be
sleeping, but dragons are very easily awakened.  If any character is
wearing metallic armor or makes a noise exceeding a whisper they will
awaken instantly.  Even if a party is completely silent, there is a
50\% chance that their scent will be enough to awaken the dragon.

Dragons can occasionally be coerced into service if they see that
there is otherwise a good chance that they will be killed.  They will
never submit gladly, however, and will try to rebel and kill their
"master" at the earliest safe opportunity.

All dragons know the generic true name of everything, profiting from
such knowledge in the ways described in the Namer College. Powerful
dragons also make it a point to learn the true names of the most
important individuals around them in case they should be needed at
some future date.

All dragons are spell casters to a greater or lesser extent, most
specializing in the College of Sorceries of the Mind.  Most dragons
are awesome magicians, knowing all spells rituals, or talents within
their College at Ranks of 15 or higher, not to mention the many
talents inherent to their species.  All dragon magic functions exactly
as the human magic of the same name.  For range purposes, all spells
are assumed to emanate from the dragon's head.  Dragons can teach
their spells to humans, but they will only do so for vast amounts of
treasure or in exchange for some highly valuable bit of knowledge.
Dragons can use their magic while flying or hovering, but not while
participating in physical combat.

All dragons know all special knowledge and general knowledge
counterspells for all colleges at Rank 15, unless noted otherwise.

If a dragon is slain, it can cast a death curse on its treasure.  The
curse can be more specific at the GM's option, but in general the
curse will be one of bad luck, the effect of which is to influence any
roll on D100 involving the character(s) adversely by 15.

The most deadly physical weapon of most dragons is their ability to
breathe fire. The breath will emerge as a cone stretching from the
dragon's mouth, A,ith the length and the base of the cone varying with
the type of dragon.  On the tactical display the cone of fire is
considered to be present until the dragon's next action (or Pass)
after breathing, with all creatures entering the cone taking damage as
if breathed upon.  To breathe fire while on the Tactical Display a
dragon must execute a Fire action.  Damage from a dragon's fire
depends on the type of dragon, but all dragon's fire will ignite
anything flammable within the cone.  Non-magical weapons or armor have
a 10\% chance of being rendered useless if caught by dragon's fire.  In
any case damage caused by a dragon's flame cannot be absorbed by
armor.

Dragons can create windstorms with their wings (by executing a Pass
action) if they are in an area large enough for their wings to reach
their full span (30 feet).  Any creature in front of a dragon creating
a windstorm and within 25 feet of the dragon itself must roll 2 x
Physical Strength or less on D100 or be blown [D100 - 10] feet.
Subtract 20 from both rolls if the creature rolling is wearing metal
armour.  All creatures will fall prone after being blown, and any
creature which is blown a distance of 10 or more feet will take [D -
4] damage, only half of which (round down) can be absorbed by armour.
In Melee Combat a draeon can attack in any or all of three ways per
Pulse without penalty.  In any of the hexes of its Strike Zone it can
attack with two claws and a bite, and it can attack any creature in a
rear hex (a hex from which a creature attacking the dragon would get
the rear bonus) with its massive tail.  If a character is hit with the
dragon's tail, the character must roll 3 x Physical Strength or be
knocked to the ground, in addition to any damage received.

All characteristics given above as well as those for specific dragcns
are for mature dragons.  Young dragons will have half the Rank of
mature Dragons in any spells, talents, and rituals.  They will breathe
with a cone of half the width, depth, and damage of fully grown
dragons, and cannot produce windstorms.  In combat, subtract 15 from
all Base Chances and 4 from the damage of immature dragons.  Very old
dragons will have the same spell capacity as mature dragons, but their
cone of flame will be 20 feet longer and 10 feet wider and will do 2
additional points of damage.  The windstorm from a very old dragon
will do 2 additional points of damage, and all characters add 20 to
their D100 rolls to see if they blow away.  In combat, very old
dragons add 15 to their Base Chance and 4 points to all damage rolls.

\subsubsection{Black Dragon}

\begin{description}
\item[Natural Habitat]  Caverns Very Rare

\item[Number] 1

\item[Description] Black dragons have reflective scales of a solid black
color.

\item[Talents, Skills and Magic] General abilities for all dragons, as noted above.  A black
dragon can also use all the talents, spell and rituals of the College
of Ensorcelments and Enchantments or Illusions at Rank 20.  A black
dragon's breath cone is 40 feet in length and 20 feet in width at the
base, and does [D + 15]. A black dragon can breathe fire [D − 6] times
per day or a minimum of 1 time.  A black dragon uses all counterspells
at Rank 20.

\item[Weapons] The Base Chance for a black dragon's bite is 50\%, with
damage [D + 12]. The two claws have a Base Chance of 40\%, with [D + 10
damage, while the tail's Base Chance is 50\%, with [D + 6] damage.


\item[Movement Rates]  Flying 850; Running: 300

\end{description}
\begin{mmstats}{}
\textbf{PS:}  220–240
& 
\textbf{MD:}  20–22
& 
\textbf{AG:}  20–22
& 
\textbf{MA:}  30–35
\\
\textbf{EN:}  70–80
& 
\textbf{FT:}  100–120  
& 
\textbf{WP:}  30–34
& 
\textbf{PC:}  28–32
\\
\textbf{PB:}  2–4
& 
\textbf{TMR:}  17/6
& 
\textbf{NA:}  Top scales absorb 10 DP
\\
\end{mmstats}

\begin{mmcomment}
 Black dragons are questers for knowledge, and they will
occasionally release those in their grasp if they can give them rare
or valuable bits of knowledge.

\end{mmcomment}

\subsubsection{Blue Dragon}

\begin{description}
\item[Natural Habitat]Caverns Very Rare

\item[Number] 1

\item[Description] Blue dragons are sky blue, making them difficult to spot
against a clear sky.

\item[Talents, Skills and Magic] General abilities for all dragons as noted above. A blue
dragon can also use all talents, rituals, spells, etc., both general
and special of the College of Illusions or of the Mind at Rank 18. A
blue dragon cannot breathe fire.

\item[Weapons] The Base Chance for a blue dragon's bite is 50\%, with damage
[D + 10]. The two claws have a Base Chance of 45\%, with damage [D +
8], while the tail's Base Chance is 55\%, with [D + 6] damage.

\item[Movement Rates]  Flying: 700; Running: 250

\end{description}
\begin{mmstats}{}
\textbf{PS:}  230–250
& 
\textbf{MD:}  16–18
& 
\textbf{AG:}  15–17
& 
\textbf{MA:}  30–35
\\
\textbf{EN:}  75–85
& 
\textbf{FT:}  100–120
& 
\textbf{WP:}  30–34
& 
\textbf{PC:}  28–32
\\
\textbf{PB:}  2–4
& 
\textbf{TMR:}  14/5
& 
\textbf{NA:}  Top scales absorb 11 DP
\\
\end{mmstats}

\begin{mmcomment}
 Blue dragons are more cunning than some of their brethren
and if they capture a character they will often let him live in
exchange for service in the outside world. Rumors, contact with
others, transport of goods, etc., will be expected of any released,
and if they attempt to evade service, the dragon's wrath will be
great.

\end{mmcomment}

\subsubsection{Golden Dragon}

\begin{description}
\item[Natural Habitat]Caverns Very Rare

\item[Number] 1

\item[Description] Golden dragons are bright gold in color, shining from a
distance in a dazzling display. For one unfamiliar with dragons,
however, there is a 50\% chance that a golden dragon will be mistaken
for a yellow dragon. Note that golden dragons do not have the stench
of other dragons.

\item[Talents, Skills and Magic] General abilities for all dragons as noted above. In
addition golden dragons can use all talents of the College of the Mind
or the College of Illusions at Rank 18, and can use all rituals or
spells, both special and general at Rank 20. Golden dragons cannot
breathe fire. Knowledge of all counterspells is at Rank 20.

\item[Weapons] The Base Chance for a golden dragon's bite is 65\%, with
damage [D + 11]. The two claws have a Base Chance of 50\%, with damage
[D + 7], while the tail has a Base Chance of 70\%, with [D + 4] damage.

\item[Movement Rates]  Flying: 850; Runnine: 300

\end{description}
\begin{mmstats}{}
\textbf{PS:}  300–320
& 
\textbf{MD:}  20–24
& 
\textbf{AG:}  18–20
& 
\textbf{MA:}  32–37
\\
\textbf{EN:}  90–100
& 
\textbf{FT:}  140–160
& 
\textbf{WP:}  32–37
& 
\textbf{PC:}  30–35
\\
\textbf{PB:}  5–7
& 
\textbf{TMR:}  17/6
& 
\textbf{NA:}   Top scales absorb 12 DP.
\\
\end{mmstats}

\begin{mmcomment}
 Golden dragons are the only draeons that can be described
as just. They will not attack unless provoked, and can be bargained
with more readily than other dragons. They generally despise evil
dragons, and will frequently attack them.

\end{mmcomment}

\subsubsection{Green Dragon}

\begin{description}
\item[Natural Habitat]  Caverns Very rare

\item[Number] 1

\item[Description] Green dragons have outer scales the color of dark pine
needles.

\item[Talents, Skills and Magic] General abilities for all dragons, as noted above. Green
dragons can also use all spells, talents, rituals, etc., of the
College of the Mind or the College of Illusions at Rank 12. The cone
of fire of their breath is 60 feet long and 30 feet wide, and does [D
+ 12] to all within the cone.  They can breathe fire D10 times on any
given day.

\item[Weapons] The Base Chance for a green dragon's bite is 60\%, and
damage is [D + 12]. The two claws have a Base Chance of 50\%,
with [D + 6] damage, while the tail's Base Chance is 70\%, with
damage [D + 4].

\item[Movement Rates]  Flying: 700; Running: 250

\end{description}
\begin{mmstats}{}
\textbf{PS:}  300–350
& 
\textbf{MD:}  19–21
& 
\textbf{AG:}  17–19
& 
\textbf{MA:}  22–25
\\
\textbf{EN:}  85–95
& 
\textbf{FT:}  120–150
& 
\textbf{WP:}  27–33
& 
\textbf{PC:}  27–30
\\
\textbf{PB:}  2–4
& 
\textbf{TMR:}  14/5
& 
\textbf{NA:}  Top scales absorb 12 DP
\\
\end{mmstats}

\begin{mmcomment}
 Green dragons are quite evil, although they are curious and
will question captives thoroughly before disposing of them. After a
green dragon dies, for the next hour or so a sip of its blood will
allow permanent comprehension and ability to speak with any normal
animal or avian without damage to the drinker.

\end{mmcomment}

\subsubsection{Red Dragon}

\begin{description}
\item[Natural Habitat]  Caverns Very Rare

\item[Number] 1

\item[Description] Red dragons are fiery colored dragons, with flecks of
gold along their scales.

\item[Talents, Skills and Magic] General abilities for dragons, as noted above. They can also
use all spells, talents, and rituals, both general and special, of the
College of the Mind at Rank 17.  The  cone of fire of  a red dragon is
80 feet long and  40 feet wide at  the base, and does [D  + 15] to any
creature in the cone. The  breath weapon can be  used [D + 3] times in
any given day.

\item[Weapons] A red dragon's bite's Base Chance is 70\%, and damage is [D +
10]. The two claws have a Base Chance of 50\%, with [D + 4] damage,
while the tail's Base Chance is 60\%, with damage [D + 2].

\item[Movement Rates]  Flying: 750;Running: 250

\end{description}
\begin{mmstats}{}
\textbf{PS:}  250–300
& 
\textbf{MD:}  18–20
& 
\textbf{AG:}  16–18
& 
\textbf{MA:}  25–30   
\\
\textbf{EN:}  10–90
& 
\textbf{FT:}  110–140  
& 
\textbf{WP:}  30–35
& 
\textbf{PC:}  21–30
\\
\textbf{PB:}  2–4
& 
\textbf{TMR:}  15/5
& 
\textbf{NA:}  Top scales absorb 12 DP
\\
\end{mmstats}

\begin{mmcomment}
 Red dragons will play games with those they encounter until
they tire of their pitiful struggles and then slowly kill them and
take all their treasure. If a character somehow impresses a red
dragon, there is a 40\% chance they will be left alive.

\end{mmcomment}

\subsubsection{Yellow Dragon}

\begin{description}
\item[Natural Habitat]  Caverns Very Rare

\item[Number] 1

\item[Description] Yellow dragons have yellowish scales. Note that when
seen from distances of 100 feet or more there is a 50\% chance that
this dragon will be mistaken for a golden dragon, and vice-versa.

\item[Talents, Skills and Magic] General abilities for all dragons, as noted above. Yellow
dragons also use all spells, talents, rituals, etc., of the College of
the Mind or the College of Illusions at Rank 15. Their breath's cone
of flame is 60 feet long and 30 feet wide and does [D + 12]. They can
breathe fire [D + 1] times on any given day. Yellow dragons use all
counterspells at Rank 12.

\item[Weapons] The Base Chance for a yellow dragon's bite is 60\%, with
damage [D + 10]. The two claws have a Base Chance of 45\%, with [D + 5]
damage, while the tail's Base Chance is 65\%, with [D + 3] damage.


\item[Movement Rates]  Flying: 700, Running: 300

\end{description}
\begin{mmstats}{}
\textbf{PS:}  280–320
& 
\textbf{MD:}  20–22
& 
\textbf{AG:}  18–20 
& 
\textbf{MA:}  24–27
\\
\textbf{EN:}  80–90
& 
\textbf{FT:}  110–140
& 
\textbf{WP:}  26–32
& 
\textbf{PC:}  27–30
\\
\textbf{PB:}  2–4
& 
\textbf{TMR:}  14/6
& 
\textbf{NA:}  Top scales absorb 11 DP
\\
\end{mmstats}

\begin{mmcomment}
 Yellow dragons have a particular fondness for gold over
gems and other items of value.

\end{mmcomment}
\end{mmgroup}

