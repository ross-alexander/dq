\begin{mmgroup}{Fantastical monsters}

Fantastical Masters include a number of beings of legend, not often
seen by humans and related species.  They tend to make their homes in
inaccessable areas where few men go.  These species include: centaurs,
chimaeras, giant amoebas, gorgons, manticores, minotaurs, nagas,
sphinxes, and unicorns. Due to their rarity, live specimens are
usually of great value in the marketplace.

\subsubsection{Centaur}

\begin{description}
\item[Natural Habitat] Woods, Marsh, Rough, Caverns, Plains Uncommon

\item[Number] 2–20 (4)

\item[Description] Centaurs are half-man, half-horse. They are human down
to the hips, but they join the body of a horse where the neck would
normally be. Their lower half has hide, just as a horse's, while their
top half is that of a normal man.

\item[Talents, Skills and Magic] Centaurs can have all the abilities and skills of a human.
Centaurs in general are good with bows and at hunting, and have an
affinity for healing and the art of prophecy.

\item[Weapons] Centaurs use weapons as do men. They will usually have Rank
in one or more of their weapons.

\item[Movement Rates] Running: 600

\end{description}
\begin{mmstats}{}
\textbf{PS:}  10–30   
& 
\textbf{MD:}  5–20
& 
\textbf{AG:}  10–23
& 
\textbf{MA:}  5–23
\\
\textbf{EN:}  12–20
& 
\textbf{FT:}  20–30
& 
\textbf{WP:}  7–26
& 
\textbf{PC:}  10–30
\\
\textbf{PB:}  12–17
& 
\textbf{TMR:}  12
& 
\textbf{NA:}  Hide absorbs 3 DP
\\
\end{mmstats}

\begin{mmcomment}
 Centaurs cannot resist alcohol and become violent when
drunk.  Centaurs will only rarely let a human ride them, and only then
at pressing need.  They eat raw flesh (including human flesh), and
will often abduct young maidens for food and other purposes.
\end{mmcomment}

\subsubsection{Chimaera}

\begin{description}
\item[Natural Habitat] Woods, Rough, Caverns, Ruins Very Rare

\item[Number] 1–3 (1)

\item[Description] The chimaera has tHe head of a goat, the foreparts of a
lion, and the rear section of a dragon.  Chimaeras are large (up to 12
feet long) and breath fire. They are 3-hex monsters.

\item[Talents, Skills and Magic] The chimaera can brcath a cone of fire. Other than that, it
has no special skills or magical ability.

\item[Weapons] The chimaera has a fire breath that it can use in Ranged and
Melee combat. The range of the cone of breath is 50 feet and at the
base the cone is 20 feet in diameter. All within the cone suffer [D +
15] damage. A chimera must execute a Fire action to breath in this
fashion. In Melee Combat and Close Combat, the chimera has a bite like
that of a huge lion (Base Chance of 75\% + 8 damage).

\item[Movement Rates]  Running: 500

\end{description}
\begin{mmstats}{}
\textbf{PS:}  28–32
& 
\textbf{MD:}  25–28
& 
\textbf{AG:}  15–20
& 
\textbf{MA:}  None
\\
\textbf{EN:}  20–22
& 
\textbf{FT:}  30–34
& 
\textbf{WP:}  14–19
& 
\textbf{PC:}  13–20
\\
\textbf{PB:}  3–7
& 
\textbf{TMR:}  10
& 
\textbf{NA:}  Hide absorbs 8 DP
\\
\end{mmstats}

\begin{mmcomment}
 Chimaera thrive on ruin, and the area surrounding one of
their lairs will be a burned wasteland. In the area surrounding the
lair, or occasionally in the lair itself, there may be victims with
some treasure (25\%, 1–6 bodies with 100-600 Silver Pennies, 25\% each
has something else of value), but otherwise chimaera do not hoard
wealth as do dragons.
\end{mmcomment}

\subsubsection{Giant amoeba}

\begin{description}
\item[Natural Habitat] Caverns, Ruins Uncommon

\item[Number]   1–6 (1)


\item[Description] A giant amoeba is a shapeless, flowing creature between
6 inches and 6 feet in diameter.

\item[Talents, Skills and Magic] A giant amoeba can sense any organic material within 25
feet, and will move toward the closest such material that it can
sense. Giant amoeba are able to cat anything they come in contact
with. They can slip under doors and through very small cracks.

\item[Weapons] A giant amoeba does not attack, per se, but rather attempts
to consume anything in its way. If a giant amoeba is ever in the same
hex on the tactical display as any living creature, that creature
takes 2 DP per Pulse until it leaves the hex occupied by the
amoeba. Note that if a creature is fully consumed, any weapons and
other non-organic materials will be left behind, although all bones
will be consumed.

\item[Movement Rates]  Crawling: 50

\end{description}
\begin{mmstats}{}
\textbf{PS:}  None
& 
\textbf{MD:}  None
& 
\textbf{AG:}  3–4
& 
\textbf{MA:}  None
\\
\textbf{EN:}  10–12  
& 
\textbf{FT:}  20–24
& 
\textbf{WP:}  6–8
& 
\textbf{PC:}  6–8
\\
\textbf{PB:}  3–5
& 
\textbf{TMR:}  1
& 
\textbf{NA:}  None
\\
\end{mmstats}

\begin{mmcomment}
 If a giant amoeba is reduced to 0 endurance as a result of
the attacks of normal (non-magical) weapons, the amoeba merely splits
into two amoebas, each with half the size, endurance, and fatigue of
the original amoeba. Magical weapons and magical attacks affect the
amoeba normally.
\end{mmcomment}

\subsubsection{Gorgon (medusa)}

\begin{description}
\item[Natural Habitat] Woods and Wilderness (lairs in caverns) Very rare 

\item[Number]  1–3 (1)

\item[Description] Gorgons are physically humanoid, but boast a headful of
writhing green snakes of a venomous nature. They also have hypnotic,
burning red eyes. Gorgons like to appear as comely maidens and often
wear the attire of human females.  They have large brazen claws and
hog-like teeth. They specialize in enticing males who they then turn
to stone.

\item[Talents, Skills and Magic] Gorgons possess no special skills or magic as a rule, but
may learn human skills and magic. They have the special talent of
turning those that look directly into their eyes to stone. Any
character facing a gorgon must roll four times Willpower or less each
pulse that they face the beast or they succumb to her blandishments,
looks into her eyes and is turned to stone.

\item[Weapons] In addition to her eyes, the Gorgon may Melee Attack with
claws (Base Chance of 50\%, + 4 damage, Rank of 1–5) or Close Combat
using claws, teeth and hair (Base Chance of 30\%, [D + 0] damage, but
possible poisoning as from an asp bite and no Rank). Gorgons may
attack using hair, teeth and claws in the same pulse. The gorgon may
attempt to turn a character to stone any time.


\item[Movement Rates] Running: 250

\end{description}
\begin{mmstats}{}
\textbf{PS:}  10–13
& 
\textbf{MD:}  12–15  
& 
\textbf{AG:}  10–14
& 
\textbf{MA:}  15–18
\\
\textbf{EN:}  10–14   
& 
\textbf{FT:}  15–19
& 
\textbf{WP:}  16–20
& 
\textbf{PC:}  16–18
\\
\textbf{PB:}  Always 0
& 
\textbf{TMR:}  5
& 
\textbf{NA:}  None
\\
\end{mmstats}

\begin{mmcomment}
 The gorgon's eyes only become visible at a range of 100
feet and she cannot turn a character to stone beyond that range. The
attempt to turn a character to stone is automatic whenever a character
faces the gorgon's front and requires no action.
\end{mmcomment}

\subsubsection{Manticore}

\begin{description}
\item[Natural Habitat] Rough, Caverns Rare

\item[Number] 1–6 (1)

\item[Description] Manticores have the body of a lion, bat-like wings, and
the head of a human, although larger to fit their bodies. At the tip
of their tail they have up to 12 spikes, which they can launch as
weapons.

\item[Talents, Skills and Magic] Manticores have no magical properties, and no special
abilities other than the ability to launch their tail spikes.

\item[Weapons] Manticores can use their tail spikes in Ranged Combat as if
they were heavy crossbows. They are able to launch up to 6 of the
spikes at any one time as long as the spikes are all aimed at spots
within 6 feet of each other. In Melee Combat, the manticore can attack
with its two claws (Base Chance of 60\%, [D + 5] damage). Once their
tail spikes are exhausted (they regenerate in about a day) manticores
try to enter Close Combat as soon as possible, where they can use
their claws.

\item[Movement Rates] Flying: 500; Running: 350

\end{description}
\begin{mmstats}{}
\textbf{PS:}  28–32
& 
\textbf{MD:}  20–25
& 
\textbf{AG:}  26–30
& 
\textbf{MA:}  None
\\
\textbf{EN:}  12–14
& 
\textbf{FT:}  20–25
& 
\textbf{WP:}  12–18
& 
\textbf{PC:}  12–18
\\
\textbf{PB:}  3–6
& 
\textbf{TMR:}  10/7
& 
\textbf{NA:}  Fur absorbs 8 DP
\\
\end{mmstats}

\begin{mmcomment}
 Manticores like to hunt, and their favorite prey is
man. They will lie in wait for a party, and then send their spikes
whirling into it. If the manticores lair is found, there is a chance
(30\%) that it will have dragged bodies with treasure on them into its
cave.
\end{mmcomment}

\subsubsection{Minotaur}

\begin{description}
\item[Natural Habitat] Caverns, Woods, Rough. Very Rare

\item[Number] 1–6 (1)

\item[Description] Minotaurs are humanoid, with the head of a bull and a
very hairy hide. They have a tail, just like that of a bull.

\item[Talents, Skills and Magic] The minotaur has no special magical abilities or
talents. They are tool users and will sometimes use simple weapons.

\item[Weapons]  A minotaur can attack by butting with his horns,
biting, or attacking A,ith a weapon. Butt: Base Chance of 40\%,
[D + 3] damage. Bite: Base Chance of 30\%, [D − 1] damage. A
minotaur will hold Rank 1–5 with whatever weapon it uses. The
minotaur can use any combination of two of these attacks in
any one pulse A,ithout penalty. in Close Combat the minotaur
can use only his bite, but the Base Chance goes up to 50\%.


\item[Movement Rates]  Running: 300

\end{description}
\begin{mmstats}{}
\textbf{PS:}  22–26
& 
\textbf{MD:}  18–20
& 
\textbf{AG:}  14–17
& 
\textbf{MA:}  None
\\
\textbf{EN:}  14–16
& 
\textbf{FT:}  22–25  
& 
\textbf{WP:}  14–16
& 
\textbf{PC:}  18–20
\\
\textbf{PB:}  4–7
& 
\textbf{TMR:}  6
& 
\textbf{NA:}  Hide absorbs 6 DP
\\
\end{mmstats}

\begin{mmcomment}
  Minotaurs are particularly vicious, and will attack
virtually anything that their dim intelligence tells them they have
even a mediocre chance of beating. These beasts generally like
the dark, and will only rarely be found in the open after sunup.
\end{mmcomment}

\subsubsection{Naga}

\begin{description}
\item[Natural Habitat] Crypts, Marsh  Very Rare

\item[Number] 1–6 (1)

\item[Description]  Nagas are humanoid above the waist, and have
the body of a serpent below. Male nagas have the upper half Of
a man, while nagians (female nagas) have the upper half of a
woman. Both types will usually be 10–12 feet long.

\item[Talents, Skills and Magic] Nagas are frequently (85\%) members of one of the Colleges of
Thaumaturgies. If a naga is a magic-user, it will have Rank 2–8 with
each of the General Knowledge spells, talents, and rituals, and will
have Rank 1–5 with those Special Knowledge spells and rituals that
they know (Usually 5–10 will be known). Nagas can also read the minds
of any that they can see, understanding both the thoughts and
intentions of the subject. This talent cannot be resisted.

\item[Weapons] Naga will use ordinary edged weapons 50\% of the time, and if
they do use a weapon, they will have Rank 4–6 with it. If they do not
use a weapon, they can attack with a bite or a constriction
attack. The bite has a Base Chance of 55\% and does [D + 4] damage,
while the constriction has a Base Chance of 40\% and does [D + 8]
damage. The bite can be used in either Close or Melee Combat, while
the constriction can only be used in Close Combat. If a nagas bite
penetrates an enemy's armor (i.e. does damage to the character's
Fatigue or Endurance), then the victim takes 2 additional DP per Pulse
for D10 Pulses because of the nagas poison. Only an antidote
specifically designed for naga venom will neutralize this poison. A
naga can also spit this poison up to a range of 40 feet. The spittle
has a Base Chance of 30\% (modified for range as an ordinary hurled
weapon) and does [D + 4] damage.

\item[Movement Rates]  Swimming: 400; Crawling: 300

\end{description}
\begin{mmstats}{}
\textbf{PS:}  20–25
& 
\textbf{MD:}  17–21
& 
\textbf{AG:}  12–16
& 
\textbf{MA:}  16–20
\\
\textbf{EN:}  25–32
& 
\textbf{FT:}  20–25  
& 
\textbf{WP:}  20–24
& 
\textbf{PC:}  19–23
\\
\textbf{PB:}  13–17
& 
\textbf{TMR:}  8/6
& 
\textbf{NA:}  Scales absorb 5 DP
\\
\end{mmstats}

\begin{mmcomment}
 Nagas are often the guardians and keepers of knowledge.
They seek to preserve powerful knowledge from the use of those who
would not use it properly, and at the same timc they try to deliver it
to those who could best use it for the cause of good. This knowledge
might be magical in nature, or of some other type. Nagas will use
force to defend the knowledge that they guard (which will usually be
in the form of a written tome) but will warn intruders beforehand, and
allow them a chance to get away.
\end{mmcomment}

\subsubsection{Sphinx}

\begin{description}
\item[Natural Habitat] Rough, Woods Very Rare

\item[Number] 1

\item[Description] A sphinx has the body of a winged lion, with the head
and breasts of a woman. A sphinx is usually about 12 feet tong. A
sphinx has large, sharp teeth, and is a two-hex monster.

\item[Talents, Skills and Magic] Most sphinxes are accomplished members of a College of
Magic, usually one of the Thaumaturgies. They will know all General
Knowledge spells at Rank 6–9, and will know all Special KnoA,Iedge
spells at Rank 3–6. In addition, a sphinx will know D10 counterspells
from other Colleges. Sphinxes also have excellent senses of smell.
They will be able track as if they had Rank 8 in the Ranger ability,
and they will be able to detect the presence of hidden or invisible
characters 75\% of the time.

\item[Weapons] A sphinx can attack three times (once with a bite, and twice
with its claws) in the same Pulse without penalty. The bite has a Base
Chance of 75\% and does [D + 8] damage. The claws have a Base
Chance of 60\% and do [D + 4] damage.


\item[Movement Rates]  Running: 500; Flying: 600

\end{description}
\begin{mmstats}{}
\textbf{PS:}  30–35
& 
\textbf{MD:}  22–24
& 
\textbf{AG:}  17–19
& 
\textbf{MA:}  12–22
\\
\textbf{EN:}  40–50
& 
\textbf{FT:}  60–75
& 
\textbf{WP:}  20–23
& 
\textbf{PC:}  17–19
\\
\textbf{PB:}  4–6
& 
\textbf{TMR:}  10/12
& 
\textbf{NA:}  Hide absorbs 6 DP
\\
\end{mmstats}

\begin{mmcomment}
 Sphinxes are proverbial riddle-lovers. They love to learn
new riddles, and will sometimes let a passerby live in exchange for a
good one. They also like to ask riddles, however.  When a sphinx asks
a riddle, it will state what will happen to a character who does not
answer the riddle successfully, and what reward (usually just free
passage) will be given to those who do.  A sphinx will always try to
keep its word as to what it will do if the riddle is answered,
although there is a 2\% chance that it will simply kill itself if the
riddle is answered correctly.
\end{mmcomment}

\subsubsection{Unicorn}

\begin{description}
\item[Natural Habitat] Woods, Plains Rare

\item[Number] 1–8 (2)

\item[Description] Unicorns are white equines with a single, long horn
coming out of their forehead. They have a singic black, 2 foot long
horn set in a deer's head, very thick feet, and the tail of a boar.

\item[Talents, Skills and Magic] Unicorns are immune to poison, and a character who possesses
one of their horns is also immune. They are also almost impossible to
trap as they are very intelligent and wary. They have 5 times the
strength of an average human. They are unable to cast spells in the
usual sense.

\item[Weapons] In Melee Combat, a unicorn uses its horn (Base Chance of
60\%, [D + 7] damage). In Close Combat, it can attack with its hooves
as a Warhorse.

\item[Movement Rates] Running: 600

\end{description}
\begin{mmstats}{}
\textbf{PS:}  55–60
& 
\textbf{MD:}  None
& 
\textbf{AG:}  16–19
& 
\textbf{MA:}  None
\\
\textbf{EN:}  25–30
& 
\textbf{FT:}  50–60
& 
\textbf{WP:}  20–25
& 
\textbf{PC:}  25–30
\\
\textbf{PB:}  18–20
& 
\textbf{TMR:}  12
& 
\textbf{NA:}  Hide absorbs 4 DP
\\
\end{mmstats}

\begin{mmcomment}
 Unicorns are virtually untameable by ordinary men, but a
unicorn can occasionally be tamed by a virgin (40\%) as unicorns love
purity and innocence.
\end{mmcomment}
\end{mmgroup}

