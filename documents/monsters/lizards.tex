\begin{mmgroup}{Lizards, snakes and insects}

\subsection{Lizards and Kindred}
The following are included in this section: basilisks, crocodiles,
giant land turtles, gila monsters, hydras, land iguanas, salamanders,
suarime, and wyverns.  Except for wyverns and suarime, these species
will be unintelligent.

\subsubsection{Basilisk (Cockatrice)}

\begin{description}
\item[Natural Habitat] All habitats except Plains and Oceans Uncommon

\item[Number] 1–2 (1)

\item[Description] The Basilisk is a fat reptilian creature about 5 feet
long and 2 feet in height. These are slow, heavily-armoured lizards
with limited intelligence.  They have strong jaws with two-inch
fangs.  They are man-eaters, but infinitely prefer fish when they can
get it.  Basilisks are usually brownish in colour with lighter
underbellies.

\item[Talents, Skills and Magic] A basilisk has no skills or magical abilities to speak of,
but does possess the special talent of turning a target to stone with
a glance.  Anyone who is within 50 feet of the basilisk may be
attacked in this manner.  The basilisk expends a Fire Action and the
figure against whom the action is directed undergoes an attack as if
from a Thrown Weapon (BC of 60\%). A basilisk breathes a cone of
poison gas 5 feet by 3 feet wide as its base.

\item[Weapons] The basilisk does not use weapons, but may bite (Base Chance
of 40\%, Damage + 3) during Close Combat and may use its gaze in
Ranged, Melee and Close Combat and breath in Melee or Close Combat.
Any hit scored with its breath does D + 10 Damage due to poisoning
(not absorbed by armour or shields). If a basilisk's gaze is reflected
back at itself, it is killed.

\item[Movement Rates] Running: 125

\end{description}
\begin{mmstats}{}
\textbf{PS:}  6–12
& 
\textbf{MD:}  None
& 
\textbf{AG:}  7–11
& 
\textbf{MA:}  None
\\
\textbf{EN:}  12–14
& 
\textbf{FT:}  15–20
& 
\textbf{WP:}  12–16
& 
\textbf{PC:}  15–20
\\
\textbf{PB:}  5–7
& 
\textbf{TMR:}  2
& 
\textbf{NA:}  Skin absorbs 6 DP
\\
\end{mmstats}

\begin{mmcomment}
 Basilisks are solitary creatures, but they are willing to
serve others in exchange for lavish supplies of food (12 pounds or
more per day).
\end{mmcomment}

\subsubsection{Crocodile}

\begin{description}
\item[Natural Habitat] Marsh, Lakes, Rivers Common

\item[Number] 1–50 (20)

\item[Description]Crocodiles are heavily scaled lizards with small sharp teeth.  They
attain lengths of 8 feet, and weights of up to 180 pounds.  Their
scales are a very dark greenish-brown that blends in well with the
muddy water that they love to inhabit.

\item[Talents, Skills and Magic]Crocodiles have no special talents, skills or magic.

\item[Weapons] Crocodiles cannot attack except in Close Combat and they
attack with two claws (Base Chance of 50\%, [D − 6] Damage) and
their bite (Base Chance of 10\%, [D + 1] Damage).

\item[Movement Rates] Running: 50; Swimming: 150

\end{description}
\begin{mmstats}{}
\textbf{PS:}  17–19
& 
\textbf{MD:}  5–8
& 
\textbf{AG:}  7–9
& 
\textbf{MA:}  None
\\
\textbf{EN:}  7–9
& 
\textbf{FT:}  15–18
& 
\textbf{WP:}  7–9
& 
\textbf{PC:}  8–10
\\
\textbf{PB:}  7–9
& 
\textbf{TMR:}  1/3
& 
\textbf{NA:}  Scales absorb 6 DP
\\
\end{mmstats}

\begin{mmcomment}
 Crocodiles often lurk just below the surface of murky
waters, waiting for a tidbit to enter the water.  On land, however,
crocodiles are rather timid, and they will slip off into the water if
they sense something approaching.  If a crocodile's jaws are grasped
while they are still closed, it only takes a PS of 12 to hold them
closed, rendering the Crocodile's bite useless.  The crocodile's skin
is used to make primitive armour (equal to leather) and the teeth
(about 60) are valuable (100 Silver Pennies each) as charms.
\end{mmcomment}

\subsubsection{Giant Land Turtle}

\begin{description}
\item[Natural Habitat] Marsh (or beach) Rare

\item[Number] 1–30 (6)

\item[Description]Giant Land Turtles have the form of an ordinary turtle, with a thick
green shell and claws instead of webbed digits. They are about 5 feet
long, and weigh about 700 pounds.

\item[Talents, Skills and Magic] Giant Land Turtles can withdraw their head, tail, and limbs
inside their shell in times of danger.  They have no magical or other
special abilities.  If the Turtle is not withdrawn into its shell,
there is a 80\% chance that any blow will strike the shell anyway. If
the turtle is inside its shell, all strikes will be softened by the
shell.

\item[Weapons] The Land Turtle can only attack by biting in Close
Combat. Its Base Chance is 25\% and its Damage is [D − 2].

\item[Movement Rates] Running or Swimming: 100

\end{description}
\begin{mmstats}{}
\textbf{PS:}  20–25
& 
\textbf{MD:}  None
& 
\textbf{AG:}  5–7
& 
\textbf{MA:}  None
\\
\textbf{EN:}  15–17
& 
\textbf{FT:}  22–24
& 
\textbf{WP:}  10–11
& 
\textbf{PC:}  13–15
\\
\textbf{PB:}  7–9
& 
\textbf{TMR:}  2
& 
\textbf{NA:}  Shell absorbs 8 DP
\\
\end{mmstats}

\begin{mmcomment}
 Despite their name, land turtles spend much of their time
in the water, where they will frequently be found.  A land turtle is
capable of carrying a large burden (up to 400 lbs.)  if one is willing
to keep a pace that the turtle can follow.
\end{mmcomment}

\subsubsection{Gila Monster}

\begin{description}
\item[Natural Habitat] Waste Rare

\item[Number] 1–8 (1)

\item[Description] Gila Monsters are black and yellow lizards with short, thin
limbs and a striped stubby tail.

\item[Talents, Skills and Magic] The gila monster has no special talents, skills, or magic.

\item[Weapons] Gila monsters can bite in Close Combat, but they cannot
attack in Ranged or Melee Combat.  The Base Chance for their bite is
50\%, and it does [D − 3] Damage.  If the bite penetrates any
armour that might be worn to do actual damage to Fatigue or Endurance,
the target takes 2 DP per pulse (not absorbed by armour) for the next
D10 pulses, or until an antidote to the Gila monster's poison is
administered.

\item[Movement Rates] Running: 100

\end{description}
\begin{mmstats}{}
\textbf{PS:}  3–4
& 
\textbf{MD:}  8–10
& 
\textbf{AG:}  7–8
& 
\textbf{MA:}  None
\\
\textbf{EN:}  4–5
& 
\textbf{FT:}  8–10
& 
\textbf{WP:}  7–9
& 
\textbf{PC:}  15–17
\\
\textbf{PB:}  7–9
& 
\textbf{TMR:}  2
& 
\textbf{NA:}  Hide absorbs 2 DP
\\
\end{mmstats}

\subsubsection{Hydra}

\begin{description}
\item[Natural Habitat] Marsh, Caverns Very Rare

\item[Number] 1–3 (1)

\item[Description] A Hydra is a nine-headed snake.  They are 12 to 15 feet
long, and have thick green scales.  Hydras also have a foul smell and
venomous breath.  Hydras are four-hex creatures.

\item[Talents, Skills and Magic] If a hydra is hit in combat for four or more points of
damage (after subtracting for the defensive benefits of the hydra's
scales) there is a 70\% chance that one of the hydra's heads has been
destroyed.  Two pulses after a head is destroyed, two more grow back,
and on the beginning of the next pulse after that they can attack in
combat. One of the hydra's original nine heads will be immortal.  This
head cannot be killed, and does not regenerate as do the others.
Instead, if a hydra has no Endurance remaining, the head is assumed
to have been cut off.  If the head is cut off, it can no longer move
or attack except in Close Combat.  The only way to kill one of the
hydra's mortal heads is to burn it while it is regenerating (a
successful strike with a torch will do this). Each time a head
regenerates, the Hydra gains three points of Endurance.  (Note that
this will occasionally mean that a hydra will have more Endurance
points at the end of a battle than before). In any event, if a hydra's
Endurance is ever reduced to zero or below, all of the heads die
except the immortal one mentioned above.

\item[Weapons] A hydra can attack once with each of its heads.  Up to six
heads can attack without penalty in either Close or Melee Combat.  The
Base Chance for one of a hydra's heads is 55\%, and each bite
does [D + 2] Damage.  In addition, if a bite penetrates target's
armour to do damage to Fatigue or Endurance, the target takes 5 DP per
pulse for the next D10 pulses due to the hydra's poison, which is
deadly.  Only antidotes specifically designed for hydra poison will be
effective against their venom.

\item[Movement Rates] Crawling: 200

\end{description}
\begin{mmstats}{}
\textbf{PS:}  18–22
& 
\textbf{MD:}  19–24
& 
\textbf{AG:}  14–16
& 
\textbf{MA:}  None
\\
\textbf{EN:}  30–35
& 
\textbf{FT:}  40–45
& 
\textbf{WP:}  18–23
& 
\textbf{PC:}  14–17
\\
\textbf{PB:}  4–6
& 
\textbf{TMR:}  4
& 
\textbf{NA:}  Scales  absorb 7 DP
\\
\end{mmstats}

\begin{mmcomment}
 Hydras are vicious, but they are not overly intelligent.
They will attack anything that approaches their lair.  A hydra's
poison lasts even after the creature dies, and can be absorbed through
the skin without a puncture.
\end{mmcomment}

\subsubsection{Land Iguana}

\begin{description}
\item[Natural Habitat] Woods and Waste Uncommon

\item[Number] 1–4 (2)

\item[Description] Iguanas are large lizards, sometimes reaching more than 3 feet
in length.  They are sandy to brown in colour, and have ridges along
their back.  They have a short, thick tail, and wrinkled skin around
their neck.  Giant iguanas may up to 3 times normal size and have
double or triple PS, EN, and FT.

\item[Talents, Skills and Magic] Iguanas have no magical abilities or special talents.

\item[Weapons] Iguanas can only attack in Close Combat.  They get one
attack with their bite, which has a Base Chance of 50\%, and does
[D + 4] Damage.

\item[Movement Rates] Crawling: 250

\end{description}
\begin{mmstats}{}
\textbf{PS:}  9–11
& 
\textbf{MD:}  None
& 
\textbf{AG:}  14–16
& 
\textbf{MA:}  None
\\
\textbf{EN:}  4–6
& 
\textbf{FT:}  8–10
& 
\textbf{WP:}  6–8
& 
\textbf{PC:}  10–12
\\
\textbf{PB:}  6–8
& 
\textbf{TMR:}  5
& 
\textbf{NA:}  Hide absorbs 3 DP
\\
\end{mmstats}

\subsubsection{Salamander}

\begin{description}
\item[Natural Habitat] Waste (particularly deserts) Rare

\item[Number] 1–2 (1)

\item[Description] A salamander is a three foot long lizard, reddish brown in
colour, with fiery red eyes.

\item[Talents, Skills and Magic] Salamanders have the ability to set things on fire by
concentrating their gaze. The action is deliberate, in that something
will not be burnt unless the salamander wishes to burn it.  Only
flammable items can be ignited. If a salamander concentrates its gaze
on a living creature, the creature takes [D + 12] Damage.  The gaze
can be resisted, and only one creature can be stared at at any one
time.  Treat the gaze as a Fire action on the Tactical Display.

\item[Weapons] A salamander can use its gaze in Close, Ranged, and Melee
Combat (range: 200 feet).  In addition, a salamander can make a bite
attack in Close Combat with a Base Chance of 40\%, doing [D + 2]
Damage.

\item[Movement Rates] Running: 350

\end{description}
\begin{mmstats}{}
\textbf{PS:}  14–17
& 
\textbf{MD:}  8–10
& 
\textbf{AG:}  17–20
& 
\textbf{MA:}  None
\\
\textbf{EN:}  12–14
& 
\textbf{FT:}  15–20
& 
\textbf{WP:}  21–24
& 
\textbf{PC:}  18–21
\\
\textbf{PB:}  5–7
& 
\textbf{TMR:}  7
& 
\textbf{NA:}  Scales absorb 4 DP
\\
\end{mmstats}

\begin{mmcomment}
 Salamanders love to set things on fire in a seemingly
random fashion.
\end{mmcomment}

\subsubsection{Suarime (Lizard Man)}

\begin{description}
\item[Natural Habitat] Marsh, Caverns (near water) Rare

\item[Number]   1–50   (8)

\item[Description] Suarime are basically humanoid, but they are reptilian in
outward appearance.  They have heavy scales along the entire body, and
have a long, heavy tail that they can use as a weapon to knock down
their victims.  They also have claws and a long forked tongue. They
are about 7 feet tall, and are greenish-yellow in colour.

\item[Talents, Skills and Magic] Suarime can fight normally under water, but they must come
up for air eventually, although they can hold their breath for periods
of more than 5 minutes.  They have their own language, but will rarely
(5\%) speak anything comprehensible to men. They do not normally
use magic, although intelligence varies widely.

\item[Weapons] Lizard men generally use simple weapons like spears or
clubs. The larger the weapon, the more the suarime prefer it as they
greatly enjoy using their strength to the utmost.  Suarime will use
shields if they find them or capture them.  Their claws have a Base
Chance of 35\% of doing [D + 1] Damage.


\item[Movement Rates] Swimming: 300; Running: 100

\end{description}
\begin{mmstats}{}
\textbf{PS:}  23–26
& 
\textbf{MD:}  8–11
& 
\textbf{AG:}  8–12
& 
\textbf{MA:}  10–15
\\
\textbf{EN:}  14–16
& 
\textbf{FT:}  20–24
& 
\textbf{WP:}  14–18
& 
\textbf{PC:}  10–14
\\
\textbf{PB:}  8–11
& 
\textbf{TMR:}  6/2
& 
\textbf{NA:}  Scales absorb 6 DP
\\
\end{mmstats}

\begin{mmcomment}
 Suarime will eat anything and they feed on marsh birds and
underwater creatures, but they have a fondness for human flesh.
\end{mmcomment}

\subsubsection{Wyvern (Mere Dragon)}

\begin{description}
\item[Natural Habitat] Rough (hills mostly), Woods, Marsh Uncommon

\item[Number] 1–5 (2)

\item[Description] Wyverns are distant cousins of dragons, but are smaller and not
blessed with the intelligence of dragons.  Usually, 6 to 10 feet tall,
the wyvern is portrayed as a one-hex character with its tail extending
into its rear hex a short distance (just enough so that it can knock a
character standing in that hex off their feet). Wyverns are slate gray
in colour and have tough armoured hides.

\item[Talents, Skills and Magic]Wyverns, unlike their larger cousins, are non-magical. Their
shriveled front limbs are not suitable for grasping much except
already subdued prey.  The wyvern's tail contains a scorpion-like
sting which may be used to infect a target in the hex the wyvern is
facing with poison (the sting is used in an over-the-head-attack). It
may not be used to attack characters behind it.

\item[Weapons] In addition to its tail which may be used in Melee (Base
Chance of 45\%, quick-acting poison instead of Damage, no Rank)
the wyvern may bite in Melee and Close Combat (Base Chance of
40\% Damage of [D + 4], no Rank).  A wyvern may not sting and
bite in the same pulse.  A wyvern can attempt to knock down a
character in his rear hex using his tail.  This type of attack is
executed like a Shield Attack.

\item[Movement Rates] Running: 75; Flying: 150

\end{description}
\begin{mmstats}{}
\textbf{PS:}  20–30
& 
\textbf{MD:}  10–12
& 
\textbf{AG:}  12–16
& 
\textbf{MA:}  8–10
\\
\textbf{EN:}  25–35
& 
\textbf{FT:}  30–40
& 
\textbf{WP:}  10–16
& 
\textbf{PC:}  18–25
\\
\textbf{PB:}  3–5
& 
\textbf{TMR:}  1/3
& 
\textbf{NA:}  Hide absorbs 8 DP
\\
\end{mmstats}

\begin{mmcomment}
 Wyverns do not know magic, but crave magical items and will
often be found to be hoarding or wearing same.  Dragons despise
wyverns and wyverns fear dragons and the two will never be found in
each other's company.  Wyverns are, by nature, somewhat cowardly.
\end{mmcomment}
\subsection{Snakes}
All snakes included in this section are non-intelligent and extremely
hostile.  Most are poisonous.  They include: asps, king cobras,
mambas, pythons, and spitting najas. Snakes tend to lie in wait for
prey and will usually strike only from ambush or if startled.

\subsubsection{Asp}

\begin{description}
\item[Natural Habitat]  Rough, Plains Rare

\item[Number]  1–7 (1)

\item[Description] The asp measures up to 20 feet in length. It has a triangular
head, flattened towards the rear, and a short, thin tail.

\item[Talents, Skills and Magic] Asps have no talents, skills or magic.

\item[Weapons] The asp can only attack in Close Combat (Base Chance of
65\%, [D − 3] Damage). If they do any effective damage, the
damage is not scored against their victim but rather they suffer 2 DP
per pulse until they take an antidote to the venom.

\item[Movement Rates]  Crawling: 150

\end{description}
\begin{mmstats}{}
\textbf{PS:}  2–3
& 
\textbf{MD:}  None
& 
\textbf{AG:}  16–19 
& 
\textbf{MA:}  None
\\
\textbf{EN:}  1–2
& 
\textbf{FT:}  3–4
& 
\textbf{WP:}  14–18
& 
\textbf{PC:}  14–17
\\
\textbf{PB:}  8–10
& 
\textbf{TMR:}  3
& 
\textbf{NA:}  None
\\
\end{mmstats}

\begin{mmcomment}
 These snakes hibernate together during the winter, and thus
very large groups may be found during hibernation.
\end{mmcomment}

\subsubsection{King cobra}

\begin{description}
\item[Natural Habitat] Plains, Woods, Marsh, Rough Rare

\item[Number] 1–8 (1)

\item[Description] Growing to 20 feet, the king cobra is the largest of all
poisonous snakes. It is usually dark brown in colour, With a
collapsible hood behind its head with a sort of horseshoe marking on
its back. The king cobra is the mortal enemy of the mongoose.

\item[Talents, Skills and Magic] Cobras possess no talents, skills or magic.

\item[Weapons] Despite its size, the king cobra cannot attack unless it is
in Close Combat. In Close Combat it attacks via its bite (Base Chance
of 75 \%, + 4 Damage). Damage done does not count, but if any actual
damage would have been inflicted, the victim is poisoned, and suffers
2 DP per Pulse, as per nerve venom.

\item[Movement Rates]  Crawling: 200

\end{description}
\begin{mmstats}{}
\textbf{PS:}  20–25
& 
\textbf{MD:}  None
& 
\textbf{AG:}  15–18
& 
\textbf{MA:}  None
\\
\textbf{EN:}  12–14
& 
\textbf{FT:}  15–20
& 
\textbf{WP:}  14–18
& 
\textbf{PC:}  12–17
\\
\textbf{PB:}  7–9
& 
\textbf{TMR:}  4
& 
\textbf{NA:}  None
\\
\end{mmstats}

\subsubsection{Mamba}

\begin{description}
\item[Natural Habitat] Woods, Marsh Rare

\item[Number] 1–4 (1)

\item[Description]  These snakes are not very large (less than 3 feet),
but their poison fangs grow to great size. They come in either
green or black, with the former a forest species, and the latter a
marsh snake.

\item[Talents, Skills and Magic] Mambas possess no talents, skills or magic.

\item[Weapons] The mamba cannot attack in Melee Combat. In Close Combat it
can bite (Base Chance 50\%, - 2 Damage).  Damage is only used to
determine if the snake did in fact penetrate armour with its fangs for
the purpose of injecting its poison. Mamba poison is among the most
deadly found in nature: a victim takes 4 DP per Pulse until an
antidote is taken.

\item[Movement Rates]  Crawling: 100

\end{description}
\begin{mmstats}{}
\textbf{PS:}  2–3
& 
\textbf{MD:}  None
& 
\textbf{AG:}  12–15
& 
\textbf{MA:}  None
\\
\textbf{EN:}  4–5
& 
\textbf{FT:}  6–8
& 
\textbf{WP:}  14–18
& 
\textbf{PC:}  12–16
\\
\textbf{PB:}  8–11
& 
\textbf{TMR:}  2
& 
\textbf{NA:}     None
\\
\end{mmstats}

\subsubsection{Python}

\begin{description}
\item[Natural Habitat] Woods, Marsh Rare

\item[Number]  1–2 (1)

\item[Description] The python is green and black, and sometimes reaches a
length of 33 feet.

\item[Talents, Skills and Magic] The Python can climb trees (large ones) although slowly. It
has no magical abilities, skills or talents.

\item[Weapons] Pythons may only attack in Close Combat. Pythons attack by
biting (Base Chance of 65\%, + 6 Damage). If the bite penetrates
armour, it hangs on, and at the next opportunity wraps it self around
its adversary, crushing the life out of it.  Wrap: Base Chance of 80\%,
+ 8 Damage per Pulse the snake squeezes, no roll needed to hit once
initial squeeze has been made. Once the snake is squeezing, it can no
longer bite until it has squeezed its prey to death.

\item[Movement Rates]  Crawling: 150

\end{description}
\begin{mmstats}{}
\textbf{PS:}  45–50
& 
\textbf{MD:}  None
& 
\textbf{AG:}  8–12
& 
\textbf{MA:}  None
\\
\textbf{EN:}  25–30
& 
\textbf{FT:}  30–35
& 
\textbf{WP:}  12–16
& 
\textbf{PC:}  14–18
\\
\textbf{PB:}  6–9
& 
\textbf{TMR:}  3
& 
\textbf{NA:}  Scales absorb 3 DP
\\
\end{mmstats}

\subsubsection{Spitting naja}

\begin{description}
\item[Natural Habitat] Rough, Woods Rare

\item[Number] 1–2 (1)

\item[Description] The spitting naja is a form of Cobra, without the hood,
but with the ability-to spit their venom. Their scales are usually
dark brown in colour.

\item[Talents, Skills and Magic] The spitting naja possesses no talents, skills or magic.

\item[Weapons] In Melee Combat, spitting najas can only spit (Base Chance
of 40\%), If they hit, (aiming at the eye) the person hit is blinded
until the eye is thoroughly washed. Unless the eye is washed
promptly, the blindness becomes permanent. In Close Combat, the naja
gets a bite (Base Chance of 65\%, - 2 Damage for purposes of armour
penetration). The bite's damage is not actually sustained, but is
rather used to determine if the snake has penetrated armour so as to
allow its venom to work. The venom does 1 DP per Pulse (in addition to
blinding the victim) until an antidote is administered.

\item[Movement Rates]  Crawling: 150

\end{description}
\begin{mmstats}{}
\textbf{PS:}  10–12
& 
\textbf{MD:}  None
& 
\textbf{AG:}  16–18
& 
\textbf{MA:}  None
\\
\textbf{EN:}  8–10
& 
\textbf{FT:}  12–17
& 
\textbf{WP:}  12–16
& 
\textbf{PC:}  11–16
\\
\textbf{PB:}  8–11
& 
\textbf{TMR:}  3
& 
\textbf{NA:}  Scales absorb 1 DP
\\
\end{mmstats}
\subsection{Insects and Spiders}
The species included in this section tend to be non-lethal to
human-sized beings individually, but most will be found, if at all, in
large numbers.  They include the Black Widow Spider, the Fire Ant, the
Killer Bee, scorpions and tarantulas.

\subsubsection{Black widow spider}

\begin{description}
\item[Natural Habitat] Waste, Rough Very Rare

\item[Number] 1–4 (1)

\item[Description] Black widows are small, black spiders with thin hairless
legs and a red hourglass marking on their backs. They are 2–3 inches
long.

\item[Talents, Skills and Magic] Black widows have no special talents, skills or magical
abilities. They are not tool users, but they do spin webs.

\item[Weapons] A Black Widow spider can only attack in Close Combat, using
its bite with a Base Chance of 30\%. If a hit is indicated, do not
check for damage, but instead follow this procedure: Roll D10; if the
die roll is greater than the bitten creature's Armour Protection
Rating, then the creature has been bitten and suffers the effects of
the spider's poison: otherwise there is no effect. A black widow's
poison does 3 DP/Pulse for D10 Pulses until an antidote is applied.

\item[Movement Rates]  Running: 75

\end{description}
\begin{mmstats}{}
\textbf{PS:}  1
& 
\textbf{MD:}  None
& 
\textbf{AG:}  18–20
& 
\textbf{MA:}  None
\\
\textbf{EN:}  1
& 
\textbf{FT:}  None
& 
\textbf{WP:}  4–6
& 
\textbf{PC:}  10–12
\\
\textbf{PB:}  3–5
& 
\textbf{TMR:}  1
& 
\textbf{NA:}  None
\\
\end{mmstats}

\subsubsection{Fire ant}

\begin{description}
\item[Natural Habitat] Plains Uncommon

\item[Number]  500–5000 (500)

\item[Description] A Fire Ant is a bright red ant about 2 inches long.

\item[Talents, Skills and Magic] Fire Ants have no magic, skills, talents or other special
abilities. They are not tool users, but they will use twigs and leaves
to cross bodies of water.

\item[Weapons]A Fire Ant can only attack in Close Combat. It bites with a Base
Chance of 25\%. If the bite hits, roll D10. If the number rolled
is more than the bitten character's Armour Protection Rating, the
character takes 2 DP. Otherwise there is no effect.

\item[Movement Rates]  Running: 150

\end{description}
\begin{mmstats}{}
\textbf{PS:}  1  
& 
\textbf{MD:}  None
& 
\textbf{AG:}  11–13
& 
\textbf{MA:}  None
\\
\textbf{EN:}  1
& 
\textbf{FT:}  None 
& 
\textbf{WP:}  5–7
& 
\textbf{PC:}  10–12
\\
\textbf{PB:}  2–4
& 
\textbf{TMR:}  3
& 
\textbf{NA:}  None
\\
\end{mmstats}

\begin{mmcomment}
 Fire Ants tend to form into columns that eat through
anything in their way. These insects dislike the smell of oil, and if
it is put in the ants' path, they will go around it if possible.
\end{mmcomment}

\subsubsection{Killer bee}

\begin{description}
\item[Natural Habitat] Woods, Plains Uncommon

\item[Number] 1–300 (200)

\item[Description] A killer bee looks like a normal bee except that it is
about an inch and a half long.

\item[Talents, Skills and Magic] Killer bees have no magic abilities or special talents or
skills. They are not tool users, but do build hives.

\item[Weapons]Killer bees can only attack in Close Combat in which they can sting
with a Base Chance of 50\%. If a bee succeeds in stinging roll
D10. If the roll is more than the armour protection rating of the stung
character, the character takes D-6 Damage (not absorbed by armour). As
soon as a bee hits a character (not necessarily penetrating armour via
the die roll above) it dies.

\item[Movement Rates]  Flying: 500

\end{description}
\begin{mmstats}{}
\textbf{PS:}  1
& 
\textbf{MD:}  None
& 
\textbf{AG:}  20–22
& 
\textbf{MA:}  None
\\
\textbf{EN:}  1
& 
\textbf{FT:}  None
& 
\textbf{WP:}  7–9
& 
\textbf{PC:}  15–17
\\
\textbf{PB:}  6–8
& 
\textbf{TMR:}  10
& 
\textbf{NA:}   None
\\
\end{mmstats}

\subsubsection{Scorpion}

\begin{description}
\item[Natural Habitat] Waste, Rough Rare

\item[Number]  1–20 (1)

\item[Description] A scorpion is a black-coloured insect about 4 inches
long. The most prominent feature of a scorpion is its tail, which
stretches over its back.

\item[Talents, Skills and Magic] Scorpions have no special talents, skills or magic.

\item[Weapons]A Scorpion can only attack in Close Combat, in which it uses its tail
with a Base Chance of 65\%. If the tail hits, roll D10. If the
die roll is more than the Armour Protection Rating of the character
stung, the character takes 4 DP / Pulse for D5 Pulses, or until an
antidote is applied.

\item[Movement Rates]  Crawling: 150

\end{description}
\begin{mmstats}{}
\textbf{PS:}  1
& 
\textbf{MD:}  None
& 
\textbf{AG:}  18–20
& 
\textbf{MA:}  None
\\
\textbf{EN:}  1
& 
\textbf{FT:}  1
& 
\textbf{WP:}  8–10
& 
\textbf{PC:}  11–13
\\
\textbf{PB:}  4–5
& 
\textbf{TMR:}  3
& 
\textbf{NA:}  None
\\
\end{mmstats}

\begin{mmcomment}
 An alchemist can use a Scorpion's tail to distill poison,
and so a scorpion can be sold for about 50 Silver Pennies in a major
town.
\end{mmcomment}

\subsubsection{Tarantula}

\begin{description}
\item[Natural Habitat] Waste Rare

\item[Number]  1–6 (1)

\item[Description] Tarantulas are large, very hairy spiders about 4 inches
across.

\item[Talents, Skills and Magic] Tarantulas have no special talents, skills, or magic
abilities. They do not tool users and do not build webs.

\item[Weapons]Tarantulas only attack in Close Combat, biting with a Base Chance of
25\%. If a creature is bitten, roll D10, and if the roll is
greater than or equal to the bitten creature's Armour Protection
Rating, the creature suffers D-4 Damage due to the tarantula's poison.

\item[Movement Rates]  Running: 75

\end{description}
\begin{mmstats}{}
\textbf{PS:}  1
& 
\textbf{MD:}  None
& 
\textbf{AG:}  16–18
& 
\textbf{MA:}  None
\\
\textbf{EN:}  1 
& 
\textbf{FT:}  None
& 
\textbf{WP:}  4–6
& 
\textbf{PC:}  9–11
\\
\textbf{PB:}  2–4
& 
\textbf{TMR:}  1
& 
\textbf{NA:}  None
\\
\end{mmstats}
\end{mmgroup}

