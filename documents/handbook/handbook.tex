% ----------------------------------------------------------------------
%
% DQ Players Handbook
%
% 2024-04-09: Updated to latest latex packages
%
% ----------------------------------------------------------------------

\documentclass[twoside,a4paper]{article}

\usepackage{handbook}

% \usepackage{draftcopy}

\title{A players Guide to Dragon Quest\\Ver: 1.4.2}
\author{Jono Bean\\Ross Alexander\\Andrew Withy\\Terry Spencer}
\date{\Large\today}

\begin{document}

\thispagestyle{empty}
\begin{center}
  {
  \fontspec{Desdemona}[Color=Black]\fontsize{50pt}{60pt}\selectfont
  A GUIDE TO THE \\
  ADVENTURERS \\
  GUILD OF SEAGATE. \\
  }

  \vspace{30mm}
  
\includegraphics{../2020e/cover.pdf}
  
  \vspace{30mm}

{\Huge March 1997}
\end{center}

\clearemptydoublepage

\maketitle

\setcounter{tocdepth}{2}
\tableofcontents

\newpage

\begin{multicols}{2}

\section{A Player's Guild}

\subsection{History of the Campaign}

In August of 1981 a single game-master (GM) Robert Leyland started a
campaign based on the then new DragonQuest (DQ) game.  Since that time
and with the help of more than a few GMs the campaign has expanded to
now encompass roughly 50--60 players, over 100 active characters,
roughly 15 hard working GMs and approximately as many worlds as 15
GMs can dream up.

\subsection{Aim of this Guide}

With such a large group of active role-players in a multi-GM campaign,
a new player can be somewhat overwhelmed.  This guide is intended to
make the entry of a new player into the campaign both easier and more
enjoyable.  In addition it is hoped that this guide will assist the
reader in understanding the reasons behind the GM's decisions when
they may appear, on the surface anyway, to be a bit obscure.

\subsection{Aims of the Campaign}

\begin{enumerate}

\item
To have fun!

\item
To promote good role playing as opposed to simple gaming --- not
because we believe gaming isn't fun, but because we believe
role-playing is more fun.  We aim to do this not by penalising gaming
but by encouraging good role-playing.

\end{enumerate}

\subsection{The Game-masters}

The GM's job is to mediate and guide the character through
adventures.  It is not an easy job, nor is it always done well.  But
without the GMs it is impossible to have a campaign, so if you think
you can do better, try it --- we desperately need all the GMs we can
get to spread the load, and to provide variety for the Players.

With the difficulties of GMing it helps if players obey certain rules
of courtesy.

\begin{enumerate}
\item
Try and tolerate other people's faults.  If they really get on your
nerves try and talk to them, please don't shout. Many GMs have a
headache from staying up late writing adventures, and shouting only
makes it worse.

\item
Try and have consideration for the people you are with.  We know it's
hard to remember but everybody has faults, even you, so please try and
minimise them.  If you do screw up try apologising - it usually makes
everyone feel better (and it stops them shouting).

\item
If you have a disagreement with a GM over a rule, or interpretation,
say so.  GM's are not always right, but please say it only once (and
please don't shout).  If the GM stands by the ruling, stop arguing and
get on with the game.  If it really is important bring it up again at
the end of the session. If you still can't settle it, talk to some of
the other GM's and a solution will be found.

\item
If you can't make it to a game session let the GM know so he doesn't
waste time waiting for you.  It is rude and inconsiderate to keep
people waiting.  If you can't turn up to most of the sessions of a
game, or can't be bothered, don't go on the adventure.  It is
extremely frustrating for a GM to spend time creating an adventure
only to have it ruined by players who don't turn up.

\item
Don't intimidate other players or play their characters for them.  If
you can't find a way to give advice in character, then don't.  If
someone really needs help, offer it but don't force it on them.  It's
their right to make mistakes too.

\end{enumerate}

\subsection{The Adventurers Guild}

The Adventurers Guild of Seagate is primarily a game mechanism devised
with great cunning to allow characters to meet and organise adventures
with some vague kind of logical consistency.

As a result of players and characters requesting protection from
dishonest (gasp!)  adventurers, the Guild has developed a system of
screening new members and dispensing justice that, while at times may
infringe on `modern freedoms', does protect members, or at least
punish members who break the law.  From the player's point of view it
is worthwhile remembering that the Guild has been doing this for 25
years and while the player may occasionally fool the GM, the character
never fools the Guild.  While this sounds heavy-handed please
understand it is for the players' protection since players who find
their characters ripped off without any justice stop playing. And it
seems that invariably it is those players who whine loudest about
Guild security who are the ones who come to the GMs for protection
from dishonest characters.

Partly as a result of Guild security, the Guild can appear
authoritarian, but the council are ex-adventurers and do try to act in
the best interests of the members.

The Guild is a non-profit organisation in both intent and fact. This
is not true of many of the Guild members, both active and retired
adventurers, who provide skills and services to the members for a
profit although some are Guild subsidised.  Guild services are given
with restrictions, not to prevent adventurers using them, but because
the Guild just hasn't got the time, money, or manpower to provide
services freely.  Despite the above comments, new and old members are
encouraged to use Guild facilities. If they can't help they may
suggest outside people who can.

Guild representatives and council members in general only give part of
their time to the Guild and the remainder of their time is devoted to
obtaining a living usually by selling services to the Guild members,
so don't expect free service as a right of membership.  Remember,
Guild reps are Guild members too, and if you can ask them for help,
they can also ask you for help.

For the player, remember any dealings with the Guild must be done
through a GM and possibly many GMs, otherwise we can't keep track of
what's happening.  In particular some Guild reps are the
responsibility of specific GMs and if possible these GMs are the
people to talk to if you want something from that Guild rep.

Both the Guild and the GMs have a librarian, both roles being
fulfilled by the same person.  The librarian is trying to gather all
the information that the Guild that has collected over the past years
concerning players, characters, GMs, places, events, adventure scribe
notes, membership forms etc.  It is a big job, so make their life
easier and any time you fill out something for Guild or GM's records,
don't lie.  Not only is it not possible for your character to lie to
the Guild and get away with it (despite the player's ability to lie to
the GM) but if you do, then after the Balrog is finished with the last
offender, we'll send it to get you.

\subsection{The World}

So now you know all about the Guild, it's time to step out into the
world.  Which world you say, yup --- the rumours you heard were true,
most of the GM's have at least one world and/or plane that they
proudly call their own, so you can expect to play anywhere and any
when.  But amidst all this confusion lies one stable consistent place,
Seagate and the Guild.  Well alright, Seagate has undergone some
pretty drastic reality ripples but the Guild is constant, at least
until we have to move. Seagate (at present) is a sea port to the south
of the Western Kingdom.  Southeast lies the Brastor Holdings, to the
north lies civilised lands, and to the sound lies the Five Sisters and
across the sea to the open frontier.  The continent is called Alusia.

\subsection{Character Generation}

To generate a Character for use in the DQ campaign, you will have to
generate a new character with one of the DQ GMs. Generally it is not
allowed for players to ``move'' characters into DQ from other campaigns
or systems.

Generating a character with a GM can be fun and enjoyable. Before you
contact the GM you should think about what type of character you would
like to play. After you have the basic idea, approach one of the GMs
and gain their assistance to generate the character.

You can have more than one character in DQ, but we recommend that you
start a second character no sooner than six months after starting
play. In this time you may gain a better understanding for the system,
and what type of character you would like to play next.

In order to play, one must have a character.  The most important
aspect of the process of character generation is the development of a
defined persona for your character, as opposed to a set of numbers.
To try and help players do this, new characters are rolled up with a
GM and then interviewed ``in character'' by one of three GMs playing
members of the guild membership tribunal.  Players are encouraged to
develop a history and a set of basic motivations for their character.
Just in case you think this is another blurb from the front of the
rule book, to be ignored like so many before it, the GM helping you
roll up your character may insist that you explain the choices you
make with reference to your character's persona and history, as will
the guild membership tribunal --- so start thinking about it now.

The most fundamental choices are those of race and whether or not to
be a magic user.  You must roll dice to determine race but may choose
magic college freely.  Ask your friends or GM to precis the races and
magic colleges for you if you are unsure, but be aware that, no matter
how unbiased they try to be, any GM will have more experience of some
areas than others and hence cannot help but be a little biased.


\subsection{The Character Tribunal}

At Guild meetings there are a number of GMs with the particular task
of validating a newly generated character. This service is to help
make sure that nothing has gone wrong with the character generation.
It is also to get the character to join the Guild and to welcome new
players.

\subsection{Guild Meetings}

Guild meetings are held four times a year.  They are normally held in
the Lower Common Room at Auckland University. They start at 1pm and
finish at about 4pm on the second Sunday of March, June, September and
December.

At Guild meetings you will be able to get updates for your rulebook.
It is up to you to maintain your own rulebook.

Another service available at the Guild meetings is ``The Seagate Times''
Newsletter.

\subsection{Getting a Rulebook}

You can place an order for a rule book at a Guild meeting.  As the
rules are inclined to change from time to time, they are now being
released in a loose leaf format. The purpose of this is so that you
can update your copy of the rules when it becomes dated. With this in
mind you may have to provide your own bindings for the rulebook.

\subsection{Grievance Tribunal}

The grievance tribunal is to help maintain fun and enjoyment within
the game.  If you feel you have been unfairly treated by your GM and
feel you cannot resolve the problem directly with the GM, then you may
take your grievances to the tribunal.  The tribunal represents all GMs
and will adjudicate on the matter.  This should be seen as an option
of last resort.

\subsection{GM List}

This is an incomplete, unordered list of current GMs and offices as at \today.

\begin{dqtblr}{colspec={lll}}
\textbf{Name} & \textbf{Phone} & \textbf{email address} \\
Jono Bean	& 3020477	& jono@utf.gen.nz \\
Scott Whitaker	& 6231101	& rottgar@clear.net.nz \\
Struan Judd	& 5247594	& struanj@fpnet.co.nz \\
Anne Judd	& 5247594	& sheba@sfmc.sf.org.nz \\
Keith Smith	& 2765069	& phaeton@ihug.co.nz \\
Jacqui Smith	& 2765069	& flamis@ihug.co.nz \\
Ross Alexander	& 3768024	& r.alexander@auckland.ac.nz \\
Martin Dickson	& 8493642	& martin@crusader.co.nz \\
Ian Wood	& 8345149	& \\
William Dymock	& 5710749	& \\
Adam Tennant	&		& a.tennant@auckland.ac.nz \\
\end{dqtblr}

\bigskip

\begin{tabular}{ll}
\textbf{Grievance Tribunal} & Ian Wood \\
	& William Dymock \\
	& Andrew Withy \\
\\
\textbf{Membership Tribunal} & William Dymock \\
	& Jon McSpadden \\
\\
\textbf{Seagate Times} & Ross Alexander \\
	& Terry Spencer \\
\\
\textbf{Secretary} & Keith Smith \\
\\
\textbf{Editor} & Ross Alexander \\
\end{tabular}

\bigskip

\subsection{World Wide Web site}

There is a WWW site containing both player and character
information available from
\begin{flushleft}
\hspace{1cm}\texttt{http://stimpy.math.auckland.ac.nz/dq}
\end{flushleft}
This site should be up from the middle of March 1997.

\newpage

% ----------------------------------------------------------------------
%
% About the Guild
%
% ----------------------------------------------------------------------

\section{About the Guild}

\subsection{Purpose}

The Adventurers Guild of Seagate was set up sometime in November 1963,
in response to a need for a common meeting ground for Adventurers.
This enabled the transmission of knowledge and information in relative
safety, resulting in a close-knit and more effective band of
Adventurers.

Since then it has evolved into a fully integrated service
organisation, providing many important functions for the Adventurer.
These include a clearing house for treasure, healing and resurrection
facilities, a library, magic services, skills services, and many
others.

One of the more important functions of the Guild is to provide
arbitration in disputes and ensure justice on behalf of Guild
members. To assist in this regard, the Guild will draft, for a fee
depending on the complexity of the terms, any contract required by a
member, and guarantee arbitration and enforcement of such a
contract. Copies of the contract may be obtained from the Guild on
request.

\subsection{Structure}

The Guild is organised on a ``level'' system.  There are 5 levels in
all, comprising all forms of membership.

\begin{description}
\item[1. Guild Council] These members are the highest level, and form
the Guild Council. The Council comprises five guild members, these
being selected from Deans by the Council. Hence they will also be head
of their particular branch of magic or skill (see 2). The Council
elects the Council Chairman from amongst their number, this position
being the highest attainable in the Guild. The Council may have no
more than two members who have not been an Adventurer member. In this
way, the Guild remains ``by Adventurers, for Adventurers''.

\item[2. Deans] These members are the Heads of the various
departments.  The departments comprise the magic colleges, skills and
services.

\item[3. Department Staff] These members are the other officials of
the guild such as assistant department heads and others with varying
degrees of responsibility.

\item[4. Adventurers] These members form the core of the Guild. They
are the adventurers who are not involved in the running of the Guild,
but utilise the services of the Guild. The Guild exists primarily for
these members.

\item[5. Staff] These members form the working staff of the Guild such
as stable-hands, kitchen staff, and Department staff. They earn a wage
in return for their services, and are full members, hence they may
take advantage of any Guild service.

\end{description}

\subsection{Membership}

\subsubsection{Entry}

Entry to the Guild must be through Adventurer or Staff levels. Each
level recruits from the one below. The only exception is staff, whose
members may move directly to Department Staff. Adventurers and Staff
levels are considered equivalent.

\subsubsection{Membership Agreement}

On acceptance as a probationary member, the prospective member must
sign the standard declaration, the ``The Adventurers Guild: Membership
Agreement''.

\subsubsection{Probationary Membership}

All new Adventurer level members are probationary Guild members for
the first six months. Most probationary members are accepted as full
Guild members at the end of the six months unless they have been come
to the attention of the Guild Grievance Tribunal.  The six months will
start again from that Tribunal date.

\subsubsection{Membership Fees}

The minimum membership charge for a year is 500sp. If a member is
incapable of paying this minimum fee, then the member concerned may
work off this amount in employment at the Guild.

\subsubsection{Guild Tax}

Maintaining a complex such as members currently have at their disposal
inevitably costs money. Therefore, it is necessary to levy a tax on
members' incomes. This tax applies on all gross income generated on
adventure or dealing with the Guild or its members.  The current Guild
taxation rate is 10\%, but this may vary at any time.

\subsubsection{Guild Adventure}

If an Adventure involving Guild Members uses Guild services or
facilties then it is considered to be a Guild Organised Adventure.
The Guild will insist that a contract be signed by every party member
on adventure.  All adventures with a guild-backed contract are
considered to be Guild organised, and the members are obliged to
uphold the contract.

\subsubsection{Valuation and Divinations}

If, on returning with treasure, the party wishes a valuation on any
items, there will be a charge of 2\% of the valued amount to defray
the cost of obtaining divinations, paying the merchants, etc. This
charge is in addition to the tax.


\subsection{Location and Facilities}

The Adventurers Guild is located south of the city of Seagate and on
the other side of the Sweetwater river within the Duchy of Carzarla.

The Guild premises contain:

\begin{enumerate}
\item Residential area, including: sleeping
      quarters (varying quality and cost),
      dining hall, kitchen, bathhouse, and staff
      quarters.

\item Records section, comprising: Historical,
      Geographical, Legal and Social Library,
      Guild membership records, and
      Adventure scribe notes.

\item Stables and stable-hands quarters.
\item Offices \& meeting room for the Council.
\item General Meeting hall and common room.
\item Vaults.
\item Store rooms.
\item Pawn Shop.
\item Departmental rooms, comprising:
      Offices for Head and any other officials,
      Soundproof Ritual and Spell casting
      rooms (checked 3 hourly when
      occupied), Laboratories, an outside area
      for training and practise, and Common
      room.
\item Hospital.
\item Arena.
\end{enumerate}

\raggedcolumns

\pagebreak

% ----------------------------------------------------------------------
%
% The Guild Heirarchy
%
% ----------------------------------------------------------------------

\section{The Guild Hierarchy}

\subsection{Level 1 - The Council}

\begin{flushleft}
Herkam the Enchanter (Chairman), \\
Wegan the Inscrutable, \\
Kali the Nameless, \\
Maya the Illusionist \\
The Late Graf Grendal von Gracht \\
\end{flushleft}

\subsection{Level 2 - Departments}

\subsubsection{Colleges}

\begin{tabular}{ll}
\multicolumn{2}{l}{Ensorcelments and Enchantments} \\
Mind			& Binder \\
Namer			& Illusion \\
Air			& Fire \\
Earth Pacifistic	&  Earth Druidic \\
Celestial Dark		&  Celestial Shadow \\
Celestial Star		&  Celestial Solar \\
Water			&  Necromancy \\
Rune			& \\
\end{tabular}

\subsubsection{Skills}

\begin{tabular}{ll}
Alchemy		& Armourer \\
Astrology	& Beastmaster \\
Courtesan	& Cook \\
Healer		& Herbalist \\
Mechanician	& Merchant \\
Military Sci.	& Navigator \\
Philosopher	& Ranger \\
Spy		& Thief \\
Troubadour	& Warrior \\
Weaponsmith \\
\end{tabular}

\subsubsection{Other service departments}

\begin{tabular}{l}
Records and Library \\
Legal \\
General Services \\
\end{tabular}

\raggedcolumns
\columnbreak

% ----------------------------------------------------------------------
%
% Goods and Services
%
% ----------------------------------------------------------------------

\section{Goods and Services available}


The following services are available at the Guild, and the information
below is given to help GMs with the Guild NPCs.  Often the detailed
NPCs are not on duty.

\subsubsection{Guild Lodgings service}

\begin{flushleft}
\emph{Location}: East of the kitchens (off the map). \\
\emph{Number of Staff}: 13 \\
\emph{Cost}: special (detailed in The Cost Of Living). \\
\emph{Time}: N/A. \\
\end{flushleft}

Marc of Seagate, runs the Guild Lodgings as detailed in the cost of
living (\S\ref{guildlodgings}).

\subsubsection{Pawning and Banking service}

\begin{flushleft}
\emph{Location}: Between the Healers \& Arena. \\
\emph{Number of Staff}: 2. \\
\emph{Cost}: special (given below). \\
\emph{Time}: N/A. \\
\end{flushleft}

RockFist of the IronFist clan; RockFist is a female Dwarf, in her
second century. She has PB:13.  She is a capable Rank 7 Merchant.
Guild members can pawn items at 80\% of Guild valuation. They may buy
the items back within a year at the full value. Otherwise, they will
be sold on the open market.

The Guild will bank money and/or valuables at a fee of 2 Silver
Pennies per 500 ounces per month.  If money and/or valuables deposited
with the Guild has a weight not divisible by 500, round up to the
nearest five-hundred weight.  A guildmember pays one-half the price to
bank with the guild.

The Guild will place money and/or valuables in safekeeping until the
depositor redeems their property, or until the value of the deposit
covers the banking fee.  The Guild does not extend credit when it
comes to banking fees.

\subsubsection{Stable service}

\begin{flushleft}
\emph{Location}: South of the Gates. \\
\emph{Number of Staff}: 2. \\
\emph{Cost}: 50sp per week of hire. \\
\emph{Time}: 4am till 8pm. \\
\end{flushleft}

Carlos of Seagate is a human male, in his late twenties. He has
PB:13. He is a capable rider and ostler. Guild members can hire horses
for 50sp per week. This price includes riding gear, (saddle, saddle
blanket, saddle bags etc) and insurance for the animal.

\subsection{Skills}

\subsubsection{Copying and Translations service}

\begin{flushleft}
\emph{Location}: Second floor of the Guild Library. \\
\emph{Number of Staff}: 2. \\
\emph{Cost}: 10sp/rank/page. \\
\emph{Time}: 11 - (Rank of translator / 2) hours per page. \\
\end{flushleft}

Cathrin of Slippery Rock is a female human, in her late 40s. She is
5'8" tall and has PB:16. Speaking and writing 28 different languages
is Cathrin's strong point. Cathrin gave up adventuring (Mind Mage) 14
years ago, after being the only surviving member of a party.

\subsubsection{Cartography service}

\begin{flushleft}
\emph{Location}: Second floor of the Guild Library. \\
\emph{Number of Staff}: 2. \\
\emph{Cost}: N/A. \\
\emph{Time}: N/A. \\
\end{flushleft}

Jodl Al-Hywn of Anguise. Jodl is a male human in his late thirties. He
is 5'10" and has a PB of 14. Jodl arrived in the beginning of 1995
from Tac, as the head of the new Cartography department. He will pay
scribe fees (1,000 silver pennies) to any adventurer who brings him
new maps.

\subsubsection{Astrology Readings service}

\begin{flushleft}
\emph{Location}: NW corner of healers building. \\
     Incense can be smelt for 20 feet around. \\
\emph{Number of staff}: 2 \\
\emph{Cost}: (Fatigue \x rank \x 20) sp per attempt. \\
\emph{Time}: 18 or 6 hours. \\
\end{flushleft}

Frederick Toadswart is human in his early 50s. He is short and balding
with black hair.  He is rank 10, and occasionally receives visions in
his sleep. He is greatly respected by some of the older guild
members. He has been in poor health since sickness affected him in
late 1993. He has PB: 9.

\subsubsection{Beast Master training and service}

\begin{flushleft}
\emph{Location}: Ground floor next to the Guild stables. \\
\emph{Number of staff}: 2 \\
\emph{Cost}: 500sp per month of training. \\
\emph{Time}: Var. \\
\end{flushleft}
Master Bowgone Worm-Rider of the Flatlands. He is a 40 year old male
from the Sea of Grass. He stands 6'4" tall and weights in at about 320
lbs. He is said to be a Were- Bear, and has worked at the Guild for 8
years.  He speaks rank 7 common, and a few of the languages from the
Sea of Grass, but doesn't speak of his home lands. He is good with
normal animals, and has been known to help with more unusual types of
beasts. He is a good and helpful teacher of few words. He has PB: 12.
He is a Rank 7 Beast Master.

\subsubsection{Guild Healer service}

\begin{flushleft}
\emph{Location}: Hospital building in Courtyard. \\
\emph{Number of staff}: 7 \\
\emph{Costs}:\\
\begin{tabular}{ll}
Rank	& Cost (on success) \\
0 to 4	& Free to members \\
5 and 6	& 550 sp \\
7	& 1,100 sp \\
8	& 4,500 sp \\
9	& 4,700 sp \\
10	& not available \\
\end{tabular}
\end{flushleft}

Mistress Jessica Laude-Foot is one of the Guild healers. She is a Rank
9 healer, and is life aspected. She hasn't left the Guild complex in
many years. She has the responsibility of running the Guild Hospital.
She is human, and is 39 years old. She looks in very good health apart
from the stress of the job. She has PB: 21.

Resurrection base chance information: \\

Mistress Jessica's BC normally is (rank + greater enchantment +
life-aspected bonuses) =88\% + patient (full normal) EN.
\begin{flushleft}
   99\% maximum for a Rank 9 Healer. \\
    9\% minumum for a Rank 9 Healer. \\
 +5\% if patient is life-aspected. \\
 -5\% if patient is death-aspected. \\
 -1\% for each year of prolonged life. \\
 -1\% for each day of regeneration required. \\
-10\% damage equal or greater than 2 \x EN. \\
-10\% per resurrection attempt. \\
\end{flushleft}
This list is a approximation please see the Healer skill for more
details.

Opter of Superstition Mountains. He is a Dwarf, and looks about 45
years old. He is 4'4" tall, and is always very well dressed. He speaks
Common and Dwarven at rank 8. He is a Rank 5 healer. He has PB: 15.

\subsubsection{Guild Healing Potion service}

\emph{Location}: Basement of the Hospital building in the
Courtyard of the Guild. \\
\emph{Number of staff}: 2 \\
\emph{Open}: 24 hours \\
\begin{dqtblr}{colspec={Xr}}
\emph{Costs}					& \emph{Wgt} \\
   500 sp for a 10 pt healing potion		& 0.2 lbs \\
 1,000 sp for a 20 pt healing potion		& 0.3 lbs \\
 1,000 sp for Waters of Healing, Rk 8		& 0.5 lbs \\
 1,000 sp for Waters of Strength, Rk 6		& 0.5 lbs \\
\end{dqtblr}

\emph{Magic Items}: It is rumoured that the Guild has a large magic
cauldron that the Healing potions are extracted from. The cost pays
for the rare ingredients that are used in the cauldron to make the
healing potions. The ingredients and recipe are a closely guarded
secret known only to a few people.

Trudy Reddingbow is a human female who stands 5'11" tall. She is a
Rank 2 healer, and spends most of her time working in the Hospital
above this basement that is used for the sale of Guild Healing
Potions. Trudy speaks Orcish, Elven, Dwarven and Common at rank
8. She has PB: 12.

\subsubsection{Geas and Curse Removal service}

\begin{flushleft}
\emph{Location}: Basement with the casting chambers \\
\emph{Number of staff}: 3 \\
\emph{Costs}: 100sp / hour / attempt (no guarantees). \\
\emph{Time}: Minor, 6 hours. Greater, 18 hours. \\
\end{flushleft}

Claudia Singer is from Southern Brastor Holdings, and she is a human
female who stands 5'11" tall. She has rank 15 in Remove Curse (MA 24),
rank 28 in Geas, and rank 2 in Healer. Claudia speaks Orcish, Elven,
Dwarven and Common at rank 8. She has PB: 13.

\emph{Minor Curse:} ((15 - MA) + 75 + 12 + 15)\% (117 - MA of curse) \\
\emph{Major Curse:} ((27 - MA) + 30 + 12 + 15)\% (84 - MA of curse) \\

Carlos Fastwing is a human male in his late thirties. He has rank 12
in Remove Curse (MA 23), rank 13 in Geas, and is a Namer (DA: 115\%,
Divination rank 4) He speaks; Common, Elven, and Hill Giant at rank
8. He has PB: 10.

\emph{Minor Curse:} ((15 - MA) + 60 + 12 + 15)\% (102 - MA of curse) \\
\emph{Major Curse:} ((26 - MA) + 24 + 12 + 15)\% (77 - MA of curse) \\

These assume a Rank 11 Enchantment, purification and a true silver
triangle.  Note that major curses with MA greater than the MA of the
Adept cannot be removed.

\subsubsection{Potions: Other than healing potions}

\begin{flushleft}
\emph{Location}: Ground floor - casting chambers \\
\emph{Number of staff}: 2 \\
\emph{Costs}: EM of the spell or talent \x 20. \\
\emph{Time}: 3 days each (also may vary). \\
\end{flushleft}

Garth Winston, is a old human male in his late fifties and he would
stand 5'8" if he stood upright which he does not. He is almost totally
deaf, and if it were not for a large brass hearing horn, he would not
be able to hear at all. He walks slowly and dodders a lot.  He has PB:
10. He is an Alchemist of rank 7.

Specific purpose potions may be purchased if the suitable ingredients,
both physical and magical are available and time permits. Garth spends
a lot of his time making potions for the Guild so there is a 20\%
chance he will be unavailable at any time.

\subsubsection{Other Skills}

Other skills are available and should be negotiated with the
Department concerned.  (See section 2)

Members should be aware that not all departments are represented at
all times. In general however, most departments can be contacted by
leaving a message on the notice boards located in the common room.

\subsection{Magic}

\subsubsection{Greater Enchantments}

\begin{flushleft}
\emph{Location}: Basement of the casting chambers \\
\emph{Number of staff}: 1 \\
\emph{Costs}: 1,000sp per 1\% (maximum of 12\%) \\
\emph{Time}: 1 hour. BC: 107\% \\
\emph{Open}: 7am till 7pm \\
\end{flushleft}

Cathrin Thunderfoot is the Guild Greater Enchanter. Cathrin is a Hill
Giant, in her late seventies and stands about 8'10" tall. She speaks
Common, Dwarven and Hill Giant at rank 8, and Elven, Orcish at Rank
6. She is PB: 12.

\subsubsection{Lesser Enchantments}

\begin{flushleft}
\emph{Location}: Basement of the casting chambers \\
\emph{Number of staff}: 2 \\
\emph{Costs}: 600sp for a 3 month lesser. \\
\emph{Time}: 1 minute. BC: 105\% \\
\emph{Open}: 7am till 7pm \\
\end{flushleft}
Rob Darktunnel is the Guild Enchanter. He is a Dwarf and is in his
late thirties and he stands about 3'10" tall. He speaks Common,
Dwarven at rank 8, and Elven, Orcish at Rank 6. He is PB: 12.

\subsubsection{Shadow Wings Service}

\begin{flushleft}
\emph{Location}: Tower of Lord of the Bats \\
\emph{Number of staff}: 1 \\
\emph{Costs}: 500sp per attempt (no guarantees). \\
\emph{Time}: N/A \\
\emph{Open}: 6pm till 6am \\
\end{flushleft}
Vance ``Lord of the Bats'' is the Guild Celestial mage. He is human
and is in his late forties and he stands about 5'10" tall. He is only
open at night and can cast Rank 16 Shadow Wings (at night BC
112\%). He can only attempt 8 casts (Specials) before running out of
Fatigue. He has PB: 8.

\subsubsection{Location and Crystal of Vision}

\begin{flushleft}
\emph{Location}: Tower of Lord of the Bats \\
\emph{Number of staff}: 1 \\
\emph{Location Costs}: 1,000 sp per hour of locating
a person or object already studied plus 250sp
per 30 mins. to study a person or object for
location if not already done. \\
\emph{Crystal Costs}: 500sp to 2,000sp for the
crystal plus 3,500sp per rank in Crystal of Vision. \\
\emph{Time}: 1 hour. \\
\emph{Open}: 6am till 6pm \\
\end{flushleft}
Martin the Farseeing is a large male halfling who stands 3'4" tall and
is in his late forties. He is a happy-go-lucky halfling that enjoys
good company as much as good food. His offices open during the day and
he can make Rank 13 Crystals of Vision, and can cast Rank 12
Locate (BC 70\%). He has a FT:20 He has PB: 19.

\subsubsection{Prot. from Magical/Normal Fire}

\begin{flushleft}
\emph{Location}: Between Arena \& landing area  \\
\emph{Number of staff}: 1 \\
\emph{Costs}: 100sp per rank, maximum of Rank 11. \\
\emph{Time}: 1 minute.
\emph{Base Chance}: 101\% \\
\emph{Open}: 6am till 6pm \\
\end{flushleft}
Festa is the Guild Fire mage. He is Elf and is about 280 years old,
and he stands about 5'10" tall. He is open all day. He has Rank 11 in
Protection from Magical Fire, and Protection from Normal Fire. Both
spells have a BC 101\%. He has FT: 22 and PB: 8 (scarred).

\subsubsection{Strength of Stones}

\begin{flushleft}
\emph{Location}: Tower of Lord of the Bats \\
\emph{Number of staff}: 1 \\
\emph{Costs}: 1200sp per attempt. \\
\emph{Time}: 1 minute.
\emph{Base Chance}: 85\% \\
\emph{Open}: 6am till 6pm normally (Anytime one week before and after guild meetings). \\
\end{flushleft}
Nigel of Seagate is one of the Guild Earth mages. He is a human, in
his late thirties and stands about 5'9" tall. He dresses in simple
dark brown robes. He can cast up to Rank 10 Strength of Stones at a BC
85\%. He has a FT:22 and a PB:16.

\subsubsection{Other Magic Services}

Other magic services should be negotiated directly with the Department
concerned. (See section 2).

\subsubsection{Invested Items}

The Guild does not sell Invested Items.  Therefore, to assist members
in the purchase of such items elsewhere, we advise that the cost of
ingredients necessary to perform the investment ritual is as follows:
\begin{flushleft}
Experience Multiple  \x  Rank  \x  Charges / 2
\end{flushleft}
It is to be hoped that with this knowledge the likelihood of our
members being ``ripped off'' will be decreased. However, we remind
members that the charge of successfully performing the ritual is not
100\%, and prices on the open market will therefore reflect this.


\raggedcolumns

\pagebreak

% ----------------------------------------------------------------------
%
% Teaching
%
% ----------------------------------------------------------------------


\section{Teaching Facilities available}

\subsection{Skills}

Guild members can get training in all skills available at the Guild
when facilities and suitable teaching staff allow.

\subsubsection{Skills available at the Guild.}

\begin{tabular}{ll}
Alchemy		& Armourer \\
Astrology	& Beastmaster \\
Courtesan	& Cook \\
Healer		& Herbalist \\
Mechanician	& Merchant \\
Military Sci.	& Navigator \\
Philosopher	& Ranger \\
Spy		& Thief \\
Troubadour	& Warrior \\
Weaponsmith \\
\end{tabular}

We have no representative of the Assassin skill. If a member wishes to
learn or train in the Assassin skill, then they will have to make
their own arrangements outside of the Guild.  Because this is an
unlawful skill members are advised to take care when pursuing it.

Fighting skills are taught by the Warrior Department and lessons in
weapons and armour maintenance are provided on request.  Through this
department the member may purchase weapons, armour and general battle
equipment. If the required items are not available at the Guild, the
department can usually provide suitable contacts outside of the
Guild. In general, the member is encouraged to visit the department
concerned and make their own arrangements.

\subsection{Magic}
\label{spellprices}

The special knowledge spells available for each represented college
are listed below. We have no representative of the colleges of Greater
Summoning or Witchcraft.  All General Knowledge Spells and Rituals
cost 300sp each to re-learn from the guild.

\subsubsection{Non-college specific}

\begin{dqtblr}{colspec={Xr}}
Counterspells	&  2,000 \\
Curse Removal	& 10,000 \\
Geas		&  8,000 \\
\end{dqtblr}

Note: Major Curse is Not taught.

\subsubsection{Binding and Animating Magics}

\begin{dqtblr}{colspec={lXr}}
2	& Binding Golems		& Soon \\
3	& Bubble of Force		& Soon \\
4	& Deactivating Golems		& 3,000 \\
5	& Frictionless Floor		& 6,000 \\
6	& Instilling Flight		& 6,000 \\
7	& Making			& 3,000 \\
10	& Programming Animation		& 10,000 \\
11	& Unbinding			& 40,000 \\
R1	& Ancient Divination		& 2,000 \\
R5	& Investment			& Soon \\
R7	& Shaping Iron Golem		& Soon \\
R8	& Shaping Stone Golem		& 10,000 \\
\end{dqtblr}

\bigskip

As there is at present only one Tutor for this College available, all
training and other services (such as the casting of Spells) must be
booked. Other services offered by this department include:

Training in Artisan skills, Toymaking, Leatherworking, Weaving,
Cooping.

\begin{dqtblr}{colspec={Xr}}
Casting of; \\
General Knowledge Spell:		& 200 sp \\
Special Knowledge Spell:		& 600 sp \\
General Knowledge Rituals:		& 1,000 sp \\
Special Knowledge Rituals:		& Price on asking \\
\end{dqtblr}

\bigskip

All costs are per attempt and do not include material, if any.

Note for GMs: The Guild Binder, Hyram Tallfellow, is run by Bryan
Holden. Please direct all enquiries regarding Binders availability to
train or perform other services to Bryan.


\subsubsection{E \& E}

\begin{dqtblr}{colspec={lXr}}
1	& Ventriloquism	& 500 \\
2	& Bolt of Energy	& 2,500 \\
3	& Opening		& 1,000 \\
4	& Enchanted Weapon	& 1,000 \\
5	& Web of Entanglement	& 2,000 \\
6	& Mage Lock		& 500 \\
7	& Enhance Enchantment	& 4,000 \\
8	& Levitation		& 2,000 \\
9	& Enchant Armour	& 2,000 \\
10	& Wizard's Eye		& 5,000 \\
11	& Slowness		& 5,000 \\
12	& Quickness		& 6,000 \\
	& Ward			& 8,000 \\
	& Investment		& 100,000 \\
\end{dqtblr}

\bigskip

\subsubsection{Illusion}

\begin{dqtblr}{colspec={lXr}}
1	& Animal		& 1,500 \\
2	& Bolt			& 2,000 \\
3	& Deep Pockets		& 6,000 \\
4	& Disguise		& 12,000 \\
5	& Hallucination	& 7,000 \\
6	& Heroism		& 10,000 \\
7	& Innocence		& 4,000 \\
9	& Metamorphosis	& 12,000 \\
10	& Minor Illusion	& 6,000 \\
11	& Mist			& 1,000 \\
12	& Multiple Images	& 5,000 \\
14	& Orchestra		& 2,500 \\
R1	& Aura			& 10,000 \\
R2	& Terrain		& 5,000 \\
	& Ward			& 8,000 \\
\end{dqtblr}

\bigskip

\subsubsection{Mind}

\begin{dqtblr}{colspec={lXr}}
1	& Mental Attack	& 2,500 \\
2	& Telepathy		& 4,000 \\
3	& Phantasm		& 5,000 \\
4	& Molec. Disruption	& 8,000 \\
6	& Force Shield		& 1,500 \\
7	& Healing		& 2,000 \\
8	& Invisibility		& 3,000 \\
9	& Telekinesis		& 1,000 \\
11	& Mind Speech		& 15,000 \\
	& Ward			& 8,000 \\
	& Investment		& 100,000 \\
\end{dqtblr}

\bigskip

\subsubsection{Naming Incantations}

\begin{dqtblr}{colspec={lXr}}
1	& Charming		& 500 \\
2	& Compelling Obedience	& 13,000 \\
4	& Banishing		& 2,000 \\
R1	& Divination		& 1,000 \\
	& Ward			& 4,000 \\
	& Investment		& 75,000 \\
\end{dqtblr}

\bigskip

\subsubsection{Air}

\begin{dqtblr}{colspec={lXr}}
1	& Air Blast		& 4,000 \\
2	& Arrow Flight		& 1,500 \\
3	& Avian Control	& 2,000 \\
4	& Barrier of Wind	& 3,000 \\
5	& Conjuring Air	& 1,500 \\
6	& Flying		& 12,000 \\
8	& Gliding		& 4,000 \\
9	& Knockout Gas		& 15,000 \\
10	& Lightning Bolt	& 10,000 \\
11	& Lightning Strike	& 4,000 \\
12	& Resist Cold		& 2,000 \\
13	& Shaping Cloud	& 2,500 \\
15	& Whispering Wind	& 5,000 \\
16	& Windstorm		& 10,000 \\
17	& Wind Walking		& 12,000 \\
18	& Ball of Lightning	& 8,000 \\
19	& Thunderclap		& 9,000 \\
R1	& Air Spring		& 2,500 \\
R2	& Conjuring Air Elem.	& 60,000 \\
R3	& Control Weather	& 25,000 \\
R4	& Summon \& Bind Clouds & 8,000 \\
	& Ward			& 8,000 \\
\end{dqtblr}

\bigskip

\subsubsection{Celestial}

\begin{dqtblr}{colspec={lXrl}}
	&				&	& \texttt{sssd} \\
1	& Healing			& 2,000 & \texttt{-***} \\
2	& Create Light/Dark Sword	& 1,500 & \texttt{*---} \\
3	& Bolt of Starfire		& 2,500 & \texttt{****} \\
4	& Meteorite Shower		& 2,000 & \texttt{-***} \\
5	& Star/Shadow Wings		& 5,000 & \texttt{*-**} \\
6	& Web of Light/Dark		& 2,500 & \texttt{****} \\
7	& Fear				& 1,000 & \texttt{--*-} \\
8	& Increased Gravity		& 10,000 & \texttt{-***} \\
10	& Solar Flare			& 10,000 & \texttt{*---} \\
10	& Falling Star			& 10,000 & \texttt{-*--} \\
10	& Blackfire			& 10,000 & \texttt{---*} \\
10	& Shadow Walking		& 10,000 & \texttt{--*-} \\
	& Ward				& 10,000 & \texttt{****} \\
	& Investment			& 100,000 & \texttt{****} \\
\end{dqtblr}
\texttt{*} Yes available. \\
\texttt{-} No not available. \\      
Sol Star Shad Dark is the order for the availability. \\

\bigskip

\subsubsection{Earth-Druidic}

\begin{dqtblr}{colspec={lXr}}
1	& Earth Hammer		& 2,000 \\
2	& Hands of Earth	& 4,000 \\
3	& Strength of Stone	& 5,000 \\
4	& Armour of Earth	& 2,500 \\
5	& Diamond Weapon	& 1,000 \\
6	& Gem Creation		& 1,000 \\
7	& Animal Growth	& 1,500 \\
8	& Enchanting Plants	& 500 \\
9	& Bind Animals		& 4,000 \\
10	& Earth Elemental	& 50,000 \\
11	& Sinking Doom		& 50,000 \\
12	& Wall of Stone	& 4,000 \\
13	& Wall of Iron		& 4,000 \\
14	& Tunnelling		& 8,000 \\
15	& Trollskin		& 8,000 \\
	& Ward			& 10,000 \\
	& Investment		& 100,000 \\
\end{dqtblr}

\bigskip

\subsubsection{Earth-Pacifistic}

\begin{dqtblr}{colspec={lXr}}
\hspace{8mm}\= \kill
2	& Hands of Earth	& 2,500 \\
3	& Strength of Stone	& 4,000 \\
4	& Armour of Earth	& 2,500 \\
7	& Animal Growth	& 1,000 \\
8	& Plant Enchantment	& 1,000 \\
12	& Wall of Stone	& 2,000 \\
13	& Wall of Iron		& 2,000 \\
14	& Tunnelling		& 5,000 \\
15	& Trollskin		& 5,000 \\
18	& Earth Transformation	& 10,000 \\
	& Ward			& 5,000 \\
	& Investment		& 100,000 \\
\end{dqtblr}

\bigskip

\subsubsection{Fire}

\begin{dqtblr}{colspec={lXr}}
1	& Wall of Fire		& 2,500 \\
2	& Bolt of Fire		& 1,500 \\
3	& Ball of Fire		& 1,000 \\
4	& Web of Fire		& 2,500 \\
5	& Self-immolation	& 8,000 \\
6	& Weapon of Flame	& 2,000 \\
7	& Speak to Fire Creat.	& 3,000 \\
8	& Hellfire		& 30,000 \\
9	& Dragon Flames		& 50,000 \\
13	& Fire Flight		& 5,000 \\
R1	& Summon Fire Element.	& 10,000 \\
R2	& Flame Sight		& 10,000 \\
R3	& Summon Efreet		& 60,000 \\
R4	& Creating Drought	& 20,000 \\
	& Ward			& 10,000 \\
	& Investment		& 100,000 \\
\end{dqtblr}

\bigskip

\subsubsection{Water}

\begin{dqtblr}{colspec={lXr}}
1	& Control Aquatic Life		& 2,000 \\
5	& Liquid Purification		& 4,000 \\
6	& Liquid Transmutation		& 2,000 \\
7	& Maelstrom			& 20,000 \\
8	& Rainstorm			& 12,000 \\
10	& Walk on Water		& 3,500 \\
11	& Waters of Healing		& 5,000 \\
12	& Waters of Strength		& 10,000 \\
13	& Waters of Vision		& 5,000 \\
14	& Waterspout			& 40,000 \\
15	& Wave Riding			& 8,000 \\
R1	& Summon Water Elemental	& 60,000 \\
\end{dqtblr}

\bigskip


\subsubsection{Necromantic Conjurations}

\begin{dqtblr}{colspec={lXr}}
1	& Agony			& 20,000 \\
2	& Animation of the Dead	& 5,000 \\
3	& Binding Greater Undead	& 15,000 \\
4	& Bone Construction		& 5,000 \\
5	& Dark Vision			& 1,000 \\
6	& Hand of Death		& 10,000 \\
7	& Life Draining		& 15,000 \\
8	& Mass Fear			& 10,000 \\
9	& Necrosis			& 30,000 \\
10	& Petit Mort			& 2,000 \\
12	& Spectral Weapon		& 4,000 \\
15	& Wall of Bones		& 3,000 \\
16	& Wraithcloak			& 2,000 \\
R2	& Life Prolonging		& 5,000 \\
R4	& Shaping Flesh Golem		& 50,000 \\
\end{dqtblr}

\bigskip


\subsubsection{Rune}

\begin{dqtblr}{colspec={lXr}}
1	& Runewall		& 2,000 \\
3	& Rune Shield		& 2,000 \\
4	& Visitation		& 4,000 \\
5	& Truth		& 3,000 \\
6	& Banishment		& 10,000 \\
7	& Smite		& 2,000 \\
8	& Runeweapon		& 2,000 \\
R1	& Casting Runes	& 25,000 \\
R3	& Create Rune Portal	& 30,000 \\
R4	& Binding		& 60,000 \\
\end{dqtblr}

\subsubsection{Witchcraft/Wicca}

\begin{dqtblr}{colspec={lXr}}
1	& Blessing Crops		& 1,000 \\
2	& Blessing/Curse Unborn Child	& 2,000 \\
3	& Blessing Livestock		& 1,000 \\
4	& Blighting Crops		& 1,500 \\
5	& Cat Vision			& 2,500 \\
6	& Controlling Animals		& 6,000 \\
7	& Converse with Animals	& 2,500 \\
9	& Creating Restorative		& 10,000 \\
10	& Earth Tremor			& 20,000 \\
12	& Evil Eye			& 2,500 \\
13	& Hellfire			& 40,000 \\
14	& Installing Flight		& 6,000 \\
15	& Mass Fear			& 15,000 \\
17	& Skin Change			& 12,500 \\
18	& Virility			& 1,500 \\
19	& Wall of Thorns		& 3,000 \\
R1	& Controlling Weather		& 10,000 \\
R5	& Summoning Animals		& 5,000 \\
	& Ward				& 10,000 \\
	& Investment			& 100,000 \\
\end{dqtblr}

\bigskip


The Guild has made arrangements with a local coven for the training of
Guild Wicca. As the coven is located near the town of Slippery Rock,
guild members must make their own arrangements for the days ride there
and back.

\end{multicols}
\newpage


% ----------------------------------------------------------------------
%
% Cost of Living
%
% ----------------------------------------------------------------------


\begin{multicols}{2}
\section{The Cost of Living}
\label{costliving}

All characters have to expend some money to keep themselves when not
on an adventure.

Under this system there are four options available to a player. The
player just choices one of the options for each or all of their
characters.  The options are: Living in a Household, Living in
Lodgings, Guild Employment, Guild Lodgings.

\subsection{Living in a Household}

The necessities, comforts, and luxuries enjoyed by a person represent
their standard of living.

If players would like their characters to live in a Household other
than at the Guild or at an Inn then the following is provided as a
guide. All of the Households below offer characters different
services.  At subsistence, life is poor, but as you spend more money
you end up paying for grounds to be kept, servants to clean, stables
to be maintained, you're yearly clothes are paid for.

This document assumes that a Household is an extended family unit of 4
to 8 people i.e.\ two adults, a grandparent or two children or so.

This is the simplest option for the player. All you need do is choose
a level of life style for the character and pay for it. You can use
the free time as you see fit. This is normally used for training.


\subsubsection{Subsistence}

\begin{idesc}
\item[Income] 1095 sp per annum (3 per day, 21 per week, 90 per month)

\item[Status] menials, prisoners, slaves, unskilled labourers,
beggars, recluses, etc.

\item[Food] turnips \& sauerkraut, or beet soup, or potatoes \&
onions, or carrots \& acorns, cereal, black bread. Little or no meat
in a non-rural area.  Drink is buttermilk, water (if drinkable),
watered wine, young beer.

\item[Housing] perhaps none; maybe a tiny room or hut for a family;
perhaps a blanket; no furniture.

\item[Clothing] sackcloth, homespun, rude leathers, nudity.

\end{idesc}

\subsubsection{Poor}

\begin{idesc}

\item[Income] 3650 sp per annum (10 per day, 70 per week, 300 month)

\item[Status] most landed peasants, minor craftsmen, civilized
soldiers of rank correspondent to sergeant or even lieutenant,
servants of those of wealth, peddlers, captains of boats and very
small ships, assistant sorcerers, etc.

\item[Food] as per the previous income tier, plus gruel, barley bread,
cheese and eggs. Meat or fish at least 1--2 times weekly.  Beer, ale,
cheap wine and porter are available.

\item[Housing] single dwelling, protects animals too.  Furniture
includes a chest or two, a table and benches, rushlights, straw
mattresses and blankets.

\item[Clothing] linens, wool and leather, shoes, one new suit of
clothes a year; winter clothes will be warm and protective.

\end{idesc}


\subsubsection{Middle Class}

\begin{idesc}

\item[Income] 8760 sp per annum (24 per day, 168 per week, 720 per month)

\item[Status] master craftsmen, servants who command other servants or
who have other independent responsibilities, professionals, minor
merchants, traders, middle rank mechanicians, healers, alchemists,
astrologers or mages, poor adventurers.

\item[Food] as per previous tier, but of better quality, plus white
bread and meat or fish with most meals. Table wines.

\item[Housing] One or two storey house. Furniture includes a chest or
two, a table and benches, rushlights, straw mattresses on good beds
and blankets, several tables and benches, 2 chairs, 2 wall
decorations, or plastered walls.

\item[Clothing] good quality linens and cottons, some use of expensive
dyes, a little fur trim, some jewelry.  This is the first tier for
which style is important.

\item[Other] 1--2 servants and 1--2 mounts.

\end{idesc}


\subsubsection{Wealthy}

\begin{idesc}

\item[Income] 20,000 sp per annum (55 per day, 385 per week, 1650 per month)

\item[Status] senior craftsmen, moderately successful merchants,
knights, thanes, other minor nobility, outstanding healers,
alchemists, astrologers and mages, average adventurers.

\item[Food] as per previous tier, but of much better quality,
plentiful meat and fish with all meals. Good wines.

\item[Housing] Greathouse, hall, manorhouse, fortified manorhouse, or
large 2--3 storey townhouse. Fine furniture.

\item[Clothing] high quality linens and cotton, some silks and satins,
use of expensive or imported dyes, fur trim, good jewellery.

\item[Other] 3--7 servants, 2--5 horses and/or a carriage, perhaps
hunting animals.

\end{idesc}

\subsubsection{Rich}
\begin{idesc}
\item[Income] 60,000 sp per annum (165 per day, 1155 per week, 4950
per month)

\item[Status] barons, less wealthy counts or earls, sheiks,
secretaries and factotums to royalty, successful merchants, veteran
adventurers.

\item[Food] as per previous tier, plus plentiful beef or other
herbivores, fowl, shellfish, etc. Home made pastries and breads. Fine
wines. Imported fruits and delicacies.

\item[Housing] large multi-roomed dwelling or castle housing the
family and servants, guards, etc.  Furnishings now include thrones,
chairs, valuable artworks, tapestries, panelling, and fine rugs.

\item[Clothing] imported silks and satins, and decorative trims
including silver and gold thread. Such folk frequently set (or always
quickly follow) the latest fashion. Gold jewelry with gemstones.

\item[Other] Includes several to many servants and mounts, other
staff, wagons, guards, ships, animals, etc.

\end{idesc}

There are higher standards of living but most of the time they are
unimportant to the average adventurer. If you desire better living
standards than those given above please just talk to a GM and they
will be able help you.

\subsection{Living in Lodgings}

If players would like their characters to live in a Inn, Tavern or
Boarding House and not in their own Household then the following is
provided as a guide.

All you need do is choose a level of life style for the character and
pay for it.


\subsubsection{Lower Class Lodgings}

\begin{idesc}

\item[Cost] 1,056 sp per annum (3 per day, 21 per week, 88 month)

\item[Status] most landed peasants, minor craftsmen, servants of those
of wealth, peddlers, captains of large boats and very small ships,
assistant sorcerers, etc.

\item[Food] turnips \& sauerkraut, or beet soup, or potatoes \&
onions, or carrots \& acorns, or a cereal, black bread, barley bread,
cheese and eggs. Meat or fish at least 1-2 times weekly. Drink is
buttermilk, water, cheap wine, young beer, ale.

\item[Room] Single room, furniture includes a small chest, a table and
bench, rushlights, straw mattresses and blankets.

Clothing: your own responsibility.

\end{idesc}

\subsubsection{Middle Class Lodgings}


\begin{idesc}

\item[Cost] 2,160 sp per annum (6 per day, 42 per week, 180 month)

\item[Status] master craftsmen, servants who command other servants
or who have other independent responsibilities, professionals, minor
merchants, traders, middle rank mechanicians, healers, alchemists,
astrologers or mages, poor adventurers.

\item[Food] as per previous tier, but of better quality, plus white
bread and meat or fish with most meals. Table wines.

\item[Lodgings] Generally a two storey Inn, Tavern or Boarding house
with larger private rooms. Furniture: as previous tier plus, a good
bed, several tables and benches, 2 chairs, 2 wall decorations, and
plastered walls.

\item[Clothing] you are expected to have: good quality linens and
cottons, some use of expensive dyes, a little fur trim, some
jewelry. This is the first tier for which style is important.

\end{idesc}


\subsubsection{Wealthy Class Lodgings}

\begin{idesc}

\item[Cost] 3,180 sp per annum (9 per day, 63 per week, 265 month)

\item[Status] senior craftsmen, moderately successful merchants,
knights, thanes, other minor nobility, outstanding healers,
alchemists, astrologers and mages, average adventurers.

\item[Food] as per previous tier, but of much better quality,
plentiful meat and fish with all meals. Good wines.

\item[Housing] a fortified 2-3 storey large Inn, Tavern or Boarding
house, with a courtyard. Private rooms furniture is as previous tier
plus, a better bed, several tables and benches, 4 chairs, 6 wall
decorations, plastered walls. Clothing: you are expected to have, high
quality linens and cotton, some silks and satins, use of expensive
and/or imported dyes, fur trim, good jewelry.

\item[Other] your room has a servant on call, and there
may be a carriage available.

\end{idesc}


\subsection{Guild Lodgings}
\label{guildlodgings}

Some Guild members desire to live at the Guild, paying for good room
and board.

A Guild member can live at the Guild. There is only one standard, and
that is 'good'. The cost is 5 sp per day. The Guild will provide
good, room, and as detailed below. This is intended as a helpful
service to members.

\begin{idesc}

\item[Food] turnips \& sauerkraut, or beet soup, or potatoes \&
onions, or carrots \& acorns, or a cereal, black bread, cheese and
eggs. Meat or fish at least daily. Drink is buttermilk, water and
young beer.

\item[Room] A single room; furniture includes a small chest, a lamp, a
good straw mattress and blankets.  Clothing: your own responsibility.

\end{idesc}

\subsubsection{Guild Employment}

All Guild members desire to live in the compassionate and cordial arms
of the Guild, working for only their up-keep.

A Guild member (but not probationary member) can be employed by the
Guild if they don't have a job.  The member dosn't have to expend any
money for living expenses. The Guild will provide part time employment
(eight hours a day, seven days a week, and only half the normal time
available for training), food, room, and clothing as detailed
below. This is to help guild members that have had a lot of time, and
have done as much training as they can afford considering that they
have had to pay for living expenses in the past. Guild staff do not
have to pay the membership fee.

\begin{idesc}

\item[Food] turnips \& sauerkraut, or beet soup, or potatoes \&
onions, or carrots \& acorns, or a cereal, black bread, cheese and
eggs. Meat or fish at least 1--2 times weekly. Drink is buttermilk,
water and young beer.

\item[Room] Small single room; furniture includes a small chest, a
lamp, straw mattress and blankets.

\item[Clothing] Guild worker's uniform (Black).

\end{idesc}

\raggedcolumns

\newpage

% ----------------------------------------------------------------------
%
% rankings
%
% ----------------------------------------------------------------------

\section{A Guide to Ranking}

At times, ranking can appear difficult and unclear.  However, it is
generally worthwhile.

All characters must have a ranking book, so that you have a record of
your character's progression and training. Ranking (training) should
be done before the first night of play in each session. This book
needs to be available for your GM to check, at which point they will
assist you in confirming that your ranking is correct, and then sign
it off.

All abilities cost Experience Points (EP) and/or Time to rank. The
total training time available is that between the conclusion of your
last adventure and the start of your current adventure, minus any time
spent engaged in other activities.

\subsection{Experience and Ranking}

The ultimate reward for adventuring is getting better at adventuring.
Don't expect treasure --- if you want to earn money become an
accountant.  The mechanisms used for monitoring how much better your
character gets are experience points (EP) and ranks (RK).

Experience is awarded on the basis of how well the character is
role-played and how much the character contributes in terms of ideas,
skills and effort.  An averagely role-played character with average
skills, who contributed effort and ideas averagely, should receive
roughly 1450 EP per session.  The exact EP awarded is at the GM's
discretion and will be decreased for uncooperative play or reducing
the enjoyment of the game for the other players.  It will
correspondingly be increased for excellent role-playing and enhancing
the gaming pleasure of all concerned, including the GM.  No experience
is given solely for killing monsters (so if you can find a way around
a fight it will save time and lives) or by gaining treasure.

Experience points are required to advance in anything.  Time spent
training is required to increase proficiency in spells, skills and
weapons.  Adventuring time is required to advance in characteristics
and talents.

EP is spent as per the rules but note the following :-
\begin{itemize}

\item
Spells must be cast at least five times before the next rank can be
achieved.  Casting chambers are available in the Guild, and monitored
on request.

\item
Talents may be ranked only once per game adventuring week.

\item
Weapon skills take 1 week of training to reach Rank 0, and 2 weeks \x
(rank to be achieved) to improve.

\item
Characteristics may only be increased a maximum of 5 above the
original value (when you rolled your character up) and never above
racial maximum.  Perception may be taken to racial maximum
irrespective of starting value.

\item
A characteristic may be increased only once per adventure. This does
not include occasions when a character has had a characteristic
reduced by more than one, for example dying twice in one session.

\item
Skill ranks 8, 9, and 10 must be ratified by a GM and in general
required significant use of the skill.

\item
You may rank any combination of two things at the same time, providing
you don't rank magic (i.e.\  spells or rituals) at the same time as
non-magic (i.e.\ weapons or skills).

\item
Namers may rank 1 name in addition to other forms of ranking.  They
may also substitute ranking names for any other form of ranking.
Hence it is possible for a Namer to be ranking up to three names at
any one time.

\item
All expenditure of EP and time should be logged for your own records
and should be approved by a GM prior to play.
\end{itemize}


\subsection{Campaign Time}

Time in the campaign happens at the same rate as that in the ``real''
world i.e.\ one real year equals one game year. However, the calendar in
DQ is set 9 months behind the real world, so that you can make use of
last year's diaries. One week of training will deduct one week from
your training total. If your time is devoted solely to training, you
may train in two related abilities per day; one in the morning, one in
the afternoon. All Weapons, Skills, Languages, and Adventuring Skills
are deemed to be related, as are all Spells, Rituals, and Names.

\subsection{Experience Points}

The only way to gain EP after character generation is by role-playing
with a recognised GM. You will gain EP on adventure for good
role-playing, for using your skills sensibly, and for entertainment.
Divide this EP by your Racial EM modifier. Add this to your unspent EP
total. This is your available EP for Ranking.

\subsection{Ill-gotten Gains}

It was fun, but what do you get out of it? Firstly, the GM will give
you a slip of paper with the character's EP total gained on the
adventure, or write this directly into your ranking book. They should
also specify the start and end dates (game time \& real time) of the
adventure. All monetary or magical gains should be written up and
signed by your GM.


\subsection{Ranking Book}

Clarity is the essence of all good ranking books.  Numerical accuracy
also assists. We suggest that you use a column format, with column
headings of Ability Name; Rank already achieved; Rank(s) being
trained; EP spent; start and end dates; and silver's spent.


\subsubsection{Example Ranking}

{
  \newfontfamily{\rankingtt}[Scale=0.7,Color=blue]{Courier Prime}
  \rankingtt
\begin{tblr}{colspec={Xlllllll},rowsep=0.1mm,hline{3}={blue}}
Name        & Old   & New  & EP    & Start & End   & SP   & EM \\
            & Rk    & Rk   & cost  & Date  & Date  &      &  \\
Healer      & 1     & 2    & 1600  & 1/1   & 14/1  & 300  & N/A \\
PC.         & 10    & 11   & 750   & N/A   & N/A   & N/A  & N/A \\
E.S.P       & 5     & 6    & 600   & 15/1  & 20/1  & N/A  & 100 \\
Total       &       &      & 2950  & 1/1   & 20/1  & 300 \\
\end{tblr}}

The reason why we stress the importance of this is that on the first
night of the session, everyone will want to start playing as soon as
possible, and anything that helps this will be appreciated by all.

\subsection{How to Rank}

All adventurers can learn any skill or weapon, and any magic within
their college. Learning costs EP, time and money.  The following can
be ranked: Skills, Weapons, Spells, and Languages, Names, Rituals,
Talents, Adventurering Skills and Characteristics.

\subsection{Skills}

All skills are assumed to be unranked (i.e.\ unknown) initially. The
first level of competence is Rank 0, and will take eight weeks to
learn. Each subsequent rank will take that number of weeks to reach
(eg; to get to Rank 8 from Rank 7 will take 8 weeks). The EP cost for
ranking Skills is listed in the rulebook.  Some skills require minimum
Characteristic requirements to Rank, or impose EP penalties (or
discounts) for exceptional Characteristics.  You may get an EP
discount of 10\% if you are trained properly. If this is done at the
Guild, it costs 150sp \x (Rank to be achieved min. 1). If no training
is available at all, a 25\% EP penalty is applied.  Achieving Ranks
8,9 and 10 is difficult.  You must find and complete a special task
relating to your skill, with the assistance of a GM, for each of these
Ranks. Rank 10 is the maximum achievable Rank in all Skills.

\subsection{Weapons}

All weapons are assumed to be unranked initially.  Rank 0 in a weapon
takes 1 week. All higher ranks take 2 \x (Rank to be achieved)
weeks. Weapons have individual maximum Ranks between 2 and 10.  EP
costs are detailed in the rulebook.  All Weapons require minimum PS
and MD Characteristics. If you do not fulfil both requirements, you
may not rank a weapon.  You may not get an EP discount for training,
but if no trainer is available, you may not increase in Rank.  The
cost of a trainer is 10 \x (Rank to be achieved squared) silver
pennies.

\subsection{Spells}

If you are an Adept (i.e.\ cast magic), you can rank your spells.
Each spell has an EM, or Experience Multiplier. This is multiplied by
the Rank that you wish to achieve, to give a total EP cost. If the
Adept has MA $>$ 15, (MA - 15) \x 5\% of the EP cost of General Spells
may be discounted. Training time for spells is (Rank to be achieved)
days.  All spells must be cast five times before each new rank is
achieved. As these spells may backfire, these rolls will need to be
made at some stage.  Learning a new spell to Rank 0 takes (EM / 100
rounded up + 1) weeks, but no experience points. If the spell is
taught by the Guild, they will charge you as laid out in
\S\ref{spellprices} of this Handbook.  You can not have more spells
and rituals below rank 6 than your MA characteristic.  Rank 20 is the
Maximum Rank achievable with any Spell (except Geas).

\subsection{Languages}

Spoken Languages have the same time requirement as weapons. They have
EP costs as set out in Table. EP discounts are as for Skills. Note
that knowing the Philosopher Skill may grant an EP discount.

Learning Written Languages is a one-off cost of 1000 EP and 4
weeks. Various discounts may apply, as specified in the rulebook under
languages and philosopher.

\subsection{Names}

Anyone can rank Names, but only Namers and E\&Es gain any normal
advantage from this.  Learning a Generic Name takes one day; learning
an Individual Name takes one week. Ranking a Generic Name takes (Rank
to be achieved) weeks; Ranking an Individual Name takes the same
number of months. No EP is required. Namers get special advantages in
Ranking - refer to the College of Naming Incantations.

\subsection{Rituals}

Rituals are learnt and ranked just like spells, except that Ranking
time is (Rank to be achieved) weeks, rather than days.  MA discount
applies to General Knowledge Rituals.

\subsection{Talents}

For each week of actual, out in the field, adventuring, you can rank
each of your talents once.  No training time is required to rank
Talents. Like Spells and Rituals, each Talent has an EM. No MA
discounts apply to any Talents.

\subsection{Adventuring Skills}

Adventuring Skills are skills used every day by adventurers to
survive, and thus are continually honed. These skills include
Orienteering, Riding, Swimming, Stealth and Climbing.

If you have extensively used an adventuring skill while on adventure,
you may rank this skill without any time requirements. Otherwise,
Ranking time is as per normal skills.

Adventurers are assumed to start off with Rank 0 or more in all these
skills, unless specifically told otherwise. Experience Point costs are
set out in the rulebook.  The Maximum Rank in all these Skills is 10.

\subsection{Characteristics}

\begin{dqtblr}{colspec={Xcc}}
\textbf{Stat} & \textbf{First Point} & \textbf{Extra Points} \\
Fatigue		& 2500 & 2500 \\
Endurance	& 5000 & 2500 \\
Perception	& 1000 & 750 \\
All others	& 5000 & 5000 \\
\end{dqtblr}

\bigskip

Characteristics, or Stats, are generated initially. PS, MD, AG, MA,
WP, EN and FT may not be increased more than five points.  No Stat may
go over Racial Maximum.  Perception may be increased more than the
limit of five times, to a limit of Racial Maximum. You can only
increase a Stat once after each adventure (not including recovering
lost Stats). Recovering lost EN costs 2,500 EP.

\end{multicols}

%% \subsection{Quick Sheet for Ranking}

\begin{table}
{\small
\begin{dqtblr}{colspec={llllll}}
Type         		& To Learn to Rk 0	& Time to Rank   		& With trainer    	& No trainer 		& Notes: \\
Weapons     		& 1 week 		& 2 X rank weeks going to	& needed       		& may not increase	& Trainer needed \\
Skills      		& 8 weeks		& 1 X rank weeks going to  	& 10\% EP discount  	& +25\% EP cost         & Training from a book is noraml +0\% EP \\
Languages   	 	& 8 weeks		& 2 X rank weeks going to  	& needed 		& may not increase     	\\
Lang.(written)   	& 4 weeks  		& N/A $^a$          									\\
Adventure Skills      	& (already learnt)  	& 1 \x rank weeks going to.$^b$	& needed      		& may not increase      & Trainer needed \\
Spells       		& EM/100 +1 wks.$^c$    & rank days going to     	& N/A    		& N/A         		& Must be cast 5 times to be able to rank \\
Rituals      		& EM/100 +1 wks.$^c$    & rank weeks going to    	& N/A    		& N/A      		& No need to be cast 5 times to be ranked \\
Names        		& Generic 1 day     	& rank weeks going to    	& N/A    		& N/A          		& Names cost no EP \\
Names        		& Individual 1 week 	& rank months going to   	& N/A    		& N/A          		& Names cost no EP \\
Talents      		& (already learnt)  	& N/A (see earlier)       	& N/A    		& N/A          		\\
Characteristics.      	& N/A      		& no time needed    	   	& N/A         		& N/A          		& Max of +1 per stat per session (3 months)$^e$ \\
\end{dqtblr}}


\begin{tabular}{ll}
$^a$       & Learning written languages is a one-off cost and is not ranked. \\
$^b$       & If used extensively on adventure no time is required (to rank it 1 rank).  \\
$^c$      & EM of the spell or ritual divided by 100 round up +1 weeks to learn. \\
$^d$      & For each point of MA over 15, General Knowledge Spells \& Rituals costs 5\% less EP. \\
$^e$      & A stat can only be increased to a maximum of 5 points above the starting point of stat. \\
\end{tabular}
\caption{Quick Sheet for Ranking}
\end{table}

\begin{table}
%%\subsection{Spell and Ritual Time Table}

This Table is used to work out the time that is needed to rank a Spell
(in days) or Ritual (in weeks) from the rank you already have (Current
Rank) to the new rank you wish to achieve.

\smallskip

{\scriptsize
  \begin{dqtblr}{colspec={X|llllllllllllllllllll},
      cell{1}{2}={c=20}{c,font=\bfseries},
    hline{1,2,3,6,9,12,15,18,21}={0.2mm,green4}}
\textbf{Current}	& New Rank in Spell or Ritual \\
\textbf{Rank}& 1& 2	& 3	& 4	& 5	& 6	& 7	& 8	& 9	& 10	& 11	& 12	& 13	& 14	& 15	& 16	& 17	& 18	& 19	& 20 \\
0	& 1	& 3	& 6	& 10	& 15	& 21	& 28	& 36	& 45	& 55	& 66	& 78	& 91	& 105	& 120	& 136	& 153	& 171	& 190	& 210 \\
1	&	& 2	& 5	& 9	& 14	& 20	& 27	& 35	& 44	& 54	& 65	& 77	& 90	& 104	& 119	& 135	& 152	& 170	& 189	& 209 \\
2	&	&	& 3	& 7	& 12	& 18	& 25	& 33	& 42	& 52	& 63	& 75	& 88	& 102	& 117	& 133	& 150	& 168	& 187	& 207 \\
3	&	&	&	& 4	& 9	& 15	& 22	& 30	& 39	& 49	& 60	& 72	& 85	& 99	& 114	& 130	& 147	& 165	& 184	& 204 \\
4	&	&	&	&	& 5	& 11	& 18	& 26	& 35	& 45	& 56	& 68	& 81	& 95	& 110	& 126	& 143	& 161	& 180	& 200 \\
5	&	&	&	&	&	& 6	& 13	& 21	& 30	& 40	& 51	& 63	& 76	& 90	& 105	& 121	& 138	& 156	& 175	& 195 \\
6	&	&	&	&	&	&	& 7	& 15	& 24	& 34	& 45	& 57	& 70	& 84	& 99	& 115	& 132	& 150	& 169	& 189 \\
7	&	&	&	&	&	&	&	& 8	& 17	& 27	& 38	& 50	& 63	& 77	& 92	& 108	& 125	& 143	& 162	& 182 \\
8	&	&	&	&	&	&	&	&	& 9	& 19	& 30	& 42	& 55	& 69	& 84	& 100	& 117	& 135	& 154	& 174 \\
9	&	&	&	&	&	&	&	&	&	& 10	& 21	& 33	& 46	& 60	& 75	& 91	& 108	& 126	& 145	& 165 \\
10	&	&	&	&	&	&	&	&	&	&	& 11	& 23	& 36	& 50	& 65	& 81	& 98	& 116	& 135	& 155 \\
11	&	&	&	&	&	&	&	&	&	&	&	& 12	& 25	& 39	& 54	& 70	& 87	& 105	& 124	& 144 \\
12	&	&	&	&	&	&	&	&	&	&	&	&	& 13	& 27	& 42	& 58	& 75	& 93	& 112	& 132 \\
13	&	&	&	&	&	&	&	&	&	&	&	&	&	& 14	& 29	& 45	& 62	& 80	& 99	& 119 \\
14	&	&	&	&	&	&	&	&	&	&	&	&	&	&	& 15	& 31	& 48	& 66	& 85	& 105 \\
15	&	&	&	&	&	&	&	&	&	&	&	&	&	&	&	& 16	& 33	& 51	& 70	& 90 \\
16	&	&	&	&	&	&	&	&	&	&	&	&	&	&	&	&	& 17	& 35	& 54	& 74 \\
17	&	&	&	&	&	&	&	&	&	&	&	&	&	&	&	&	&	& 18	& 37	& 57 \\
18	&	&	&	&	&	&	&	&	&	&	&	&	&	&	&	&	&	&	& 19	& 39 \\
19	&	&	&	&	&	&	&	&	&	&	&	&	&	&	&	&	&	&	&	& 20 \\
\end{dqtblr}}
\caption{Spell and Ritual Time Table}
\end{table}

% ----------------------------------------------------------------------
%
% Dates
%
% ----------------------------------------------------------------------

\include{dates}

\section{Map of the Duchy of Carzala}

\fbox{\includegraphics[width=175mm]{carzala.png}}

\begin{multicols}{3}

\section{Price List}
\label{pricelist}

\subsection{Coin valuation}

The value of a coin is determined by its weight and metal of which it
is made.

\begin{tabular}{l@{\hspace{0.5em}}c@{\hspace{1.0em}}l}
\textbf{Name} & \textbf{Weight} & \textbf{Conv} \\
Copper farthing (cf) & 1/5 oz & \\
Silver penny (sp) & 1/20 oz & 4 cf \\
Gold schilling (gs) & 1/20 oz & 12 sp \\
Truesilver guinea (tg) & 1/10 oz & 21 gs \\
\end{tabular}

Other common coins include the halfpenny, threepence, and sixpence.
The values and weights of these coins correspond to those of the
Silver Penny.


\subsection{Measures}

\begin{tabular}{lll}
16 ounces		& $=$ &	1 pound \\
14 pounds		& $=$ & 1 stone \\
112 pounds		& $=$ &	1 hundredweight \\
2000 pounds		& $=$ &	1 ton \\
\\
2 pints			& $=$ &	1 quart \\
4 quart			& $=$ &	1 gallon \\
8 gallons		& $=$ & 1 bushel \\
\\
12 inches		& $=$ & 1 foot \\
36 inches		& $=$ & 1 yard \\
1760 yards		& $=$ & 1 mile \\
5280 feet		& $=$ & 1 mile \\
\end{tabular}

\subsection{Prices}

All food prices are for a rural area that produces that type of food.
In another area the price may be 150\% or more than that shown.  Fish
prices are for coastal areas.  Inland that prices may rise as high as
500\% or become unavailable.

{\scriptsize
\setcounter{secnumdepth}{1}

\subsubsection{Vegetables - 1 pound}

				\hfill In season / out \\
Beans         			\hfill 1 cf / 1 cf \\
Peas           			\hfill 1 cf / 2 cf \\
Beets          			\hfill 1 cf / 1 cf \\
Carrots        			\hfill 1 cf / 1 cf \\
Lettuce        			\hfill 2 cf / ---- \\
Watercress     			\hfill 2 cf / ---- \\
Lentils        			\hfill 2 cf / 2 cf \\
Onions         			\hfill 1 cf / 1 cf \\
Cabbage        			\hfill 1 cf / 2 cf \\
Turnips        			\hfill 1 cf / 1 cf \\
Parsnips       			\hfill 1 cf / 1 cf \\
Parsley (bunch)			\hfill 2 cf / 3 cf \\
Cucumbers      			\hfill 1 sp / 2 sp \\
Garlic         			\hfill 2 cf / 3 cf \\
Potatoes       			\hfill 1 cf / 1 cf \\
Squash         			\hfill 1 cf / 2 cf \\
Sauerkraut     			\hfill 2 cf / 2 cf \\

\subsubsection{Fruit - 1 pound}

				\hfill In season or dried / Out \\
Apples				\hfill 2 cf / 3 cf \\
Apricots			\hfill 3 cf / 3 sp \\
Cherries			\hfill 3 cf / ---- \\
Crabapples			\hfill 1 cf / 2 cf \\
Plums				\hfill 1 cf / 2 cf \\
Strawberries			\hfill 3 cf / 3 sp \\
Lemons				\hfill 2 cf / 3 cf \\
Pears				\hfill 3 cf / ---- \\
Grapes				\hfill 1 cf / 2 cf \\
Berries				\hfill 1 cf / ---- \\
Oranges				\hfill 2 cf / 2 sp \\
Figs				\hfill 2 sp / 3 sp \\
Dates				\hfill 3 cf / 5 cf \\
Dried = 150\% of In season price \\
Candied = 200\% of In season price \\


\subsubsection{Dairy and Misc.}

Eggs				\hfill 3 cf / dozen \\
Common Cheese				\hfill 2 cf / pound \\
Good Cheese				\hfill 1 sp / pound \\
Fine Chesse				\hfill 10 cf / pound \\
Soft Chesse				\hfill 2 sp / pound \\
Cottage Cheese				\hfill 1 cf / pound \\
Milk				\hfill 1 cf / quart \\
Butter				\hfill 3 cf / pound \\
Salted Butter				\hfill 1 sp / pound \\
Buttermilk				\hfill 1 cf / quart \\
Curds \& Whey				\hfill 1 cf / pound \\
Cream				\hfill 1 sp / pint \\


\subsubsection{Preserved Food - 1 pound}

Salt Pork				\hfill 1 sp \\
Salt Beef				\hfill 5 cf \\
Salt Fish				\hfill 3 cf \\
Smoked Salmon				\hfill 10 cf \\
Salami				\hfill 3 sp \\
Smoked Sausages				\hfill 6 cf \\
Spiced Sausages				\hfill 3 sp \\
Hard Biscuits				\hfill 3 cf \\
Black Bread				\hfill 1 sp \\


\subsubsection{Seasonings and Misc. - 1 pound}

Sea Salt				\hfill 6 cf \\
Rock Salt				\hfill 3 sp \\
Honey				\hfill 10 cf \\
Lump Sugar				\hfill 5 sp \\
Pepper				\hfill 330 sp \\
Coffee (100 cups)				\hfill 25 sp \\
Tea (225 cups)				\hfill 56 sp \\
Cocoa Beans				\hfill 25 sp \\
Assorted Nuts				\hfill 1 sp \\


\subsubsection{Herbs, Nuts, Spices - 1 ounce}

Almonds				\hfill 1 cf \\
Aniseed				\hfill 1 cf \\
Aloes				\hfill 2 cf \\
Arsenic				\hfill 10 cf \\
Belladona			\hfill 60 sp \\
Basil				\hfill 3 cf \\
Chamomile			\hfill 1 cf \\
Catnip				\hfill 1 cf \\
Caper				\hfill 1 cf \\
Coriander			\hfill 1 cf \\
Cloves				\hfill 2 cf \\
Chives				\hfill 1 cf \\
Cinnamon			\hfill 10 cf \\
Endive				\hfill 1 cf \\
Elderflower			\hfill 2 cf \\
Fennel				\hfill 4 sp \\
Foxglove			\hfill 1 cf \\
Ginger				\hfill 12 sp \\
Hazelnuts			\hfill 1 cf \\
Hemlock				\hfill 10 cf \\
Hensbane			\hfill 1 cf \\
Hyssop				\hfill 3 cf \\
Horehound			\hfill 1 cf \\
Juniper				\hfill 2 cf \\
Jasmine				\hfill 10 cf \\
Lime				\hfill 1 cf \\
Mistletoe			\hfill 10 cf \\
Mushroom (poisonous)		\hfill 1 sp \\
Mace				\hfill 1 cf \\
Mint				\hfill 1 cf \\
Marjoram			\hfill 1 cf \\
Nutmeg				\hfill 1 cf \\
Nightshade			\hfill 10 cf \\
Oregano				\hfill 1 cf \\
Olives				\hfill 1 cf \\
Opium				\hfill 120 sp \\
Poppy Seed			\hfill 10 cf \\
Rosemary			\hfill 60 sp \\
Rue				\hfill 1 cf \\
Saffron				\hfill 120 sp \\
Sage				\hfill 1 cf \\
Sulphur				\hfill 1 cf \\
Thyme				\hfill 1 cf \\
Wormwood			\hfill 2 cf \\
Wolfbane			\hfill 10 cf \\

\subsubsection{Meat - 1 pound}

Veal				\hfill 1 sp \\
Beef Steak				\hfill 3 cf \\
Beef Roast				\hfill 2 cf \\
Beef, other				\hfill 2 cf \\
Pork, Loin				\hfill 1 sp \\
Pork, other cut				\hfill 3 cf \\
Ham				\hfill 5 cf \\
Bacon				\hfill 3 cf \\
Suckling Pig, 25lb				\hfill 35 sp \\
Mutton				\hfill 2 cf \\
Lamb				\hfill 5 cf \\
Chicken				\hfill 2 cf \\
Chicken, 5lb				\hfill 10 cf \\
Small Game Birds			\hfill 1 sp \\
Duck					\hfill 3 cf \\
Duck, 5lb				\hfill 3 sp \\
Goose					\hfill 1 sp \\
Goose, Fatted 15lb			\hfill 13 sp \\
Swan					\hfill 6 cf \\
Swan, 10lbs				\hfill 14 sp \\
Venison					\hfill 10 cf \\
Wild Boar				\hfill 9 cf \\


\subsubsection{Fish - 1 pound}

Common Fish				\hfill 2 cf \\
Good Fish				\hfill 1 sp \\
Common Shellfish			\hfill 3 cf \\
Clams/Crabs				\hfill 1 sp \\
Shrimps					\hfill 2 sp \\

\subsubsection{Bread - 1 pound}

White					\hfill 3 cf \\
Whole Wheat				\hfill 2 cf \\
Rye					\hfill 1 cf \\
White Rolls				\hfill 1 sp \\
White Trencher				\hfill 2 cf \\
Brown Trencher				\hfill 1 cf \\


\subsubsection{Beverages - Pints}

Cider					\hfill 1 cf \\
Wine (Bad)				\hfill 2 cf \\
Wine (Poor)				\hfill 2 sp \\
Wine (Good)				\hfill 10 sp \\
Wine (Fine)				\hfill 50-1000 sp \\
Young Beer				\hfill 1 cf \\
Beer					\hfill 2 cf \\
Double Beer				\hfill 3 cf \\
Ale					\hfill 2 cf \\
Stout					\hfill 3 cf \\
Mead					\hfill 1 sp \\
Brandy (Good)				\hfill 20 sp \\
Brandy (Fine)				\hfill 50-1000 sp \\
Port (Poor)				\hfill 3 cf \\
Port (Good)				\hfill 15 sp \\
Port (Fine)				\hfill 20-1000 sp \\
Coffee (cup)				\hfill 2 cf \\
Tea (cup)				\hfill 2 cf \\
Porter					\hfill 2 cf \\
Rum, Vodka, Gin, Whisky			\hfill 3 sp+ \\
Vinegar, 1 gallon			\hfill 2 cf \\


\subsubsection{Grain \& Feed - bushel (50lbs)}

Horses require 10lbs of grain and 20lbs of hay per day, or 15lbs of
wheat a day.  Grain may also be used to make bread, in which case
milling cost are incurred.  The cost of flour is 2 x the cost for the
grain for equal weight.

Wheat				\hfill 7 sp \\
Oats				\hfill 5 sp \\
Barley				\hfill 6 sp \\
Rye				\hfill 4 sp \\
Hay				\hfill 3 sp \\
Straw				\hfill 7 cf \\

\subsubsection{Gems - carat (1/120th oz.)}

Most costs are in carats, of which there are 120 in an ounce. A 120
carat (1 oz.) diamond measures approx 8 cc (2cm x 2cm x 2cm).	\hfill Cut gem
prices presume that a high level of skill was used but that nothing
fancy was done.


Raw Diamond			\hfill 120 sp \\
Cut Diamond			\hfill 360 sp \\
Raw Ruby			\hfill 110 sp \\
Cut Ruby			\hfill 300 sp \\
Raw Emerald			\hfill 100 sp \\
Cut Emerald			\hfill 240 sp \\
Raw Sapphire			\hfill 100 sp \\
Cut Sapphire			\hfill 240 sp \\
Cut Jade (1/10 pound)		\hfill 130 sp \\
Opal				\hfill 55 sp \\
Pearl				\hfill 115 sp \\
Amber (1/10 pound)		\hfill 60 sp \\
Semi-Precious			\hfill 25 sp \\

\subsubsection{Essences \& Perfumes - 1 ounce}

Black Lotus			\hfill 600 sp \\
Orchid				\hfill 250 sp \\
Musk				\hfill 50 sp \\
Poppy				\hfill 50 sp \\
Frankincense			\hfill 300 sp \\
Myrrh				\hfill 300 sp \\
Sunflower			\hfill 12 sp \\
Peony				\hfill 8 sp \\
Rose				\hfill 60 sp \\
Black Rose			\hfill 500 sp \\
Lavender			\hfill 18 sp \\
Cherry Blossom			\hfill 12 sp \\
Purple Rose			\hfill 200 sp \\
Black Poppy			\hfill 240 sp \\

\subsubsection{Oils}

Cooking Oil, Average		\hfill 2 sp / gallon \\
Cooking Oil, Good		\hfill 5 sp / gallon \\
Olive Oil, Average		\hfill 7 sp / gallon \\
Olive Oil, Good			\hfill 10 sp / gallon \\
Lard				\hfill 1 cf / pound \\

\subsubsection{Adventure Equipment}

\begin{tabbing}
\hspace{0.8\linewidth}\= \kill
Oil Lamp, Aladdin Type			\> 1lb			\' \` 6 sp \\
Oil Lantern				\> 3lb			\' \` 20 sp \\
Torch, Pitch Coated			\> 1lb			\' \` 1 sp \\
Leather Backpack, 20lb			\> 2lb			\' \` 3 sp \\
Leather Backpack, 40lb			\> 5lb			\' \` 8 sp \\
Leather Backpack, 60lb			\> 8lb			\' \` 20 sp \\
Sack, 10lb				\> 1/2lb		\' \` 1 sp \\
Sack, 20lb				\> 3/4lb		\' \` 2 sp \\
Sack, 40lb				\> 1lb			\' \` 3 sp \\
Leather Shoulder Pouch			\> 1lb			\' \` 2 sp \\
Quiver (20 Arrows)			\> 3lb			\' \` 6 sp \\
Tarp					\> 8lb			\' \` 12 sp \\
Flint \& Steel				\> 1/2lb		\' \` 3 sp \\
Tinderbox				\> 1lb			\' \` 10 sp \\
Leather Tent, 2 person			\> 50lb			\' \` 80 sp \\
Leather Tent, 4 person			\> 100lb		\' \` 150 sp \\
Leather Tent, 8 person			\> 150lb		\' \` 250 sp \\
Canvas Tent, 2 person			\> 30lb			\' \` 55 sp \\
Canvas Tent, 4 person			\> 60lb			\' \` 110 sp \\
Canvas Tent, 8 person			\> 90lb			\' \` 220 sp \\
Fish Hook \& Line			\> 0			\' \` 2 sp \\
Fish Net, Small				\> 1lb			\' \` 10 sp \\
Fish Net, Medium			\> 5lb			\' \` 25 sp \\
Blanket, Light				\> 2lb			\' \` 25 sp \\
Blanket, Heavy				\> 5lb			\' \` 55 sp \\
Sleeping Furs				\> 10lb			\' \` 165 sp \\
8'x4' Canvas Hammock			\> 2lb			\' \` 35 sp \\
8'x4' Leather Hammock			\> 5lb			\' \` 80 sp \\
Canvas Camp Bed, Single			\> 15lb			\' \` 20 sp \\
Leather Camp Bed, Single		\> 20lb			\' \` 50 sp \\
Mattress w/o Straw			\> 2lb			\' \` 2 sp \\
Sleeping Straw /week			\> 5lb			\' \` 1 cf \\
Knockdown Wooden Bed			\> 40lb			\' \` 110 sp \\
Fine Woolen Sheet			\> 1lb			\' \` 65 sp \\
Linen Sheet				\> 1lb			\' \` 165 sp \\
Silk/Satin Sheet			\> 1lb			\' \` 350 sp \\
Mosquito Netting			\> 0			\' \` 55 sp \\
Canvas Bucket, 5 gallon			\> 3/4lb		\' \` 5 sp \\
Leather Bucket, 5 gallon		\> 1lb			\' \` 8 sp \\
Wooden Bucket, 5 gallon			\> 2lb			\' \` 2 sp \\
Metal Bucket, 5 gallon			\> 2lb			\' \` 15 sp \\
Canteen, pint				\> 1/4lb		\' \` 3 sp \\
Canteen, quart				\> 1/2lb		\' \` 5 sp \\
Canteen, 1/2 gallon			\> 1lb			\' \` 8 sp \\
Waterskin, 1 gallon			\> 1lb			\' \` 8 sp \\
Waterskin, 5 gallon			\> 2lb			\' \` 15 sp \\
Pewter Mess Kit				\> 1lb			\' \` 45 sp \\
Lamp Oil, pint (24hrs)			\> 1 1/4lb		\' \` 65 sp \\
Physiker's Kit				\> 1lb			\' \` 20 sp \\
52 Card Common Deck			\> 1/10lb		\' \` 10 sp \\
78 Card Tarot Deck			\> 1/5lb		\' \` 30 sp \\
Trap container				\> 5lb			\' \` 860 sp \\
Iron rations, average (1 day)		\> 1lb			\' \` 1 sp \\
Iron rations, good (1 day)		\> 1lb			\' \` 3 sp \\
\end{tabbing}

\subsubsection{Containers}

All weights given are for empty containers, unless otherwise shown.  1
Gallon of water weighs 10 pounds.  There are 8 pints or 4 quarts to
the gallon.  1 dose is approximately 1/5 of a pint and weights 1/4 of
a pound.

\begin{tabbing}
\hspace{0.8\linewidth}\= \kill
Wooden Cask, 12 gallon			\> 10lb			\' \` 25 sp \\
Wooden Cask, 25 gallon			\> 25lb			\' \` 35 sp \\
Wooden Cask, 50 gallon			\> 50lb			\' \` 50 sp \\
Pottery Jar, 1 pint			\> 1lb			\' \` 2 sp \\
Pottery Jar, 1/2 gallon			\> 2lb			\' \` 5 sp \\
Pottery Jar, 1 gallon			\> 4lb			\' \` 8 sp \\
Pottery Crock, 5 gallon			\> 10lb			\' \` 25 sp \\
Wine Amphora, 12 gallon			\> 35lb			\' \` 40 sp \\
Glass Jar, 1 pint			\> 3/4lb			\' \` 40 sp \\
Glass Jar, 1 quart			\> 1lb			\' \` 55 sp \\
Glass Jar, 1/2 gallon			\> 2lb			\' \` 100 sp \\
Glass Jar, 1 gallon			\> 4lb			\' \` 165 sp \\
Glass Vial, 1 dose			\> 1/4lb			\' \` 25 sp \\
Glass Vial, 5 dose			\> 1lb			\' \` 65 sp \\
Tin Vial, 1 dose			\> 1/4lb			\' \` 10 sp \\
Tin Vial, 5 dose			\> 1lb			\' \` 25 sp \\
Pottery Vial, 1 dose			\> 1/4lb			\' \` 3 sp \\
Pottery Vial, 5 dose			\> 1lb			\' \` 5 sp \\
\end{tabbing}

NB: Crystal vials cost x5 Glass vials; Silver vials cost x5 Tin vials;
Fine Porcelain vials cost x5 Pottery vials.

\subsubsection{Cooking/Eating Equipment}

\begin{tabbing}
\hspace{0.8\linewidth}\= \kill
Small Iron Pan				\> 2lb			\' \` 9 sp \\
Large Iron Pan				\> 5lb			\' \` 24 sp \\
Iron Pot, 1 quart			\> 2lb			\' \` 9 sp \\
Iron Pot, 1/2 gallon			\> 3lb			\' \` 15 sp \\
Iron Pot, 1 gallon			\> 8lb			\' \` 30 sp \\
Iron Kettle, 5 gallon			\> 25lb			\' \` 100 sp \\
Iron Cauldron, 10 gallon		\> 50lb			\' \` 265 sp \\
Iron Cauldron, 25 gallon		\> 125lb		\' \` 330 sp \\
Iron Cauldron, 50 gallon		\> 200lb		\' \` 400 sp \\
Copper cooking gear			\> var			\' \` +25\% \\
Wooden Spoon				\> 0			\' \` 1 sp \\
Pewter Knife/Spoon			\> 1/10lb		\' \` 5 sp \\
Silver Knife/Spoon			\> 1/10lb		\' \` 35 sp \\
Gold Knife/Spoon			\> 1/10lb		\' \` 420 sp \\
Pewter Fork				\> 1/10lb		\' \` 6 sp \\
Silver Fork				\> 1/10lb		\' \` 45 sp \\
Gold Fork				\> 1/10lb		\' \` 540 sp \\
Wooden Soup Ladle			\> 1/5lb		\' \` 7 cf \\
Pewter Soup Ladle			\> 1/4lb		\' \` 10 sp \\
Silver Soup Ladle			\> 1/4lb		\' \` 75 sp \\
Gold Soup Ladle				\> 1/4lb		\' \` 720 sp \\
Wooden Plate/Bowl			\> 1/4lb		\' \` 5 cf \\
Pewter Plate/Bowl			\> 1/3lb		\' \` 10 sp \\
Enamelled Tin Plate/Bowl		\> 1/3lb		\' \` 9 sp \\
Silver Plate/Bowl			\> 1/2lb		\' \` 80 sp \\
Gold Plate/Bowl				\> 1/2lb		\' \` 900 sp \\
Earthenware Plate/Bowl			\> 1/2lb		\' \` 2 sp \\
Porcelain Plate/Bowl			\> 1/2lb		\' \` 55 sp \\
Earthenware/Wooden Mug			\> 1/4lb		\' \` 5 cf \\
Pewter Mug				\> 1/3lb		\' \` 24 sp \\
Porcelain Cup				\> 1/3lb		\' \` 50 sp \\
Silver Goblet				\> 1/3lb		\' \` 55 sp \\
Gold Goblet				\> 1/3lb		\' \` 650 sp \\
Porcelain Goblet, Fancy			\> 1/4lb		\' \` 60 sp \\
Glass Goblet				\> 1/4lb		\' \` 110 sp \\
Fine Crystal Goblet			\> 1/4lb		\' \` 250 sp \\
Pewter Salt Cellar			\> 1lb			\' \` 80 sp \\
Silver Salt Cellar			\> 1lb			\' \` 230 sp \\
Gold Salt Cellar			\> 1lb			\' \` 2500 sp \\
Drinking Horn, Plain			\> 1/5lb		\' \` 15 sp \\
Drinking Horn, Silver			\> 1/3lb		\' \` 45 sp \\
\end{tabbing}
Silver Items may be alloy/plate for 50\% cost \\
Gold Items may be alloy/plate for 30\% cost \\

\subsubsection{Household Goods}

\begin{tabbing}
\hspace{0.8\linewidth}\= \kill
Rushlight x5 (1 hr. each)		\> 1lb			\' \` 1 cf \\
Candle, Tallow (1hr)			\> 1/5lb		\' \` 2 cf \\
Candle, Wax (1hr)			\> 1/5lb		\' \` 3 cf \\
Candelabra, Wood			\> 1/2lb		\' \` 3 sp \\
Candelabra, Iron			\> 1 1/2lb		\' \` 15 sp \\
Candelabra, Bronze			\> 1 1/2lb		\' \` 24 sp \\
Candelabra, Silver			\> 1 1/2lb		\' \` 300 sp \\
Candelabra, Gold			\> 2lb			\' \` 3500 sp \\
For Candelabra, add 1/2lb of weight and approx \\
+30\% per additional spike to cost. \\
Glassed Candle Lantern			\> 2lb			\' \` 45 sp \\
Glassed Oil Lantern			\> 2lb			\' \` 110 sp \\
Wooden Stool			\> 5lb			\' \` 4 sp \\
Wooden Chair			\> 10lb			\' \` 6 sp \\
Wooden Bench, 2 person			\> 15lb			\' \` 6 sp \\
Wooden Bench, 4 person			\> 30lb			\' \` 8 sp \\
Great Seat			\> 50lb			\' \` 30 sp \\
Plain Throne			\> 55lb			\' \` 60 sp \\
Fancy Throne			\hfill 60lb			\> 100 sp+			\' \` \\
Wooden Table			\> 50lb+			\' \` 20 sp+ \\
Wooden Chest (Small)			\> 10lb			\' \` 10 sp \\
Wooden Chest (Large)			\> 20lb			\' \` 24 sp \\
Metal Chest (Small)			\> 20lb			\' \` 48 sp \\
Good Bed			\> 100lb+			\' \` 55 sp+ \\
Fine Bed			\> 150lb+			\' \` 265 sp+ \\
Bed Hangings			\> 10lb			\' \` 55 sp+ \\
Fine Bed Hangings			\> 20lb			\' \` 250 sp+ \\
Seat Cushion			\> 1lb			\' \` 12 sp \\
Fine Seat Cushion			\> 1lb			\' \` 25 sp+ \\
Small Metal Mirror			\> 1lb			\' \` 30 sp \\
Large Metal Mirror			\> 20lb			\' \` 85 sp \\
Soap, Plain			\> 1/4lb			\' \` 5 sp \\
Soap, Perfumed			\> 1/4lb			\' \` 25 sp \\
Brazier, Small Iron			\> 2lb			\' \` 20 sp \\
Brazier, Tripod Iron			\> 20lb			\' \` 170 sp \\
Brazier, Small Bronze			\> 2lb			\' \` 60 sp \\
Brazier, Tripod Bronze			\> 20lb			\' \` 395 sp+ \\
Charcoal			\> 10lb			\' \` 3 sp \\
\end{tabbing}

\subsubsection{Tools \& Miscellaneous}

\begin{tabbing}
\hspace{0.8\linewidth}\= \kill
Wood Saw, Iron Blade			\> 3lb			\' \` 40 sp \\
Wood Saw, Steel Blade			\> 3lb			\' \` 120 sp \\
Wood Saw, Double Ended			\> 20lb			\' \` 280 sp \\
Hammer, Carpenters			\> 2lb			\' \` 12 sp \\
Wooden Mallet				\> 2lb			\' \` 2 sp \\
Hatchet					\> 2lb			\' \` 25 sp \\
Wood Axe				\> 4lb			\' \` 45 sp \\
Adze					\> 2lb			\' \` 30 sp \\
Auger					\> 2lb			\' \` 30 sp \\
Iron Drill Bits x5			\> 1lb			\' \` 25 sp \\
Steel Drill Bits x5			\> 1lb			\' \` 55 sp \\
Wood Chisel				\> 1lb			\' \` 17 sp \\
Masonry/Stone Chisel			\> 1lb			\' \` 22 sp \\
Rock Drill				\> 6lb			\' \` 30 sp \\
Crowbar					\> 5lb			\' \` 15 sp \\
Crowbar, Heavy				\> 10lb			\' \` 28 sp \\
Finishing Nails x250			\> 1lb			\' \` 55 sp \\
Large Nails x100			\> 1lb			\' \` 35 sp \\
Iron Spikes x10				\> 1lb			\' \` 25 sp \\
Iron Wedges x3				\> 1lb			\' \` 6 sp \\
Stake, Wooden				\> 1lb			\' \` 1 cf \\
Rough Cut Lumber (1cu.ft.)		\> 50lb			\' \` 1 sp \\
Clean Cut Lumber (1cu.ft.)		\> 50lb			\' \` 2 sp \\
Wood Glue, 1 pint			\> 1 1/2lb			\' \` 4 sp \\
Paint, White 1 gallon			\> 10lb			\' \` 10 sp \\
Paint, Coloured 1 gallon		\> 10lb			\' \` 15 sp+ \\
Metal Yardstick				\> 2lb			\' \` 24 sp \\
Wooden Yardstick			\> 1lb			\' \` 3 sp \\
Carpenter's Square			\> 2lb			\' \` 7 sp \\
Carpenter's Level			\> 2lb			\' \` 32 sp \\
Waxed 100' Tape/Cord			\> 1lb			\' \` 20 sp \\
Plumbob					\> 2lb			\' \` 12 sp \\
Iron 100' Measuring Chain		\> 10lb			\' \` 50 sp \\
String 100'				\> 1/2lb			\' \` 6 cf \\
Pick Axe				\> 6lb			\' \` 55 sp \\
Shovel					\> 4lb			\' \` 24 sp \\
Sickle					\> 2lb			\' \` 22 sp \\
Scythe					\> 4lb			\' \` 34 sp \\
Pitchfork				\> 4lb			\' \` 30 sp \\
Iron Plough Blade			\> 25lb			\' \` 185 sp+ \\
Full Mouldboard Plough			\> 100lb+		\' \` 800 sp+ \\
Hammer, Blacksmiths			\> 3lb			\' \` 25 sp \\
Hammer, Armourers			\> 3lb			\' \` 24 sp \\
Cold Chisel				\> 2lb			\' \` 22 sp \\
Portable Anvil, Light			\> 15lb			\' \` 85 sp \\
Portable Anvil, Heavy			\> 25lb			\' \` 165 sp \\
Standard Anvil				\> 50lb+			\' \` 250 sp \\
Bellows, Small				\> 5lb			\' \` 40 sp \\
Bellows, Large				\> 25lb			\' \` 70 sp \\
Forge, Portable				\> 100lb			\' \` 300 sp \\
Forge					\> 1000lb			\' \` 450 sp \\
Tongs/Pliers, Small			\> 2lb			\' \` 15 sp \\
Tongs/Pliers, Large			\> 6lb			\' \` 28 sp \\
Iron Ingot				\> 25lb			\' \` 34 sp \\
Copper Ingot				\> 25lb			\' \` 110 sp \\
Lead Ingot				\> 25lb			\' \` 25 sp \\
Tin Ingot				\> 25lb			\' \` 28 sp \\
Other Base Metals			\> 25lb			\' \` 20 sp \\
Silver Ingot				\> 5lb			\' \` 1600 sp \\
Gold Ingot				\> 5lb			\' \` 19200 sp \\
Platinum Ingot				\> 5lb			\' \` 20800 sp \\
TrueSilver Ingot			\> 1lb			\' \` 40320 sp \\
Pulley 100lb, 1.5:1			\> 3lb			\' \` 30 sp \\
Pulley 100lb, 2:1			\> 4lb			\' \` 55 sp \\
Pulley 100lb, 3:1			\> 5lb			\' \` 80 sp \\
Pulley 100lb, 4:1			\> 6lb			\' \` 165 sp \\
Pulley 100lb, 5:1			\> 7lb			\' \` 250 sp \\
Pulleys per extra 100lb			\> 1/2lb		\' \` +10\% \\
Rope 50', 1/2"				\> 3lb			\' \` 10 sp \\
Rope 50', 1"				\> 5lb			\' \` 20 sp \\
Rope 50', Silk				\> 2lb			\' \` 55 sp \\
Chain 1'				\> 2lb			\' \` 30 sp \\
Chain, Heavy 1'				\> 5lb			\' \` 50 sp \\
Wooden Ladder per 10'			\> 25lb			\' \` 25 sp \\
Rope Ladder per 10'			\> 4lb			\' \` 18 sp \\
Grappling Hook				\> 1lb			\' \` 15 sp \\
Hourglass				\> 3lb			\' \` 350 sp+ \\
Minuteglass (1, 3 or 5)			\> 1/4lb		\' \` 75 sp+ \\
Sundial					\> 50lb			\' \` 220 sp+ \\
Ornate Sundial				\> 100lb		\' \` 550 sp+ \\
Pocket Sundial				\> 1/4lb		\' \` 55 sp+ \\
Waterclock				\> 50lb+		\' \` 700 sp+ \\
Spring Clock				\> 5lb			\' \` 1100 sp+ \\
Pendulum Clock				\> 100lb+		\' \` 1500 sp+ \\
Spyglass				\> 5lb			\' \` 300 sp+ \\
Spectacles				\> 1/10lb		\' \`250 sp+ \\
\end{tabbing}

\subsubsection{Writing Materials}

\begin{tabbing}
\hspace{0.8\linewidth}\= \kill
3' x 3' Paper Sheet			\> 1/10lb			\' \` 3 cf \\
3' x 3' Parchment Sheet			\> 1/10lb			\' \` 2 sp \\
3' x 3' Vellum Sheet			\> 1/10lb			\' \` 3 sp \\
Writing Tablet, Slate			\> 1lb			\' \` 3 sp \\
Writing Tablet, Wax			\> 1lb			\' \` 5 sp \\
Quill Pen, Average			\` 0 2 cf \\
Quill Pen, Good			\` 0 3 cf \\
Stylus (for Wax Tablet)			\> 1/10lb			\' \` 3 sp \\
Chalk, Stick			\> 1/10lb			\' \` 1 cf \\
Ink \& Pot			\> 1/5lb			\' \` 1 sp \\
Portable Writing Desk			\> 10lb			\' \` 18 sp \\
Large Writing Desk			\> 100lb+			\' \` 55 sp+ \\
Bindery Glue \& Pot			\> 1/2lb			\' \` 2 sp \\
Set of Book Covers			\> 5lb			\' \` 12 sp \\
Book (x50 A4 pages)			\> 7lb			\' \` 120 sp+ \\
Book (x50 A3 pages)			\> 14lb			\' \` 250 sp+ \\
Illuminated Book			\> 0			\' \` +150\% \\
Printed Book			\> 0			\' \` 25\% \\
Map Case			\> 1lb			\' \` 10 sp \\
Seal, Personal			\> 1/4lb			\' \` 25 sp \\
Seal, Silver			\> 1/4lb			\' \` 100 sp \\
Seal, Gold			\> 1/4lb			\' \` 250 sp \\
Sealing Wax (Red)			\> 1/10lb			\' \` 6 sp \\
Sealing Ribbon (Red)			\> 1/10lb			\' \` 5 sp \\
Coloured Ink \& Pot			\> 1/5lb			\' \` 5 sp \\
4" x 4" Sheet of Gold Leaf			\> 0 3 sp			\' \` \\
\end{tabbing}

\subsubsection{Transport Gear}

\begin{tabbing}
\hspace{0.8\linewidth}\= \kill
Bit/Bridle			\` 20 sp \\
Riding Saddle			\` 105 sp \\
Sidesaddle			\` 155 sp \\
Saddle Blanket			\` 15 sp \\
Saddle Roll			\` 3 sp \\
10lb Saddle Bag			\` 5 sp \\
20lb Saddle Bag			\` 12 sp \\
300lb Packsaddle			\` 80 sp \\
Horseshoe			\` 6 sp \\
5lb Nosebag			\` 4 sp \\
Draft Harness			\` 35 sp \\
Horse Collar			\` 65 sp \\
Ox Yoke			\` 45 sp \\
Spurs, Plain			\` 18 sp \\
Spurs, Silvered			\` 50 sp \\
Riding Crop			\` 15 sp \\
Driving Whip			\` 25 sp \\
Pony Cart (500lb)			\` 85 sp \\
Cart (1 Horse, 1000lb)			\` 150 sp \\
Sm. Wagon (2 Horse, 1t)			\` 275 sp \\
Lg. Wagon (4 Horse, 3t)			\` 450 sp \\
Open Coach (2 Horse)			\` 500 sp \\
Closed Coach (2 Horse)			\` 650 sp \\
Open Coach (4 Horse)			\` 900 sp \\
Closed Coach (4 Horse)			\` 1050 sp \\
6' Rowboat (4 Seat)			\` 110 sp \\
12' Longboat (8 Seat)			\` 350 sp \\
16' Longboat (10 Seat)			\` 500 sp \\
Canoe (2 Seat)			\` 60 sp \\
Folding Leather Boat (4 Seat)			\` 250 sp \\
Paddle			\` 8 sp \\
Oar			\` 15 sp \\
Galley Oar			\` 25 sp \\
Ships (see Ship Supplement)			\` \\
\end{tabbing}

\subsubsection{Cloth \& Clothing}

\begin{tabbing}
\hspace{0.8\linewidth}\= \kill
Hat / Hood				\> 3/4lb		\' \` 3 sp \\
Short Cloak				\> 2lb			\' \` 5 sp \\
Long Cloak				\> 5lb			\' \` 8 sp \\
Jacket					\> 2lb			\' \` 6 sp \\
Full Length Coat			\> 5lb			\' \` 8 sp \\
Tunic					\> 3/4lb		\' \` 3 sp \\
Shirt					\> 3/4lb		\' \` 2 sp \\
Dress					\> 3lb			\' \` 5 sp \\
Skirt					\> 1 1/2lb		\' \` 3 sp \\
Hose					\> 1/2lb		\' \` 5 sp \\
Long Pants				\> 1 1/4lb		\' \` 5 sp \\
Short Pants				\> 1lb			\' \` 1 sp \\
Sandals					\> 1/2lb		\' \` 1 sp \\
Slippers				\> 1/4lb		\' \` 3 sp \\
Walking Shoes				\> 1lb			\' \` 3 sp \\
Low Boots				\> 2lb			\' \` 4 sp \\
High Boots				\> 3lb			\' \` 6 sp \\
Hip Boots				\> 4lb			\' \` 10 sp \\
Belt					\> 1/2lb		\' \` 2 sp \\
Girdle					\> 1lb			\' \` 4 sp \\
Money Belt				\> 3/4lb		\' \` 4 sp \\
Weapons Belt				\> 1lb			\' \` 5 sp \\
Baldric					\> 2lb			\' \` 8 sp \\
Scarf					\> 1/4lb		\' \` 1 sp \\
Gloves					\> 1/2lb		\' \` 2 sp \\
Mittens					\> 1/2lb		\' \` 1 sp \\
Winter Wear				\> +25\%		\' \` +20\% \\
Undyed Homespun (Subsistence)					\` x1 \\
Undyed Linen/Cotton						\` x3/2 \\
Dyed Linen/Cotton (Poor)					\` x2 \\
Undyed Good Linen/Cotton					\` x3 \\
Dyed Good Linen/Cotton (Middle Class)				\` x4 \\
Dyed Fine Linen/Cotton (Wealthy+)				\` x8 \\
Silk/Satin (Rich+)					 	\` x15 \\
Fur Trim							\` +50\% \\
Lightly Embroided						\` +20\% \\
Richly Embroided			 			\` +50\% \\
Silver Thread							\` +50\% \\
Gold Thread							\` +100\% \\
Leather Items: \\
\hspace{0.4cm} Poor						\` x1 \\
\hspace{0.4cm} Middle Class					\` x2 \\
\hspace{0.4cm} Wealthy						\` x4 \\
\hspace{0.4cm} Rich						\` x8+ \\
\end{tabbing}

\subsubsection{Accomodation and Meals}
\begin{tabbing}
A private room in a inn						\\
\hspace{0.4cm} Poor						\` 2 sp \\
\hspace{0.4cm} Middle Class					\` 4 sp \\
\hspace{0.4cm} Wealthy						\` 6 sp \\
\hspace{0.4cm} Rich						\` 12 sp \\
A common room in a inn							\\
\hspace{0.4cm} Poor						\` 1 sp \\
\hspace{0.4cm} Middle Class					\` 2 sp \\
A small meal (breakfast, small lunch)					\\
\hspace{0.4cm} Poor						\` 1 cp \\
\hspace{0.4cm} Middle Class					\` 2 cp \\
\hspace{0.4cm} Wealthy						\` 3 cp \\
\hspace{0.4cm} Rich						\` 2 sp \\
A large meal (large lunch, dinner)					\\
\hspace{0.4cm} Poor						\` 2 cp \\
\hspace{0.4cm} Middle Class					\` 1 sp \\
\hspace{0.4cm} Wealthy						\` 3 sp \\
\hspace{0.4cm} Rich						\` 6 sp \\
\end{tabbing}

\subsubsection{Sewing/Weaving Equipment}

\begin{tabbing}
\hspace{0.8\linewidth}\= \kill
Needles (5)				\> 1/10lb	\' \` 8 sp \\
Thread 150'				\> 1/10lb	\' \` 2 sp \\
Coloured Thread 150'			\> 1/10lb	\' \` 3 sp \\
Silver Thread 150'			\> 1/10lb	\' \` 12 sp \\
Gold Thread 150'			\> 1/10lb	\' \` 30 sp \\
Shears					\> 1lb		\' \` 20 sp \\
Weaving Loom				\> 25lb		\' \` 40 sp \\
Spindle					\> 1lb		\' \` 1 sp \\
Carding Comb				\> 1lb		\' \` 2 sp \\
\end{tabbing}

\subsubsection{Animals \& Livestock}

War Horse			\hfill 1050 sp \\
Palfry				\hfill 720 sp \\
Drafthorse			\hfill 600 sp \\
Pony				\hfill 385 sp \\
Donkey/Mule			\hfill 200 sp \\
Mustang				\hfill 480 sp \\
Quarterhorse			\hfill 900 sp \\
Ox				\hfill 280 sp \\
Bull				\hfill 300 sp \\
Cow				\hfill 250 sp \\
Calf				\hfill 125 sp \\
Pig				\hfill 100 sp \\
Suckling Pig			\hfill 34 sp \\
Sheep				\hfill 24 sp \\
Lamb				\hfill 14 sp \\
Chicken				\hfill 2 sp \\
Duck				\hfill 10 cf \\
Goose				\hfill 12 sp \\
Swan				\hfill 13 sp \\
Monkey				\hfill 55 sp \\
Parrot				\hfill 3 sp \\
Dog, Hunting			\hfill 20 sp \\
Dog, War			\hfill 35 sp \\
Weasel				\hfill 2 sp \\
}

\end{multicols}


\section{THE ADVENTURERS GUILD: MEMBERSHIP AGREEMENT}

\bigskip

{\fontspec{TeX Gyre Termes}\fontsize{8}{9pt}\selectfont

{\bfseries
THIS AGREEMENT made this \hspace{15em} day of \hspace{16em} A.P.

BETWEEN the Adventures Guild of Seagate, hereinafter called "The Guild"

\vspace{5mm}

\hspace{5mm}   AND  \hspace{5mm}        (name)

\vspace{5mm}

\hspace{5mm}     (sex)
\hspace{5mm}     (status)  \hspace{5mm}   (college)

\vspace{5mm}

\hspace{5mm}   hereinafter called "The Member".

\vspace{5mm}


PARTICULARS:}

\begin{enumerate}

\item    Definitions.
\begin{enumerate}
\item "Treasure" means any monies or goods gained while actively pursuing "the Mission".
\item "The Mission" means the activities performed under Guild auspices.
\item "The Party" means a group with whom the Member is on a Mission.
\end{enumerate}

\item \begin{enumerate}
\item  The Guild does not condone activities such as killing, theft,
extortion, etc. These activities (and others) are illegal in most
cultures we interact with. Any crimes committed against a society will
be answerable to that society. The Guild will not stand in the way of
the normal course of justice.
\item The Guild cannot supplant or override the laws of the land.
\item  All members are subject to the laws of the Duchy of Carzarla.
\end{enumerate}

\item    THE GUILD RULES:

The Member is subject to Guild justice if they:

\begin{enumerate}
\item  attack another member without reasonable provocation,
\item  kill another member,
\item  steal from another member, including the withholding of treasure,
\item  otherwise harm another member, including the use of magical coercion or control,
\item fail to comply with an "Adventuring Agreement" provided that such is
currently in effect and binding on the Member,
\item fail to make all reasonable attempts to regain, if necessary, and
return another Party Member who is dead, incapacitated and/or
captured, to the Guild Headquarters in Seagate,
\item  desert the Party voluntarily; if accidentally separated from the
Party the Member is obliged to make all reasonable attempts to rejoin
the Party,
\item  fail to follow instructions given by a duly appointed Leader of
the Party of which the Member is part.
\end{enumerate}
\item Guild justice may include: fines, forfeiture of part or whole of
shares of Treasure, suspension, expulsion, offering of a reward for
capture, turning over to Civil Authorities and/or the bringing of a
civil case.
\item 10\% of all Treasure gained by the Member or 200sp per annum,
whichever is greater, is due to the Guild.
\end{enumerate}
\vspace{5mm}

{\bfseries
DECLARATION:
\vspace{5mm}

I agree to abide by the rules of the Guild as above, and as may be,
from time to time, laid down by the council, and to accept their
rulings in matters of arbitration and justice.}

\vspace{5mm}

Signed

\vspace{15mm}

(by the Member)


\vspace{5mm}

(for and on behalf of the Guild)
}

\pagebreak


%% ----------------------------------------------------------------------

\section{Useful Tables}

\fbox{
\begin{minipage}{\linewidth}
\subsection{Combat Equation Summary}

\begin{multicols}{2}
\begin{description}

\item[Inititive value for engaged figures]
PC + Modified AG + maximum Rank with any prepared weapon.

\item[Initiative for non-engaged figures]
D10 + (PC + 1 \x Rank of Military Scientist).

\item[Strike chance with ranked weapon]
Weapon's Base Chance + attacker's modified Manual Dexterity + (4 \x
Rank with weapon) - Opponent's Defense.

\item[Figure's defense]
Modified AG + Shield defense + Magic.

\item[Repulse a charge attack]
D10 verses Rank of repulser's prepared weapon.

\item[Withdraw from close combat] (D10 + total friendly Physical
Strength - total hostile Physical strength) $>$ 10.

\item[Strike chance to trip] 40\% + attacker's Modified Manual
Dexterity + (4 \x Rank with weapon) - opponent's defense
(\emph{Damage:} D10).

\item[Restrain]
3 \x ((PS + AG of attacker) - (PS + AG of defender)).

\item[Shield rush] 40\% + attacker's modified MD + (4 \x Rank with
shield) - oppenent's defense (\emph{Damage:} [D - 2]).

\item[Disarm] -20 to strike chance.

\item[Entangle] Same as normal strike chance with weapon
(\emph{Damage:} [D - 4]).

\item[Knockout] Must roll under (15\% \x strike chance) + 10.

\item[Avoid weapon break or drop] 3 \x modified MD on D100.

\item[Stun recovery] (2 \x WP) + current Fatigue.

\item[Parry result] D10 + evader's Rank - attacker's Rank. \\
1,2,3 or less: Successful parry; evader must pass next action. \\
4,5,6,7: Disarm, 1 EN damage. \\
8 or greater: Disarm plus a riposte; evader may melee attack, 1 EN damage. \\
\end{description}
\end{multicols}
\end{minipage}}

\subsection{Fatigue, Encumbrance and Movement Charts}
\label{tables:encumbrance}
\label{tables:tmr}

\begin{tabular}[t]{cc}
{\fontsize{8}{11pt}\selectfont
\begin{tabular}[t]{l@{\hspace{1.8em}}c@{\hspace{1.8em}}c@{\hspace{1.8em}}c@{\hspace{1.8em}}c@{\hspace{1.8em}}c@{\hspace{1.8em}}c@{\hspace{1.8em}}c@{\hspace{1.8em}}c@{\hspace{1.8em}}c}
\textbf{PS} & \multicolumn{8}{c}{\textbf{Weight of Load (lbs)}} & \textbf{Max} \\ \hline
3-5	&0	&0	&0	&10	&18	&25	&35	&40	&50 \\ \hline
6-8	&0	&0	&10	&15	&20	&30	&50	&60	&75 \\ \hline
9-12	&0	&10	&15	&20	&30	&50	&70	&80	&100 \\ \hline
13-17	&10	&15	&20	&30	&50	&70	&90	&100	&125 \\ \hline
18-20	&15	&20	&30	&40	&60	&90	&120	&130	&150 \\ \hline
21-23	&20	&30	&50	&60	&80	&120	&160	&170	&200 \\ \hline
24-27	&30	&40	&60	&70	&100	&140	&180	&190	&225 \\ \hline
28-32	&40	&50	&80	&90	&120	&160	&200	&210	&250 \\ \hline
33-36	&50	&60	&100	&120	&160	&200	&240	&250	&275 \\ \hline
37-40	&60	&70	&120	&150	&190	&225	&270	&290	&325 \\ \hline
\multicolumn{10}{l}{\textbf{Rate of Exercise}} \\
Light	&0	&0	&0	&$\frac{1}{2}$&$\frac{1}{2}$&1	&2	&3	&5 \\ \hline
Medium	&0	&0	&$\frac{1}{2}$&$\frac{1}{2}$&1	&1	&3	&4	&6 \\ \hline
Hard	&$\frac{1}{2}$&$\frac{1}{2}$&1	&1	&2	&3	&5	&6	&8 \\ \hline
Strenuous&2	&2	&3	&3	&4	&5	&6	&7	&9 \\ \hline
\multicolumn{10}{l}{\textbf{Agility Loss in Combat}} \\
Loss	 &0	&1	&2	&3	&5	&7	&9	&10	&12 \\ \hline
\end{tabular}}
&
\begin{minipage}[t]{2.5in}
\begin{tabularx}{2.5in}[t]{cXc}
\textbf{Modified} & \\
\textbf{Agility} & & \textbf{TMR} \\ \hline
 $<$ 1		& & 0 \\ \hline
  1 -- 2	& &  1 \\ \hline
  3 -- 4	& & 2 \\ \hline
  5 -- 8	& & 3 \\ \hline
  9 -- 12	& & 4 \\ \hline
 13 -- 17	& & 5 \\ \hline
 18 -- 21	& & 6 \\ \hline
 22 -- 25	& & 7 \\ \hline
 26 -- 27	& & 8 \\ \hline
$>$ 27		& & \dag \\ \hline
\end{tabularx}

\dag TMR = 9 + 1 for every two points of AG over 28, for example AG 32 gives 11 TMR.
\end{minipage} \\
\end{tabular}

\begin{multicols}{2}

{\setlength\leftmargini{0pt}
\begin{description}
\setlength\itemsep{0pt}
\item[Weight of Load (lbs)] The weight, in pounds, that a character is
carrying, rounded off to the nearest entry on the appropriate Physical
Strength row (if the weight is exactly between two entries, use the
greater one).  Note: A mount can carry weight for a character while
they are riding.

\item[Max] The maximum load, in pounds, that a character can carry for
a sustained period of time.

\item[Agility Points Lost] The temporary Agility Point loss suffered
by a character toting the given weight in combat.

\end{description}}

\end{multicols}

\rule{\linewidth}{0.6mm}

\newpage

\section{Non Magic User Character Sheet}
\newpage

\section{Magic User Character Sheet}
\newpage

\section{Character Equipment Sheet}
\newpage

\end{document}
