\documentclass{article}

\usepackage{dq2020e}

\title{Ships\\Version 1.0}
\author{Martin Dickson}
\date{\today}

\begin{document}

\maketitle

\tableofcontents

\pagebreak

\begin{multicols}{3}

\section{Introduction}

The following pages have, with the exception of this cover-sheet, been
lifted from 'Bireme and Galley', the excellent naval supplement for
'Chivalry and Sorcery'.  Because of this a very few of the ship's
statistics, such as turning radius, will be in game system notation
and should be ignored when used for Dragonguest.  The general period
of the DQ game is assumed to be approx. AD 1485, with some areas up to
50 years in advance of this, socially and economically, and other,
backward areas, as early as 1285.  The Viking type ships, though of an
earlier period on Earth, are included as certain Nordic cultures still
seem to flourish in Alusia.  The prices and statistics given for the
ships in the DQ rules are dubious to say the least, and should
hereinafter be ignored.  The prices for the new ships are detailed
below, appropriately modified for use in DQ.

\smallskip

\begin{dqtblr}{colspec={lXl}}
	& Type of Vessel	& Cost to Build  \\
31.	&  Galea Sottila	& 35,000 sp \\
32.	&  Galea Tarida	& 125  sp/ton \\
33.	&  Sagitta	& 7,000  sp \\
34.	&  Vaccheete	& 1000-4000 sp \\
35.	&  Galee di Mercanzia	& 150  sp/ton \\
36.	&  "Ordinary" Galley	& 40,000 sp \\
37.	&  Galleas	& 150,000  sp \\
38.	&  Usciere or Nef	& 125  sp/ton \\
39.	&  Medieval Med. Merchant	& 100  sp/ton \\
40.	&  Venetian Carrack	& 150  sp/ton \\
41.	&  Viking Warboat	& 6,000  sp \\
42.	&  Viking Longship	& 9,000  sp \\
43.	&  Viking "Serpent" Type	& 37,500 sp \\
44.	&  Sm. Nordic Merchant	& 60 sp/ton \\
45.	&  Lge. Nordic Merchant	& 75 sp/ton \\
46.	&  Small  Cog	& 90 sp/ton \\
47.	&  Large  Cog	& 100  sp/ton \\
48.	&  Great  Cog	& 125  sp/ton \\
49.	&  Caravel	& 100  sp/ton \\
50.	&  Carrack	& 150  sp/ton \\
\end{dqtblr}

Auxiliary Boats not otherwise detailed:

\smallskip

\begin{dqtblr}{colspec={Xr}}
Rowboat   & 500 sp \\
Longboat  & 1,500  sp \\
Canoe     & 500 sp \\
\end{dqtblr}

The building cost of a vessel includes only the hull, masts, sails, running
gear, oars, and ram, if any.  If special armaments and fittings are desired,
these must be purchased at extra cost.  The prices of some of these are listed
below, and descriptions will be available in a future supplement:

\smallskip

\begin{dqtblr}{colspec={Xr},row{1}={font=\bfseries}}
Type of Armament/Fitting  	& Cost \\
Ballista (bolt thrower)    	& 500  sp \\
Springal (spear thrower)   	& 750  sp \\
Onager (small catapult)    	& 500  sp \\
Catapult                   	& 1,000 sp \\
Greek Fire Projector       	& 10,OOO sp \\
Corvus                     	& 175  sp \\
Grappling iron \&  line     	& 10  sp \\
3' x 4' x 3" mantlet       	& 25  sp \\
\end{dqtblr}

\section{The Italians:  A.D. 1100 - 1500}

From the beginning of the Crusades to the destruction of Islamic naval
power at Lepanto in 1571.  Italian naval designs predominated in the
Mediterranean.  The great advance in naval architectu was the
"galley".  During the Dark Ages the Tialians used ships of Byzantine
design, but in the late llth century they had departed from the dromon
and were launching vessels with a single bank or oars.  The galleys
placed three rowers on the same bench at first, each man working his
own oar.  Later, a "saloccio" arrangement placed several men on the
same oar, and the sweep was made much longer.  The deck of the galley
was usually 2 to 3 feet above the waterline, over which was fixed the
talero or massive rowing frame which ran for most of the length of the
ship.  The talero carried the oars and tholepins, increasing rowing
efficiency and power in the same manner as the outriggers of earlier
oared warships.  It also increased the beam above the waterline by as
much as 8 feet, providing a fighting area of considerable size.

The rowers of Christian galleys were protected by a line of square
mantlets hung on the guardrails that ran the length of the telaro.
This practice was not imitated by the Muslims, an oversight which
exposed Moorish and Turkish rowers to murderous missile fire.  A
platform was placed between the longitudinal members of the talero
(the 'apostis) and the side of the ship, permitting the soldiers to
man the sides without interfering with the rowers.  Mechanical
artillery, then cannon, would be mounted in the bow.  The sides of
important vessels were sometimes covered with raw leather hides or
felt to protect against Greek Fire and incendiaries, such ships being
termed "in barbotte".  Finally, a spur ram replaced the submerged
prow, the intention becoming to break up the telaro of the enemy and
cripple him before boarding.

Galley crews often numbered 250 men or more,for the favorite tactic
was the boarding action.  Archery, musketry, and artillery were
employed in preparation for storming the enemy decks.  Ramming was a
method of crippling the enemy and killing his rower/fighters while the
attacker maneuvered for the best position from which to board.  The
vessel with the largest fighting crew had the advantage.  This was
especially true of Christian vessels.  The inboard oarsmen had swords
and half-pikes, while the others were archers or stone-throwers.  The
soldiers wore excellent armour and were armed with an assortment of
deadly melee weapons and crossbows or muskets.

Islamic galleys tended to be rowed by slaves and captives.  Thus both
crew efficiency and fighting stength of Moorish and Turkish galleys
tended to be inferior to those of Christian nations.  Islamic fighting
crews tended to be lightly armoured, as well, but they did not lack in
numbers of archers.  When cannon were mounted in galleys, the Islamic
nations again accepted an inferiority in weight of shot, for it was
their practice to ship 3 bow guns to the usual 5 of a Christian
vessel.

\subsection{The Galea Sotilla:  13th -- 15th Century}

Italian galley designs quickly proved popular throughout the
Mediterranean and the galea sotilla or "ordinary galley" became the
main ship of battle in most fleets.  The design is characterized by
the one man-one oar concept which predominated until the 14th century,
after which three men on a single sweep or the saloccio arrangement
was adopted.

\subsection{The Galea Tarida:  13th -- 15th Century}

The galea tarida served to carry troops, horses, munitions,
provisions, and seige engines.  Almost twice as wide as the galea
sotilla, it was significantly slower under oars and sail.  In battle,
the galea tarida was placed in the second line or reserve.  Such
vessels appear to be unique to the Christian navies, although there
may have been a few in Islamic navies as well.

\subsection{The Sagitta:  13th -- 16th Century}

The sagitta was a small galley, an auxiliary rowing vessel possessing
greater agility than the ordinary galleys of the fleet.  It was used
as a dispatch and reconnaissance vessel.

\subsection{The Vacchette:  13th -- 16th Century}

The vachette or "little cow" was an auxiliary vessel designed to
accompany t hj large ships.  Apparently, a vachette was attached to
each ordinary galley and assisted in provisioning and carrying
messages between vessels in the fleet.  It also accompanied its mother
ship into battle, using the larger bulk of the galley as cover from
artillery and missile fire.  Once battle was joined, it could launch
boarding parties against the unengaged sides of the enemy or elqe pick
up friendly troops in the water.

\subsection{Galee di Mercanzia:  13th -- 15th Century}

The merchant galleys were built for commerce but were often
requisitioned for war as they were designed to protect themselves.
Specifications for such vessels were carefully laid down by the
governments of Venice, Genoa, and Pisa, for in them the richest
cargoes were carried.  Italian practice was to form convoys of such
vessels under escort of a powerful squadron of ordinary galleys and
auxiliaries. Such merchant fleets were often quite safe against
piracy, as they could hold their own against all but a battle fleet.
The merchant galley was somewhat slow and sluggish under oars, but it
made good speed under sail.  Thus its position in battle was in the
reserve or second line.

\subsection{``Ordinary'' Galley:  15th -- 16th Century}

The ordinary galleys cruising the Mediterranean during the Renaissance
differed from the earlier versions in that they had fewer oars, their
being three men to each oar.  The ships were larger and heavier as
well.  Armaments included cannon mounted in the bow of the vessel and
large contingents of musketmen.

At the battle of Lepanto in 1571, the Christian galleys typically
mounted 5 bow guns: one 36-pdr, two 9-pdr, two 4 1/2-pdr.  The
heaviest gun, usually weighing around 3 tons, was mounted on the
centerline. The 9-pdr guns flanked the 36-pdr, and the 4 1/2-pdr guns
were placed outboard of the 9-pdr guns.  Turkish galleys mounted only
3 guns in the bow, the 4 1/2-pdr cannon being omitted.  Also,
Christian galleys mounted three 4 1/2-pdr guns on each broadside, a
practice rarely encountered in Turkish vessels.  Thus the Christians
at Lepanto had a total weight of fire of 18,450 pounds for 205
galleys, plus the 326 pounds of all-around fire from each of their six
galleasses.  The Turks had about 250 galleys, but because of the
smaller number of guns aboard each vessel, their total weight of fire
was only 13,500 pounds.

Galleys rarely fired more than twice in battle before close action was
joined, although there have been instances of such vessels standing
off and engaging in prolonged gun duels.  At the battle of Jiddah in
1517, the Muslim galleys remained under the cover of their fortress
guns and bombarded the advancing Portugese at long range.  Indeed, the
lesson of Jiddah is that the galley could operate with devastating
effect where it had access to nearby, bases whose shore batteries
could be used to augment the firepower of the fleet.

Early naval cannon tended to be no longer than 25 calibres (25 times
the diameter of the bore).  Stone shot was preferred because of the
large holes it made, but the high cost drove all nations to use iron
shot.  It is a fallacy to believe that iron shot is superior to stone.
Economics alone became the reason for abandoning stone.  Similarly,
economics argued against the continuance of the great galley fleets of
the 15th and 15th centuries, and sailed warships were adopted as an
economy measure. In the Mediterranean, with its light airs, the galley
was the logical ship of war, but far too expensive.

\subsection{The "Capitanas" and "Patronnes":  16th - 17th Century}

Although not presented in the Data Tables, the capitanas (lst
flagships) and onnes (2nd flagships) may be desired by some players.
These vessels were larger than ordinary galleys, but possessed much
the same appearance and rowing and sailing characteristics.  A typical
capitana is outlined in Chapman's Naval Architecture.  This Maltese
capitana was 179.5 feet long, 25 feet in beam, and had a draft of 8.3
feet.  A forecastle protected the bow guns and provided additional
fighting space above the main deck.  The sides were several feet
higher than in the case of ordinary galleys.  There were 30 oars on a
side, with 7 men on the aft oars and 6 on the forward oars, yielding a
total rowing crew of 390.  About 400 officers, soldiers, and sailors
completed the crew.  Displacement would have been between 400 and 500
tons, nndmaximum speed under oars would have been comparable to that
of an ordinary galley, about 6.5 knots.  In addition to the standard
forward battery, additional 9-pdr guns vould have been mounted on the
broadside as well as a number of 4 1/2-pdr or 2-pdr swivel guns.  The
patronnes were slightly smaller, with 28 or 29 oars to the side, a
rowing crew of 364 to 376, and a fighting/sailing crew of 300 to 350.

\subsection{The Galleass:  16 -- 17th Century}

The galleass had a short period of life, compared with the galley, due
to the vast costs involved in building and maintaining them.  The
tactical purpose of the galleass was to deliver a heavy weight of shot
against the enemy, but the number and weight of the guns made the
vessel sluggish under the relatively limited power of the oars. To
reflect this sluggishness, require a complete turn to accelerate to
any speed level above slow cruise.  The advantage of the galleass was
the height of its sides and the ability to fire all around.

The galleass presented in the Data Tables is based upon the model in
the Arsenel in Venice.  With 33 guns, 6 would fire astern, 11 ahead,
and 9 on each broadside.  The heaviest guns (1 × 36-pdr, 2 × 9-pdr, 2
× 4 1/2-pdr or 6-pdr) fired directly forward from under the
forecastle.  The lighter guns varied from 9-pdrs to 2-pdrs.  After
Lepanto, galleasses mounted more and heavier guns.  The "Royale" of
Louis XIV had six 36-pdr guns in the main forward battery, three
24-pdr guns firing on each broadside and many lighter guns of 9-pdr,
6-pdr, 4 1/2-pdr and 2-pdr weight of shot, giving an all-around
weignt of fire of 788 pounds!

The effectiveness of the heaviest guns against galleys is readily
seen, but even a 4 1/2-pdr had a good chance of penetrating the 3 to 4
inches of planking in most galleys.  Since galleasses tended to have
much thicker planking, they withstood small shot better and could
fulfill their tactical role by barging into the midst of a formation
of galleys to endure their fire and shatter them with their superior
gun batteries.  Also the talero of the galleass was placed higher off
the water than in the galley, and was heavily reinforced, making it
far less susceptible to ramming attacks.


\subsection{The Usciere:  13th -- 16th Century}

The usciere were large sailing vessels with ports in their round
sterns to permit the loading and unloading of horses and heavy cargo.
In addition to their role as military transports, they were also used
as floating fortresses by erecting fighting castles on the decks with
flying bridges to throw over the sea walls of fortified ports.  The
ship presented in the Data Tables was constructed by Louis IX in 1268
for his crusade.

Similar "navi" or "nefs" were constructed for commercial purposes,
both in the Mediterranean and the Atlantic, but they rarely were in
excess of 1000 to 1500 tons displacement.

\subsection{Italian Merchantman:  12th -- 16th Century}

Much smaller merchant vessels were used for the coastal trade,
although they were capable of long voyages.  These ships were found,
with local modifications, throughout the European nations along the
Mediterranean.  The illustration shows an Italian vessel of the 12th
Century.  Later vessels had a poop-deck similar to Atlantic shipping.

\subsection{Venetian Carracks:  15th -- 16th Century}

The carrack carried an exceptional quantity of sail.  It first made
its appearance in the 15th century in both the Atlantic and the
Mediterranean.  The ship presented in the Data Tables is an Italian
type which ranged from 200 to 1000 tons displacement.  Few of the
larger vessels would have been built, due to the cost.  With the
introduction of guns to naval warfare, the carrack became the logical
choice to carry such armaments.  There was, after all, little
difference between merchantmen and warships at this time, for
merchants had to protect themselves from pirates and act as
auxiliaries and transports in war@e anyway.  Besides, few nations
could afford to maintain vessels solely for making war.  The carrack
was fast, strongly-built, and capable of mounting a gun battery.  The
guns were small, however, mostly 2-pdr -and 4-pdr cannon, with perhaps
a few 6-pdr or 9-pdr guns in the largest vessels.  Carracks under 400
tons likely carried only a few guns, if any.

\end{multicols}

\begin{multicols}{3}
  
\section{The Atlantic and Baltic Regions}

The Atlantic is a hostile sea.  The Baltic is inhospitable too.  Such
waters are characterized by strong winds, high seas, and long periods
of gale and storm.  There are few good harbours and anchorages, and
many of these tended to be defended by the ships and fortifications of
men suspicious of all strangers.  Shipping was designed for extended
voyages, emphasizing seaworthiness, large cargo space for provisions
and commodities in transit, and structural strength.  Since all but
the smallest vessels find oars to be inefficient and difficult to use
in a high sea, the wind became the main source of motive power. (Of
course, the earlier vessels on the lines of the Viking sips were
excellent sea boats and used oars; but speed was obtained at the
expense of cargo space and shelter from the elements.)

Th@ typical vessel of the Atlantic and Baltic was built wider, taller,
and much sturdier than its counterparts in the Mediterranean.
Virtually all of the larger vessels were decked.  Since draft aids a
sailing ship to "bite" deep into the sea and prevent drift, and
because naval construction was necessarily heavier to provide
strength, the ships drew more water than Mediterranean shipping.

With weather and sea dictating so much of a ship's characteristics,
warships tended to resemble the merchantmen of the time, tubby, round
ships, relatively unresponsive in maneuveing but capable of surviving
the elements.  Fighting towers --- the forecastles and sterncastles
--- were erected to give archers and crossbowmen positions of
advantage when firing at enemy crews.  Because relatively few men were
needed to work the hsip under sail, warhip crews tended to be largely
composed of fighting men.  Also, since ramming is not very effective
when under sail, and backing away from a rammed ship is impossible
without oars, tactics concentrated upon laying alongside the enemy and
either raking her decks with archery fire or grappling and boarding.
Seaborne artillery tended to be conspicuous by its absence,
furthermore, until the appearance of gunpowder.  Few vessels
mounted ballistae or catapults, close combat being preferred


\subsection{The Northern Peoples (A.D. 400 -- 1200)}

Throughout the Dark Ages the cry, "The Vikings are upon you!"  chilled
the hearts of even the bravest men and provided impetus to the
development of the whole feudal system.  When the Viking movement
began, it derived its impact on Atlantic Europe less from the numbers
of the raiders than from the siaiple fact that they were the finest
warriors in Europe.  The Franks and Saxons no longer the fierce
fighters who occupied the Roman Empire, and the soldiers of Byzantium
generally preferred to sidestep a battle unless there was a good
prospect of victory.  A valorous Dane or Norwegian thought his life
well spent if he took his enemy with him to Hall.

Initially, the Viking raids were small, but they increased in size as
the Northmen became experienced in the activity.  In England, first
the Saxons gained a foothold and then came to dominate the
Britons. The Danes came later, seizing and fortifying the Island of
Thanet in the mouth of the Thames, then moving inland to establish the
Danelaw.  In France, the Vikings ranged up and down the navigable
rivers, looting and pillaging almost at will.  In both lands, they
eventuilly were granted territory or else seized it and stayed as
permanent set tlers.   Against the Viking attacks no effective reply
could be made until the victims also developed shipping, as did King
Alfred of England, to pursue them on the sea.

\subsection{The Viking Warboat:  A.D. 400 -- 1200}

The typical Viking warboat was "clinker-built", about 70 -- 80 feet
long, and carried 60 -- 70 men.  Drawing less than 3 feet of water and
possessing superb lines, such vessels were capable of speeds between
7.0 and 7.5 knots for shot spurts, with two men on each oar.  For
seaworthiness they had no equal for many centuries.  Because the
yardarm was capable of turning freely through 360' the vessel was also
capable of sailing very close to the wind, making it far less likely
to be drivin ashore by adverse winds. It was, in all, an excellent
seaboat suited to long voyages through hostile waters.

\subsection{The Viking Longship: A.D. 700 -- 1200}

The longship is nothing more than a large warboat capable of carrying
a larger crew.  Its advantage lay in its relatively high sailing
speeds and larger rowing crew.  To reflect the effect of extra crewmen
for relief at the oars, increase slow cruising times by 3 hours and
standard cruising times by 2 hours.

\subsection{``The Great Serpent'': A.D. 1000}

There were apparently a 1 imited number of large "kings ships' built
by the No@thmen from the tenth century onward. The most celebrated of
these was tht Long Serpent of King Olaf Tryggevessc The Saga of King
Olaf notes that such vessels had high poops and forecastles, and
although the description places 574 men aboard her, even correcting
for exaggeration a reasonable crew of 400 emerges. Comparable
Mediterranean vessels, the dromons of Byzantium shipped about 300.
How- ever, such vessels were few in number.

\subsection{Nordic Merchantmen:  A.D. 400 - 1200}

The Northmen were great sailors and great traders, and their vessels
ranged far afield.  The Atlantic proved to be no insurmountable
barrier, and colonies were planted in Iceland, Greenlanc and even
Vineland in America. Trade was conducted all along the Atlantic coast
and into the Mediteranean.  Smaller vessels were taken up the great
river systems of Russia and North Eastern Europe as well, all the way
to far Mikklegard (Byzantium).  Most of these vessels were small,
probably averaging 55 to 60 feet in length, with a displacement of 25
to 30 tons.  However, by 700 AD, vessels up to 90 feet long and 70
tons displacement were being constructed for longer voyages.  These
ships were apparently the model on which the nations of the Atlantic
and Baltic founded their later sailing designs, and the Small Cog (46)
betrays its Nordic lines and marks the transition.


\subsection{THE ATLANTIC AND BALTIC NATIONS (A.D. 900 - 1450)}

Under the assault of the Vikings, the natins of the Atlantic Regions
had to develop their own seapower to. counter the menace. Alfred the
Great bult a fairly strong navy and kept England safe for a time. The
peoples of the Low Countries and Northern Germany also took to the
sea.  By the end of the 22th century, the characgeristic poop deck
began to appear on some vessels.  The designers have refrained from
distinguishing between too many types, as this would have been playing
minor variations on the same tune.  The essential differences between
English aad Hanseatic zDgsar2,simply too small to consider.  Of
course, there were many vessels of only a few tons burden, but these
can be simulated by scaling down the size of the t'ypes given and
reducing speeds by 0.5 knots at slow speeds, and 1.0 to 2.0 knots at
high speeds to take into account their poorer sailing characteristics.

Atlantic vessels were quite capable of sailing in Mediterranean
waters, although they suffer a 20\% reduction in speed in light
breezes because of intrinsic design.differences and variations in sail
handling, compared to Mediterranean vessels.  Atlantic vessels were
the product of nations too poor to be able to maintain pure warship
types, so they performed the dual roles of merchantmen and ships of
war. In fact, with the incidence of chant ship hd to be a warship to
survive.  Tactics emphasized to thin enemy ranks, and finally boarding
the enemy for close castle" and "aftercastle" suggest their function
in battle as a good number of the ships even had bulwarks cut in the
shape

In Mediterranean warfare, cannon were used to "soften up" the to brush
aside smaller, annoying opposition. In the Atlantic piracy as high as
it was, a mer- maneuvering for advantage, archery combat.  The very
terms "fore- floating fortresses, and indeed of battlements.
enemy for a boarding action, or else naval tactics quickly turned to
the gun duel as a viable alternative to boarding, and deck actions
tended to occur only after a numder of broadsides had been exchanged.

\subsection{Small Cogs:  A.D. 900 - 1450}

Initially, the cog bore a striking resemblance to Viking vessels, for
they were clinker-built and imitated the general lines of the raiders.
They differed in their dependence upon wind for motive power and their
higher sides.  By A.D. 1000, they had become considerably broader in
beam than comparable Viking merchantmen, and some might have had as
much as 100 tons' displacement.  The vessel depicted as a cog of 1300
is based upon the ship in the Seal of Dover.  The beginnings of the
forecastle and sterncastle can be seen in the castellated platforms at
bow and stern.  The bowsprit is used to carry the bow lines which keep
the sail tuat when sailing cose to the wind.  Larger vessels probably
would not need the bowsprit, but a short (65-foot), tubby vessel like
the one depicted would require one to give the bow lines room to work.
Such vessels also shipped a windlass near the stern to haul o n the
halyard.  Steering is still done with the steering oar, mounted on the
starboard (steerboard) side, so maneuverability left something to be
desired.  Within 50 years, the appearance of such vessels had changed
significantly (see the illustration of the Large Cog, c. 1370) and the
rudder was often fitted at the stern.

\subsection{Large Cogs:  A.D. 1100 -- 1450}

Large cogs were built in limited numbers at first, and they resembled
the early small cogs.  By 1350, tonnages of up to 500 tons had been
reached, and the vessels had the more characteristic if modern" look,
with full decks, forecastles and sterncastles, and fairly
sophisticated rigging.  Like the smaller types, they were
exceptionally broad in the beam.

\subsection{Great Cogs:  A.D. 1200 -- 1450}

Relatively few great cogs or "nefs" were built becasue of the expense
and also the risk of financial disaster if one should be lost.
However, the merchant fleets of most nations contained a number of
these vessels, usually 500 - 600 tons in displacement, although cogs
of 1000 tons were sometimes built.  (See 38.  The UsciLere)

\subsection{The CaTaveli  A..D, 1250 -- 1600}

The first mention of "caravela" appears in a 13th century document
referring to fishing vessels.  Apparently this design proved so
successful that it was developed and enlarged so that it became a
fair-sized ship averaging 60 to 100 tons in displacement.  It enjoyed
a brief period of glory in the early period of exploration, but by
1588 it had been relegated to the status of a dispatch boat.

\subsection{The Carrack:  A.D. 1400 -- 1550}

The carrack was the first "modern" ship in all of its components. All
that happened after its appearance was a refinement in equipment, a
reduction in the size of sails (and an increase in number) to improve
handling, and superior hull designs.  It was very fast, despite the
fact that most carracks tended to be large vessels with broad beams
and deep drafts.  By Tudor times, they were fitted with small cannon,
often of the breech-loading bype, and sometimes a swivel gun was
placed in the mizzen top.  (See: 40.  Venetian Carracks)

\end{multicols}

\section{The Tables}

The following data tables give the important physical, fighting, and
sailing characteristics of the typical vessels of different nations
and time periods from about 1200 B.C. to A.D. 1500.  The individual
entries may require some explanation:

\begin{description}
\item[Type of Vessel] name of ship.

\item[Seakeeping] environment for which a ship was designed.  Most
  vessels are either Atlantic or Mediterranean ships, but a few are
  capable of sailing in the waters of either region without penalty
  and are designated as Atlantic-Mediterranean or
  Mediterranean-Atlantic types.

\item[Length at Waterline] ship's length from prow to stern at the
  waterline.  Most ships will be longer at deck level.

\item[Beam at Waterline] ship's breadth at the waterline.  Vessels may
  be wider on deck as, for instance, in a galley with an outrigger or
  any vessel in which the curve of the hull is continued outward above
  the waterline.

\item[Draft] depth of water drawn by the hull. Clearly, this is an
  averaged figure, as ships with lighter or heavier load displacements
  will draw more or less water.



\item[Displacement] weight of water displaced ' by the ship's hull,
  cargo, and crew.  This figure is much higher than the actual weight
  of the hull alone.  A Greek trireme, for instance, weighed no more
  than 67\% of its displacement: 43--50\% for the hull and up to 17\%
  for masts, sails, anchors, etc.  The remainder is for crew and
  cargo.

\item[Freeboard] height of main deck or else thegunwales of undecked from
  the waterline. If a ship is decked, add 2 - 3.5 feet to the figure
  given.  The choice of giving deck height was made to provide
  wargamers with a figure on which to compute height advantages.

\item[Lower Oars] height of lowest bank of oars on a galley from the
  waterline.

\item[Upper Deck] height of foredecks, afterdecks, fighting castles,
  etc., from the waterline.

\item[Structure] relative size of ship compared to others/structural
  strength calss.  The first number is called the "size number", the
  following letter is the "contruction type".

\item[Deck Type] class of deck, if any.

\item[Ram Type] type of bow. A bow is the usual cutting edge found on
  most vessels and is generally ineffective against all but the
  smallest boats except in unusual circumstances. A ram is a pointed,
  often armoured beak carried at or below the waterline to pierce the
  hull of enemy vessels.  A spur is a ram carried above the waterline
  to smash oars and unship the outrigger or apostis, crippling an
  oared vessel severely.  A spur can also cause serious hull damage.


\item[Crew] standard complement of men carried aboard the typical
  vessel.  The crewsizes should be kept more or less within the limits
  given, as the figures reflect actual practice.  Seriouc alterations
  in ship performance occlir when the crew numbers are significantly
  raised.

\item[Officers] Captain, Navigator, Rowing Officers, and Sailing
  Masters.

\item[Soldiers] trained fighting men, usually heavily armed and
  armoured.

\item[Seamen] members of the crew assigned to general shiphandling.

\item[Rowers] number of trained oarsmen aboard the vessel.  Depending
  on the type of ship and the period, these may be fighting men,
  freemen non-combatants (usually) or slaves.

\item[Turning Radius] track to be used on the Turn Radius Inlay when
  altering course/Turn Radius Table to be used when computing
  responsiveness in a turn at speed.

\item[Banks of Oars] number of horizontal rowing banks on oared
  vessels.

\item[Number of Oars] maximum number of oars that could be employed
  throughout the ship.  To compute the number of oars per bank, divide
  the maximum by the number of oarbanks.  To compute the number of
  oars on one side of a vessel for a particular bank, divide the total
  oars forthat rowing bank by 2.


\item[Slow Cruise] average roving speed for 3 watches (12 hours),
  given in knots/hour.

\item[Standard Cruise] average rowing speed for 2 watches (8 hours),
  given in knots/hour.

\item[Battle Cruise] average rowing speed for 4 hours, given in
   knots/hour, with full crew rowing.

\item[Raming Speed] standard fast battle speed to lay alongside an
  enemy nearby to ram or board.  Speed is given in knots/hour, but few
  crews can maintain it for more than 1/2 hour.

\item[Racing] maximum speed possible, given in knots per hour, with
  all crew members row- ing at maximum speed.  After 10 - 20 minutes,
  depending on the type of crew and ship, all rovers will collapse
  into total exhaustion.

\item[Masts] number of masts shipped.  Galleys usually have detachable
  masts in the Ancient Period.

\item[Max. Sail Area] possible maximum sail carried by the vessel.
  These estimates are based upon the containing parallelpipedon (L x W
  at waterline) and the probable maximum sail that would have been
  carried.

\item[Light Breeze] speed of ship in knots/hour in mild breezes under
  sail.

\item[Good Breeze] speed of ship in knots/hour in steady, fairly
  strong breezes.

\item[Good Wind] speed of ship in knots/hour in steady, brisk winds.

\item[Strong Wind] speed of ship in knots/hour in forceful winds.

\item[Max. Under Sail] maximum theoretical safe speed of ship in
  knots/hour.  Faster speeds will strain masts, carry gear away, and
  cause galleys to ship water unless in very calm seas.  In the
  Mediterranean, such ideal sailing conditions are obtained only close
  inshore with a land breeze (in late evening and early morning
  hours).

\item[Provisions] number of days a ship can maintain the crew on full
  rations of food and water.  Five days' provisions plus the weight of
  a man is an average of 270 pounds: Weight of man, 150 lbs.;
  clothing, 12 lbs.; arms and armour, 28 lbs; food, 3 lvs. per day;
  firewood, 2 lbs. per day; water, 1 gal. per day, weighing 11 lbs. in
  casks or earthenware.  (See Rodgers, Greek and Roman Naval Warfare,
  p. 44) Using these figures, wargamers can compute provisions
  required for longer voyages or cargoes to be carried by transports
  of the supply fleet.

\item[Cargo] probable cargo in tons that can be carried by
  merchantmen.

\item[Time of Service] dates in which the vessel appeared and was
  extensively used.

\end{description}

\end{document}
