\section{Agents (Version 1.1)}

Many of the Powers of the DQ universe use mortals as Agents.  These
Agents serve in many diverse capacities depending on the particular
aims and goals of their Power: as priestly leaders of covens or sects
dedicated to the Power; as spies; provocateurs; instruments of justice
or revenge; scourges; and recruiters.  Not all of the mortal followers
of a Power are necessarily Agents, some may simply be "lay followers",
lacking the dedication or certain skills or abilities demanded by the
Power of its Agents.

\subsection{The Pact}

The link between a Power and an Agent is in the form of a contract
whereby the Agent, a mortal sympathetic to the cause, goals and
motivations of a particular Power, agrees to further that Power's
interests in the material world and to keep the tenets of its "faith",
in return for a measure of guidance and protection, and usually for
some security in their eventual fate in the afterlife.  This is not a
contract to be entered into lightly, as it may influence the actions
and general behaviour of the Agent, and it is not easy to break such a
contract, once made, without risking the direst wrath of the renounced
Power.

\subsubsection{Becoming an Agent}
The process for becoming an Agent is solemn and ancient.  The
character must first invoke the desired Power using an Ancient
Invocation.  Should the Power choose to appear, the character must
state that they wish to become an Agent for the Power.  They must also
offer a token to the Power.  This initial interaction between the
character and the Power is very serious, and both the GM and the
character's player should carefully consider their actions.  If the
token is acceptable to the Power, the character possess those skills
or abilities required by the Power, and agrees to the Powers
strictures relating to the behaviour of its Agents, the Power accepts
them as an Agent.  If the token is unacceptable, the reaction of the
Power will depend on its personality and the reasons for the gift
being unacceptable.  Reactions may range from a sympathetic declining
of the token, and perhaps a hint as to a more acceptable one, through
to an admonition to never Invoke the Power again, on pain of death.

\subsubsection{Tokens}

The exact form that a token takes may vary widely.  The acceptability
of a token will depend on its value to the character and the
personality and motivations of the Power.  It may take the form of an
object, an entity, a physical or spiritual attribute of the character,
an oath or undertaking, or even an insubstantial "gift".  Even the
most benevolent Powers dislike mockery, and the choice of token to be
offered is often difficult.

EX. A Fire Mage wishing to become an Agent of Aim, might choose to
offer a captive Water Mage as a sacrifice, or might perhaps set fire
to a navigators guild and offer it to Aim as a "gift".

\subsubsection{Renouncing Status}

A character may renounce their status of Agent.  Renouncing Agent
status is a perilous task, only ever undertaken in extreme
circumstances.  The process of renunciation is ancient and invariable.
First the Agent Invokes their patron Power by the use of an Ancient
Invocation.  If a Minion is sent, the Agent informs the Minion that
they are renouncing their status as Agent.  The Power will then appear
in their insubstantial form, if they have not already done so.  The
Agent then states that they are renouncing their status of Agent, and
demands the return of the token they offered when they became an
Agent.  If the token was an object, or some other physical thing, the
Power must return it immediately.  If it was a service, an oath or
some other form of insubstantial offering, the Power simply
acknowledges its return.  The character immediately loses their Call
Patron talent, their Familiar or Companion, and any other abilities
they may have gained as a result of becoming an Agent.  They do not
regain any abilities that they lost due to their prohibition by the
Power, but are freed from any strictures on their behaviour.  The
Power then returns to its home dimension, and the character is no
longer an Agent of that Power.  The Power's reaction to the desertion
of an Agent will vary considerably, depending on the Power's
personality, ethics and motives.  Some Powers may direct their Agents,
cults or sects to hunt down and slay the deserter, others may seek to
punish or kill the offender personally, when next summoned to the
material world.  Still others may feel that the loss of their great
patronage is punishment in itself.  When deciding exactly what form of
action a Power takes against an ex-Agent, the GM should carefully
consider the reasons for the renunciation of Agent status, the
circumstances surrounding it, and the personality of the Power.
Needless to say, it is not possible to force an Agent to renounce
their status.  Such a control will certainly be noticed by the Power
and will invalidate the renunciation, not to mention incurring the
Power's wrath.  Particularly unwise Agents may use their Call Patron
talent to summon their Power so that they may renounce their status.
This will work, but it must be noted that when the process of
renunciation is complete, the presumably annoyed Avatar of the Power
will be physically present, and may immediately take direct action.

\subsection{Restrictions}

\subsubsection{Ancient Invocations}

Only through the use of an Ancient Invocation may a character become
an Agent.  If a character invokes a Power without the use of an
Ancient Invocation, they may not become an Agent.  The reaction of a
Power to such a mistaken Invoking will vary with the Power's
personality.  Many will direct the character to one of their present
Agents, or to a cult or sect dedicated to the Power and led by one or
more Agents, both of whom must perforce possess an Ancient Invocation.
Some Powers will require the character to go on a quest to find an
Ancient Invocation, though they may give some clue as to where to find
one. Others may be sympathetic but unhelpful, or angry, or simply
condescending as to the character's amateur behaviour.

\subsubsection{Prohibited Abilities}

An Agent may not possess abilities prohibited by its Power.  Should a
character be accepted as an Agent, they will immediately lose all
magic, Skills and abilities prohibited by the Power.  In a few rare
cases, some form of compensation will be offered by the Power, beyond
those abilities usually granted by it, but this is certainly not the
norm.  It should be noted that a state of at least cold war, exists
between many of the Powers, and the College of Summoning Magics.
Those very few of the Powers who will accept Summoners as Agents, are
so noted in their individual descriptions.  In short, if it does not
positively state that a Power will accept a Summoner as an Agent, then
they will reject them, or require that they quit that College.

\subsection{Benefits}

\subsubsection{Call Patron}

An Agent receives the ability to call their Patron. The ability
received is a special form of invocation that may only be used by
Agents.  The ability works as a Racial Talent (EM: 300), and operates
as follows:

The Agent may, by Invoking their patron Power, attempt to call their
plight to their patron's attention and gain help, guidance or
communication.  The exact nature of the help rendered is at the GM's
discretion, and may range from nothing, through to a full
manifestation of the patron's Avatar and accompanying Minions.  More
usually it will consist of the arrival of one or more Minions of the
patron, or an insubstantial manifestation of the Power itself.  The
ability takes at least one Pulse to enact, during which time the Agent
must verbally Invoke the Power, and may not engage in any other
activity that requires verbalisation.  For example, an Agent could
Call Patron whilst engaged in combat, but not while spellcasting.

The Base Chance of the Power responding in some way is their Response
percentage to an Invocation, as listed in their description, +3\% per
Rank achieved with this talent. An Ancient Invocation may be used to
increase the chance, but will require its full time and any materials.
From +20 to -20 may be added at the GM's discretion to the chance of
the Power responding, depending on the situation, the Power's
personality and the Agent's actions during the Invocation.

EX. After slaying one opponent and beginning on their second, an Agent
of Alloces uses their Call Patron talent.  Alloces is the "Warrior
Duke", and likes nothing better than bloodshed and carnage.  The GM
determines that the actions of the Agent are worth +10, increasing to
+15 once they slay their next opponent.

If the Power does not respond, the Agent may continue to Call Patron
on the next and subsequent Pulses. Each extra Pulse that they call
increases the chance of the Power responding by 1\%.  This roll is made
each Pulse until either the Power appears or the Agent ceases to call.
Once the Power responds they may require a service, item or oath of
the Agent after, or even before, they offer any aid.  This requirement
will depend on the urgency and severity of the Agent's need, how
closely they have been following the strictures concerning their
behaviour, and how often they have called for help.  Most patrons will
ask for no service, or perhaps only a token one, if the Agent has only
called as a last resort, and has in all ways been faithful, but may
levy a very severe charge if the calling is for a trivial situation or
the Agent has in any way been false to their patron.

\subsubsection{Special Abilities}

Some Powers grant various special abilities to their Agents.  These
special abilities are listed in the description of the individual
Powers, are granted to a character immediately they become an Agent,
and are immediately lost if the character renounces their pact.

\subsubsection{Familiars}

An Agent may gain a Familiar.  All of the Powers, unless stated
otherwise in their description, grant their Agents the use and
companionship of a Minor Minion to serve as a Familiar.  This Familiar
is a combination of guide, teacher, student, friend, henchling and
political officer.  The Familiar feels great loyalty to the Agent,
second only to its Power, and will serve and help them in all ways.
The Familiar begins as a standard Minor Minion of the patron Power.
Unless otherwise stated in the Power's description, the Familiar's
alternate (animal) form may be of any small non-sentient and
unenchanted creature.  The Agent may ask for a particular form for
their Familiar, but the Power makes the final decision.  The only
additional ability of the Familiar is that it knows all of the
languages known by the Agent but at one less Rank.  If this reduces
the Rank below 0, the Familiar does not have that language.  There is
a link between the Familiar and the Agent that keeps the Familiar in
the material world.  If the Agent dies, the Familiar is immediately
dispelled back to its own dimension.  If the Agent is resurrected,
they may collect their Familiar by Invoking their patron.  The Power
will then despatch the Familiar in response to the Invocation.
Depending on the circumstances surrounding the Agent's death, some
small token may have to be paid to regain the Familiar.  If the
Familiar "dies", it is dispelled back to its own dimension.  The Agent
may usually only automatically gain a new Familiar five years after
the issue of the last one, and only if their Familiar has died. Often
however, it is possible to gain a replacement Familiar in return for a
considerable service to the Power.  If the Agent wishes to have the
same Familiar back, they will usually have to perform another service,
in addition to the one required to get any new Familiar.  Much of this
depends on the past behaviour of the Agent and the circumstance
surrounding the "death" of the Familiar.

\subsubsection{Companions}

Some of the Powers, particularly the Elohim, grant a Companion to
their Agents, rather than a Familiar. Instead of accompanying the
Agent, as does a Familiar, the Companion may be summoned into the
material world by the Agent by Invoking it.  This Invocation will
always succeed, and need not be rolled for.  The Companion will take
(D-2, min. 1) Pulses to appear, and will then do the bidding of the
Agent.  The Companion will accompany the Agent if requested to do so,
or will return to their home if dismissed by the Agent.  The major
difference between a Companion and a Familiar, is that Companions
iseldom have an inconspicuous form, and will be noticed and cause a
major reaction.  There is a link between the Agent and the Companion,
and the Companion will assist and guard the Agent in all ways, putting
their concerns second only the Companion's patron Power.  If the Agent
is slain whilst the Companion is in the material world, the Companion
is immediately dispelled back to their home dimension.  If the Agent
is resurrected, they may again call forth the Companion.  If the
Companion is "slain" they return to their home dimension and may not
again be called forth by the Agent.  A new Companion can usually only
be automatically gained five years after the issue of the last one,
and only if the old Companion is "dead".  An Agent may however gain a
new Companion before the five year limit, through completing some
great service for their Power.  A newly assigned Companion is in all
ways a standard Minor Minion for the Power, except that it may only
speak languages designated by its Power, at least one of which will be
known to the Agent.  The Ranks of the languages are also designated by
the Power.  One language will usually be at Rank 8 or higher.

\subsubsection{Other Benefits}

Agents generally gain the respect of all lay worshippers of their
Patron.  This may lead to other benefits not detailed here.  Agents of
those Powers who have structured organisations, such as the Powers of
Light, may be able to gain free board and training at the places
dedicated to their Patron

EX. Michaelines, Agents of the Elohim Archangel Michael, will
certainly be able to get their weapon training free of charge at any
of Michael's Chapter houses or monasteries.

\subsubsection{Vengeance}

The death of an Agent may open a link to the patron Power.  When an
Agent dies in a manner that is irressurectable, or after the time in
which a dead Agent could have been resurrected has expired, a portal
is opened to the patron Power, so as to allow them to collect the
essential lifeforce of the Agent.  As a result of this portal, the
patron Power may send an Avatar to the material world.  Powers are, as
a rule, somewhat put out by the untimely demise of their Agents, and
may seek restitution from the Agents slayers.  If a character kills an
Agent and then vacates the area, the patron may very well appear after
100 hours, but decide against investigating.  However, standing over
the smoking remains of an Agent when their patron appears, may be
hazardous to the health.
