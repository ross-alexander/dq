\documentclass{article}
\usepackage[gyre]{dq}

\title{GM's Guild}
\author{Ross Alexander}
\date{\today}

\begin{document}

\maketitle
\tableofcontents

\pagebreak

\begin{multicols}{2}

\section{Introduction}

This kit contains all publications, modifications, rule
interpretations, forms, notes etc.\ needed to run an adventure in the
multi-GM DQ campaign.

This kit should be returned after each session for updating and
reissue.  hence kits will only be in the possession of a GM who is
presently running an adventure and will always be up-to-date as at the
beginning of that session.

Where there are spare copies of forms these may be used freely,
however please ensure that you leave at least one.  Copies may be
freely made.

To maintain time consistency try to run your adventure in chunks of
three months real time (10--12 weeks or sessions).

The First Session Checklist should be followed on the first session of
an adventure.  Please observe the Ten Commandments.  These provide a
means of ensuring some consistency between GMs, and assist with game
balance.

\section{First Session Checklist}

\subsection{Part 1 --- Player}

Interview each player and do the following.

\begin{enumerate}

\item
If you so desire, fill in the GMs character record (this is so you
have a record of each character's vital statistics during play).

\item
Fill in the Adventure record (for Librarian).

\item
Verify player's character is a Guild member and has a copy of the
House rules.  Check the tribunal's approval.  Check that experience is
properly recorded and spent.

\item
Do any character generation required.

\end{enumerate}

\subsection{Part 2 --- Character}

With the entire party do the following.  The Guild Representative will
be present to ensure that the adventure is properly set up.  The
representative will remind the party of the House rules, suggest that
a leader be appointed, and demand that a scribe be appointed.

\begin{itemize}

\item
Introductions, descriptions, etc.

\item
Appoint scribe and give him/her the Adventure record you started in
part 1.  Appoint leader if so desired.

\item
Introduce the adventure.

\item
Preparation and departure.

\end{itemize}

\section{House Rules}

The Guild does not condone activities such as killing, theft,
extortion etc.  These activities (and others) are illegal in most
cultures we interact with.  Any crimes committed against society will
be answerable by the members concerned to that society.  The Guild
will not stand in the way of the normal course of justice.

You do not have to like, or even wish to adventure with, your fellow
guild members, but it is expected that you will avoid attacking,
deserting, endangering, injuring, stealing from, or withholding
treasure from your fellow guild members, and to avoid provoking other
members into performing such activities.  A breach of the standards of
behaviour expected between guild members will result in a full
investigation using all the powers and abilities of the guild.  The
matter will be dealt with severely!

\section{The Ten Commandments}

\begin{enumerate}

\item
Experience shell be awarded according to the described in the document
entitled ``Experience Point Awards.''

\item
Thou shalt observe the Laws of Shaping, which are:
\begin{enumerate}
\item
Law of Balance \\
The only shaped items that may be without flaws are those of minor
nature, for example, amulets, +5 +1 swords, etc.  Flaws cannot
be removed without also removing the useful abilities.  The more
powerful the magic, the worse will be the flaws.  The difference
between the powers and the flaws will represent the value (see
commandment 3).

\item
Law of Non-indestructibility \\
No shaped item is indestructible --- there will always be a way to
destroy it.  If any part of a shaped item is broken the magic is lost.

\item
No item will work always under all circumstances.  This is not a flaw
--- just a hole.

\end{enumerate}

\item
The absolute maximum value of any shaping is 50,000sp.

\item
The maximum net treasure that will be allowed per adventure is
dependent on the experience level: 1000sp for low level, 2000sp for
medium, and 3000sp for high level.  This is per session (approximately
4 hours of play time).

\item
The Guild will loan money to a member to buy back items in the
treasure split if the adventurer has insufficient funds.  Earlier
debts must be settled before further debts may be incurred in another
treasure split.  Note --- the Guild will loan only as much as is
required.

\item
Thou shalt avoid giving extra ranks or characteristics.  If you must,
then the maximums shall be as follows: characteristics 1, weapons 8
weeks, spells 1000ep.

\item
Time on all planes effectively passes as the same rate.

\item
Items, if of a post-renaissance technology, will either not work or
decay into uselessness. All non-technological items must be converted
into DQ terms so as not to destroy the flavour of the game.

\item
Thou shalt turn away characters if they will unbalance a party (reason
should prevail however).

\item
Thou shalt always adjudicate for the enjoyment of the majority.  Think
about this when rewarding individuals.  Adventuring with superman is
boring!

\end{enumerate}

\newpage

\section{Experience Points Awards}


Experience points are a measure of a character's ability to learn or
improve Statistics, Skills and Magic.  Experience is awarded to a
player's character based on the following four categories. Note that a
Session is considered to be 3.5 to 4 hours of real time play and
Sessions of greater or lesser duration should have their Experience
Awards adjusted accordingly.

\subsection{Attendance and Preparation}

\subsubsection{POTENTIAL AWARD: 0-500 EP / Session}

This category is awarded for turning up to game sessions on time and
prepared to play.  The full award should be given to characters whose
players turn up on time with character sheets completed and the
necessary accessories for play.  Courtesy by a player in informing the
Game-master when they may be late or unable to play due to other
commitments should result in some award.  Beginning players should be
given leeway in respect of not having complete character sheets.  A
player who is constantly distracted or distracts others when they are
trying to play is not properly attending should be given a reduced
award.

\subsubsection{GUIDELINES}

Full award for turning up on time and prepared to play.  75\% award
for incomplete character sheet.  50\% award for no character sheet or
unable to attend but supplying character sheet.  25\% award for
expected non-attendance without supplying character sheet.  No award
for unexpected non-attendance.  Lateness should result in an
appropriate adjustment to the entire experience award for the session
dependent on the degree of lateness.  (i.e.\ arriving 1 hour late
should result in 75\% of the total `normal' award for that session).

\subsection{Role-playing}

\subsubsection{POTENTIAL AWARD:  0-500 EP / Session}

This category is awarded for good role-playing, both characterisation
and player's enjoyment.  It is possible to play an obnoxious or
annoying character well but in such a fashion that other players
`enjoy' the character. This category is totally up to the Game-master's
judgement but should be based on consistency of motive and action and
enjoyment by the players and Game-master.  Take into account how well
the player differentiates between being in and out of character and
stays in-character when they should be.

\subsubsection{GUIDELINES}

Full award for believable and consistent characterisation which was
enjoyable by most present.  Half award for adequate role-playing or
good role-playing that annoyed players.  No award when a block of wood
could have done better or they managed to seriously and continually
annoy the Game-master or most the players.

\subsection{Contribution}

\subsubsection{POTENTIAL AWARD: 0-500 EP / Session}

This category is awarded for taking part in the action of the session,
either by providing ideas (or non-ideas) or by appropriate or
innovative use of the abilities or possessions of the character.  How
much did the player and / or their character contribute to the game
compared with how much they were able to contribute.

\subsubsection{GUIDELINES}

Full award for consistently and interestingly being a major
contributor to the action when possible and refraining from
contributing when unable to do so in-character.  Half award for
contributing occasionally when able to and also occasionally when
unable to.  No award for consistently contributing when unable to and
hindering other player's or character's appropriate contributions.

\subsection{Quest Level}

\subsubsection{POTENTIAL AWARD: 0-1500 EP / Session}

This category is awarded based solely on the difficulty and complexity
of the quest.  The adjudged level of difficulty and risk should be
announced at the beginning of the quest and the Game-master should
endeavour to abide by this level.  This means not making things harder
when the party is able to overpower the quest and also not being easy
on younger and less experienced characters when the quest seems too
tough for them.

\subsubsection{QUEST LEVELS:}
\begin{description}

\item[0 EP Very Low Level]
The characters will be confronted with little personal risk, with death
unlikely, low ability opponents, e.g.\  animals, low experience NPC's.
Problems will have readily obtainable solutions with little, low Rank
magic involved.

\item[300 EP Low Level] The characters will be confronted with some
personal risk with opponents of some ability, e.g.\ humanoid monsters,
NPC's with some experience.  Problems will have obtainable solutions
with some low Rank magic and a little, medium Rank magic involved.

\item[600 EP Medium Level] The characters will be confronted with
definite personal risk, and possible death, with opponents of equal
ability, e.g.\ lesser undead, monsters, experienced NPC's.  Problems
will require thought and include magic, though little magic above
medium Rank.

\item[900 EP High Level] The characters will be confronted with high
personal risk, including the likelihood of death, with tough
opponents, e.g.\  greater undead, devils, very experienced mages.
Problems will be difficult and involve much magic and/or ingenuity,
but not entirely high Rank magic.

\item[1200 EP Very High Level] The characters will be confronted with
death, possibly irresurrectable, with very tough opponents, e.g.\
powerful groups, individual dragons or Powers.  Problems will require
thought, ingenuity and magic to solve and will involve any (and
probably all) magic.

\item[1500 EP Extreme Level (Certain Death or World-Saving)] The
characters will be confronted with almost certain death, probably
irresurrectable, with virtually unbeatable opponents, e.g.\ groups of
Powers, Empires.  Problems will be all but unsolvable and will involve
any and all magic plus stuff not readily possible within the rules.
\end{description}

\end{multicols} 

\pagebreak

\section{Light and Dark Aspect}

\begin{multicols}{3}

The College of Celestial Magics is an elemental College.  They are
involved in the study and manipulation of the fifth and sixth Elements
: Light; and Darkness (there is a philosophical disagreement as to
whether Darkness is in fact a true element, or is just the absence of
Elemental Light; but that argument is outside the scope of is
discussion).

This Elemental connection understood, it is very easy to see that the
so-called ``Dark and Light alignment'' is no more than a form of
elemental aspect, in much the same way that someone might have the
aspect ``Autumn Air'' or ``Summer Fire''.  To avoid the confusion that
arises from the use of the term ``alignment'' (definition: place in
relation of agreement or alliance with others), the word aspect (in
it's astrological sense) should be used.

There is no direct connection between the possession of a Light or
Dark aspect and the self-styled ``Powers of Light and Darkness''.
Persons of Light aspect are not necessarily ``good'' nor persons of Dark
aspect ``evil''.  The aspect refers only to the persons position with
regard to the Elements of Light and Dark.  It is possible that
uneducated plebs may confuse darkness and evil, as night time may be
associated with strange doings and the unknown.

Crepuscular (i.e., of twilight; appearing or active in twilight)
should be considered Dark aspected in the same way as Celestial Shadow
Weavers.

Should a character or NPC fall into one or more of the following
categories, they are considered \emph{Dark} aspected.
\begin{itemize}
\item Nocturnal or Crepuscular creatures
\item Those with Lunar aspects (including Shapechangers)
\item The following Adepts:
\begin{itemize}
\item Celestial Dark Mages
\item Celestial Shadow Weavers
\end{itemize}
\end{itemize}

Should a character or NPC fall into one or more of the following
categories, they are considered \emph{Light} aspected.
\begin{itemize}
\item Celestial Star Mages
\item Celestial Solar Mages
\end{itemize}

The order of precedence among these categories, from lowest to highest
is:
\begin{itemize}
\item Creature
\item Aspect
\item College
\end{itemize}

For example, a human (light), who is Lunar aspected (dark), becomes a
Star Mage (light) and are thus light aspected.  Should they lose their
college, they would become dark aspected again.

\subsection{Monsters --- Classification for Light and Dark}

\subsubsection{Demons and begins of Mana}

Demons and the other beings of mana have no direction connection with
the Elements of Light and Darkness, unless they are Celestial Mages.

\subsubsection{Common Land Mammals}
\begin{itemize}

\item Apes, Pre-humans \& Humans
\begin{itemize}
\item Baboon (L)
\item Gorilla (L)
\item Neanderthal (L)
\item Orang-Outang (L)
\item Sasquatch (L)
\item Humans (L)
\end{itemize}

\item Felines
\begin{itemize}
\item Cheetah (D)
\item House cat (D)
\item Leopard (D)
\item Lion (D)
\item Sabretooth tiger (D)
\item Tiger (D)
\item Wild cat (D)
\end{itemize}

\item Great Land Mammals
\begin{itemize}
\item Bear (L)
\item Boar (L)
\item Camel (L)
\item Elephant (L)
\item Ox (L)
\item Stag (L)
\item Wooly Mammoth (L)
\end{itemize}

\item{Small Land Mammals}
\begin{itemize}
\item Dingo (D)
\item Hyena (D)
\item Jackal (D)
\item Mongoose (D)
\item Rat (D)
\item Weasel (D)
\item Wolf (D)
\end{itemize}
\end{itemize}

\subsubsection{Avians}

\begin{itemize}
\item Common Avians
\begin{itemize}
\item Buzzard (L)
\item Eagle (L)
\item Goshawk (L)
\item Owl (D)
\end{itemize}

\item Fantastical Avians
\begin{itemize}
\item Gargoyle (D)
\item Gryphon (L)
\item Harpy (D)
\item Hippogriff (L)
\item Pegasus (L)
\item Phoenix (L)
\item Roc (L)
\end{itemize}
\end{itemize}

\subsubsection{Aquatics}

\begin{itemize}
\item Fish
\begin{itemize}
\item Barracuda (L)
\item Manta Ray (L)
\item Pike (L)
\item Piranha (L)
\item Shark (L)
\end{itemize}

\item Aquatic Mammals
\begin{itemize}
\item Dolphin (L)
\item Great White Whale (L)
\item Killer Whale (L)
\item Mer-folk (L)
\end{itemize}

\item Others
\begin{itemize}
\item Eel (D)
\item Kraken (L)
\item Octopus (L)
\item Squid (L)
\end{itemize}
\end{itemize}

\subsubsection{Lizards, Snakes \& Insects}

\begin{itemize}

\item Lizards and Kindred
\begin{itemize}
\item Basilisk (L)
\item Crocodile (L)
\item Giant Land Turtle (L)
\item Hydra (L)
\item Land Iguana (L)
\item Salamander (L)
\item Suarime (L)
\item Wyvern (L)
\end{itemize}

\item Snakes
\begin{itemize}
\item Asp (D)
\item King Cobra (D)
\item Mamba (D)
\item Python (D)
\item Spitting Naga (D)
\end{itemize}

\item Insects and Spiders
\begin{itemize}
\item Spider (D)
\item Fire Ant (D)
\item Killer Bee (L)
\item Scorpion (D)
\end{itemize}
\end{itemize}

\subsubsection{Giants, Fairies \& Earth dwellers}

\begin{itemize}
\item Giants
\begin{itemize}
\item All Giant subspecies (L)
\item Ogres (L)
\item Titan (L)
\item Troll (D)
\end{itemize}

\item Fairy Folk
\begin{itemize}
\item Brownie (L)
\item Dryad (L)
\item Elf (L)
\item Fossergrim (L)
\item Leprechaun (L)
\item Nixie (L)
\item Nymph (L)
\item Pixie (L)
\item Satyr (L)
\item Sylph (L)
\end{itemize}

\item Earth Dwellers
\begin{itemize}
\item Dwarf (D)
\item Gnoll (D)
\item Gnome (D)
\item Goblin (D)
\item Halfling (D)
\item Hobgoblin (D)
\item Kobold (D)
\item Orc (D)
\end{itemize}
\end{itemize}

\subsubsection{Fantastical Monsters}

\begin{itemize}
\item Centaur (L)
\item Chimaera (L)
\item Giant Amoeba (D)
\item Gorgon (L)
\item Manticore (L)
\item Naga (L)
\item Sphinx (L)
\item Unicorn (L)
\end{itemize}

\subsubsection{Creatures of Night \& Shadow}

\begin{itemize}
\item Bat (D)
\item Dire Wolf (D)
\item Doppelganger (D)
\item Weres (D)
\end{itemize}

\subsubsection{Summonables}

\begin{itemize}
\item Djinn (*)
\item Efreet (*)
\item Elementals (*)
\item Hellhound (D)
\end{itemize}

* These creatures, being elemental in nature themselves, are solely
tied to their particular element and are in no way connected with
elemental Light or Darkness.

\subsubsection{Undead}

\begin{itemize}

\item Lesser Undead
\begin{itemize}
\item Ghost (D)
\item Ghoul (D)
\item Revenant (D)
\item Skeleton (aspect of their Animator)
\item Zombie (aspect of their Animator)
\end{itemize}

\item Greater Undead
\begin{itemize}
\item Night-Gaunt (D)
\item Spectre (D)
\item Vampire (D)
\item Wight (D)
\item Wraith (D)
\end{itemize}
\end{itemize}

\subsubsection{Dragons}

\begin{itemize}
\item Black (D)
\item Blue (D)
\item Golden (L)
\item Green (D)
\item Red (D)
\item Yellow (D)
\end{itemize}

\subsubsection{Riding Animals}

\begin{itemize}
\item Donkey (L)
\item Draft Horse (L)
\item Mule (L)
\item Mustang (L)
\item Palfrey (L)
\item Pony (L)
\item Quarter horse (L)
\item War horse (L)
\end{itemize}

\end{multicols}


\begin{multicols}{2}

\setcounter{secnumdepth}{3}

\section{The Powers (Version 2.0)}

\subsection{Overview}

This document is not a list of the various Gods, Demons and other
Powers within the DQ universe. Rather it contains information for GMs
on the nature of Powers, their impact and interactions with the game
world, and their abilities and limitations. It includes a design
system for minions, those lesser beings that the Great Powers often
send out to do their bidding, and also information on the Powers'
mortal Agents and the pacts that bind the two together.

\subsection{Background}

The DQ universe, and most of the things in it, was created through the
intervention of various Great Powers, commonly known as Gods, or
Celestials. These beings still exist into the present age, and though
their ancient Covenant prevents, or at least reduces, their direct
meddling in the affairs of mortals, they remain a powerful force,
whose presence may be felt in day to day life, through the works and
deeds of their mortal worshipers, followers and Agents.

\subsection{Types of Great Power}

\subsubsection{Gods}

Gods are the greatest of the immortal beings and most have existed
from the very beginning of the present universe, and predate all
mortals by an immeasurable time. They are in many ways similar to
their lesser cousins the Demons, with two important differences.
Firstly, Gods very seldom send Avatars into the material worlds,
preferring instead to work through more subtle means such as visions
and prophecies. Some do use mortal Agents, but very seldom offer much
in the way of direct aid. Some even have servant Demons who do their
more direct bidding. Secondly, the power of Gods is in some way
influenced by the quantity and faith of their mortal worshipers,
although little is known about this process even by the most learned
sage, and unsurprisingly scant information is forthcoming from the
Gods themselves.

Some Gods are concerned with particular animals, objects or concepts,
and have few interests outside of these, whilst others have more
wide-ranging goals and ambitions. The Gods vary widely in their
natures, ranging from concepts and stances that mortals would call
ethical and good, through to the blackest evil.  There are various
hierarchies and groupings (often referred to as pantheons), amongst
the Gods, but usually each employs quite different minions.

Following a devastating war amongst the Gods when the universe was
newly formed - a war that saw the destruction of many material planes,
and Gods alike - the remaining Celestials made a solemn covenant that
they would never again make war on one another directly.  Respect for
the Covenant is another issue on which the Gods differ, some will
never willingly disobey it, while others will do everything except
defy it openly.

GM Note: GMs should exercise caution before introducing whole new
pantheons into Alusia. Two pseudo-historic pantheons are already known
to exist; a Norse one, and a Celtic one. These have been used by GMs
as the ``Old Gods'' of the Dwarves and Elves respectively.


\subsubsection{Demons}

Since the creation of the material universe, various lesser Powers,
usually called Demons have come into being. These entities, unlike
Gods, were once mortal creatures, who have managed to continue their
personal existence beyond death to become an immortal. It is believed
that some Demons have gone on to become ``true'' Gods, although almost
nothing is known about how this might be accomplished. Demons exist on
the spirit-plane adjacent to a material world or a group of such
worlds. They tend to take a much greater interest in the worlds that
they lived in when mortal, and summoning them is possible if the
correct rituals are known.

The earliest Alusian Demons date from the time of the first mortal
race, the Dragons, but the majority appeared at the time of the fall
of the Elven Empire. There are two major factions amongst this group
of Demons, and they continue beyond death, the struggle they started
in life.

\subsubsection{The Powers of Darkness}

This large faction of Demons are those immortals that were once the
most potent and dire of the Drow mages, or are a ``chimera'' of several
lesser mages, whose life essences have fused together. The Powers of
Darkness (PoD) are all generally chaotic in outlook and, by mortal
standards, unethical in nature. Many have retained the titles of their
temporal power: Duke, Prince, etc., so whilst they appear to have a
hierarchy, it is mostly a convention dating from their time as
mortals. The minions of these Demons are well known to mortals, and
include: Succubi/Incubi, Devils, Imps and Hellhounds. The Powers of
Darkness pursue their various goals in the material world through the
use of mortal Agents, although it is not unknown for them interfere
directly.

\subsubsection{The Elohim}

Also known as the ``Powers of Light''. The Elohim were once a mortal
group, convened in the twilight days of the Elven Empire, in an
attempt to reverse the decadent trends that ultimately destroyed elven
civilisation. Their apotheosis was a planned and deliberate act, so
that they might stand against the Powers of Darkness, as some form of
protection for mortals in later ages of the world. The Elohim obey a
strict hierarchical order. The highest are the Archangels, then the
Bene Elim, Malakin and Aishim, and finally the servant Elohim, the
Erelim, Kerubim and Seraphim. The Elohim, being opposed to the highly
magical drow Demons, and seeing the excessive use of magic as the
prime cause for their ancient fall, will seldom use Mages as
Agents. Their Agents are often known as Clerics or Priests, and it is
generally through them that the Elohim pursue their goals, although
they may occasionally directly ``help'' mortals, for their own good.

\subsubsection{Other Demons}

Other mortal beings as well have undergone the apotheosis into a Demon
upon their death. The most notable are the ``Emperor Demons'', great
dragons from the first age of the world, whose power was subsequently
broken by the upstart demonic drow, and their spirits imprisoned in
material worlds. It is still possible today for a strong enough mortal
(or a group of like-minded ones) to carry their personal existence
beyond death, to become a new Demon.

\subsection{Avatars}

The differences outlined aside, the Powers are reasonably equivalent
in their attributes, and abilities. Whilst Gods certainly have move
power and a wider influence than Demons, they are limited by their
covenant in the amount of raw power that they may manifest in the
material worlds. Each Power has a sphere of influence, goals and
motivations, a form or forms that they commonly appear in, a code of
ethics that they expect their followers to adhere to, along with a
number of common physical properties.

\subsubsection{Manifestation}

When a Power manifests in the material world, the manifestation is
only a small portion of its total being. This manifestation, or Avatar
cannot be killed, though it may be disrupted sufficiently so as to
cause it to dissipate back to the home dimension of the Power. Each
Avatar created by a Power may have slightly different portions of the
Power inherent in it, thus giving it somewhat different attributes and
abilities. A Power may have only one Avatar manifest in any material
world at any one time.  Avatars may be summoned to the material world
by a variety of Spells or Rituals, or by the Call Patron talent of an
Agent.  A Power may not send an Avatar to the material world without a
summoning or some other form of "gateway" or "portal" being available.
In some circumstances, a Power may cause minions to manifest in the
material world without such a "portal".

\subsubsection{Appearance}

A Power has a number of options regarding the appearance of its
Avatar.  An Avatar may appear in any of the forms used by the Power,
and with or without any of all of the accoutrements associated with
the Power.  In addition, the Avatar may be accompanied by a number of
Beast minions (usually no more than 5).  These will be of the type
commonly used by the Power.

\subsubsection{Death \& Damage}

The damage and "death" of an Avatar is handled differently to that of
a mortal.  Avatars may only be harmed by magic, weapons of magical
nature, and by silvern metals.  When an Avatar has taken sufficient
damage that it would have been slain, or made unconcious or insensate,
had it been mortal, it is dissipated back to its home dimension. Most
Avatars cannot be Stunned. If an Avatar is dissipated due to damage
taken, or by having the spell that summoned it to the material world
dispelled, counterspelled or dissipated, the Power that formed it may
not usually send another Avatar to that material world for a period of
about thirty days.  If the Avatar is sent back by its summoner, it may
not reappear in the material world until the next day.

\subsubsection{Magical Abilities}
The magical abilities of an Avatar are different to those of mortals.
Avatars that use magic are not restricted to the standard College
system, but rather possess magic which is consistent with their
Sphere.  An Avatar may possess any number of Talents, Spells and
Rituals, from any number of Colleges.  The magic that a Power
possesses should be consistent with their Sphere.  The individual
Talents, Spells and Rituals that a Power possesses may be considered
as Colleged for the purposes of Counterspells, Protections, and the
like.  When casting Talents, Spells and Rituals, the Avatar uses the
Base Chance modifiers of the College to which the ability belongs.

All Avatars have this lack of restriction on their magical abilities,
though some possess far more magic than others.  This difference is
determined by the Power's Sphere.  The Avatars of scholarly or
mystical Powers will possess much magic, while those of more
physically orientated Powers will have less.

\subsubsection{Cold Iron}
Avatars are not prevented from exercising their magical abilities by
the presence of Cold Iron.

\subsubsection{Ranks}
The magical abilities of an Avatar are usually practised at Rank 20,
although the Avatars of particularly physical Powers may practise
their magic at lower Ranks.

\subsubsection{Physical Forms}
An Avatar may use its magical abilities regardless of the physical
attributes of its form.  An Avatar's form need not be capable of
vocalisation, or of complex manual manipulation for the Avatar to
exercise its magical abilities.  However, if the form can vocalise
and/or complete intricate hand manoeuvres, it must do so to cast its
magic.  It should be noted that in most cases the outward form of a
Power is entirely cosmetic, and that while an animal of the type
portrayed may not be possessed of the correct organs required for
vocalisation, this muteness does not apply to a Power in that form
unless explicitly so stated under the description of the individual
Power.

\begin{example}
Samigina, "Marquis of Dead Souls", may choose to appear as a small ass
or as a human.  Whilst in ass form he may cast his magic without the
use of hand gestures, if however, he takes on human form, he may not
choose to omit the hand gestures while casting.  Whilst in ass form he
speaks with a braying voice, despite the fact that asses, as a rule,
are incapable of speech.
\end{example}

\subsubsection{Changing Form}
Avatars may change forms as a Pass action.

\subsubsection{Weapon Skills}
Avatars will always possess (maximum Rank + 1) with any weapons that
are part of their Power's accoutrements. An Avatar will commonly
possess maximum Rank in up to 5 other weapons and high ranks in up to
10 more. There are exceptions to this rule, and some particularly
war-like Powers will manifest Avatars who possess maximum Rank (or
better) in almost any known form of weaponry.

\subsubsection{Languages}
Avatars will usually possess maximum Rank in many languages (at least
one of which will be known by their summoner), and will sometimes
possess maximum Rank in any other languages in the same language group
or groups.  The Avatars of particularly scholarly Powers may possess
more languages than this, some possessing all known languages and
dialects at maximum Rank.

\subsubsection{Skills}
Avatars will generally possess all the skills within the sphere of
their Power at Rank 10, or in some cases, at even greater Rank. If the
Power that the Avatar represents is particularly tied to a skill, the
Avatar may possess as high as Rank 15 with that skill. The exact uses
of certain skills at above Rank 10 are left to the discretion of the
individual GM.

\subsubsection{Special Abilities}
Many Avatars are possessed of special talents or abilities that are
connected to their Power's sphere, appearance or other attributes.
Avatars may be horrendously ugly, inspiring fear in their viewers, or
may dazzle the eye with their power or beauty.  They may have special
teaching abilities that violate the standard rules, or they be able to
bestow abilities on their followers or Agents.  All of the special
abilities available to an Avatar must be detailed under the individual
Power.

\subsubsection{Minions}
A Power possesses a legion of lesser beings. The legions of a Power,
(i.e. their supernatural servants and followers) can generally be
divided into four categories: Greater minions, Lesser minions, Minor
minions, and Beasts.  In addition to these four types of supernatural
follower a Power may have one or more types of natural Animal minion
associated with it. Both the Powers of Darkness and the Elohim have
"standardised" minions; the PoD have Incubi/Succubi, Devils and Imps;
the Elohim have Erelim, Kerubim and Seraphim. The Demons commonly use
Hellhounds, whilst the Elohim often use sentient Pegasi. The
individual Demons employ a variety of natural animals, the exact type
being dependant on their personal "sphere".  The differences between
the minions of the PoD and the Elohim, and those of the Celestials, or
other Demons, who have no "standards", are less than they might at
first seem.  Whilst there are obvious cosmetic differences between an
Imp and a Seraph, the amount of power that each wields is roughly
equal. Their attributes are within similar ranges, and the only real
difference is in their appearance and special abilities.  Imps are
tremendously ugly, possessed of a poisonous attack with their tail,
and cause fear, whilst Seraphim are inhumanly beautiful and gain some
measure of defence due to their dazzling aura.

\subsubsection{Format}
The format in which a Power is described is standard.

\begin{Description}
\item[Name] Common Name of the Power
\item[Title] Common Title or Sobriquet
\item[Type] God or Demon
\item[Group] The group or faction to which this power belongs: Elohim, Powers of Darkness, Elven gods, other?
\item[Response chance] The percentage chances that the Power will respond to an invocation / respond to any use of their name (only if not 0\%).
\item[Description] In what forms does the Power commonly appear? What accoutrements does the Power appear with?  What affect does the Power have, merely by its presence?  List: Species, race, sex, colour, clothes, weapons, companions, language, accent, what does the Power look like, sound like, smell like, emotional effects; fear, lust, anger, depression, etc.
\item[Sphere] To what area of power is it linked? Where do its aims and goals lie?
\item[Personality] What is the Power like ethically, morally, what is its position on: truth, justice, theft, murder, good, evil, life, death, love, hate, fun, pain, etc.
\item[Greater minions] What type/form/look do the Power's Greater minions take?  (Is this a standard variety, or is there a variation from the standard, or is it a special type of minion specific to this Power?)
\item[Lesser minions] As for Greater minions.
\item[Minor minions] As for Greater minions.
\item[Beast minions] As for Greater minions.
\item[Animal minions] What animals, if any, are closely associated with the Power.
\item[Agents] What Skills, Colleges, Weapon abilities, etc. does the Power insist/prohibit its Agents from having?  How must the Power's agents behave/not behave? what ethics/morality is expected of them?  What must they attempt to do/not do?  How does the Power expect its Agents to serve it?  Does the Power grant Familiars or Companions?  Do these always have some particular form or ability?
\item[Areas] What geographical areas/economic types/political arenas is the Power most/least likely to be found in?
\item[Colours \& symbols] are there any colours, symbols, calls, songs, etc. associated with/prohibited by the Power?
\item[Avatar abilities] Does the Power's Avatar commonly possess some special ability?  What statistic values, or range of values, are associated with the Avatar? Strength, Dexterity, Agility, Magical Aptitude, Willpower, Endurance, Fatigue, Perception, Physical Beauty, Movement Rates, Natural Armour? What natural weapons and attack forms does the Avatar possess?
\item[Avatar summoning] Can the Avatar only be/not be summoned into a particular type of area?  Must some particular type of "sacrifice" be made?
\end{Description}

\begin{example}
\begin{Description}
\item[Name] Aim: "The Fire Duke"

\item[Type] Demon

\item[Group] Powers of Darkness

\item[Response] 15\%

\item[Description] Aim always appears as a man with three heads dressed
in crimson robes with the design of flames burning up from the hem.
His central head is human, the left one that of a serpent, and the
right is that of a calf.  He bears two stars of the forehead of his
human head.  In his left hand he carries a ball of eternally blazing
fire.  He rides a huge lizard with scales of midnight blue.  Wherever
he goes, Aim is surrounded by billowing clouds of sulphurous, red
tinged smoke.  All three heads may speak, the human one with manic
laughter, the serpent with sibilant menace and the calf with a
mournful lowing.  Aim never appears with or wears armour, nor does he
appear with weapons.  Whilst he will use small one-handed weapons, he
is not primarily a fighter.

\item[Sphere] Aim's sole interest is in Fire and he possesses all magic
related to it.  He will gladly begin a blazing inferno merely for the
pleasure of watching it burn, and believes that one day the entire
universe will be consumed in a conflagration of cosmic proportions.

\item[Personality] Aim is almost entirely insensitive to the desires of
mortals, wishing only to see his beloved fire propagated.  To Aim,
fire is the basis of all things, "creation through destruction",
whilst water is the slayer of life.  Aim will immolate Adepts of the
College of Waters Magics without a moments hesitation.  Otherwise, his
attitude to mortals is one of complete disinterest, except for his
Agents, in whose fiery ambitions he delights.

\item[Greater minions] Sentient Fire Elementals (equivalent to those
summoned at Rank 20, except with an MA of 0, otherwise standard).

\item[Lesser minions] Devils (only ever of the Fire College, otherwise standard).

\item[Minor minions] Imps (never Water College, otherwise standard).

\item[Beast minions] Hellhounds (standard).

\item[Animal minions] Salamanders (standard).

\item[Agents] Aim will only accept Adepts of the College of Fire Magics
as Agents.  His Agents must create fire wherever and whenever possible
and are prohibited from extinguishing a fire for any reason.  They
must never associated with Adepts of the College of Water Magics, and
are encouraged to slay them whenever possible.  Upon becoming a Agent
of Aim the mage immediately loses the Extinguish Fire Spell (along
with any Ranks they may have had in it), and are forbidden from ever
relearning it.  In addition the Agent immediately gains great insight
into the workings of the Bolt of Fire Spell.  This spell will
thereafter have its EM lowered by 100, and become General Knowledge,
for them alone. Should the Agent lose the spell for some reason, they
may always relearn it with its lower EM. Aim grants his Agents
Familiars which are always Salamanders in their animal form.

\item[Areas] Cults of Aim worshippers are almost invariably found in
large cities.  They go under various "fire" related names such as "The
Children of the Flame" or the "Red Redemption".  The cults have been
known to offer to redeem the souls of the proprietors and customers of
gambling and bawdy houses for large cash sums.  Many choose to pay, as
it is not unknown for those that do not to have their establishment
"cleansed by fire". Most civil authorities are strongly opposed to the
activities of the cults of Aim and will take measures to eradicate
them.

\item[Colours \& symbols] Agents of Aim dress as Fire Mages of the most
extreme sort, in blazing crimsons and oranges, more often than not
decorated with depictions of roaring fire.  Both the symbol of a ball
of fire, and the slogan "creation through destruction" are common.

\item[avatar abilities] Aim is a master of Fire magic and a potent
Alchemist.  He is also a skilled Military Scientist. He can
automatically set fire to any combustible object merely by touching it
with the ball of fire in his left hand.  The ball may not be thrown.
He is neither a great scholar nor a fighter.  Aim is extremely ugly
and can inspire fear in those that view him. His lizard is in all
ways, except colour, a giant Salamander

\item[Movement Rate] (yards per minute): Run: 250

\begin{tabular}{llll}
PS: 22	& MD: 24	& AG: 23	& MA: 30 \\
EN: 25	& FT: 35	& WP: 34	& PC: 26 \\
PB: 3 	& TMR: 5	& NA: 3 DP \\
\end{tabular}

\item[Weapons] Aim may bite in Close Combat or Melee with his non-human
heads. If the serpent head does effective, the target also suffers an
additional D-2 damage per Pulse, for D10 Pulses, from poison.  The
ball of fire causes D+8 damage per pulse that a target is in contact
with it. The damage done by the ball is magical fire damage and may be
resisted, for half damage

\item[Avatar summoning] Aim may only be summoned into an area where
fire could burn, and if there is not a fire burning in the vicinity he
will insist that one is lit immediately, or will create one himself.
\end{Description}
\end{example}


\subsection{Minions}

Most supernatural minions are, in effect, tiny fragments of their
Patron power. The way in which minions work has been standardised for
ease of use by GMs.

\subsubsection{Classes}

The classes into which a Power's minions are divided are standard.
The standard classes for minions are: Greater, Lesser, Minor, and
Beast.  Animal minions are mortal, rather than supernatural, and
sections 5.2 through 5.6 inclusive, do not apply to them.  Whilst not
all Powers possess minions corresponding to all of the standard
classes, those that they do have will fall into one of those classes.
Some Powers may also be said to have "standard" minions, that is to
say, they may use minions that are also found naturally or are used by
more than one of the Powers.  "Standard" minions may be found detailed
in the Bestiary.

\subsubsection{Death \& damage}
The damage and "death" of a Minion is handled differently to that of a
mortal.  Minions may only be harmed by magic, weapons of magical
nature, and by silvern metals.  When a Minion has taken sufficient
damage that it would have been slain, had it been mortal, it is
dissipated back to its home dimension.  If a Minion is dissipated due
to damage taken, or by having the spell that summoned it to the
material world dispelled, counterspelled or dissipated, it may not
usually reappear in that material world for a period of thirty days.
If the Minion is sent back by its summoner, it may not reappear in the
material world until the next day.

\subsubsection{Magical abilities}

The magical abilities of a Minion are different to those of
mortals. Minions are not bound by the Magical Aptitude restriction
that limits the number of lowly ranked abilities that a mortal Mage
may possess. Minions may not possess magic from more than one
College. Unless stated otherwise in the description of their
particular Power, Minions may be of any College. The magical abilities
of a Minion are practised at the Rank listed in their description.

\subsubsection{Cold iron}

Minions are not prevented from exercising their magical abilities by the presence of Cold Iron.

\subsubsection{Physical form}

A Minion may use its magical abilities regardless of the physical
attributes of its form.  A Minion's form need not be capable of
vocalisation, or of complex manual manipulation to exercise its
magical abilities.  However, if the form can vocalise and/or complete
intricate hand manoeuvres, it must do so to cast its magic.  A Minion
that possesses magical abilities may use them in all of its forms.  It
takes a Pass action for a Minion to change from one form to another.

\subsubsection{Experience}
Minions may become more powerful over time.  Minions that are sentient
may gain experience and increase their abilities, as may mortals.  In
this way, over an extremely long period, Minor Minions may be
"promoted" to Lesser, Lesser to Greater, and, over several aeons, a
Greater Minion may even become a Power in its own right.  The time
taken for a Minion to achieve these increases in its home dimension is
beyond even the lifespan of Elves, and even if the same Minion is
encountered more than once in a character's life, little or no
difference will be noted.  The only exception to this is when a Minion
is assigned to the material world for an extended length of time, such
as an Agent's Familiar or Companion. These posts are usually given to
Minions who have pleased the Power in some way, or who require
training of some kind.  Minions in the material world may learn and
train in exactly the same manner as mortals.  The masters of Minions
who are assigned as Familiars or Companions, may transfer up to 10\% of
the Experience Points they gain, to their Familiar or Companion.
These Experience Points may then be spent by the Minion.  Minions who
are operating independently may gain their own experience.  It must be
noted that the abilities listed for Minions in the Bestiary, or
elsewhere, are for generic Minions and in no way limit a particular
Minion's learning capacity.

\subsubsection{Generic types}
The generic type for a Minion is created in the format detailed below,
and has abilities and Attributes within the designated ranges.  It may
have magical abilities as detailed, and in addition, may be possessed
of one or more Special Abilities.  The Special Abilities of a Minion
may be of almost any nature and are commonly such things as:
Attributes beyond the specified ranges (eg. extreme ugliness or
beauty), spells possessed as Talents, poisonous attacks, knowing a
Skill, or divinatory ability.  All Minions of a generic type will have
the abilities and Special Abilities of that type, and Attributes the
same as, or very similar to, those listed as generic.
 
\subsubsection{Format}

The format for a generic type is:

\begin{Description}
\item[Name] the Generic type of the Minion.
\item[Class] the type of Minion: Greater, Lesser, Minor, Beast.
\item[Description] what the Minion looks, smells, sounds, feels like, what the generic personality type of the Minion is.  What forms the Minion may take.  As a general rule, Greater Minions are far more impressive than Lesser ones, and Lesser more than Minor.  If the Minion is a Beast, is it sentient?  The personality of a Minion is usually that of its patron Power, or of a facet of that patron.
\item[Talents, skills \& magic] what Skills, magical abilities, and Special Abilities the Minion possesses.
\item[Movement rates] the rates of movement of the Minion's various forms.
\item[Attributes] the Attributes, or Attribute ranges for individual Minions.
\item[Weapons] the facility of the Minion with its form's natural weapons, along with any other weapon skills it can possess.
\end{Description}

\subsubsection{Minion powers}

The number and range of abilities, Skills, Special Abilities, Attributes, and weapon skills of a generic type are dependant on its class:


\begin{Description}
\item[Greater Minions]

\begin{Description}
\item[Forms] 1-3 
\item[Skills] usually practised at Rank 10.  May have 0-2 Skills, more if taken as Special Abilities.
\item[Magic] 0-1 College only, usually possess all College magic at Ranks 8-15. May also know non-College abilities such as Ward or Geas.
\item[Special abilities] 0-5
\item[Movement] 0-3 modes of locomotion per form.  A movement rate of beyond 500 yards per minute is a Special Ability.
\item[Attributes] maximum value of 30.  Beyond this value the Attribute is a Special Ability.  Physical Beauty causing Awe or Fear is a Special Ability.  Counting statistics above a value of 30 as 30, Primary statistics should not exceed 160.  Fatigue should be reflective of Endurance.  Perception should be very high, (25-30). TMR may be figured from Movement Rate and should also reflect Agility.  Greater Minions may have 0-6 points of Natural Armour as standard, more than 6 points constitutes a Special Ability.
\item[Weapons] Any natural weapons possessed by their forms at Ranks 7-10 (but not over max).  0-10 other weapon skills at Rank 0-max. Natural weapons that do more than D+4 damage are Special Abilities.
\end{Description}

\item[Lesser Minions]
\begin{Description}
\item[Forms] 1-2
\item[Skills] 0-1 Skills.
\item[Magic] 0-1 College only.  May possess all College magic at Ranks 6-12. Non-college magic only as a Special Ability.
\item[Special abilities] 0-4 
\item[Movement] 0-3 modes of locomotion per form.  A movement rate beyond 500 yards per minute is a Special Ability.
\item[Attributes] maximum value of 30, but no more than half of their primary statistics should exceed 26.  Fatigue should be reflective of Endurance.  Perception should be high, (20-26). Physical Beauty causing Awe or Fear is a Special Ability.  Primary statistics should not exceed 130.  TMR may be figured from Movement Rate and should also reflect Agility.  0-3 points of Natural Armour standard.
\item[Weapons] Any weapons natural to the form at Ranks 6-10 (but not over max).  0-5 other weapons at Rank 0-10.
\end{Description}

\item[Minor Minions]
\begin{Description}
\item[Forms] 1-2
\item[Skills] are only possessed as Special Abilities. If possessed are Rank 0-7.
\item[Magic] 0-1 College only. May possess all College magic at Ranks 0-12.
\item[Special abilities] 0-3
\item[Movement] 0-2 modes of locomotion with each form.  A movement rate of beyond 500 yards per minute is a Special Ability.
\item[Attributes] maximum value of 25.  Primary statistics should not exceed 100.  FT may be figured from EN as for Humans.  Perception will be around 15, and PB may not exceed 25, or cause Awe or Fear except as the result of a Special Ability.  TMR may be figured from Movement Rate and should also reflect Agility. Natural Armour 0-2 standard.
\item[Weapons] Any weapons natural to the form at Ranks 4-8 (but not over max).  0-2 other weapons at Ranks 0-5.
\item[Comments] As these Minions are granted to characters as Familiars and Companions, random methods may be offered for some of their abilities, when designing the generic type.
\end{Description}

\item[Beast Minions]
\begin{Description}
\item[Forms] 1
\item[Skills] None if non-sentient, else are only possessed as Special Abilities. 1 Skill per Special Ability.  If possessed are Rank 0-8.
\item[Magic] None if non-sentient, else 1 College only, possessed as a Special Ability. May have Rank 0-9 with general magics and Rank 0-5 with special magics.
\item[Special abilities] 0-4
\item[Movement] 0-2 modes of locomotion.  A movement rate of more than 500 yards per minute is a Special Ability.
\item[Attributes] maximum value of 25.  Primary statistics should not exceed 110. FT should reflect EN. TMR may be figured from Movement Rate and also reflect AG.  Perception is often high, around 20. PB may not exceed 25, or cause Awe or Fear except as the result of a Special Ability. Natural Armour 0-5 is standard.
\item[Weapons] Any weapons natural to their form at Rank 5-10 (but not over max). Will not usually be able to use other weaponry, but if allowed by the form may have 0-3 weapons at Rank 0-5. Natural weapon damage in excess of D+4 is a Special Ability.
\end{Description}

\item[Animal Minions]
\begin{Description}
\item[Special abilities] 0-1.  The only Special Ability that an Animal Minion may have, beyond the normal abilities of the animal itself, is sentience. Sentient Animal Minions have an MA of 0. The Animal Minions sent to spy on, or communicate with an Invoker are usually sentient.
\item[Comments] Animal Minions may be of any type of material world creature that is normally non-sentient. They will have all of the normal abilities associated with that creature.  The animal may be an Enchanted one.  
\end{Description}
\end{Description}
\begin{example}
"Aim, The Fire Duke" uses Salamanders as Animal Minions.  These Salamanders retain the ability to cause fire with their gaze.
\end{example}

\subsubsection{Special ability limit}

The generic type of a Greater, Lesser, Minor, or Beast Minion may have
more Special Abilities than the number listed, provided that such
abilities are exclusive.

\begin{example}
Hellhounds, a Beast Minion type, are listed in the Bestiary as
having a PC of 25-30, a PB of 4-6, and 3 other Special Abilities.  A
Hellhound may be found that had a Perception of more than 25 or a
Physical Beauty of less than 5, but not both.  In either case the
number of Special Abilities possessed by an individual Hellhound does
not exceed the maximum of 4 listed for the Beast type Minion.
\end{example}

\section{Agents (Version 1.1)}

Many of the Powers of the DQ universe use mortals as Agents.  These
Agents serve in many diverse capacities depending on the particular
aims and goals of their Power: as priestly leaders of covens or sects
dedicated to the Power; as spies; provocateurs; instruments of justice
or revenge; scourges; and recruiters.  Not all of the mortal followers
of a Power are necessarily Agents, some may simply be "lay followers",
lacking the dedication or certain skills or abilities demanded by the
Power of its Agents.

\subsection{The Pact}

The link between a Power and an Agent is in the form of a contract
whereby the Agent, a mortal sympathetic to the cause, goals and
motivations of a particular Power, agrees to further that Power's
interests in the material world and to keep the tenets of its "faith",
in return for a measure of guidance and protection, and usually for
some security in their eventual fate in the afterlife.  This is not a
contract to be entered into lightly, as it may influence the actions
and general behaviour of the Agent, and it is not easy to break such a
contract, once made, without risking the direst wrath of the renounced
Power.

\subsubsection{Becoming an Agent}
The process for becoming an Agent is solemn and ancient.  The
character must first invoke the desired Power using an Ancient
Invocation.  Should the Power choose to appear, the character must
state that they wish to become an Agent for the Power.  They must also
offer a token to the Power.  This initial interaction between the
character and the Power is very serious, and both the GM and the
character's player should carefully consider their actions.  If the
token is acceptable to the Power, the character possess those skills
or abilities required by the Power, and agrees to the Powers
strictures relating to the behaviour of its Agents, the Power accepts
them as an Agent.  If the token is unacceptable, the reaction of the
Power will depend on its personality and the reasons for the gift
being unacceptable.  Reactions may range from a sympathetic declining
of the token, and perhaps a hint as to a more acceptable one, through
to an admonition to never Invoke the Power again, on pain of death.

\subsubsection{Tokens}

The exact form that a token takes may vary widely.  The acceptability
of a token will depend on its value to the character and the
personality and motivations of the Power.  It may take the form of an
object, an entity, a physical or spiritual attribute of the character,
an oath or undertaking, or even an insubstantial "gift".  Even the
most benevolent Powers dislike mockery, and the choice of token to be
offered is often difficult.

EX. A Fire Mage wishing to become an Agent of Aim, might choose to
offer a captive Water Mage as a sacrifice, or might perhaps set fire
to a navigators guild and offer it to Aim as a "gift".

\subsubsection{Renouncing Status}

A character may renounce their status of Agent.  Renouncing Agent
status is a perilous task, only ever undertaken in extreme
circumstances.  The process of renunciation is ancient and invariable.
First the Agent Invokes their patron Power by the use of an Ancient
Invocation.  If a Minion is sent, the Agent informs the Minion that
they are renouncing their status as Agent.  The Power will then appear
in their insubstantial form, if they have not already done so.  The
Agent then states that they are renouncing their status of Agent, and
demands the return of the token they offered when they became an
Agent.  If the token was an object, or some other physical thing, the
Power must return it immediately.  If it was a service, an oath or
some other form of insubstantial offering, the Power simply
acknowledges its return.  The character immediately loses their Call
Patron talent, their Familiar or Companion, and any other abilities
they may have gained as a result of becoming an Agent.  They do not
regain any abilities that they lost due to their prohibition by the
Power, but are freed from any strictures on their behaviour.  The
Power then returns to its home dimension, and the character is no
longer an Agent of that Power.  The Power's reaction to the desertion
of an Agent will vary considerably, depending on the Power's
personality, ethics and motives.  Some Powers may direct their Agents,
cults or sects to hunt down and slay the deserter, others may seek to
punish or kill the offender personally, when next summoned to the
material world.  Still others may feel that the loss of their great
patronage is punishment in itself.  When deciding exactly what form of
action a Power takes against an ex-Agent, the GM should carefully
consider the reasons for the renunciation of Agent status, the
circumstances surrounding it, and the personality of the Power.
Needless to say, it is not possible to force an Agent to renounce
their status.  Such a control will certainly be noticed by the Power
and will invalidate the renunciation, not to mention incurring the
Power's wrath.  Particularly unwise Agents may use their Call Patron
talent to summon their Power so that they may renounce their status.
This will work, but it must be noted that when the process of
renunciation is complete, the presumably annoyed Avatar of the Power
will be physically present, and may immediately take direct action.

\subsection{Restrictions}

\subsubsection{Ancient Invocations}

Only through the use of an Ancient Invocation may a character become
an Agent.  If a character invokes a Power without the use of an
Ancient Invocation, they may not become an Agent.  The reaction of a
Power to such a mistaken Invoking will vary with the Power's
personality.  Many will direct the character to one of their present
Agents, or to a cult or sect dedicated to the Power and led by one or
more Agents, both of whom must perforce possess an Ancient Invocation.
Some Powers will require the character to go on a quest to find an
Ancient Invocation, though they may give some clue as to where to find
one. Others may be sympathetic but unhelpful, or angry, or simply
condescending as to the character's amateur behaviour.

\subsubsection{Prohibited Abilities}

An Agent may not possess abilities prohibited by its Power.  Should a
character be accepted as an Agent, they will immediately lose all
magic, Skills and abilities prohibited by the Power.  In a few rare
cases, some form of compensation will be offered by the Power, beyond
those abilities usually granted by it, but this is certainly not the
norm.  It should be noted that a state of at least cold war, exists
between many of the Powers, and the College of Summoning Magics.
Those very few of the Powers who will accept Summoners as Agents, are
so noted in their individual descriptions.  In short, if it does not
positively state that a Power will accept a Summoner as an Agent, then
they will reject them, or require that they quit that College.

\subsection{Benefits}

\subsubsection{Call Patron}

An Agent receives the ability to call their Patron. The ability
received is a special form of invocation that may only be used by
Agents.  The ability works as a Racial Talent (EM: 300), and operates
as follows:

The Agent may, by Invoking their patron Power, attempt to call their
plight to their patron's attention and gain help, guidance or
communication.  The exact nature of the help rendered is at the GM's
discretion, and may range from nothing, through to a full
manifestation of the patron's Avatar and accompanying Minions.  More
usually it will consist of the arrival of one or more Minions of the
patron, or an insubstantial manifestation of the Power itself.  The
ability takes at least one Pulse to enact, during which time the Agent
must verbally Invoke the Power, and may not engage in any other
activity that requires verbalisation.  For example, an Agent could
Call Patron whilst engaged in combat, but not while spellcasting.

The Base Chance of the Power responding in some way is their Response
percentage to an Invocation, as listed in their description, +3\% per
Rank achieved with this talent. An Ancient Invocation may be used to
increase the chance, but will require its full time and any materials.
From +20 to -20 may be added at the GM's discretion to the chance of
the Power responding, depending on the situation, the Power's
personality and the Agent's actions during the Invocation.

EX. After slaying one opponent and beginning on their second, an Agent
of Alloces uses their Call Patron talent.  Alloces is the "Warrior
Duke", and likes nothing better than bloodshed and carnage.  The GM
determines that the actions of the Agent are worth +10, increasing to
+15 once they slay their next opponent.

If the Power does not respond, the Agent may continue to Call Patron
on the next and subsequent Pulses. Each extra Pulse that they call
increases the chance of the Power responding by 1\%.  This roll is made
each Pulse until either the Power appears or the Agent ceases to call.
Once the Power responds they may require a service, item or oath of
the Agent after, or even before, they offer any aid.  This requirement
will depend on the urgency and severity of the Agent's need, how
closely they have been following the strictures concerning their
behaviour, and how often they have called for help.  Most patrons will
ask for no service, or perhaps only a token one, if the Agent has only
called as a last resort, and has in all ways been faithful, but may
levy a very severe charge if the calling is for a trivial situation or
the Agent has in any way been false to their patron.

\subsubsection{Special Abilities}

Some Powers grant various special abilities to their Agents.  These
special abilities are listed in the description of the individual
Powers, are granted to a character immediately they become an Agent,
and are immediately lost if the character renounces their pact.

\subsubsection{Familiars}

An Agent may gain a Familiar.  All of the Powers, unless stated
otherwise in their description, grant their Agents the use and
companionship of a Minor Minion to serve as a Familiar.  This Familiar
is a combination of guide, teacher, student, friend, henchling and
political officer.  The Familiar feels great loyalty to the Agent,
second only to its Power, and will serve and help them in all ways.
The Familiar begins as a standard Minor Minion of the patron Power.
Unless otherwise stated in the Power's description, the Familiar's
alternate (animal) form may be of any small non-sentient and
unenchanted creature.  The Agent may ask for a particular form for
their Familiar, but the Power makes the final decision.  The only
additional ability of the Familiar is that it knows all of the
languages known by the Agent but at one less Rank.  If this reduces
the Rank below 0, the Familiar does not have that language.  There is
a link between the Familiar and the Agent that keeps the Familiar in
the material world.  If the Agent dies, the Familiar is immediately
dispelled back to its own dimension.  If the Agent is resurrected,
they may collect their Familiar by Invoking their patron.  The Power
will then despatch the Familiar in response to the Invocation.
Depending on the circumstances surrounding the Agent's death, some
small token may have to be paid to regain the Familiar.  If the
Familiar "dies", it is dispelled back to its own dimension.  The Agent
may usually only automatically gain a new Familiar five years after
the issue of the last one, and only if their Familiar has died. Often
however, it is possible to gain a replacement Familiar in return for a
considerable service to the Power.  If the Agent wishes to have the
same Familiar back, they will usually have to perform another service,
in addition to the one required to get any new Familiar.  Much of this
depends on the past behaviour of the Agent and the circumstance
surrounding the "death" of the Familiar.

\subsubsection{Companions}

Some of the Powers, particularly the Elohim, grant a Companion to
their Agents, rather than a Familiar. Instead of accompanying the
Agent, as does a Familiar, the Companion may be summoned into the
material world by the Agent by Invoking it.  This Invocation will
always succeed, and need not be rolled for.  The Companion will take
(D-2, min. 1) Pulses to appear, and will then do the bidding of the
Agent.  The Companion will accompany the Agent if requested to do so,
or will return to their home if dismissed by the Agent.  The major
difference between a Companion and a Familiar, is that Companions
iseldom have an inconspicuous form, and will be noticed and cause a
major reaction.  There is a link between the Agent and the Companion,
and the Companion will assist and guard the Agent in all ways, putting
their concerns second only the Companion's patron Power.  If the Agent
is slain whilst the Companion is in the material world, the Companion
is immediately dispelled back to their home dimension.  If the Agent
is resurrected, they may again call forth the Companion.  If the
Companion is "slain" they return to their home dimension and may not
again be called forth by the Agent.  A new Companion can usually only
be automatically gained five years after the issue of the last one,
and only if the old Companion is "dead".  An Agent may however gain a
new Companion before the five year limit, through completing some
great service for their Power.  A newly assigned Companion is in all
ways a standard Minor Minion for the Power, except that it may only
speak languages designated by its Power, at least one of which will be
known to the Agent.  The Ranks of the languages are also designated by
the Power.  One language will usually be at Rank 8 or higher.

\subsubsection{Other Benefits}

Agents generally gain the respect of all lay worshippers of their
Patron.  This may lead to other benefits not detailed here.  Agents of
those Powers who have structured organisations, such as the Powers of
Light, may be able to gain free board and training at the places
dedicated to their Patron

EX. Michaelines, Agents of the Elohim Archangel Michael, will
certainly be able to get their weapon training free of charge at any
of Michael's Chapter houses or monasteries.

\subsubsection{Vengeance}

The death of an Agent may open a link to the patron Power.  When an
Agent dies in a manner that is irressurectable, or after the time in
which a dead Agent could have been resurrected has expired, a portal
is opened to the patron Power, so as to allow them to collect the
essential lifeforce of the Agent.  As a result of this portal, the
patron Power may send an Avatar to the material world.  Powers are, as
a rule, somewhat put out by the untimely demise of their Agents, and
may seek restitution from the Agents slayers.  If a character kills an
Agent and then vacates the area, the patron may very well appear after
100 hours, but decide against investigating.  However, standing over
the smoking remains of an Agent when their patron appears, may be
hazardous to the health.

\section{Invocations (Version 1.1)}

\subsection{Power's Names}

Powers may be aware when their names are spoken.  Those Powers that
have a second "Response" percentage listed after their name are aware
when their listed common name or Individual True Name is spoken aloud.
Most Powers will not respond in any manner if their name is merely
used in casual conversation, or the like, but will do so if their name
is being spoken as part of an Invocation.  The Powers are not aware
when their titles, sobriquets, or nicknames are spoken.  Those Powers
that have a second Response percentage listed have that chance of
responding to the use of their name in casual conversation.  If their
Individual True Name is spoken, increase the chance by 25\%.  Those
Powers that have 0\% listed as their second Response percentage are
aware that their name has been used but never respond to that casual
usage.  Note that most Powers are not aware when their names are used
casually.

\subsection{Simple Invocations}

The "Response" percentage listed amongst the Power's attributes gives
the chance on D100 that the Power will respond to their Common Name
spoken as part of an Invocation.  If it is the Power's Individual True
Name that is being spoken, the percentage chance that they will
respond should be raised by 25.  The Power's response may take one of
three forms:
\begin{Enumerate}

\item
The Power may choose to despatch a Minor minion to spy on the Invoker,
ask them the purpose of their invocation, or even to slay them.  This
minion will be of the common Minor type for that Power and if
intending to speak with or slay the invoker, they will appear within
D-5 minutes.  Their appearance is almost invariably "natural"; flying
creatures will fly through a window, creeping ones may crawl from
behind furniture, etc.

\item
The Power may despatch an animal Minion to serve the same purpose as
in 1.  This animal will be of a type commonly associated with the
Power, it will be sentient, but otherwise unexceptional.

\item
The Power may choose to project an insubstantial image of themself.
This image will usually be of their common form, and has no powers
other than to communicate with the invoker or to confer upon the
invoker the status of Agent, along with the Call Patron Talent; this
latter ability only being available if the Invoker used an Ancient
Invocation.
\end{Enumerate}

\subsection{Ancient Invocations}

Powers will more often respond to Ancient Invocations.  In its
simplest form, an Invocation is merely the use of a Power's name in a
formal sense, along with the willingness for the Power to respond.  A
simple "I call upon the power of ...", or the use of the Power's name
as a chant or mantra is just as effective as the most elaborate ritual
that the Invoker can conceive.  Certain Ancient Invocations may
however have gained great power through their extreme age, and thus
have a greater chance of causing a Power to respond to their use.
These Ancient Invocations may be found in rare books in arcane
libraries, and usually only through a great deal of searching.  The GM
must determine the additional chance that an individual Invocation has
of causing the Power to respond, along with any special conditions or
ingredients that are required to perform it as well as the amount of
time required.

EX. "Ritual for Invoking Aim, (The Fire Duke), +10\%, requires a very
large bonfire, 10 minutes".

\section{The Elohim}

\subsection{History \& Background}

It was in the last years of the Elven Empire, as the power of the Drow
was rising, and elven culture was stagnating in decadent luxury, that
the group called Noldanor was formed. In elven their name meant
Knowledge of Light or Illuminated, and they appointed themselves the
task of stopping the spread of the decay that threatened their
civilisation. They believed in purity of spirit, and discipline of
mind and body. They saw the casual use of magic as one of the prime
causes for the decadence of the elven people, and counselled against
it. In the end, the task proved too large, and their numbers too
few. As the fall began, and the war spread, they fought to the last
with a bravery that has seldom since been seen in all of the ages of
the world.

Although all of the Noldanor perished in the fall, their wisdom and
beliefs did not completely leave the world, for they had planned and
trained for this outcome. The spirits of the Noldanor passed beyond,
where, because of their dedication and training, they retained traces
of their former existence, and eventually formed immortal beings, the
Powers of Light, composites of the mortal beings that once were.

In the later ages of the world these Powers have sought to guide
mortals, to prevent another fall, and to protect them against the
designs of the Powers of Darkness, Demons who seek to seduce and use
mortals for their own ends.

Whilst the Powers of Light are technically Demons themselves, the term
is never used by them to refer to themselves, but rather used as a
generic term for the Powers of Darkness and any other Powers opposed
to their cause. It is mortals who have given to the Powers of Light
the name by which they are now known: Elohim, the Lordly ones.

\subsection{Organisation}

The Elohim possess a structured order and hierarchy comprising of
seven tiers. There are four great Powers of Light, called Archangels,
each with a broad sphere of influence. Each Archangel leads a host of
lesser Powers, the Bene Elim, who fill subordinate offices within the
Archangel's sphere. These are in turn served by Elohim of decreasing
power, the Malakim and Aishim. The lowest three tiers form the
servants (or minions) of the Elohim.

The seven tiers, their titles and meanings are:

\begin{description}
\item[Archangels]
\item[Bene Elim] Mighty Ones
\item[Malakim] Kings
\item[Aishim] Flames
\item[Erelim] Valiant Servants
\item[Kerubim] Servants of Knowledge
\item[Seraphim] Servants of Light
\end{description}

\subsection{Agents}

The Elohim seek mortal Agents to spread their teachings, and to work
against the plans of the Demons. The Elohim prefer elven or human
Agents, but will accept those of any race provided they meet their
stringent criteria. Each of the Archangels looks for different
abilities in their Agents, but all four only accept mortals who are
pure of spirit, brave, dedicated, and completely committed to their
cause.

Because the Elohim see the irresponsible use of magic as one of the
leading causes for the fall of the ancient elven world, they are
deeply concerned for those that delve in the magical arts. They
counsel the limiting of magical training to only those who show
integrity, strong morality, and personal responsibility.

Most of the Elohim's mortal Agents are followers of one of the four
Archangels. Relations between the Archangels are usually cordial,
although they do not always agree on matters affecting the
Elohim. Occasionally, Agents of one of the four may find that they are
working for goals opposed to the wishes of the other Archangels, and
may find themselves in conflict with other Powers of Light Agents.
This conflict of interests seldom leads to violence, and Agents will
usually strive for some compromise.

\subsection{The Archangels}

The four great Archangels are Michael, Gabriel, Raphael, and
Uriel. Their spheres, personalities, Avatar abilities, and other
information about them is listed below.


\subsubsection{Michael: "The Sword, Sword of Light"}

\begin{Description}
\item[Type] Demon

\item[Group] Elohim

\item[Response] 15\% / -

\item[Description] Michael always appears as a stunningly handsome
young, blond, human or elven warrior, dressed all in red, and armoured
for war, wielding his great two-handed sword, Ira (Wrath), that blazes
with the fire of his righteous anger. Michael may appear with great
feathered wings, or riding a sentient pegasus.

\item[Sphere] Michael's office is that of warrior and he leads the
Elohim's fight against the Powers of Darkness. He destroys and
confounds the works of evil and the Demons whenever possible, and
charges his Agents, the Michaelines, to do the same.

\item[Personality] Michael is rash and bloodthirsty and prefers to
charge into battle, so as to defeat the unholy by strength of arms.
He impatiently awaits Nightfall, (the resumption of the War of the
Powers), and has been known to be involved in many plans to hasten its
arrival.

\item[Greater minions] Erelim 

\item[Lesser minions] Kerubim 

\item[Minor minions] Seraphim 

\item[Beast minions] Sentient Pegasi (MA 0)

\item[Animal minions] Boar

AGENTS: Michael is extremely suspicious of Mages and will very seldom accept them as Agents. He expects all his Agents to be warriors, and to fight courageously against the Powers of Darkness and their Agents and followers. Michaelines may be found investigating rumours of Demonic cults, and the doings of "Black Mages". Once discovered they will attempt to convince civil authorities to deal with cults and covens, and will lend their martial prowess in eliminating them. They also root out heresy and corruption in established churches, and are often found acting as the martial arm of the Gabrielite inquisition. They tend to view any non-Elohim Agents with the gravest suspicion.  Michaelines have a well founded reputation for being rather blood-thirsty and over zealous in the pursuit of their goals.

AREAS: Michael's Agents may be found in all areas, though they tend to be concentrated in borderlands and trouble spots. In territories where the Elohim and their Agents are officially recognised, the Michaeline Order constructs chapter houses that serve as barracks and training facilities, and are usually fortified strongholds. These chapter houses recruit "Knights of Michael", fighting lay members dedicated to the cause of the Elohim. Agents of Michael will always be able to get free board and weapons training at these chapter houses.

COLOURS \& SYMBOLS: Michaelines often wear red garments, or white robes edged with red, and their symbol is a flaming sword (usually shown point downwards). They are seldom seen not wearing armour.

AVATAR ABILITIES: Michael is a mighty Warrior, and a superb Military Scientist (even though he will tend to counsel attack in most situations). Michael is very beautiful to look at and may inspire Awe. Michael cannot cast magic but may initiate various magical talents upon himself as a Pass action. These include: Quickness, Coruscate, and ESP. Michael's Magic Resistance is special, in that he has 99% MR normally, and in addition has an automatic 52% Active Resistance that is always operating even if he cannot see the Mage that has targeted him.

Movement Rate: (yards per minute): Run: 350; Fly: 600

PS: 30	MD: 28	AG: 32	MA: 10
EN: 26	FT: 38	WP: 32	PC: 30
PB: 30	TMR: 8	NA: 3 DP (+10 pt. Plate Armour)
MR: 99/52 (Special)

Weapons: Michael always wields his two-handed sword, which is of demonic quality, and with which he is Rank 9. His sword may be considered magical and can strike any target, even those usually immune to material weapons (equivalent to Spectral Weapon Rank 20). He has Rank 10 Unarmed Combat, and may use most other weapons at maximum Rank. He wears demon crafted red plate armour of some light-weight material. Whenever he enters combat Michael becomes surrounded by a coruscating aura (equivalent to Coruscate Rank 20).

AVATAR SUMMONING: Michael will seldom manifest except in the thick of battle, where he will immediately plunge into the combat wielding his great sword.
\end{Description}

\subsubsection{Gabriel: "The Herald, The Messenger"}

\begin{Description}
\item[Type] Demon
GROUP: Elohim
RESPONSE: 15%
DESCRIPTION: Gabriel manifests as either a dazzling human youth, with almost feminine features, or as a young elven woman of flawless beauty. Gabriel is dressed in blue robes, and bears a sword and an heraldic trumpet, and may in rare circumstances appear in armour.  The trumpet has a banner attached, on which its name, Veritas (Truth), is embroidered in gold. Gabriel may appear with great feathered wings, or riding a sentient pegasus. Gabriel's voice is melodious and mellifluous.
SPHERE: Gabriel's office is that of herald, messenger, evangelist, and inquisitor.  Gabriel's agents, the Gabrielites, are charged to spread the word of the Elohim, to convert entities to the faith, and to root out heresy and evil-doing. 
PERSONALITY: Gabriel is impulsive and often makes rash decisions.  Gabriel is closely associated with Michael, and often defers to him on matters of policy. Followers of the Elohim state the Gabriel will one day signal the resumption of the War of the Powers, and the onset of Nightfall, with a mighty trumpet blast, that will be heard on all planes and dimensions. Other observers have commented that Gabriel seems all too eager to discharge that particular duty.
GREATER MINIONS: Erelim 
LESSER MINIONS: Kerubim 
MINOR MINIONS: Seraphim 
BEAST MINIONS: Sentient Pegasi (MA 0)
ANIMAL MINIONS: Hawk
AGENTS: Gabriel is suspicious of Mages, but accepts their worth  and will accept as Agents those that have unequivocally proved their austerity and self discipline. Gabrielites are often trained as scribes in their churches and monasteries, and then sent out as missionaries to preach to the heathen (those who do not follow the Elohim's teachings). They are also taught the arts of war, so as to aid in getting their point across. Gabrielites check on the purity of the doctrine in established churches, and work along side Michaelines as Inquisitors. 
AREAS: Gabrielites are found both in civilised areas, where they serve in the churches as clerks and preachers, and also in the most untamed wilderness, where they seek out new converts, and look for cults and other signs of corruption. Because of their close association with the Michaelines, Agents of Gabriel can always get free board and weapons training at Michaeline chapter houses.
COLOURS \& SYMBOLS: Gabrielites usually wears robes of blue, or white edged with blue, and their symbol is an heraldic trumpet. They usually only wear armour when dealing with the heathen.
AVATAR ABILITIES: Gabriel is a mighty Adept of the College of Bardic Magics, and possesses  the skills of Courtier, Spy, and Troubadour. Gabriel is stunningly beautiful to look at and may inspire Awe. Gabriel may Charm as a Talent.

Movement Rate: (yards per minute): Run: 350; Fly: 600

PS: 24	MD: 24	AG: 30	MA: 30
EN: 22	FT: 32	WP: 26	PC: 26
PB: 35	TMR: 8	NA: 2 DP (+10 pt. Plate Armour)
MR: 60

Weapons: Gabriel is not a great fighter, but will do so if pressed, using a hand \& a half sword at Rank 7. This sword is of demonic quality, and may be considered magical, able to strike any target, even those usually immune to material weapons (equivalent to Spectral Weapon Rank 20). Gabriel may wear demon crafted blue plate armour of some light-weight material, although this is rare. Upon entering combat Gabriel becomes surrounded by a coruscating aura (equivalent to Coruscate Rank 20).
AVATAR SUMMONING: Gabriel may be summoned into any place of learning, or anywhere that heresy has been discovered.
\end{Description}

\subsubsection{Raphael: "The Shield, Defender of the Faithful"}

\begin{Description}

\item[Type] Demon
GROUP: Elohim
RESPONSE: 15%
DESCRIPTION: Raphael usually appears as a handsome mature man, dressed in green and outfitted for war, either with large, feathered wings, or more often, seated upon a pegasus. He bears a lance and his shield Fidei (Faith).  His shield is green and carries in gold the image of a crook, or crosier, reflecting his role as a shepherd and protector. He will often bring with him a pack of hunting or war dogs.
SPHERE: Raphael's office is that of the defender of the faithful, teacher and guide.  He charges his Agents, the Raphaelites, with the protection and education of their followers.
PERSONALITY: Raphael is careful and cautious, and will seldom counsel attack, but rather a brave and stalwart defence. He often finds himself in opposition to the more insane plans of both Michael and Gabriel, and co-operating with Uriel in matters of protection.
GREATER MINIONS: Erelim 
LESSER MINIONS: Kerubim 
MINOR MINIONS: Seraphim 
BEAST MINIONS: Sentient Pegasi (MA 0)
ANIMAL MINIONS: Dog
AGENTS: Raphael will accept Mages provided they can prove their bravery, loyalty, and discipline. His Agents may be found as guards, generals, or teachers, or as questing knights who seek to rid the world of dangers to their people, such as Dragons. Raphaelites tend to be the least impetuous of the Agents of the Powers of Light, and more reserved and conservative, as befits those whose office is defence.
AREAS: Raphaelites are usually found in settled and civilised areas where they act in roles that help to protect their people. When found in wilderness areas Raphaelites will usually be on a mission to destroy a particular threat to the safety of their "flock".
COLOURS \& SYMBOLS: Raphaelites often wear robes of green, or white edged with green. Their symbol is a shield or a crook. Raphaelites often wear armour.
AVATAR ABILITIES: Raphael is a great Armourer, Mechanician, Military Scientist, Ranger, and Warrior, and possess lesser abilities as a Beast Master, Healer, and Weaponsmith. Raphael is very handsome to look at and may inspire Awe. Raphael cannot cast magic but may initiate various magical talents upon himself as a Pass action. These include: Quickness, Coruscate, and ESP. Raphael's Magic Resistance is special, in that he has 95% MR normally, and in addition he has his Active Resistance chance of 54% of "parrying" an incoming spell with his shield, so that it rebounds and targets the caster instead. He can only do this if he can see the Mage that has targeted him.

Movement Rate: (yards per minute): Run: 350; Fly: 600

PS: 28	MD: 26	AG: 28	MA: 10
EN: 32	FT: 38	WP: 34	PC: 36
PB: 27	TMR: 8	NA: 5 DP (+10 pt. Plate Armour)
MR: 95/54 (Special)

Weapons: Raphael may use most weapons at maximum Rank, and prefers knightly weapons. He carries a lance and hand \& a half sword, which are of demonic quality, and with which he is Rank 6 \& 8 respectively. These weapons may be considered magical and can strike any target, even those usually immune to material weapons (equivalent to Spectral Weapon Rank 20). He wears demon crafted green plate armour of some light-weight material. Whenever he enters combat Raphael becomes surrounded by a coruscating aura (equivalent to Coruscate Rank 20).
AVATAR SUMMONING: Raphael can be summoned whenever or wherever defence is needed. 
\end{Description}


\subsubsection{Uriel: "The Judge, Lady Justice"}

\begin{Description}

\item[Type] Demon
GROUP: Elohim
RESPONSE: 15%
DESCRIPTION: Uriel usually manifests as a beautiful, dark-haired human or elven woman, dressed in purple robes, edged in gold, and bearing a sword and her balances or scales, Jus (Law). Uriel may appear with great feathered wings, or riding a sentient pegasus. She may appear wearing purple plate armour if she feels it warranted.
SPHERE: Uriel's office is the gathering of souls, the judging of their behaviour, and the choosing of the final rewards accorded to the followers of the Elohim in the afterlife. She charges her followers, the Urielites, with the dispensing of justice, and the punishment of criminal and unethical behaviour.
PERSONALITY: Uriel is stern but fair, brooking no argument with her judgements. She does know compassion, however, and may reduce the severity of her judgement if she feels that the subject had cause for their actions.
GREATER MINIONS: Erelim 
LESSER MINIONS: Kerubim 
MINOR MINIONS: Seraphim 
BEAST MINIONS: Sentient Pegasi (MA 0)
ANIMAL MINIONS: Owl
AGENTS: Uriel will accept Mages as Agents, favouring Thaumaturges. Those of her Agents that are Mages tend to be even more fervent than usual in finding and bringing to justice criminals who use magic. Her Agents are often judges and advocates, but also jailers, guards and even executioners.
AREAS: Urielites are usually found in civilised areas. In areas where the Powers of Light are very strong, the Agents and followers of Uriel are often accorded the legal right to try and punish criminals. Urielites found in wilderness areas will almost certainly be hunting escaped criminals, and may be accompanied by Michaelines or Raphaelites.
COLOURS \& SYMBOLS: Urielites often dress in the manner of their patron, in gold edged purple robes, or in white robes edged with purple, and sometimes have their armour embossed with their symbol, a pair of scales.  Urielites wear armour when it is fitting for their job or station, or when hunting criminals.
AVATAR ABILITIES: Uriel is a practitioner of the College of Sorceries of the Mind. She is a great debater and Philosopher, Astrologer and Courtier. Uriel is dazzlingly beautiful and may inspire Awe. Uriel can always detect if someone is knowingly lying to her, and can cause a mortal to feel great pain every time that they lie. Uriel's Magic Resistance is special in that she will always resist magic that is cast at her unjustly, or with criminal intent. Otherwise her MR is 66.

Movement Rate: (yards per minute): Run: 350; Fly: 600

PS: 26	MD: 22	AG: 26	MA: 28
EN: 22	FT: 30	WP: 36	PC: 32
PB: 32	TMR: 7	NA: 2 DP (+10 pt. Plate Armour)
MR: Special

Weapons: Uriel is not a great fighter preferring to work through legal and diplomatic channels rather than open combat. If forced into combat she uses her Hand \& a half sword, which is of demonic quality, and with which she is Rank 7. Her sword may be considered magical and can strike any target, even those usually immune to material weapons (equivalent to Spectral Weapon Rank 20). She may wear demon crafted purple plate armour of some light-weight material. If she enters combat Uriel becomes surrounded by a coruscating aura (equivalent to Coruscate Rank 20).
AVATAR SUMMONING: Uriel can only be summoned into a place of justice (court, prison, etc.), or to where a crime has recently been committed. 
\end{Description}

\subsection{Subordinate Offices}

The offices of the other Elohim are specific sub-sets of the spheres of the Archangels.  For example, Azrael, "The Executioner", is a Bene Elim (or Angel) of Uriel's host, whose office is the execution or slaying of those condemned for their heinous crimes. Agents of the Powers of Light who are executioners or scourges, may have Azrael as their Patron, but still consider themselves Urielites. This "tree" of offices extends down through the Malakim to the Aishim. Thus an Agent may have as their Patron an Aishim whose office is simply the defence of a particular sacred well, but may still be considered a Raphaelite.

When creating a new Elohim, GMs should decide what office they hold.  From this it may be ascertained which host they will belong to, and from the scope of the office, what degree in the hierarchical order they hold. 

\subsection{Elohim Names}
The names chosen by the Elohim follow a somewhat standard format, offered here as an aid to GMs.  The Elohim invariably use the suffix -el with their common names.  This suffix may be roughly translated as Lordly, and is used even by the servant Elohim.  To generate an Elohim's name, choose one syllable from the Primary list, append one or two syllables from the Secondary list and add el.

PRIMARY:
A, Ab, Ak, Ap, As, At, Az, Ba, Bel, Ca, Cher, Ches, Dan, E, Ez, Ga, Gab, I, Is, Jet, Ka, Ky, Lo, Mic, Na, O, Ra, Rach, Raz, Re, Sag, Ser, U, Ve, Yah, Zaph, Zek, Zu.

SECONDARY:
a, aph, as, da, di, ek, et, f, fa, fi, g, ga, gi, ha, hi, hu, i, ki, ma, mi, na, ni, pha, qu, ra, re, ri, ru, sa, si, ub, ubi, za, zi. 

\subsection{Agents Names}

Followers of the Elohim often choose a new name upon achieving Agent
status. Most often these names follow the same format as the names of
the Elohim themselves, save that they omit the -el, or lordly suffix,
and often substitute -im, meaning servant.

\subsection{Minions of the Elohim}

\subsubsection{Erelim "Valiant Servants"}

\begin{Description}

\item[Class] Greater Minion
Description: Erelim (also sometimes known as Thrones) appear as mighty humanoids, with the height and build of hill giants. They are however extremely good-looking, and appear more like giant-sized elven and human men and women, with immense feathered wings. Erelim radiate a dazzling inner light, and glow strongly in all lighting. They may choose to assume a less obvious form, concealing their wings, shrinking to 6 foot or so, and reducing their glow. They may appear armoured for war, or wearing  short robes (well tailored and cut to reveal their powerful bodies). Their armour or clothes will tend to be in the colour of their faction. Erelim usually appear armed with a giant bow, a giant spear, and a large sword and shield. They may use these even if they assume human size.
Talents, Skills and Magic: Erelim of Uriel are almost all Mages. Some of those of Raphael's and Gabriel's factions are Mages, but Erelim of Michael are not. Erelim that are Mages all belong to one to the Thaumaturgical Colleges. They will have Ranks 12-15 with the General and Special Knowledge of their College. All have 1 or 2 skills at Rank 10, these skills varying depending on their faction. Non-Mage Erelim may also have the Warrior skill, usually also at Rank 10. The presence of an Erelim may cause Awe.
Movement Rates: (yards per minute): Run: 450; Fly: 500
PS: 35-40		MD: 18-25	AG: 20-26	MA: 5-30
EN: 30-40	FT: 45-50	WP: 24-30	PC: 25-30
PB: 27-28	TMR: 9/10	NA: 3 DP (+10 pt. Plate Armour)
MR: 24-50
Weapons: Erelim usually have giant bow, giant spear, and 1 sword and shield at maximum ranks, and Warrior Erelim will know a large number of other weapons, at lower ranks. They may choose to appear in well-crafted plate armour of some light-weight material, and carrying a giant bow, giant spear, and sword and shield of their choice. These weapons will be demon crafted and have bonuses to BC and/or damage. When Erelim enter combat they become surrounded by a coruscating aura (equivalent to Coruscate Rank 20).
\end{Description}

\subsubsection{Kerubim "Servant of Knowledge"}

\begin{Description}

\item[Class] Lesser Minion
Description: Kerubim usually appear as extremely tall (7' - 7'6"), good-looking elven and human men and women, of Herculean or amazonian stature. All have large feathered wings, although they may choose to conceal these at will. Kerubim radiate a strong inner light, and glow faintly in all lighting. They wear short robes (cut to reveal their muscled physique) or armour, in the colour of their faction, and usually appear armed with spear,  sword and shield.
Talents, Skills and Magic: Kerubim of Raphael and Uriel tend to be Mages. Some of those of Gabriel's faction are Mages, but most Kerubim of Michael are not. Kerubim that are Mages tend to belong to one to the Thaumaturgical Colleges, but only a rare few are Elementalists. Kerubim of the Entities Colleges are almost unknown. They will have Ranks 10-12 with the General and Special Knowledge of their College. Most have 1 skill at Rank 9, this skill varying depending on their faction. Non-Mage Kerubim may also have the Warrior skill, usually also at Rank 9.
Movement Rates: (yards per minute): Run: 400; Fly: 500
PS: 28-30		MD: 18-25	AG: 22-28	MA: 5-30
EN: 28-34	FT: 35-40	WP: 22-28	PC: 20-25
PB: 20-26	TMR: 8-10/10	NA: 2 DP (+9 pt. Plate Armour)
MR: 22-48
Weapons: Kerubim usually have spear, and 1 sword and shield at maximum ranks, and Warrior Kerubim will know a large number of other weapons, at lower ranks. They may choose to appear in well-crafted plate armour of some light-weight material, and carrying a spear, sword and shield of their choice. When Kerubim enter combat they become surrounded by a coruscating aura (equivalent to Coruscate Rank 15).
\end{Description}

\subsubsection{Seraphim "Servant of Light"}

\begin{Description}

\item[Class] Minor Minion
Description: There are many more Seraphim than any other type of Elohim, for they are the manifestations of the individual souls of the original Noldanor, who died many aeons ago, and also of the many Agents of the Powers of Light who have fought for their cause throughout the later ages of the world. Seraphim usually appear as tall and handsome elven and human men and women, although some are of other races. All have large feathered wings, radiate an inner light and well-being, and glow faintly in dim lighting. They wear robes or armour in the colour of their faction and usually appear armed with sword and shield.
Talents, Skills and Magic: Many Seraphim are members of one of the Colleges of Magic, but as many are not Mages. Those that are tend to belong to one to the Thaumaturgical Colleges, but some are Elementalists. Seraphim of the Entities Colleges are very rare, but not unknown. They will have Ranks 6-12 with the General and Special Knowledge of their College. Most have 1 skill at Rank 7, this skill varying depending on their faction. Non-Mage Seraphim may also have the Warrior skill, usually at Rank 8.
Movement Rates: (yards per minute): Run: 350; Fly: 500
PS: 18-26		MD: 18-25	AG: 22-27	MA: 5-27
EN: 20-26	FT: 22-24	WP: 22-27	PC: 15-25
PB: 18-25	TMR: 7-10/10	NA: 1 DP (+8 pt. Plate Armour)
MR: 22-48
Weapons: Seraphim usually have 1 sword and 1 shield at maximum ranks, and Warrior Seraphim will know a large number of other weapons, at lower ranks. They may choose to appear in well-crafted plate armour of some light-weight material, and carrying a sword and shield of their choice. When Seraphim enter combat they become surrounded by a coruscating aura (equivalent to Coruscate Rank 12).
\end{Description}


\end{multicols}

\end{document}
