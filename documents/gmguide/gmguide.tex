\documentclass{article}

\usepackage[gyre]{dq}

\title{GM's Guild}
\author{Ross Alexander}
\date{\today}

\begin{document}

\maketitle
\tableofcontents

\pagebreak

\begin{multicols}{2}

\section{Introduction}

This kit contains all publications, modifications, rule
interpretations, forms, notes etc.\ needed to run an adventure in the
multi-GM DQ campaign.

This kit should be returned after each session for updating and
reissue.  hence kits will only be in the possession of a GM who is
presently running an adventure and will always be up-to-date as at the
beginning of that session.

Where there are spare copies of forms these may be used freely,
however please ensure that you leave at least one.  Copies may be
freely made.

To maintain time consistency try to run your adventure in chunks of
three months real time (10--12 weeks or sessions).

The First Session Checklist should be followed on the first session of
an adventure.  Please observe the Ten Commandments.  These provide a
means of ensuring some consistency between GMs, and assist with game
balance.

\section{First Session Checklist}

\subsection{Part 1 --- Player}

Interview each player and do the following.

\begin{enumerate}

\item
If you so desire, fill in the GMs character record (this is so you
have a record of each character's vital statistics during play).

\item
Fill in the Adventure record (for Librarian).

\item
Verify player's character is a Guild member and has a copy of the
House rules.  Check the tribunal's approval.  Check that experience is
properly recorded and spent.

\item
Do any character generation required.

\end{enumerate}

\subsection{Part 2 --- Character}

With the entire party do the following.  The Guild Representative will
be present to ensure that the adventure is properly set up.  The
representative will remind the party of the House rules, suggest that
a leader be appointed, and demand that a scribe be appointed.

\begin{itemize}

\item
Introductions, descriptions, etc.

\item
Appoint scribe and give him/her the Adventure record you started in
part 1.  Appoint leader if so desired.

\item
Introduce the adventure.

\item
Preparation and departure.

\end{itemize}

\section{House Rules}

The Guild does not condone activities such as killing, theft,
extortion etc.  These activities (and others) are illegal in most
cultures we interact with.  Any crimes committed against society will
be answerable by the members concerned to that society.  The Guild
will not stand in the way of the normal course of justice.

You do not have to like, or even wish to adventure with, your fellow
guild members, but it is expected that you will avoid attacking,
deserting, endangering, injuring, stealing from, or withholding
treasure from your fellow guild members, and to avoid provoking other
members into performing such activities.  A breach of the standards of
behaviour expected between guild members will result in a full
investigation using all the powers and abilities of the guild.  The
matter will be dealt with severely!

\section{The Ten Commandments}

\begin{enumerate}

\item
Experience shell be awarded according to the described in the document
entitled ``Experience Point Awards.''

\item
Thou shalt observe the Laws of Shaping, which are:
\begin{enumerate}
\item
Law of Balance \\
The only shaped items that may be without flaws are those of minor
nature, for example, amulets, +5 +1 swords, etc.  Flaws cannot
be removed without also removing the useful abilities.  The more
powerful the magic, the worse will be the flaws.  The difference
between the powers and the flaws will represent the value (see
commandment 3).

\item
Law of Non-indestructibility \\
No shaped item is indestructible --- there will always be a way to
destroy it.  If any part of a shaped item is broken the magic is lost.

\item
No item will work always under all circumstances.  This is not a flaw
--- just a hole.

\end{enumerate}

\item
The absolute maximum value of any shaping is 50,000sp.

\item
The maximum net treasure that will be allowed per adventure is
dependent on the experience level: 1000sp for low level, 2000sp for
medium, and 3000sp for high level.  This is per session (approximately
4 hours of play time).

\item
The Guild will loan money to a member to buy back items in the
treasure split if the adventurer has insufficient funds.  Earlier
debts must be settled before further debts may be incurred in another
treasure split.  Note --- the Guild will loan only as much as is
required.

\item
Thou shalt avoid giving extra ranks or characteristics.  If you must,
then the maximums shall be as follows: characteristics 1, weapons 8
weeks, spells 1000ep.

\item
Time on all planes effectively passes as the same rate.

\item
Items, if of a post-renaissance technology, will either not work or
decay into uselessness. All non-technological items must be converted
into DQ terms so as not to destroy the flavour of the game.

\item
Thou shalt turn away characters if they will unbalance a party (reason
should prevail however).

\item
Thou shalt always adjudicate for the enjoyment of the majority.  Think
about this when rewarding individuals.  Adventuring with superman is
boring!

\end{enumerate}

\newpage

\section{Experience Points Awards}


Experience points are a measure of a character's ability to learn or
improve Statistics, Skills and Magic.  Experience is awarded to a
player's character based on the following four categories. Note that a
Session is considered to be 3.5 to 4 hours of real time play and
Sessions of greater or lesser duration should have their Experience
Awards adjusted accordingly.

\subsection{Attendance and Preparation}

\subsubsection{POTENTIAL AWARD: 0-500 EP / Session}

This category is awarded for turning up to game sessions on time and
prepared to play.  The full award should be given to characters whose
players turn up on time with character sheets completed and the
necessary accessories for play.  Courtesy by a player in informing the
Game-master when they may be late or unable to play due to other
commitments should result in some award.  Beginning players should be
given leeway in respect of not having complete character sheets.  A
player who is constantly distracted or distracts others when they are
trying to play is not properly attending should be given a reduced
award.

\subsubsection{GUIDELINES}

Full award for turning up on time and prepared to play.  75\% award
for incomplete character sheet.  50\% award for no character sheet or
unable to attend but supplying character sheet.  25\% award for
expected non-attendance without supplying character sheet.  No award
for unexpected non-attendance.  Lateness should result in an
appropriate adjustment to the entire experience award for the session
dependent on the degree of lateness.  (i.e.\ arriving 1 hour late
should result in 75\% of the total `normal' award for that session).

\subsection{Role-playing}

\subsubsection{POTENTIAL AWARD:  0-500 EP / Session}

This category is awarded for good role-playing, both characterisation
and player's enjoyment.  It is possible to play an obnoxious or
annoying character well but in such a fashion that other players
`enjoy' the character. This category is totally up to the Game-master's
judgement but should be based on consistency of motive and action and
enjoyment by the players and Game-master.  Take into account how well
the player differentiates between being in and out of character and
stays in-character when they should be.

\subsubsection{GUIDELINES}

Full award for believable and consistent characterisation which was
enjoyable by most present.  Half award for adequate role-playing or
good role-playing that annoyed players.  No award when a block of wood
could have done better or they managed to seriously and continually
annoy the Game-master or most the players.

\subsection{Contribution}

\subsubsection{POTENTIAL AWARD: 0-500 EP / Session}

This category is awarded for taking part in the action of the session,
either by providing ideas (or non-ideas) or by appropriate or
innovative use of the abilities or possessions of the character.  How
much did the player and / or their character contribute to the game
compared with how much they were able to contribute.

\subsubsection{GUIDELINES}

Full award for consistently and interestingly being a major
contributor to the action when possible and refraining from
contributing when unable to do so in-character.  Half award for
contributing occasionally when able to and also occasionally when
unable to.  No award for consistently contributing when unable to and
hindering other player's or character's appropriate contributions.

\subsection{Quest Level}

\subsubsection{POTENTIAL AWARD: 0-1500 EP / Session}

This category is awarded based solely on the difficulty and complexity
of the quest.  The adjudged level of difficulty and risk should be
announced at the beginning of the quest and the Game-master should
endeavour to abide by this level.  This means not making things harder
when the party is able to overpower the quest and also not being easy
on younger and less experienced characters when the quest seems too
tough for them.

\subsubsection{QUEST LEVELS:}
\begin{description}

\item[0 EP Very Low Level]
The characters will be confronted with little personal risk, with death
unlikely, low ability opponents, e.g.\  animals, low experience NPC's.
Problems will have readily obtainable solutions with little, low Rank
magic involved.

\item[300 EP Low Level] The characters will be confronted with some
personal risk with opponents of some ability, e.g.\ humanoid monsters,
NPC's with some experience.  Problems will have obtainable solutions
with some low Rank magic and a little, medium Rank magic involved.

\item[600 EP Medium Level] The characters will be confronted with
definite personal risk, and possible death, with opponents of equal
ability, e.g.\ lesser undead, monsters, experienced NPC's.  Problems
will require thought and include magic, though little magic above
medium Rank.

\item[900 EP High Level] The characters will be confronted with high
personal risk, including the likelihood of death, with tough
opponents, e.g.\  greater undead, devils, very experienced mages.
Problems will be difficult and involve much magic and/or ingenuity,
but not entirely high Rank magic.

\item[1200 EP Very High Level] The characters will be confronted with
death, possibly irresurrectable, with very tough opponents, e.g.\
powerful groups, individual dragons or Powers.  Problems will require
thought, ingenuity and magic to solve and will involve any (and
probably all) magic.

\item[1500 EP Extreme Level (Certain Death or World-Saving)] The
characters will be confronted with almost certain death, probably
irresurrectable, with virtually unbeatable opponents, e.g.\ groups of
Powers, Empires.  Problems will be all but unsolvable and will involve
any and all magic plus stuff not readily possible within the rules.
\end{description}

\end{multicols} 

\pagebreak

\include{aspect}

\begin{multicols}{2}

\setcounter{secnumdepth}{3}

\include{powers}
\include{agents}
\include{invocations}
\include{elohim}
\end{multicols}

\end{document}
