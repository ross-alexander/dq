\section{The Powers (Version 2.0)}

\subsection{Overview}

This document is not a list of the various Gods, Demons and other
Powers within the DQ universe. Rather it contains information for GMs
on the nature of Powers, their impact and interactions with the game
world, and their abilities and limitations. It includes a design
system for minions, those lesser beings that the Great Powers often
send out to do their bidding, and also information on the Powers'
mortal Agents and the pacts that bind the two together.

\subsection{Background}

The DQ universe, and most of the things in it, was created through the
intervention of various Great Powers, commonly known as Gods, or
Celestials. These beings still exist into the present age, and though
their ancient Covenant prevents, or at least reduces, their direct
meddling in the affairs of mortals, they remain a powerful force,
whose presence may be felt in day to day life, through the works and
deeds of their mortal worshipers, followers and Agents.

\subsection{Types of Great Power}

\subsubsection{Gods}

Gods are the greatest of the immortal beings and most have existed
from the very beginning of the present universe, and predate all
mortals by an immeasurable time. They are in many ways similar to
their lesser cousins the Demons, with two important differences.
Firstly, Gods very seldom send Avatars into the material worlds,
preferring instead to work through more subtle means such as visions
and prophecies. Some do use mortal Agents, but very seldom offer much
in the way of direct aid. Some even have servant Demons who do their
more direct bidding. Secondly, the power of Gods is in some way
influenced by the quantity and faith of their mortal worshipers,
although little is known about this process even by the most learned
sage, and unsurprisingly scant information is forthcoming from the
Gods themselves.

Some Gods are concerned with particular animals, objects or concepts,
and have few interests outside of these, whilst others have more
wide-ranging goals and ambitions. The Gods vary widely in their
natures, ranging from concepts and stances that mortals would call
ethical and good, through to the blackest evil.  There are various
hierarchies and groupings (often referred to as pantheons), amongst
the Gods, but usually each employs quite different minions.

Following a devastating war amongst the Gods when the universe was
newly formed - a war that saw the destruction of many material planes,
and Gods alike - the remaining Celestials made a solemn covenant that
they would never again make war on one another directly.  Respect for
the Covenant is another issue on which the Gods differ, some will
never willingly disobey it, while others will do everything except
defy it openly.

GM Note: GMs should exercise caution before introducing whole new
pantheons into Alusia. Two pseudo-historic pantheons are already known
to exist; a Norse one, and a Celtic one. These have been used by GMs
as the ``Old Gods'' of the Dwarves and Elves respectively.


\subsubsection{Demons}

Since the creation of the material universe, various lesser Powers,
usually called Demons have come into being. These entities, unlike
Gods, were once mortal creatures, who have managed to continue their
personal existence beyond death to become an immortal. It is believed
that some Demons have gone on to become ``true'' Gods, although almost
nothing is known about how this might be accomplished. Demons exist on
the spirit-plane adjacent to a material world or a group of such
worlds. They tend to take a much greater interest in the worlds that
they lived in when mortal, and summoning them is possible if the
correct rituals are known.

The earliest Alusian Demons date from the time of the first mortal
race, the Dragons, but the majority appeared at the time of the fall
of the Elven Empire. There are two major factions amongst this group
of Demons, and they continue beyond death, the struggle they started
in life.

\subsubsection{The Powers of Darkness}

This large faction of Demons are those immortals that were once the
most potent and dire of the Drow mages, or are a ``chimera'' of several
lesser mages, whose life essences have fused together. The Powers of
Darkness (PoD) are all generally chaotic in outlook and, by mortal
standards, unethical in nature. Many have retained the titles of their
temporal power: Duke, Prince, etc., so whilst they appear to have a
hierarchy, it is mostly a convention dating from their time as
mortals. The minions of these Demons are well known to mortals, and
include: Succubi/Incubi, Devils, Imps and Hellhounds. The Powers of
Darkness pursue their various goals in the material world through the
use of mortal Agents, although it is not unknown for them interfere
directly.

\subsubsection{The Elohim}

Also known as the ``Powers of Light''. The Elohim were once a mortal
group, convened in the twilight days of the Elven Empire, in an
attempt to reverse the decadent trends that ultimately destroyed elven
civilisation. Their apotheosis was a planned and deliberate act, so
that they might stand against the Powers of Darkness, as some form of
protection for mortals in later ages of the world. The Elohim obey a
strict hierarchical order. The highest are the Archangels, then the
Bene Elim, Malakin and Aishim, and finally the servant Elohim, the
Erelim, Kerubim and Seraphim. The Elohim, being opposed to the highly
magical drow Demons, and seeing the excessive use of magic as the
prime cause for their ancient fall, will seldom use Mages as
Agents. Their Agents are often known as Clerics or Priests, and it is
generally through them that the Elohim pursue their goals, although
they may occasionally directly ``help'' mortals, for their own good.

\subsubsection{Other Demons}

Other mortal beings as well have undergone the apotheosis into a Demon
upon their death. The most notable are the ``Emperor Demons'', great
dragons from the first age of the world, whose power was subsequently
broken by the upstart demonic drow, and their spirits imprisoned in
material worlds. It is still possible today for a strong enough mortal
(or a group of like-minded ones) to carry their personal existence
beyond death, to become a new Demon.

\subsection{Avatars}

The differences outlined aside, the Powers are reasonably equivalent
in their attributes, and abilities. Whilst Gods certainly have move
power and a wider influence than Demons, they are limited by their
covenant in the amount of raw power that they may manifest in the
material worlds. Each Power has a sphere of influence, goals and
motivations, a form or forms that they commonly appear in, a code of
ethics that they expect their followers to adhere to, along with a
number of common physical properties.

\subsubsection{Manifestation}

When a Power manifests in the material world, the manifestation is
only a small portion of its total being. This manifestation, or Avatar
cannot be killed, though it may be disrupted sufficiently so as to
cause it to dissipate back to the home dimension of the Power. Each
Avatar created by a Power may have slightly different portions of the
Power inherent in it, thus giving it somewhat different attributes and
abilities. A Power may have only one Avatar manifest in any material
world at any one time.  Avatars may be summoned to the material world
by a variety of Spells or Rituals, or by the Call Patron talent of an
Agent.  A Power may not send an Avatar to the material world without a
summoning or some other form of "gateway" or "portal" being available.
In some circumstances, a Power may cause minions to manifest in the
material world without such a "portal".

\subsubsection{Appearance}

A Power has a number of options regarding the appearance of its
Avatar.  An Avatar may appear in any of the forms used by the Power,
and with or without any of all of the accoutrements associated with
the Power.  In addition, the Avatar may be accompanied by a number of
Beast minions (usually no more than 5).  These will be of the type
commonly used by the Power.

\subsubsection{Death \& Damage}

The damage and "death" of an Avatar is handled differently to that of
a mortal.  Avatars may only be harmed by magic, weapons of magical
nature, and by silvern metals.  When an Avatar has taken sufficient
damage that it would have been slain, or made unconcious or insensate,
had it been mortal, it is dissipated back to its home dimension. Most
Avatars cannot be Stunned. If an Avatar is dissipated due to damage
taken, or by having the spell that summoned it to the material world
dispelled, counterspelled or dissipated, the Power that formed it may
not usually send another Avatar to that material world for a period of
about thirty days.  If the Avatar is sent back by its summoner, it may
not reappear in the material world until the next day.

\subsubsection{Magical Abilities}
The magical abilities of an Avatar are different to those of mortals.
Avatars that use magic are not restricted to the standard College
system, but rather possess magic which is consistent with their
Sphere.  An Avatar may possess any number of Talents, Spells and
Rituals, from any number of Colleges.  The magic that a Power
possesses should be consistent with their Sphere.  The individual
Talents, Spells and Rituals that a Power possesses may be considered
as Colleged for the purposes of Counterspells, Protections, and the
like.  When casting Talents, Spells and Rituals, the Avatar uses the
Base Chance modifiers of the College to which the ability belongs.

All Avatars have this lack of restriction on their magical abilities,
though some possess far more magic than others.  This difference is
determined by the Power's Sphere.  The Avatars of scholarly or
mystical Powers will possess much magic, while those of more
physically orientated Powers will have less.

\subsubsection{Cold Iron}
Avatars are not prevented from exercising their magical abilities by
the presence of Cold Iron.

\subsubsection{Ranks}
The magical abilities of an Avatar are usually practised at Rank 20,
although the Avatars of particularly physical Powers may practise
their magic at lower Ranks.

\subsubsection{Physical Forms}
An Avatar may use its magical abilities regardless of the physical
attributes of its form.  An Avatar's form need not be capable of
vocalisation, or of complex manual manipulation for the Avatar to
exercise its magical abilities.  However, if the form can vocalise
and/or complete intricate hand manoeuvres, it must do so to cast its
magic.  It should be noted that in most cases the outward form of a
Power is entirely cosmetic, and that while an animal of the type
portrayed may not be possessed of the correct organs required for
vocalisation, this muteness does not apply to a Power in that form
unless explicitly so stated under the description of the individual
Power.

\begin{example}
Samigina, "Marquis of Dead Souls", may choose to appear as a small ass
or as a human.  Whilst in ass form he may cast his magic without the
use of hand gestures, if however, he takes on human form, he may not
choose to omit the hand gestures while casting.  Whilst in ass form he
speaks with a braying voice, despite the fact that asses, as a rule,
are incapable of speech.
\end{example}

\subsubsection{Changing Form}
Avatars may change forms as a Pass action.

\subsubsection{Weapon Skills}
Avatars will always possess (maximum Rank + 1) with any weapons that
are part of their Power's accoutrements. An Avatar will commonly
possess maximum Rank in up to 5 other weapons and high ranks in up to
10 more. There are exceptions to this rule, and some particularly
war-like Powers will manifest Avatars who possess maximum Rank (or
better) in almost any known form of weaponry.

\subsubsection{Languages}
Avatars will usually possess maximum Rank in many languages (at least
one of which will be known by their summoner), and will sometimes
possess maximum Rank in any other languages in the same language group
or groups.  The Avatars of particularly scholarly Powers may possess
more languages than this, some possessing all known languages and
dialects at maximum Rank.

\subsubsection{Skills}
Avatars will generally possess all the skills within the sphere of
their Power at Rank 10, or in some cases, at even greater Rank. If the
Power that the Avatar represents is particularly tied to a skill, the
Avatar may possess as high as Rank 15 with that skill. The exact uses
of certain skills at above Rank 10 are left to the discretion of the
individual GM.

\subsubsection{Special Abilities}
Many Avatars are possessed of special talents or abilities that are
connected to their Power's sphere, appearance or other attributes.
Avatars may be horrendously ugly, inspiring fear in their viewers, or
may dazzle the eye with their power or beauty.  They may have special
teaching abilities that violate the standard rules, or they be able to
bestow abilities on their followers or Agents.  All of the special
abilities available to an Avatar must be detailed under the individual
Power.

\subsubsection{Minions}
A Power possesses a legion of lesser beings. The legions of a Power,
(i.e. their supernatural servants and followers) can generally be
divided into four categories: Greater minions, Lesser minions, Minor
minions, and Beasts.  In addition to these four types of supernatural
follower a Power may have one or more types of natural Animal minion
associated with it. Both the Powers of Darkness and the Elohim have
"standardised" minions; the PoD have Incubi/Succubi, Devils and Imps;
the Elohim have Erelim, Kerubim and Seraphim. The Demons commonly use
Hellhounds, whilst the Elohim often use sentient Pegasi. The
individual Demons employ a variety of natural animals, the exact type
being dependant on their personal "sphere".  The differences between
the minions of the PoD and the Elohim, and those of the Celestials, or
other Demons, who have no "standards", are less than they might at
first seem.  Whilst there are obvious cosmetic differences between an
Imp and a Seraph, the amount of power that each wields is roughly
equal. Their attributes are within similar ranges, and the only real
difference is in their appearance and special abilities.  Imps are
tremendously ugly, possessed of a poisonous attack with their tail,
and cause fear, whilst Seraphim are inhumanly beautiful and gain some
measure of defence due to their dazzling aura.

\subsubsection{Format}
The format in which a Power is described is standard.

\begin{Description}
\item[Name] Common Name of the Power
\item[Title] Common Title or Sobriquet
\item[Type] God or Demon
\item[Group] The group or faction to which this power belongs: Elohim, Powers of Darkness, Elven gods, other?
\item[Response chance] The percentage chances that the Power will respond to an invocation / respond to any use of their name (only if not 0\%).
\item[Description] In what forms does the Power commonly appear? What accoutrements does the Power appear with?  What affect does the Power have, merely by its presence?  List: Species, race, sex, colour, clothes, weapons, companions, language, accent, what does the Power look like, sound like, smell like, emotional effects; fear, lust, anger, depression, etc.
\item[Sphere] To what area of power is it linked? Where do its aims and goals lie?
\item[Personality] What is the Power like ethically, morally, what is its position on: truth, justice, theft, murder, good, evil, life, death, love, hate, fun, pain, etc.
\item[Greater minions] What type/form/look do the Power's Greater minions take?  (Is this a standard variety, or is there a variation from the standard, or is it a special type of minion specific to this Power?)
\item[Lesser minions] As for Greater minions.
\item[Minor minions] As for Greater minions.
\item[Beast minions] As for Greater minions.
\item[Animal minions] What animals, if any, are closely associated with the Power.
\item[Agents] What Skills, Colleges, Weapon abilities, etc. does the Power insist/prohibit its Agents from having?  How must the Power's agents behave/not behave? what ethics/morality is expected of them?  What must they attempt to do/not do?  How does the Power expect its Agents to serve it?  Does the Power grant Familiars or Companions?  Do these always have some particular form or ability?
\item[Areas] What geographical areas/economic types/political arenas is the Power most/least likely to be found in?
\item[Colours \& symbols] are there any colours, symbols, calls, songs, etc. associated with/prohibited by the Power?
\item[Avatar abilities] Does the Power's Avatar commonly possess some special ability?  What statistic values, or range of values, are associated with the Avatar? Strength, Dexterity, Agility, Magical Aptitude, Willpower, Endurance, Fatigue, Perception, Physical Beauty, Movement Rates, Natural Armour? What natural weapons and attack forms does the Avatar possess?
\item[Avatar summoning] Can the Avatar only be/not be summoned into a particular type of area?  Must some particular type of "sacrifice" be made?
\end{Description}

\begin{example}
\begin{Description}
\item[Name] Aim: "The Fire Duke"

\item[Type] Demon

\item[Group] Powers of Darkness

\item[Response] 15\%

\item[Description] Aim always appears as a man with three heads dressed
in crimson robes with the design of flames burning up from the hem.
His central head is human, the left one that of a serpent, and the
right is that of a calf.  He bears two stars of the forehead of his
human head.  In his left hand he carries a ball of eternally blazing
fire.  He rides a huge lizard with scales of midnight blue.  Wherever
he goes, Aim is surrounded by billowing clouds of sulphurous, red
tinged smoke.  All three heads may speak, the human one with manic
laughter, the serpent with sibilant menace and the calf with a
mournful lowing.  Aim never appears with or wears armour, nor does he
appear with weapons.  Whilst he will use small one-handed weapons, he
is not primarily a fighter.

\item[Sphere] Aim's sole interest is in Fire and he possesses all magic
related to it.  He will gladly begin a blazing inferno merely for the
pleasure of watching it burn, and believes that one day the entire
universe will be consumed in a conflagration of cosmic proportions.

\item[Personality] Aim is almost entirely insensitive to the desires of
mortals, wishing only to see his beloved fire propagated.  To Aim,
fire is the basis of all things, "creation through destruction",
whilst water is the slayer of life.  Aim will immolate Adepts of the
College of Waters Magics without a moments hesitation.  Otherwise, his
attitude to mortals is one of complete disinterest, except for his
Agents, in whose fiery ambitions he delights.

\item[Greater minions] Sentient Fire Elementals (equivalent to those
summoned at Rank 20, except with an MA of 0, otherwise standard).

\item[Lesser minions] Devils (only ever of the Fire College, otherwise standard).

\item[Minor minions] Imps (never Water College, otherwise standard).

\item[Beast minions] Hellhounds (standard).

\item[Animal minions] Salamanders (standard).

\item[Agents] Aim will only accept Adepts of the College of Fire Magics
as Agents.  His Agents must create fire wherever and whenever possible
and are prohibited from extinguishing a fire for any reason.  They
must never associated with Adepts of the College of Water Magics, and
are encouraged to slay them whenever possible.  Upon becoming a Agent
of Aim the mage immediately loses the Extinguish Fire Spell (along
with any Ranks they may have had in it), and are forbidden from ever
relearning it.  In addition the Agent immediately gains great insight
into the workings of the Bolt of Fire Spell.  This spell will
thereafter have its EM lowered by 100, and become General Knowledge,
for them alone. Should the Agent lose the spell for some reason, they
may always relearn it with its lower EM. Aim grants his Agents
Familiars which are always Salamanders in their animal form.

\item[Areas] Cults of Aim worshippers are almost invariably found in
large cities.  They go under various "fire" related names such as "The
Children of the Flame" or the "Red Redemption".  The cults have been
known to offer to redeem the souls of the proprietors and customers of
gambling and bawdy houses for large cash sums.  Many choose to pay, as
it is not unknown for those that do not to have their establishment
"cleansed by fire". Most civil authorities are strongly opposed to the
activities of the cults of Aim and will take measures to eradicate
them.

\item[Colours \& symbols] Agents of Aim dress as Fire Mages of the most
extreme sort, in blazing crimsons and oranges, more often than not
decorated with depictions of roaring fire.  Both the symbol of a ball
of fire, and the slogan "creation through destruction" are common.

\item[avatar abilities] Aim is a master of Fire magic and a potent
Alchemist.  He is also a skilled Military Scientist. He can
automatically set fire to any combustible object merely by touching it
with the ball of fire in his left hand.  The ball may not be thrown.
He is neither a great scholar nor a fighter.  Aim is extremely ugly
and can inspire fear in those that view him. His lizard is in all
ways, except colour, a giant Salamander

\item[Movement Rate] (yards per minute): Run: 250

\begin{tabular}{llll}
PS: 22	& MD: 24	& AG: 23	& MA: 30 \\
EN: 25	& FT: 35	& WP: 34	& PC: 26 \\
PB: 3 	& TMR: 5	& NA: 3 DP \\
\end{tabular}

\item[Weapons] Aim may bite in Close Combat or Melee with his non-human
heads. If the serpent head does effective, the target also suffers an
additional D-2 damage per Pulse, for D10 Pulses, from poison.  The
ball of fire causes D+8 damage per pulse that a target is in contact
with it. The damage done by the ball is magical fire damage and may be
resisted, for half damage

\item[Avatar summoning] Aim may only be summoned into an area where
fire could burn, and if there is not a fire burning in the vicinity he
will insist that one is lit immediately, or will create one himself.
\end{Description}
\end{example}


\subsection{Minions}

Most supernatural minions are, in effect, tiny fragments of their
Patron power. The way in which minions work has been standardised for
ease of use by GMs.

\subsubsection{Classes}

The classes into which a Power's minions are divided are standard.
The standard classes for minions are: Greater, Lesser, Minor, and
Beast.  Animal minions are mortal, rather than supernatural, and
sections 5.2 through 5.6 inclusive, do not apply to them.  Whilst not
all Powers possess minions corresponding to all of the standard
classes, those that they do have will fall into one of those classes.
Some Powers may also be said to have "standard" minions, that is to
say, they may use minions that are also found naturally or are used by
more than one of the Powers.  "Standard" minions may be found detailed
in the Bestiary.

\subsubsection{Death \& damage}
The damage and "death" of a Minion is handled differently to that of a
mortal.  Minions may only be harmed by magic, weapons of magical
nature, and by silvern metals.  When a Minion has taken sufficient
damage that it would have been slain, had it been mortal, it is
dissipated back to its home dimension.  If a Minion is dissipated due
to damage taken, or by having the spell that summoned it to the
material world dispelled, counterspelled or dissipated, it may not
usually reappear in that material world for a period of thirty days.
If the Minion is sent back by its summoner, it may not reappear in the
material world until the next day.

\subsubsection{Magical abilities}

The magical abilities of a Minion are different to those of
mortals. Minions are not bound by the Magical Aptitude restriction
that limits the number of lowly ranked abilities that a mortal Mage
may possess. Minions may not possess magic from more than one
College. Unless stated otherwise in the description of their
particular Power, Minions may be of any College. The magical abilities
of a Minion are practised at the Rank listed in their description.

\subsubsection{Cold iron}

Minions are not prevented from exercising their magical abilities by the presence of Cold Iron.

\subsubsection{Physical form}

A Minion may use its magical abilities regardless of the physical
attributes of its form.  A Minion's form need not be capable of
vocalisation, or of complex manual manipulation to exercise its
magical abilities.  However, if the form can vocalise and/or complete
intricate hand manoeuvres, it must do so to cast its magic.  A Minion
that possesses magical abilities may use them in all of its forms.  It
takes a Pass action for a Minion to change from one form to another.

\subsubsection{Experience}
Minions may become more powerful over time.  Minions that are sentient
may gain experience and increase their abilities, as may mortals.  In
this way, over an extremely long period, Minor Minions may be
"promoted" to Lesser, Lesser to Greater, and, over several aeons, a
Greater Minion may even become a Power in its own right.  The time
taken for a Minion to achieve these increases in its home dimension is
beyond even the lifespan of Elves, and even if the same Minion is
encountered more than once in a character's life, little or no
difference will be noted.  The only exception to this is when a Minion
is assigned to the material world for an extended length of time, such
as an Agent's Familiar or Companion. These posts are usually given to
Minions who have pleased the Power in some way, or who require
training of some kind.  Minions in the material world may learn and
train in exactly the same manner as mortals.  The masters of Minions
who are assigned as Familiars or Companions, may transfer up to 10\% of
the Experience Points they gain, to their Familiar or Companion.
These Experience Points may then be spent by the Minion.  Minions who
are operating independently may gain their own experience.  It must be
noted that the abilities listed for Minions in the Bestiary, or
elsewhere, are for generic Minions and in no way limit a particular
Minion's learning capacity.

\subsubsection{Generic types}
The generic type for a Minion is created in the format detailed below,
and has abilities and Attributes within the designated ranges.  It may
have magical abilities as detailed, and in addition, may be possessed
of one or more Special Abilities.  The Special Abilities of a Minion
may be of almost any nature and are commonly such things as:
Attributes beyond the specified ranges (eg. extreme ugliness or
beauty), spells possessed as Talents, poisonous attacks, knowing a
Skill, or divinatory ability.  All Minions of a generic type will have
the abilities and Special Abilities of that type, and Attributes the
same as, or very similar to, those listed as generic.
 
\subsubsection{Format}

The format for a generic type is:

\begin{Description}
\item[Name] the Generic type of the Minion.
\item[Class] the type of Minion: Greater, Lesser, Minor, Beast.
\item[Description] what the Minion looks, smells, sounds, feels like, what the generic personality type of the Minion is.  What forms the Minion may take.  As a general rule, Greater Minions are far more impressive than Lesser ones, and Lesser more than Minor.  If the Minion is a Beast, is it sentient?  The personality of a Minion is usually that of its patron Power, or of a facet of that patron.
\item[Talents, skills \& magic] what Skills, magical abilities, and Special Abilities the Minion possesses.
\item[Movement rates] the rates of movement of the Minion's various forms.
\item[Attributes] the Attributes, or Attribute ranges for individual Minions.
\item[Weapons] the facility of the Minion with its form's natural weapons, along with any other weapon skills it can possess.
\end{Description}

\subsubsection{Minion powers}

The number and range of abilities, Skills, Special Abilities, Attributes, and weapon skills of a generic type are dependant on its class:


\begin{Description}
\item[Greater Minions]

\begin{Description}
\item[Forms] 1-3 
\item[Skills] usually practised at Rank 10.  May have 0-2 Skills, more if taken as Special Abilities.
\item[Magic] 0-1 College only, usually possess all College magic at Ranks 8-15. May also know non-College abilities such as Ward or Geas.
\item[Special abilities] 0-5
\item[Movement] 0-3 modes of locomotion per form.  A movement rate of beyond 500 yards per minute is a Special Ability.
\item[Attributes] maximum value of 30.  Beyond this value the Attribute is a Special Ability.  Physical Beauty causing Awe or Fear is a Special Ability.  Counting statistics above a value of 30 as 30, Primary statistics should not exceed 160.  Fatigue should be reflective of Endurance.  Perception should be very high, (25-30). TMR may be figured from Movement Rate and should also reflect Agility.  Greater Minions may have 0-6 points of Natural Armour as standard, more than 6 points constitutes a Special Ability.
\item[Weapons] Any natural weapons possessed by their forms at Ranks 7-10 (but not over max).  0-10 other weapon skills at Rank 0-max. Natural weapons that do more than D+4 damage are Special Abilities.
\end{Description}

\item[Lesser Minions]
\begin{Description}
\item[Forms] 1-2
\item[Skills] 0-1 Skills.
\item[Magic] 0-1 College only.  May possess all College magic at Ranks 6-12. Non-college magic only as a Special Ability.
\item[Special abilities] 0-4 
\item[Movement] 0-3 modes of locomotion per form.  A movement rate beyond 500 yards per minute is a Special Ability.
\item[Attributes] maximum value of 30, but no more than half of their primary statistics should exceed 26.  Fatigue should be reflective of Endurance.  Perception should be high, (20-26). Physical Beauty causing Awe or Fear is a Special Ability.  Primary statistics should not exceed 130.  TMR may be figured from Movement Rate and should also reflect Agility.  0-3 points of Natural Armour standard.
\item[Weapons] Any weapons natural to the form at Ranks 6-10 (but not over max).  0-5 other weapons at Rank 0-10.
\end{Description}

\item[Minor Minions]
\begin{Description}
\item[Forms] 1-2
\item[Skills] are only possessed as Special Abilities. If possessed are Rank 0-7.
\item[Magic] 0-1 College only. May possess all College magic at Ranks 0-12.
\item[Special abilities] 0-3
\item[Movement] 0-2 modes of locomotion with each form.  A movement rate of beyond 500 yards per minute is a Special Ability.
\item[Attributes] maximum value of 25.  Primary statistics should not exceed 100.  FT may be figured from EN as for Humans.  Perception will be around 15, and PB may not exceed 25, or cause Awe or Fear except as the result of a Special Ability.  TMR may be figured from Movement Rate and should also reflect Agility. Natural Armour 0-2 standard.
\item[Weapons] Any weapons natural to the form at Ranks 4-8 (but not over max).  0-2 other weapons at Ranks 0-5.
\item[Comments] As these Minions are granted to characters as Familiars and Companions, random methods may be offered for some of their abilities, when designing the generic type.
\end{Description}

\item[Beast Minions]
\begin{Description}
\item[Forms] 1
\item[Skills] None if non-sentient, else are only possessed as Special Abilities. 1 Skill per Special Ability.  If possessed are Rank 0-8.
\item[Magic] None if non-sentient, else 1 College only, possessed as a Special Ability. May have Rank 0-9 with general magics and Rank 0-5 with special magics.
\item[Special abilities] 0-4
\item[Movement] 0-2 modes of locomotion.  A movement rate of more than 500 yards per minute is a Special Ability.
\item[Attributes] maximum value of 25.  Primary statistics should not exceed 110. FT should reflect EN. TMR may be figured from Movement Rate and also reflect AG.  Perception is often high, around 20. PB may not exceed 25, or cause Awe or Fear except as the result of a Special Ability. Natural Armour 0-5 is standard.
\item[Weapons] Any weapons natural to their form at Rank 5-10 (but not over max). Will not usually be able to use other weaponry, but if allowed by the form may have 0-3 weapons at Rank 0-5. Natural weapon damage in excess of D+4 is a Special Ability.
\end{Description}

\item[Animal Minions]
\begin{Description}
\item[Special abilities] 0-1.  The only Special Ability that an Animal Minion may have, beyond the normal abilities of the animal itself, is sentience. Sentient Animal Minions have an MA of 0. The Animal Minions sent to spy on, or communicate with an Invoker are usually sentient.
\item[Comments] Animal Minions may be of any type of material world creature that is normally non-sentient. They will have all of the normal abilities associated with that creature.  The animal may be an Enchanted one.  
\end{Description}
\end{Description}
\begin{example}
"Aim, The Fire Duke" uses Salamanders as Animal Minions.  These Salamanders retain the ability to cause fire with their gaze.
\end{example}

\subsubsection{Special ability limit}

The generic type of a Greater, Lesser, Minor, or Beast Minion may have
more Special Abilities than the number listed, provided that such
abilities are exclusive.

\begin{example}
Hellhounds, a Beast Minion type, are listed in the Bestiary as
having a PC of 25-30, a PB of 4-6, and 3 other Special Abilities.  A
Hellhound may be found that had a Perception of more than 25 or a
Physical Beauty of less than 5, but not both.  In either case the
number of Special Abilities possessed by an individual Hellhound does
not exceed the maximum of 4 listed for the Beast type Minion.
\end{example}
