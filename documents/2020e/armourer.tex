\begin{Skill}[1.3]{armourer}{Armourer}

\section{Restrictions}

The skill is related to that of weaponsmith, and an armourer who is a
more skilled weaponsmith expends only three-quarters of the necessary
Experience Points to acquire or improve this skill.  The reverse is
also true.

An armourer’s progress in their skill is inhibited by a low Manual
Dexterity, and aided by a high Manual Dexterity.  An armourer has an
increased Experience Point cost of 5\% for each point of Manual
Dexterity less than 16. An armourer decreases their Experience Point
cost by 5\% for each point of Manual Dexterity greater than 20.

\section{Benefits}

An armourer acquires the ability to make one category of armour every
two ranks.

Some categories require other categories as prerequisites and cannot
be learned before their prerequisites.  All armourers begin with the
cloth category at rank 0.

\begin{dqtblr}{colspec={Xm{6em}}}
Categories					& Prerequisites \\
Cloth						& None \\
Leather (leather, soft leather and furs)	& Cloth \\
Scale (scale and full scale)			& Cloth \\
Chain mail					& Cloth \\
Partial plate					& Chain  \\
Plate I (full plate and heavy plate)		& Chain \\
Plate II (improved plate, jousting armour)	& Plate I \\
Dragon skin					& Scale, Leather \\
Mithril						& Chain \\
\end{dqtblr}

Additional categories may be gained without increasing in rank by
spending 5,000 Experience Points and 4 weeks training time per
category.  These costs are discounted by 25\% if the armourer has
reached rank 8, or by 50\% if they have reached rank 10.

\precis{An  armourer  can  make  increasingly  effective 
armour as their rank increases.}

An armourer may positively affect any of the 4 attributes of armour
(Weight factor, Protection, Agility Modifier and Stealth Modifier) or
any combination thereof.  Some of the attributes are harder to affect,
and this is reflected in the number of ranks an armourer must have to
do so.  Also, some of the attributes have maximums (e.g. the Agility
Modifier may not be decreased beyond 0).  The ranks required and the
attribute maximums are:

\begin{Description}
\item[Weight] 1/2 a factor per 3 full ranks. Never lighter than WT 1.
  This attribute may not be affected for the cloth, leather or mithril
  categories.

\item[Protection] +1 per 4 full ranks.  This attribute may not be
  affected for cloth, furs or soft leather, and no more than 1
  additional point of protection may be added to hard leather.

\item[Agility Modifier] 1 per 6 full ranks.  Never better than 0.

\item[Stealth Modifier] +1\% per rank. Never better than +5\%.

\end{Description}

Note: These effects are not cumulative. For example a rank 7 armourer
could make a suit of armour with 1 less weight factor and 1\% better
stealth, or 1/2 a weight factor less and 1 point more protection, or
any of the other non-cumulative combinations.  An armourer may always
make a suit of armour at a lower effective rank than their true rank.

Armour statistics shown on the Alusian Armour Chart are for armours
manufactured with an effective rank of 0, i.e. of the mass-produced,
off the peg variety.  The armourer who made them may have been of
greater rank but the level of skill used was elementary.

The time and cost required for an armourer to construct a suit of
armour is dependent on the effective Rank used and the category of
armour.

Time The number of days required to construct a suit of armour in a
properly equipped and staffed workshop is effective rank plus the base
number for the armour listed below:

\begin{dqtblr}{colspec={Xl}}
Categories		& Time \\
Cloth or leather	& 5 days \\
Scale			& 10 days \\
Mail			& 20 days \\
Partial plate		& 25 days \\
Plate I			& 30 days \\
Plate II		& 35 days \\
Dragon skin		& 20 days \\
Mithril			& 30 days \\
\end{dqtblr}

An Armourer with greater rank than the effective rank being applied
may reduce the construction time by (Rank − Effective Rank) days
(minimum 1 day).

The fitting time for the armour (the time spent with the armourer by
the wearer-to-be) is a number of hours equal to the base number of
days (e.g. 20 hours to fit a suit of mail).  The hours are not
con-secutive and may be reduced by (Rank − Effective Rank) hours
(minimum 1).

Cost 80\% of the Base Cost as shown on the Armour Chart × (Effective
Rank + 1) silver pennies.  Note that this is the cost to the armourer,
not the sale price.

\subsection{Fixing and Modifying Armour}

The time taken to repair a suit of armour damaged by a Grievous blow,
or to modify a suit to fit a new (but appropriately sized) wearer, is
usually the same as the original fitting time.

\precis{An armourer is treated as a merchant of their armourer rank when
attempting to buy or value armour from categories with which they are
familiar.}

If the armourer is not familiar with an armour category they act as a
merchant of half their rank (rounded down).

\section{Costs}

\precis{An armourer can only perform their skill in a properly maintained
workshop.}

It costs 2000 silver pennies to construct a workshop and 500 silver
pennies per year to maintain it with tools and materials. A basic tool
kit will cost (100 + 100 × Rank) silver pennies.  It costs only 20\%
of the above amount to add to a weaponsmith’s workshop so as to make
it usable by an armourer as well. The reverse is also true.  A
workshop may be rented at the rate of 10 silver pennies a day.

\end{Skill}
