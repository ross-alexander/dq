\begin{Chapter}{Non-College Magic}

\section{Introduction}

All spells and rituals listed in this section can be learnt by Adepts
of any college.


\section{Special Knowledge Spells}

\begin{ritual}{Geas}

\range{ The  Adept  must  be  able  to  see  and  communicate with the target}
\duration{Until removed, fulfilled or target dies}
\multiple{250}
\basechance{Always successful (see below)}
\resist{Special}
\target{Sentient entity}
\begin{effects}
A geas is an obligation to complete a quest, an injunction against the
performance of a particular action, or a requirement to respond in the
same fashion to particular stimuli.

The target must acknowledge their acceptance of the geas.
Furthermore, either the Adept must believe that the target deserves
the geas, or the target must truly wish (not forced by physical or
magical means) to have an unmerited geas placed upon them. The Adept
specifies the nature of the geas in 25 words or less, and the GM will
use the most liberal interpretation of that wording to the benefit of
the target.  Rank with the geas spell does not affect the chance of
casting the spell as it is always automatically successful.  The Rank
equals the effectiveness of the geas, expressed in percentage terms.
If a geased entity directly contravenes the letter of a geas, they
have a chance of dying equal to the Rank of the geas. A geased entity
will begin to feel weak or ill when they first take an action counter
to the restriction of the geas, and will be- come increasingly
afflicted until they once more comply with the geas.

If a quest geas is fulfilled by the geased entity, they are no longer
subject to that geas.  The other two types of geas (for and against a
given action) last indefinitely.  A geas can be removed automatically
by the one who placed it. A geased entity will not attempt to free
themselves from the compulsion.  An Adept may attempt to remove a geas
if they have a higher rank than the geas in effect.  The Adept must
inscribe a triangle about the geased entity, and perform the ritual of
geas removal for 12 consecutive hours.  If the triangle is silver or
truesilver, the geased entity does not suffer the penalties for
ignoring the geas during the ritual.

The Adept attempting to remove the geas has a success chance equal to
five times the difference between their Rank with geas and the Rank of
the geas being removed. The GM rolls percentile dice: if the roll is
less than or equal to the success percentage, the geas is removed.  If
the roll is greater than the success percentage, the Rank of the geas
is increased by one.

\subsubsection{Full Geas}

An Adept with Rank greater than 15 with the geas spell has the power
of full geas.  A full geas can be placed upon an entity without their
consent, though it can be passively resisted.  Additionally, one with
the power of full geas may automatically remove (without the support
of a triangle and 12 hours of ritual) a geas which is at least 5 Ranks
less than their Rank with the spell.
\end{effects}
\end{ritual}

\begin{ritual}{Major Curse}

\range{20 feet + 15 / Rank}
\duration{Until removed or target dies}
\multiple{750}
\basechance{15\%}
\resist{Passive (unless a Death-curse)}
\target{Entity or Object}
\begin{effects}
An Adept’s Endurance value is decreased by one whenever they inflict a
major curse upon a being.  Note that when casting a Death-curse this
Endurance point loss is in addition to any possible Endurance point
loss due to resurrection. There are several types of major curses:

Affliction The Adept may choose to torment or kill their target. If
the effects of the affliction curse are intended to be deadly, the
target may not die as a direct result of the curse before (24 - Rank)
hours have passed. The following list of sample affliction curses is
provided to give the GM a guideline as to what major curses should be
allowed in their campaign.
\begin{Enumerate}
\item Target becomes totally blind, deaf or mute. 

\item Target becomes senile. 

\item Target suffers from a contagious disease (for example open
  running sores).

\item Target is transformed into a frog or other small creature.

\item Target  becomes  weakened  and  enfeebled  and 
must be helped with any physical action. 

\item Target falls into century-long sleep. 
\end{Enumerate}

\begin{Description}
\item[Ill Luck] Add two times the Rank of the major curse spell to any
  percentile roll involving the target or the use any of their
  abilities. This may not be applied favourably.

\item[Doom] A doom is a pronouncement, by the Adept, upon an event
  that will occur in the target’s future (e.g.  “You will die by the
  hand of a loved one.”).  The statement which must be indefinite will
  be true unless removed.  The GM should be careful as to what to
  allow for dooms.

\item[Death-curse] At the moment of their death, an Adept may
  automatically cast a major curse (unless backfire occurs).  The
  being at which it is cast may not resist the curse.  A Deathcurse
  must be an af-fliction, ill luck or doom.  If a doom, it will be
  gasped out with the Adept’s final breath.
\end{Description}
Note Lycanthropy is considered a major curse.
\end{effects}
\end{ritual}


\section{Special Knowledge Rituals}


\begin{ritual}{Remove Curse}

\duration{Immediate}
\multiple{500}
\resist{None}
\target{A curse}
\concentration{Standard}
\begin{effects}
Every curse is rated by the Magical Aptitude (MA) of the Adept who
cast it. If the curse is natural (such as Lycanthropy) it usually has
an MA of 20.

There are two types of curses, minor ones and major ones.  A minor
curse causes its victim to suffer from a non-fatal malediction.  Minor
curses come from various sources, for example the spells Evil Eye (G-9)
of the College of Ensorcelments and Enchantments), the Damnum Minatum
(G-1 of the Witchcraft College) and certain backfires.  Major curses
normally come from the Major Curse spell (§11.2).

When a ritual of curse removal has been completed, the GM rolls
percentile dice.  If the roll is less than or equal to the success
percentage the curse is removed.  If the roll is between one and two
times the success percentage, the curse re- mains in effect. If the
roll is equal to or greater than twice the success percentage, the MA
of the curse is increased by one. This ritual does not backfire in the
normal fashion.

\subsubsection{Minor Curse}

\basechance{(15 - MA of curse + 5 × Rank )\%}
\casttime{6 hours}
\actions{Inscribe symbol of power}

The Adept must inscribe a triangle or symbol of power about the cursed
being, and perform this ritual for six consecutive hours.

\subsubsection{Major Curse}

\basechance{(Adept’s MA - MA of curse + 2 × Rank)\%}
\casttime{18 hours}
\actions{Inscribe symbol of power}

The Adept must have a Magical Aptitude greater than that of the
curse. They must inscribe a triangle or symbol of power about the
cursed being, and perform this ritual for eighteen consecutive hours.
If the major curse is a death-curse, Base Chance is (Adept’s MA - MA
of curse + Rank)\%.

Precious Metals The use of triangles or symbols of power fashioned of
varying amounts of precious metals causes an addition to the success
percentage, per the following schedule:

\begin{tabularx}{\columnwidth}{lll}
Metal		& Add	& Cost \\
Silver		& + 3	& 1,000 sp \\
Gold		& + 7	& 10,000 sp \\
Platinum	& + 10	& 15,000 sp \\
Truesilver	& + 15	& 20,000 sp \\
\end{tabularx}

The symbol necessary for this ritual is large enough for the target to
sit in, and is inscribed in the ground. This symbol may be portable.
\end{effects}
\end{ritual}

\end{Chapter}
