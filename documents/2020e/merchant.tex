\begin{Chapter}{Merchant (Ver 1.1)}

Since  adventurers  are  highly  talented  individuals 
who  often  risk  their  lives,  and  a  person  is  usually 
compensated for the value of the work they do, the 
player  characters  will  fare  better  than  most  eco-
nomically.  A  merchant  character,  blessed  with  the 
ability  to  earn  even  more  Silver  Pennies,  has  the 
best  of  all  worlds.  Their  business  acumen  enables 
them to command a stiff price for those goods they 
vend,  and  to  acquire  that  which  they  covet  at  bar-
gain  rates.  The  merchant  is  not  often  fooled  in 
monetary  matters,  for  they  can  be  an  expert  in 
evaluating the worth of rare and costly goods. 

The  economies  of  most  DragonQuest  worlds  do 
not  promote  the  growth  of  capitalism.  Basically, 
the nobility has a vested interest in all rural lands, 
which comprise the vast majority of human-settled 
areas.  An  ambitious,  dynamic  merchant  could 
perhaps  own  the  entirety  of  a  large  town,  but  it  is 
quite  likely  that  a  jealous  duke  or  prince  would 
twist justice to break the merchant’s power. There-
fore,  it  behooves  a  merchant  to  cultivate  powerful 
allies when their holdings burgeon. 

41.1 Restrictions 
A  merchant  must  be  able  to  read  and  write  in  at 
least three languages at Rank 6 in order to use their 
assaying ability. 

41.2 Benefits 

The merchant’s ability to buy and sell a particu-
lar item is dependent upon its type. 

Any  item  will  be  classified  as  one  of  three  types: 
common,  uncommon,  and  rare  or  costly.  Items 
listed in the Players’ Handbook are of the common 
type. Jewellery set with semiprecious stones, spices 
from  another  continent,  and  fine  paintings  are 
examples  of  the  uncommon  type.  Rare  and  costly 
items  include  magic-invested  objects,  diamonds, 
roc’s eggs, giant slaves, etc. The GM must classify 
each item with which a merchant wishes to deal. 

A  merchant  can  purchase  items  at  a  cost 
cheaper than the asking price. 

If  the  result  is  odd,  the  quote  is  below  the  actual 
asking price; if even, it is above. 

Item Type 

Discount to Merchant  

[5 × Rank] %  
[2 × Rank] % 
[1 × Rank] % 

Common 
Uncommon 
Costly or Rare 
If the GM is actively playing the role of the seller, 
or  another  player  is  the  seller,  the  merchant  must 
do  their  own  haggling.  There  will  also  be  those 
items which the vendor cannot afford to sell at the 
usual discount to the merchant. The GM should use 
their discretion here. 

A  merchant  may  mark  up  the  price  of  an  un-
common or rare item. 

A  merchant  can  gain  (1.5  ×  Rank)%  above  the 
value of an uncommon item they are selling. They 
can gain (0.5 × Rank)% above the value of a costly 
or rare item they are selling. 

A  merchant  can  assay  an  item  to  determine  its 
exact worth. 

The  player  characters  will  generally  receive  a  fair 
quote on the price of basic goods, but must accept 
the word of the being with whom they are dealing 
when  conducting  a  transaction  involving  uncom-
mon,  rare  or  costly  items.  The  odds  of  the  player 
characters  being  bilked  increase  as  they  venture 
forth from their native land(s). However, if a mer-
chant is amongst them, they can assay the value of 
any item after (11 - Rank) minutes. 

The  success  percentage  for  assaying  a  common 
item is equal to the merchant’s (Perception + 12 × 
Rank)%,  to  assay  an  uncommon  item  equal  to 
(Perception  +  9  ×  Rank)%,  and  to  assay  a  rare  or 
costly  item  equal  to  (Perception  + 6  ×  Rank)%.  If 
the  GM’s  roll  is  equal  to  or  less  than  the  success 
percentage  the  merchant  is  told  the  exact  value  of 
the  item  in  question.  If  the  roll  is  greater  than  the 
success  percentage,  the  GM’s  quote  increasingly 
diverges from reality as the result approaches 100. 

A  merchant  may  use  their  skill  to  affect  transac-
tions  involving  up  to  (250  +  50  ×  Rank  Squared) 
Silver Pennies per month, or a single transaction of 
any amount. 

The merchant must buy and sell at the asking price 
for any transactions over their monthly limit. 

A merchant can specialise in a specific category 
of item assaying for every three full ranks. 

The  merchant  chooses  their  speciality  from  the 
following list (and any the GM should add): 

1.   Ancient Writings 
2.   Antiques 
3.   Archaeological Finds 
4.   Art 
5.   Books 
6.   Gems 
7.  
8.  
9.   Magic Items 
10.   Monster and Animal Products (e.g. furs, 

Jewellery 
Land 

eggs) 

11.   Precious Metals 
12.  Slaves 
When  a  merchant  assays  an  item  of  a  category  in 
which  they  specialise,  they  add  (2  ×  Rank)%  to 
their  success percentages.  It  is  possible  for  a  mer-
chant to attain a 100% chance of accurately pricing 
a speciality item (exception to 90% + Rank limit). 

If  a  merchant  wishes  to  add  additional  specialities 
without increasing in rank, they must expend 4,000 
Experience  Points  and  4  weeks  of  training  per 
speciality.  These  costs  are  discounted  by  25%  if 
the merchant has reached rank 8, or by 50% if they 
have reached rank 10. 

\end{Chapter}
