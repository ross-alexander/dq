\begin{Chapter}{Spy (Ver 2.0)}

Amongst the many professionals encountered in everyday life, there
will be a scattering of those with the covert skill of Spy.  This is a
profession dealing with obtaining and distributing information.
There are basic information gathering and remembering abilities in
common with all spies, but the methods of operation, and spheres of
influence vary from the court ambassador to the military scout to
the pub minstrel. A spy may specialise in a particular field of
operation and if they have the accompanying skills appropriate for
their cover, will perform better, with less chance of their actions
being discovered and their cover blown.

\section{Restrictions}

There are no restrictions on learning the spy ability, or to using it,
except perhaps the fear of being discovered.  Note that having learnt
the Spy skill is not the same as actually spying.

\subsection{Discovery}

If the spy catastrophically fails in their skill by rolling greater
than 30\% over the base chance for the ability, or 90\% + Rank
(whichever is the lesser BC), then they may have been discovered.  The
repercussions of this discovery is based on their situation: what
exactly they were attempting, who the discoverer is, the difficulty of
the task com- pared to their rank in the skill, whether the spy has
other skills to back up their cover etc.  The repercussions may range
from a slap in the face, being given disinformation, expulsion from an
inn/town, a beating in the back alley, through to the traditional
punishment for an exposed traitor: to be drawn and quartered,
(although nobles are sometimes beheaded). Thus, a good spy is the one
least likely to be discovered.

\section{Benefits}

All spies gain grounding in basic spy craft, including memory
enhancement, moving quietly, observing closely and communicating with
their peers.

\subsection{Enhanced Memory}

\begin{Itemize}

\item A spy may memorise and recall visual details, such as those of a
  room, a person, or a piece of parchment, etc. Memorisation requires
  (120 - 10 × Rank) seconds of undisturbed concentration, studying
  the object.

\item A spy’s chance to recall a memorised image accurately is (2 ×
  Perception + 12 × Rank)\% rolled by the GM.  If the failed recall
  attempt occurs within 1 + (1 × Rank) days of the memorisation, the
  spy merely cannot remember. After this period, their effective rank
  for recall reduces by one per subsequent day, and with a failed
  roll, erroneous information may be remembered instead.  If a spy
  fails to recall an object or place, they may not attempt to recall
  it again until they study it again.

\item A spy may also use this ability to recall spoken phrases and
  combination of sounds.  Even if a spy does not know the language
  used, they can reproduce the phrases phonetically.

\item A spy’s enhanced memory gives them an effective understanding of
  any spoken language that they have at Ranks 0-3 as if one rank
  higher.

\end{Itemize}

\subsection{Stealth}

A spy increases their chance of acting stealthily by 2\% per Rank. 

\subsection{Fieldcraft}

A spy is trained to notice, recognise, and appropriately interact with
other spies. This includes hand- ing off messages discretely, the use
of dead-drops, safe-houses and identification phrases, as well as
foiling the less subtle attempts to interfere with such exchanges.
Also, they may identify the weak willed, manipulable, morally suspect
or gullible individuals that a spy prefers to associate with, and has
a greater chance of enticing them to a course of action by coaxing,
flattery, wheedling, blackmail or other enticement.

\subsection{Optional Abilities}

In addition, a spy gains an ability chosen from the list below with
each Rank (including Rank 0).  Additional abilities may be gained
without increas- ing in rank by the expenditure of 2,500 Experience
Points and 4 weeks of training.  These costs are discounted by 25\% if
the Spy has reached rank 8, or by 50\% if they have reached rank
10. Individual Base Chances are provided for some of the various spy
abilities; for the other skills there is a generic Base Chance of 3 ×
appropriate characteristic (+ 5 / Rank), modified by difficulty.

\begin{Description}
\item[Assess] a spy can infer some information from observing the
  grouping and activities of people.  The level of information gained
  is logistical, rather than the in depth knowledge that knowing the
  associated skill would give. Example uses include:
\begin{Itemize}
\item Estimate Entourage ascertain the likely social ranking of the
  target based on the size, quality, snootiness etc of accompanying
  servants.  Evalua- tion of who they are or the implications of the
  household makeup would require Courtier.

\item Estimate Goods estimate the number of boxes, and the people,
  wagons, time and other logistics required to move them.  Evaluation
  of quality or value would require Merchant.

\item Troop Estimation estimate size, equipment and quality of a
  military force or navy.  Inferring the tactics this represents would
  require Military Sci- entist.
\end{Itemize}

\item[Befriend] over a period of days or weeks, a spy may target
  individuals to engender trust in themselves, and possibly distrust
  in others. The “friend” begins to willingly and unknowingly reveal
  information to the spy.

\item[Bribery] a spy may recognise appropriate people and which
  “gifts” they prefer, to gain information or access. They may also
  recognise situations when bribing will not work, before making the
  attempt.

\item[Codes] a spy can recognise and use simple codes.  They may
  attempt to break others’ codes and ciphers.  The length of study
  required to do this depends on the familiarity of the code’s style
  and the difficulty set by the code maker.

\item[Counterspy] a spy may perceive other current spying activity and
  recognise the other spies. Also, a spy may create convincing
  artifices that fit with available information, to spread
  disinformation, misunderstanding or confusion.

\item[Disguise] with the use of physical props and resources, the spy
  can apply makeup, dyes, false hair etc to convincingly alter the
  appearance of people.  The success of disguising race and gender
  depends on the physical similarity of the person, and how closely
  the disguised entity is inspected. A spy may only attempt to imitate
  a specific person’s appear- ance after prolonged study of the
  target.  Animals the spy is familiar with may also be disguised.

\item[Forgery] With the right materials to hand, a spy can create
  convincing replicas of personal letters, official documents and the
  like.  This includes the ability to open and re-seal letters,
  produce false seals, or move real ones to a forgery.  The chance for
  this is MD + PC + (4 × Rank). If a spy is literate with the language
  of the document, then the language rank can be added to the base
  chance for forgery.

\item[Hiding] a spy can find unlikely hiding places and conceal
  themselves for long periods of time, keeping still and quiet. The
  spy must roll WP + EN + (5 × Rank) to maintain this for extended
  periods of time.

\item[Imitation] the spy can study and copy behaviour, mannerisms, and
  accents. This will allow the spy to maintain the roles of ordinary
  people and not stand out as a foreigner.  As a guide, this takes 6 -
  (Rank/2) hours of exposure to the society, with modifiers based on
  the apparent familiarity or strangeness of the society.

\item[Information] by using other skills and knowledge a spy has an
  increased chance of getting the most relevant information they are
  seeking.  They have an increased chance of noticing disinformation
  about their area of knowledge, and distinguishing fabrication and
  pretence from fact and reality.  The spy may recognise information
  as having worth to other individuals or spies in their network, even
  though it is not useful to them.

\item[Lip Reading] a spy may understand spoken conversations outside
  of their hearing range.  A clear line of sight to the targets’ face
  and knowledge of the language being spoken (min rank 6) is required.
  The spy may garner fragments but not all of the conversation if they
  are not fluent with the language, can only see one of the
  participants, or an unlikely topic is being discussed.

\item[Pick Locks] while using appropriate tools, a spy can spend (240
  - (20 × Rank) ) seconds to attempt to pick a lock to either unlock
  or lock it. The base chance of the attempt is (MD + (4 × Rank)) - (6
  × Lock Rank). If the attempt fails, the lock resists the attempt.  A
  catastrophic failure may damage the lock.

\item[Resist Torture] being familiar with extraction techniques, a spy
  may add their Rank to their effective WP for resisting torture
  attempts.  They may also choose to apparently lower their WP by up
  to Rank to release false information.

\item[Shadowing] a spy learns the skills of following individuals at a
  distance without being observed. In addition they have an increased
  chance of noticing when they themselves are being followed, and may
  attempt to lose their followers if the appropriate terrain is
  available. To spot a tail the Spy must roll under (2 × PC) + (5 ×
  Rank) - 5 times opposing Spy’s rank.  To successfully tail another
  spy requires a roll of PC + AG + (5 × Rank) - 5 times opposing Spy’s
  rank.

\item[Sleight of Hand] a spy can palm, swap or place small objects,
  without attracting notice.  Removing objects from people requires
  the Pick Pockets skill from Thief.
\end{Description}

A GM may give the following abilities to a Spy over the course of
play, or they may be requested by a player as part of their character
knowledge and backgrounds.

\begin{Description}

\item[Network (area)] the spy has joined a spy ring in a specific
  area, and knows the specific field craft routines used by that ring.

\item[Spy Master (area)] the spy has set up a spy ring in a specific
  area, and knows where to place spies, how to store information for
  later reference, and how to manage other spies.  This ability
  requires Rank 8 plus.

\end{Description}

\end{Chapter}
