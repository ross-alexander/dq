\begin{Chapter}{Adventuring Skills (Ver 1.2)}

These skills may be ranked as with any other skill. 
The  only  differences  are  that  all  characters  start 
with  swimming,  climbing,  stealth  and  horseman-
ship  at  Rank  0,  and  if  the  skill  is  used  conspicu-
ously  during  an  adventure  it  can  be  ranked  once 
without  the  need  for  training  time,  but  there  must 
be  a  tutor  with  a  similar  skill  who  is  present  to 
advise  the  character  on  the  technique  they  should 
employ. 

29.1 Climbing 
This  skill  allows  a  character  to  climb  anything 
from walls to mountains without the aid of special-
ised  equipment, if  this is  at all  possible.  The  Base 
Chance  to  use  this  skill  is  (4  ×  MD  +  8  ×  Rank  - 
[structure height in feet / 10])%. A character using 
this  skill  should  make  a  roll  at  approximately  20’ 
intervals,  but  if  the  climb  is  especially  difficult, 
every  10’.  Note  that  the  GM  may  modify  the  for-
mula in certain instances. 

A  climber  suffers  ([Height  of  fall  (in  feet)  /  10] 
Squared) Endurance Points when they fall. 

Various  items  of  equipment  may  be  used  to  im-
prove a character’s chance of climbing as follows: 

1. Climbing Claws add 15% to BC but have no use 
for rock climbing where hands are more use. 

2. Rope allows the user to climb the structure mak-
ing  only  one  roll  but  are  only  useful  where  ropes 
may be practically used. 

29.2 Horsemanship 
A  character  will  use  horsemanship  to  direct  ani-
mals  which  they  ride.  A  character  may  use  their 
horsemanship  with  any  animal  or  monster  which 
they  would  ordinarily  ride  (such  as  horses,  don-
keys, camels, elephants, etc.). Enchanted or Fantas-
tical monsters do not necessarily fall into this cate-
gory,  and  the  GM  must  make  rulings  governing 
these situations. 

The actual occurrence must be decided by the GM 
and  should  become  worse  the  farther  the  roll  is 
above the modified percentage. 

If  the  GM  judges  the  rider  has  totally  lost  control 
of  their  mount,  the  rider  may  take  no  other  action 
until  they  have  regained  control  (presuming  they 
manage to stay mounted). 

Using horsemanship while in combat may be done 
in  combination  with  any  other  Action.  A  trained 
rider receives certain abilities as they rise in Rank: 

Rank 
3  

Rank 
5  

Rank 
7  

May use two-handed weapons 

May fire a missile weapon or cast a spell 
while moving 

May  use  two  one-handed  weapons  at 
once 

29.3 Flying 
Flying is the skill of performing aerial manoeuvres 
using  magical  flying.  As  a  rule  aerial  combat  is 
difficult. Flying is an adventuring skill. 

A character may always take off, fly, or land in an 
appropriate manner and reasonable conditions, and 
under such circumstances no roll is necessary. Note 
that  landing  appropriately  is  not  precise.  The  suc-
cess  chance  to  perform  a  complex  aerial  manoeu-
vre  with  precision is  (3  ×  AG  +  10  ×  Rank).  This 
base chance may be modified by the following: 

 
 
 

0 to -50 
+10 to -50 
0 to -m/hr 

Environmental conditions. 
Type of flight used. 
Speed. 
Flying into an obstacle causes up to [D + (relative 
speed  in  miles  per  hour  /  10)  squared]  endurance 
damage. The nature of the obstacle may reduce the 
damage.  Specific  Grievous  injuries  (normally  C 
class)  may  also  be  incurred.  See  Climbing  (§29.1) 
for falling (as opposed to flying) damage. 

The  character’s  player  will  roll  percentile  dice 
whenever their horsemanship is called into play. A 
character’s  horsemanship  is  equal  to  [(modified 
AG + WP) / 2 + Rank × 8], round down. 

As  a  rule  of  thumb, an  airborne clothed humanoid 
who  falls  through  the  air  drops  350ft  in  the  first 
pulse, 650ft in the second, and 1000ft in each sub-
sequent pulse. 

The type of mount a character is riding will modify 
their horsemanship as follows: 

Donkey 
Mustang † 
Quarterhorse 
Dire Wolf 
Draft Horse 
Elephant 
†Rating unless trained by rider; in that case, 0. 

-10   Palfrey 
-12  Warhorse † 
-10  Camel 
-10  Mule 
-5 
Pony 
 
-10 

+15 
-5 
-15 
-8 
+10 
 

The GM should also take into account the familiar-
ity  the  character  has  with  the  individual  animal 
type and apply modifiers thereby (i.e. the first time 
a  character  finds  themselves  atop  a  camel  should 
be worth at least an additional - 15). 

A  character’s  horsemanship  is  called  into  play 
whenever  they  wish  their  mount  to  perform  an 
unusual  or  difficult  action.  Any  mount  can  be 
directed  into  moving  at  a  walking  pace  or  even  a 
brisk  trot;  an  unusual  or  difficult  action  would  be 
to break into a gallop or charge, jump an obstacle, 
etc.  During  combat,  horsemanship  is  called  into 
play during every Pulse to a) keep the mount con-
trolled,  b)  regain  control  if  it  is  lost,  and  c)  direct 
the  mount  to  take  any  specific  Action.  Remember 
only a Warhorse can be directed to enter into Close 
Combat by its rider, and all other mounts will only 
attack if directly assaulted. 

A  successful  roll  will  result  in  the  mount  obeying 
the  directions  of  the  rider.  A  roll  above  the  modi-
fied percentage but less than the modified percent-
age plus the rider’s WP indicates the mount either 
does  nothing  or  continues  to  do  whatever  it  was 
doing.  A  roll  above  both  of  these  indicates  the 
mount  will  either  disobey  the  rider,  buck,  attempt 
to throw the rider, or some other unpleasant result. 

Note  that a  speed  of  one  mile  per  hour  is  equal  to 
30 yards per minute in the chase sequence and 1.5 
hexes per pulse in combat. 

A  trained  magical  flier  receives  certain  combat 
abilities as they rise in rank. 

Rank 
3  
Rank 
5  
Rank 
7  

May use two-handed weapons 

May fire a missile weapon or cast a spell 
while moving 
May use two one-handed weapons at 
once 

29.4 Stealth 
A character can use stealth to move as soundlessly 
and unobtrusively as possible. 

A  character  may  use  their  stealth  ability  only  if 
they  have  adequate  cover  (i.e.  space  in  which  to 
conceal  or  obscure  themselves)  in  the  area  they 
wish  to  traverse,  they  are  appropriately  clad  (e.g. 
not  in  plate  armour  or  luminescent  clothing),  and 
they  are  not  currently  under  observation  by  the 
entities from whom they are attempting to conceal 
their presence. 

The  GM  will  roll  percentile  dice  to  determine  if  a 
character is able to use their stealth ability success-
fully. The GM only makes such a check if there is 
a reasonable possibility that the character could be 
detected.  The  GM  makes  one  check each  time  the 
character  attempts  one  continuous  action,  or  each 
time  an  unexpected  change  of  condition  has  a 
significant  effect  upon  the  character’s  chance  of 
remaining  hidden  (e.g.  one  of  the  entities  under 
surveillance heads for a room which happens to be 
through  the  doorway  in  which  the  character  is 
hidden). The GM may modify the success percent-
age. 

104 

A  character’s  base  chance  of  using  their  stealth 
ability is (3 × Agility + 5 × Rank + Thief Rank + 2 
×  Spy  Rank  + 2  ×  Assassin Rank)%.  The  greatest 
Perception value of the entities who may be able to 
discover  the  character  using  the  stealth  ability  is 
subtracted if those entities are unaware of the char-
acter’s  presence,  or  three  times  that  Perception 
value if they are. 

29.5 Swimming 
This  skill  is  required  in  order  to  perform  any  ac-
tions  in  the  water.  All  player  characters  start  off 
with  Rank  0.  This,  under  good  conditions,  will 
allow  the  character  to  tread  water  in  order  to  stay 
afloat. The higher the rank, the more the character 
will be able to do until they are at the stage where 
they  can  swim  like  a  fish  and  survive  even  in  ad-
verse conditions. 

Base Chance 

The base chance for swimming is PS + AG + EN + 
8  ×  Rank  and  is  modified  by  the  following  (all 
adjustments cumulative): 

-1 
+5 to -25 
+10 to -25 
-10 to -50 

Wearing no or little clothing  +10 
Encumbered (per pound) 
Water Temperature 
Water Conditions 
May not swim freely 
Other  modifiers  may  be  applied  by  GM  as  appro-
priate.  An  unsuccessful  skill  roll  does  not  imply 
drowning (yet) but the character could be in serious 
trouble.  If  they  are  trying  to  float  and  the  roll  is 
failed  then  they  need  to  make  another  successful 
skill  roll  in  order  to  stay  afloat.  Two  failed  skill 
rolls  implies  they  are  underwater,  holding  their 
breath, without preparation. 

If  an  Adept  is  attempting  to  cast  then  they  can  do 
so,  within  the  restrictions  of  their  College,  if 
breathing  water  or  if  they  make  a  successful  skill 
roll.  A concentration check (3 × WP) may also be 
required in adverse conditions. 

Breath Holding 

The  base  time  a  character  can  hold  their  breath  is 
(current  EN  /  3  +  swimming  Rank  /  2)  pulses 
rounded  up.  This  time  is  doubled  if  a  Pass Action 
is used in the previous pulse to prepare. 

Drowning 

Once  that  time  is  expired  then  the  character  must 
make a 5 × WP check in order to continue holding 
their  breath.  At  the  end  of  subsequent  pulses,  the 
WP factor is reduced by 1 until the roll fails. 

At  that  point  the  character  starts  drowning,  taking 
physical  damage  at  a  rate  of  D10  EN  per  pulse 
until  death  or  rescue.  A  drowning  character  needs 
to make a 2 × (WP + swimming rank) check before 
being able to perform useful activity as above. 

Sight and Communication 

The character can see PC hexes in clear water. This 
is  halved  in  lakes  and  rivers  because  of  algae  and 
silt. 

Communication  is by  sign  language, or  a  range  of 
one hex if speaking. 

Movement Rates 

Swimming TMR = (Land TMR + Rank) / 3. Walk-
ing  on  the  bottom  (if  weighted)  =  Land  TMR  /  3. 
Swimming is generally a hard or strenuous activity 
unless the entity concerned is an aquatic. 

Characters  that  are  encumbered  by  non-buoyant 
materials descend at the following rates: 

Unencumbered to 5 lbs 
5–10 lbs encumbrance 
10–15 lbs 
15–20 lbs 
20–25 lbs 
25+ 
Unencumbered  characters  floating  to  the  surface 
(e.g. if unconscious) do so at 1 ft per pulse. 

0 ft per pulse 
1 ft per pulse 
2 ft per pulse 
3 ft per pulse 
4 ft per pulse 
5 ft per pulse 
\end{Chapter}
