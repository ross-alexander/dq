\begin{Chapter}{Troubadour (Ver 2.1)}

A troubadour is a multi-talented entertainer and performer, and a well
skilled troubadour may be an actor, poet, mimic and musician. The most
powerful ability that a troubadour can gain is the Bardic Voice,
which enables them to influence all but the deaf.

\section{Benefits}
All troubadours gain a grounding in stagecraft.  They become able to
size up an audience and determine what form of entertainment will be
the best to perform, and how to handle interjection and ridicule.

In addition, a troubadour gains 3 abilities at Rank 0, and one further
ability per Rank. All abilities are usually performed at the overall
Rank of the troubadour.  However, a troubadour may choose to
specialise.  If, upon gaining a new Rank (or an additional ability
without increasing in rank), the troubadour wishes to forego gaining a
new ability, they may specialise in one of the abilities that they
already possess. That ability then operates at (troubadour’s Rank +
1), to a maximum of (Rank + specialisation) of 10.  A troubadour may
specialise more than once with the same ability, gaining Rank + 2,
Rank + 3, etc.  Additional abilities may be gained without increasing
in rank by the expenditure of 1,000 Experience Points and 4 weeks of
training per ability.  These costs are discounted by 25\% if the
troubadour has reached rank 8, or by 50\% if they have reached rank
10.

Individual Base Chances are not provided for the various troubadour
skills; rather, there is a generic Base Chance of 3 × appropriate
characteristic (+ 5 / Rank), modified by the GM to reflect the
difficulty of the feat being attempted.

The abilities available to a troubadour are:

\begin{Description}
\item[Acrobatics] Mostly involves tumbling across the ground, but also
  performing manoeuvres after swinging from a trapeze, rope, or bar;
  jumping from a springboard, or high ledge.

\item[Acting] Portraying fictitious personalities and devising
  rationales for assumed identities.  Usually involves accentuated and
  exaggerated actions and emotions.

\item[Bardic Voice] (see below).  Note: A troubadour may not
  specialise in Bardic Voice.

\item[Comedy] The use of timing, inflection and language to cause
  merriment or laughter. Also writing both jokes and skits.  Comedy
  may also be combined or included in many other art forms.

\item[Dance] Mostly traditional, often rural dances, performed for an
  audience; also includes creating new dances.

\item[Fire Eating] Appearing to swallow and/or produce flame, usually,
  from the mouth.  To do this, a fire eater requires a special liquid,
  which may be purchased from an Alchemist for a modest fee.

\item[Juggling] Throwing and catching objects. A juggler is able to
  keep up to 1 (+ 1 / Rank) items, of equal weight and size, in the
  air at the same time.  If the items juggled are of a different size
  and/or weight, each difference counts as another item juggled.

\item[Make-up] Using props, stage makeup, and items such as wigs, fake
  beards, and wax noses, a troubadour can portray a character of a
  different age, race, sex, or profession to their own.

\item[Mime] Using only the performer’s body, and its movements, to
  convey an idea, describe a scene, tell a tale, or entertain.

\item[Mimicry] Imitating sounds and voices accurately and believably.

\item[Patter] Talking interestingly, seemingly non-stop, either as
  advertising for a show or as a misdirecting part of a performance.

\item[Play an Instrument] This ability may be taken several times with
  different instruments.  A singer is one who has play instrument
  (voice).  A troubadour can usually play similar instruments to the
  ones they have chosen at (Rank / 2).

\item[Poetry] Creating and reciting poetry, including lengthy epics
  running to hundreds of lines.

\item[Prestidigitation] Manipulation of small articles such as coins,
  eggs, or pebbles to make them move, disappear and reappear in
  unusual and entertaining ways. This ability also gives a bonus to
  the casting of all Cantrips of 2\% (+ 2 / Rank).

\item[Production] Play writing and turning a play into a successful
  production.  Includes set design and sound effects. The higher the
  Rank, the less likely it is that a major catastrophe will befall the
  production through something having been forgotten or overlooked.

\item[Puppetry] Writing a story to be performed by puppeteers, and
  performing a story or play with puppets.

\item[Stilt Walking] Balancing and walking on stilts of up to 50\% (+
  20\% / Rank) of the troubadour’s height.

\item[Storytelling] Creating and reciting stories for an audience.

\item[Sword Swallowing] Controlling the mouth, tongue and throat such
  as to be able to allow long, rigid props to pass into the throat.

\item[Tightrope] Walking Walking, balancing, and turning on a taut
  raised rope, or narrow beam.

\item[Ventriloquism] The ability to speak without moving the lips and
  make the voice seem to come from any location up to (Rank / 2) feet
  away.

\end{Description}

\section{Bardic Voice}

A troubadour may use their Bardic Voice in an attempt to influence an
audience.  Beings who are affected will see the troubadour as their
friend, and the troubadour’s words as wise and well meant.  Bardic
Voice may be used, for example, to calm a lynch mob, or to begin a
riot against a cruel tyrant.

The troubadour begins speaking to key elements in the crowd, stirring
their emotions and playing upon their beliefs and feelings. All beings
to be affected must be within earshot, and capable of understand- ing
the language used by the troubadour. When the troubadour begins to use
this ability they may enthrall up to (4 + 6 / Rank) beings, with (15 -
Rank) minutes being required to work their skill.  Once they have
spent the required time, the troubadour makes a Check to see if they
are having the desired effect.  If successful, the troubadour may
elect to use their voice again on the same crowd.  By doing this they
may double the number of beings whose attention they have captured.

Using Bardic Voice is tiring, and a troubadour must expend 4 FT each
time that they use this ability.  A troubadour may use their voice
continuously upon a crowd until they exhaust their FT, they reach
the limit of the size of audience, they fail a Bardic Voice roll or
they have doubled (Rank / 2, round down) times.  The Base Chance is
50\% (+ 5 / Rank), modified by the GM for the reasonableness of the
troubadour’s suggestions and the audience’s predisposition to certain
actions.

\end{Chapter}
